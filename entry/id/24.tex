\documentclass[]{article}
\usepackage{lmodern}
\usepackage{amssymb,amsmath}
\usepackage{ifxetex,ifluatex}
\usepackage{fixltx2e} % provides \textsubscript
\ifnum 0\ifxetex 1\fi\ifluatex 1\fi=0 % if pdftex
  \usepackage[T1]{fontenc}
  \usepackage[utf8]{inputenc}
\else % if luatex or xelatex
  \ifxetex
    \usepackage{mathspec}
    \usepackage{xltxtra,xunicode}
  \else
    \usepackage{fontspec}
  \fi
  \defaultfontfeatures{Mapping=tex-text,Scale=MatchLowercase}
  \newcommand{\euro}{€}
\fi
% use upquote if available, for straight quotes in verbatim environments
\IfFileExists{upquote.sty}{\usepackage{upquote}}{}
% use microtype if available
\IfFileExists{microtype.sty}{\usepackage{microtype}}{}
\usepackage[margin=1in]{geometry}
\usepackage{color}
\usepackage{fancyvrb}
\newcommand{\VerbBar}{|}
\newcommand{\VERB}{\Verb[commandchars=\\\{\}]}
\DefineVerbatimEnvironment{Highlighting}{Verbatim}{commandchars=\\\{\}}
% Add ',fontsize=\small' for more characters per line
\newenvironment{Shaded}{}{}
\newcommand{\AlertTok}[1]{\textcolor[rgb]{1.00,0.00,0.00}{\textbf{#1}}}
\newcommand{\AnnotationTok}[1]{\textcolor[rgb]{0.38,0.63,0.69}{\textbf{\textit{#1}}}}
\newcommand{\AttributeTok}[1]{\textcolor[rgb]{0.49,0.56,0.16}{#1}}
\newcommand{\BaseNTok}[1]{\textcolor[rgb]{0.25,0.63,0.44}{#1}}
\newcommand{\BuiltInTok}[1]{#1}
\newcommand{\CharTok}[1]{\textcolor[rgb]{0.25,0.44,0.63}{#1}}
\newcommand{\CommentTok}[1]{\textcolor[rgb]{0.38,0.63,0.69}{\textit{#1}}}
\newcommand{\CommentVarTok}[1]{\textcolor[rgb]{0.38,0.63,0.69}{\textbf{\textit{#1}}}}
\newcommand{\ConstantTok}[1]{\textcolor[rgb]{0.53,0.00,0.00}{#1}}
\newcommand{\ControlFlowTok}[1]{\textcolor[rgb]{0.00,0.44,0.13}{\textbf{#1}}}
\newcommand{\DataTypeTok}[1]{\textcolor[rgb]{0.56,0.13,0.00}{#1}}
\newcommand{\DecValTok}[1]{\textcolor[rgb]{0.25,0.63,0.44}{#1}}
\newcommand{\DocumentationTok}[1]{\textcolor[rgb]{0.73,0.13,0.13}{\textit{#1}}}
\newcommand{\ErrorTok}[1]{\textcolor[rgb]{1.00,0.00,0.00}{\textbf{#1}}}
\newcommand{\ExtensionTok}[1]{#1}
\newcommand{\FloatTok}[1]{\textcolor[rgb]{0.25,0.63,0.44}{#1}}
\newcommand{\FunctionTok}[1]{\textcolor[rgb]{0.02,0.16,0.49}{#1}}
\newcommand{\ImportTok}[1]{#1}
\newcommand{\InformationTok}[1]{\textcolor[rgb]{0.38,0.63,0.69}{\textbf{\textit{#1}}}}
\newcommand{\KeywordTok}[1]{\textcolor[rgb]{0.00,0.44,0.13}{\textbf{#1}}}
\newcommand{\NormalTok}[1]{#1}
\newcommand{\OperatorTok}[1]{\textcolor[rgb]{0.40,0.40,0.40}{#1}}
\newcommand{\OtherTok}[1]{\textcolor[rgb]{0.00,0.44,0.13}{#1}}
\newcommand{\PreprocessorTok}[1]{\textcolor[rgb]{0.74,0.48,0.00}{#1}}
\newcommand{\RegionMarkerTok}[1]{#1}
\newcommand{\SpecialCharTok}[1]{\textcolor[rgb]{0.25,0.44,0.63}{#1}}
\newcommand{\SpecialStringTok}[1]{\textcolor[rgb]{0.73,0.40,0.53}{#1}}
\newcommand{\StringTok}[1]{\textcolor[rgb]{0.25,0.44,0.63}{#1}}
\newcommand{\VariableTok}[1]{\textcolor[rgb]{0.10,0.09,0.49}{#1}}
\newcommand{\VerbatimStringTok}[1]{\textcolor[rgb]{0.25,0.44,0.63}{#1}}
\newcommand{\WarningTok}[1]{\textcolor[rgb]{0.38,0.63,0.69}{\textbf{\textit{#1}}}}
\ifxetex
  \usepackage[setpagesize=false, % page size defined by xetex
              unicode=false, % unicode breaks when used with xetex
              xetex]{hyperref}
\else
  \usepackage[unicode=true]{hyperref}
\fi
\hypersetup{breaklinks=true,
            bookmarks=true,
            pdfauthor={Justin Le},
            pdftitle={Pipes: Streaming Huffman Compression in Haskell (Part 3)},
            colorlinks=true,
            citecolor=blue,
            urlcolor=blue,
            linkcolor=magenta,
            pdfborder={0 0 0}}
\urlstyle{same}  % don't use monospace font for urls
% Make links footnotes instead of hotlinks:
\renewcommand{\href}[2]{#2\footnote{\url{#1}}}
\setlength{\parindent}{0pt}
\setlength{\parskip}{6pt plus 2pt minus 1pt}
\setlength{\emergencystretch}{3em}  % prevent overfull lines
\setcounter{secnumdepth}{0}

\title{Pipes: Streaming Huffman Compression in Haskell (Part 3)}
\author{Justin Le}
\date{June 26, 2014}

\begin{document}
\maketitle

\emph{Originally posted on
\textbf{\href{https://blog.jle.im/entry/pipes-streaming-huffman-compression-in-haskell-part-3.html}{in
Code}}.}

Let's finally finish up our Streaming Huffman Compression project by actually
implementing the ``streaming'' part :) In
\href{http://blog.jle.im/entry/streaming-huffman-compression-in-haskell-part-1-trees}{part
1} we looked at the data structures which we used to implement our compression
logic; in
\href{http://blog.jle.im/entry/streaming-huffman-compression-in-haskell-part-2-binary}{part
2} we looked at the actual compression/decompression algorithm and implemented
it. Finally, let's wrap it all up and actually implement a streaming interface!

If we were using an imperative approach, this would usually involve some sort of
loop --- read a byte, process it, write the resulting byte, read the next,
process it, write it\ldots{}it's a step of instructions that a computer will be
able to perform step-by-step.

In Haskell, when we can, we try to look for a pure, declarative approach based
on compositions of abstractions. That's what Haskell does best, after all. So
let's see what we can do!

(All of the code in this article and the ones before can be downloaded
\href{https://github.com/mstksg/inCode/tree/master/code-samples/huffman}{from
github}, so you can download it and try it out yourself!)

\hypertarget{pipes}{%
\section{Pipes}\label{pipes}}

\hypertarget{choosing-pipes}{%
\subsection{Choosing Pipes}\label{choosing-pipes}}

So we are searching for an abstraction to handle \emph{constant-space} IO
streaming --- that is, we only ever have in memory exactly what we are
processing at that moment, and nothing else. For this, there are a couple go-to
abstractions we can use that provide this (at the low level).

We can use lazy IO, which basically relies on Haskell's built in laziness
semantics that we know and love to control when IO happens. The problem here is
that your IO actions are no longer
\href{http://blog.jle.im/entry/the-compromiseless-reconciliation-of-i-o-and-purity}{first-class
members} of the language --- they are ``runtime magic''. You can no longer
really reason about when file handles are closed and exactly when reads happen.
This really is a bit antithetical to Haskell, a language where we actually have
the ability to move IO into a first-class member of the language and make it
something that we can actually reason about.

There have been many solutions developed to this problem and in modern times,
\href{https://hackage.haskell.org/package/conduit}{conduit} and
\href{http://hackage.haskell.org/package/pipes}{pipes} have emerged, built on
the backs of early coroutine-based libraries. These libraries are built on the
idea of purely assembling and ``declaring'' the IO pipeline that you want, with
each pipeline component having very explicit and comparable and
able-to-reason-with IO read/write/close semantics.

The choice between \emph{conduit} and \emph{pipes} depends a lot on what you
want to accomplish. There was a very nice
\href{http://www.haskellcast.com/episode/006-gabriel-gonzalez-and-michael-snoyman-on-pipes-and-conduit/}{Haskell
Cast} episode on this matter (and more) that I would highly recommend. Both
libraries come from very different backgrounds and histories.

This picture is slightly simplified, but \emph{conduit} focuses around safe
resource handling, and \emph{pipes} focuses on equational reasoning and applied
mathematical abstractions.

I'm picking \emph{pipes} for this tutorial, for no major reason. All of this
could be written in \emph{conduit} with little difference in code size or
expressiveness, I'm sure. I mostly chose \emph{pipes} because I wanted to
demonstrate some of the nice reasoning that pipes enables that Haskell is so
famous for. I also just wanted to learn it, myself :)

\hypertarget{before-we-go}{%
\subsection{Before we go}\label{before-we-go}}

Before you proceed, it is recommended that you read over or are at least
somewhat familiar with the excellent
\href{http://hackage.haskell.org/package/pipes-4.1.2/docs/Pipes-Tutorial.html}{pipes
tutorial}, which is a part of the actual pipes documentation. This post does not
attempt to be a substitute for it, only a ``what's next?''.

Now, we are going to be using a bit more than just plain old \emph{pipes} for
our program. In addition to the libraries used in our previous parts, we're
going to be using:

\begin{enumerate}
\def\labelenumi{\arabic{enumi}.}
\tightlist
\item
  \emph{\href{http://hackage.haskell.org/package/pipes-parse}{pipes-parse}}, for
  leftover support. We're going to be using limited leftover handling for this
  project in a couple of situations.
\item
  \emph{\href{http://hackage.haskell.org/package/pipes-bytestring}{pipes-bytestring}},
  which provides lenses for us to manipulate bytestring and byte producers in
  efficient and expressive ways.
\end{enumerate}

Today, our work with \emph{pipes} will revolve around a couple of main concepts:

\begin{itemize}
\item
  Taking two or more pipes and chaining them together to make new ones; hooking
  up input generators (``sources'', or \emph{Producers}) to pipes and to data
  consumers (``sinks'', or \emph{Consumers})
\item
  Taking producers and pipes and chains of pipes (which are themselves just
  pipes, by the way) and \emph{transforming} them into new producers and pipes.
\end{itemize}

If you've ever used bash/unix, the first concept is like using unix pipes to
``declare'' a chain of tools. You can do powerful things by just chaining simple
components.

The second concept relates to things to \emph{sudo} or \emph{time}; they take
normal bash commands and ``transform'' them into super user commands, or
``timeable'' commands.

And without any further delay, let's write \emph{encode.hs}!

\hypertarget{encoding}{%
\section{Encoding}\label{encoding}}

(Remember that you can download
\href{https://github.com/mstksg/inCode/tree/master/code-samples/huffman/encode.hs}{encode.hs}
from github and try it out yourself; just remember to also grab
\href{https://github.com/mstksg/inCode/tree/master/code-samples/huffman/Huffman.hs}{Huffman.hs},
\href{https://github.com/mstksg/inCode/tree/master/code-samples/huffman/PQueue.hs}{PQueue.hs},
and
\href{https://github.com/mstksg/inCode/tree/master/code-samples/huffman/PreTree.hs}{PreTree.hs},
and
\href{https://github.com/mstksg/inCode/tree/master/code-samples/huffman/Weighted.hs}{Weighted.hs}
from the previous parts of this tutorial!)

\hypertarget{design}{%
\subsection{Design}\label{design}}

Okay, so with the above in mind, let's sketch out a rough plan. We'll talk about
the holes in the plan later, but it's useful to see exactly what won't work, or
what is a bad idea :)

We can envision this all as a big single giant pipeline of atomic components.

As a \emph{Producer}, we have \texttt{fromHandle}, which emits
\texttt{ByteString}s read from a given file handle. As a \emph{Consumer}, we
have \texttt{toHandle}, which takes in \texttt{ByteString}s and writes them to
the given file handle.

Those in hand, we'll need:

\begin{enumerate}
\def\labelenumi{\arabic{enumi}.}
\tightlist
\item
  A pipe that turns incoming \texttt{ByteString}s into bytes\footnote{Remember,
    a \texttt{ByteString} is an efficiently packed ``chunk''/``list'' of
    \texttt{Word8}/bytes; we can use functions like \texttt{ByteString.unpack}
    and \texttt{ByteString.pack} to turn a \texttt{ByteString} into a list of
    \texttt{Word8}s or go backwards.} (\texttt{Word8}s), emitting one at a time.
\item
  A pipe that turns incoming \texttt{Word8}s into \texttt{Direction}s, by
  looking up each \texttt{Word8} in an encoding table to get a list of
  \texttt{Direction}s and emitting them one at a time.
\item
  A pipe that takes in \texttt{Direction}s and ``chunks them up'' 8-at-a-time,
  and re-emits those chunks of 8 as bytes/\texttt{Word8}'s.
\item
  A pipe that takes incoming \texttt{Word8}s and wraps them each back into
  \texttt{ByteString}s and emits them.
\end{enumerate}

Sounds simple enough, right? Basically like using unix pipes!

We'll be making two modifications to this plan before we go forward.

\hypertarget{leftovers}{%
\subsubsection{Leftovers}\label{leftovers}}

The first hole: vanilla \emph{pipes} does not have \emph{leftover support}. That
is, the stream terminates as soon as the producer terminates.

To put more technically: when a pipe is \emph{awaiting} something, there is no
guarantee that it'll ever get anything --- if the producer it is awaiting from
terminates, then that's that; no chance to respond.

This is normally not a problem, and it won't be an issue for our decoding
program. However, we run into a problem for pipe \#3 above: we need to ``clump
up'' incoming \texttt{Direction}s and emit them in groups of 8.

This spells trouble for us, because our pipe will be merrily be waiting for
eight Directions before clumping them up --- until our producer terminates
mid-clump. Then what? That final in-progress clump will be lost\ldots{}forever!

The problem is in the semantics of pipe composition with
\texttt{(\textgreater{}-\textgreater{})}.

So it's clear that using normal pipe composition
(\texttt{(\textgreater{}-\textgreater{})}) doesn't work. We're going to have to
transform our \texttt{Direction} producer in another way.

Luckily for us, this is precisely the problem that \emph{pipes-parse} was made
to solve.

We'll go into more detail about just \emph{how} it solves this later. At the
high level, instead of composing pipes with
\texttt{(\textgreater{}-\textgreater{})}, we'll \emph{transform} pipes by using
pipe transformers/functions.

So we'll modify our plan. We'll have our ``\texttt{Direction} producer'', which
consists of:

\begin{enumerate}
\def\labelenumi{\arabic{enumi}.}
\tightlist
\item
  Our \texttt{ByteString} producer.
\item
  Our \texttt{ByteString} to \texttt{Word8} pipe.
\item
  Our \texttt{Word8} to \texttt{Direction} pipe.
\end{enumerate}

And then we ``transform'' that \texttt{Direction} producer into a \texttt{Word8}
producer, which we'll call \texttt{dirsBytes}:

\begin{Shaded}
\begin{Highlighting}[]
\OtherTok{dirsBytes ::} \DataTypeTok{Producer} \DataTypeTok{Direction}\NormalTok{ m r }\OtherTok{->} \DataTypeTok{Producer} \DataTypeTok{Word8}\NormalTok{ m r}
\end{Highlighting}
\end{Shaded}

which turns a \texttt{Direction} producer into a \texttt{Word8} producer that
clumps up the \texttt{Direction}s into groups of 8 --- and if the directions run
out, pad the rest of the byte with 0's.

\emph{pipes-parse} gives us the ability to write \texttt{dirsBytes}.

\hypertarget{perfect-packing}{%
\subsubsection{Perfect Packing}\label{perfect-packing}}

The next problem.

If you've ever worked with \texttt{ByteString}s, you might have noted an
asymmetry to what we are doing. Look closely --- do you see it?

We \emph{read} \texttt{ByteString}s from the file --- entire \emph{big chunks}
of bytes/\texttt{Word8}s.

We \emph{write} individual bytes, one at a time. That is, we emit individual
\texttt{Word8}s, which we each individually wrap into singleton
\texttt{ByteString}s one at a time, which we write to the file one at a time.

This is bad!

As you might have guessed, the solution is to not use
\texttt{(\textgreater{}-\textgreater{})} and instead use a pipe transformer.

We're not going to write it ourselves using \emph{pipes-parse};
\emph{pipes-bytestring} (which we will import qualified as \texttt{PB}) actually
comes with such a transformer for us.

The only hitch is that it's ``trapped'' in a ``lens'', called \texttt{PB.pack}.

\begin{Shaded}
\begin{Highlighting}[]
\NormalTok{PB.pack}\OtherTok{ ::} \DataTypeTok{Lens'}\NormalTok{ (}\DataTypeTok{Producer} \DataTypeTok{Word8}\NormalTok{ m r) (}\DataTypeTok{Producer} \DataTypeTok{ByteString}\NormalTok{ m r)}
\end{Highlighting}
\end{Shaded}

If you are still learning lens, this basically means that \texttt{PB.pack}
contains, among other things, a function that allows you to go from a
\texttt{Word8} producer to a \texttt{ByteString} producer. The function
\texttt{view} lets us unlock that pipe transformer from the lens.

\begin{Shaded}
\begin{Highlighting}[]
\OtherTok{view ::} \DataTypeTok{Lens'}\NormalTok{ a b }\OtherTok{->}\NormalTok{ (a }\OtherTok{->}\NormalTok{ b)       }\CommentTok{-- in our case}
\end{Highlighting}
\end{Shaded}

So,

\begin{Shaded}
\begin{Highlighting}[]
\NormalTok{view PB.pack}\OtherTok{ ::} \DataTypeTok{Producer} \DataTypeTok{Word8}\NormalTok{      m r}
             \OtherTok{->} \DataTypeTok{Producer} \DataTypeTok{ByteString}\NormalTok{ m r}
\end{Highlighting}
\end{Shaded}

Cool. Anyways, \emph{pipes-bytestring} implements \texttt{view\ pack} (the
conversion function) in a way that does
\href{http://www.haskellforall.com/2013/09/perfect-streaming-using-pipes-bytestring.html}{``smart
chunking''} --- it waits until an appropriate amount of \texttt{Word8}s have
accumulated in a buffer before packing them all into a big fat
\texttt{ByteString}.

And that should be the last hole in our puzzle!

\hypertarget{down-to-it}{%
\subsection{Down to it}\label{down-to-it}}

Let's just get down to it!

First, our imports:

\begin{Shaded}
\begin{Highlighting}[]
\CommentTok{-- source: https://github.com/mstksg/inCode/tree/master/code-samples/huffman/encode.hs#L19-L46}

\CommentTok{-- General imports}
\KeywordTok{import} \DataTypeTok{Control.Applicative}\NormalTok{              ((<$>))}
\KeywordTok{import} \DataTypeTok{Control.Monad.Trans.State.Strict}\NormalTok{ (evalState)}
\KeywordTok{import} \DataTypeTok{Data.Foldable}\NormalTok{                    (sum)}
\KeywordTok{import} \DataTypeTok{Data.Map.Strict}\NormalTok{                  (}\DataTypeTok{Map}\NormalTok{, (!))}
\KeywordTok{import} \DataTypeTok{Lens.Family2}\NormalTok{                     (view)}
\KeywordTok{import} \DataTypeTok{Prelude} \KeywordTok{hiding}\NormalTok{                   (sum)}
\KeywordTok{import} \DataTypeTok{System.Environment}\NormalTok{               (getArgs)}
\KeywordTok{import} \DataTypeTok{System.IO}\NormalTok{                        (withFile, }\DataTypeTok{IOMode}\NormalTok{(..))}
\KeywordTok{import} \KeywordTok{qualified} \DataTypeTok{Data.Map.Strict}        \KeywordTok{as} \DataTypeTok{M}

\CommentTok{-- Pipes imports}
\KeywordTok{import} \DataTypeTok{Pipes}
\KeywordTok{import} \DataTypeTok{Pipes.Parse}
\KeywordTok{import} \KeywordTok{qualified} \DataTypeTok{Pipes.ByteString} \KeywordTok{as} \DataTypeTok{PB}
\KeywordTok{import} \KeywordTok{qualified} \DataTypeTok{Pipes.Prelude}    \KeywordTok{as} \DataTypeTok{PP}

\CommentTok{-- Working with Binary}
\KeywordTok{import} \DataTypeTok{Data.Binary} \KeywordTok{hiding}\NormalTok{             (encodeFile)}
\KeywordTok{import} \DataTypeTok{Data.Bits}\NormalTok{                      (setBit)}
\KeywordTok{import} \DataTypeTok{Data.ByteString}\NormalTok{                (}\DataTypeTok{ByteString}\NormalTok{)}
\KeywordTok{import} \KeywordTok{qualified} \DataTypeTok{Data.ByteString}      \KeywordTok{as} \DataTypeTok{B}
\KeywordTok{import} \KeywordTok{qualified} \DataTypeTok{Data.ByteString.Lazy} \KeywordTok{as} \DataTypeTok{BL}

\CommentTok{-- Huffman imports}
\KeywordTok{import} \DataTypeTok{Huffman}
\KeywordTok{import} \DataTypeTok{PQueue}
\KeywordTok{import} \DataTypeTok{PreTree}
\end{Highlighting}
\end{Shaded}

It's a doozy, admittedly!

Now \texttt{main}:

\begin{Shaded}
\begin{Highlighting}[]
\CommentTok{-- source: https://github.com/mstksg/inCode/tree/master/code-samples/huffman/encode.hs#L48-L60}

\OtherTok{main ::} \DataTypeTok{IO}\NormalTok{ ()}
\NormalTok{main }\FunctionTok{=} \KeywordTok{do}
\NormalTok{    args     }\OtherTok{<-}\NormalTok{ getArgs}
    \KeywordTok{let}\NormalTok{ (inp, out)  }\FunctionTok{=} \KeywordTok{case}\NormalTok{ args }\KeywordTok{of}
\NormalTok{                        i}\FunctionTok{:}\NormalTok{o}\FunctionTok{:}\NormalTok{_      }\OtherTok{->}\NormalTok{ (i,o)}
\NormalTok{                        _          }\OtherTok{->}\NormalTok{ error }\StringTok{"Give input and output files."}

\NormalTok{    metadata }\OtherTok{<-}\NormalTok{ analyzeFile inp}
    \KeywordTok{let}\NormalTok{ (len, tree) }\FunctionTok{=} \KeywordTok{case}\NormalTok{ metadata }\KeywordTok{of}
                        \DataTypeTok{Just}\NormalTok{ (l, t) }\OtherTok{->}\NormalTok{ (l, t)}
                        \DataTypeTok{Nothing}     \OtherTok{->}\NormalTok{ error }\StringTok{"Empty File"}

\NormalTok{    encodeFile inp out len tree}
\end{Highlighting}
\end{Shaded}

Just straight-forward, more or less. The error handling is kind of not too
great, but we won't go into that too deeply here :)

\hypertarget{file-metadata}{%
\subsubsection{File metadata}\label{file-metadata}}

\texttt{analyzeFile} is going to be how we build the Huffman Tree for the
encoding, as discussed in part 1. It'll go through an entire pass of the file
and count up the number of occurrences for each byte and build a Huffman
encoding tree out of it. It'll also give us the length of the file in bytes;
this is actually necessary for \emph{decoding} the file later, because it tells
us where to stop decoding (lest we begin decoding the leftover padding bits).

\begin{Shaded}
\begin{Highlighting}[]
\CommentTok{-- source: https://github.com/mstksg/inCode/tree/master/code-samples/huffman/encode.hs#L63-L74}

\OtherTok{analyzeFile ::}\NormalTok{ FilePath }\OtherTok{->} \DataTypeTok{IO}\NormalTok{ (}\DataTypeTok{Maybe}\NormalTok{ (}\DataTypeTok{Int}\NormalTok{, }\DataTypeTok{PreTree} \DataTypeTok{Word8}\NormalTok{))}
\NormalTok{analyzeFile fp }\FunctionTok{=}\NormalTok{ withFile fp }\DataTypeTok{ReadMode} \FunctionTok{$}\NormalTok{ \textbackslash{}hIn }\OtherTok{->} \KeywordTok{do}
    \KeywordTok{let}\NormalTok{ byteProducer }\FunctionTok{=}\NormalTok{ PB.fromHandle hIn }\FunctionTok{>->}\NormalTok{ bsToBytes}
\NormalTok{    fqs }\OtherTok{<-}\NormalTok{ freqs byteProducer}
    \KeywordTok{let}\NormalTok{ len  }\FunctionTok{=}\NormalTok{ sum fqs}
\NormalTok{        tree }\FunctionTok{=}\NormalTok{ evalState (listQueueStateTable fqs }\FunctionTok{>>}\NormalTok{ buildTree) emptyPQ}
\NormalTok{    return }\FunctionTok{$}\NormalTok{ fmap (len,) tree}
  \KeywordTok{where}
\OtherTok{    freqs ::}\NormalTok{ (}\DataTypeTok{Monad}\NormalTok{ m, }\DataTypeTok{Ord}\NormalTok{ a) }\OtherTok{=>} \DataTypeTok{Producer}\NormalTok{ a m () }\OtherTok{->}\NormalTok{ m (}\DataTypeTok{M.Map}\NormalTok{ a }\DataTypeTok{Int}\NormalTok{)}
\NormalTok{    freqs }\FunctionTok{=}\NormalTok{ PP.fold f M.empty id}
      \KeywordTok{where}
\NormalTok{        f m x }\FunctionTok{=}\NormalTok{ M.insertWith (}\FunctionTok{+}\NormalTok{) x }\DecValTok{1}\NormalTok{ m}
\end{Highlighting}
\end{Shaded}

First, we use
\href{http://hackage.haskell.org/package/base-4.7.0.0/docs/System-IO.html\#v:withFile}{\texttt{withFile}}
from System.IO, which gives us a file handler for a given filepath; we can pass
this handler onto functions that take file handlers. \texttt{withFile} actually
handles most of the IO-based error handling and cleanup we would ever need for
our simple use cases of \emph{pipes}.

Now we run into real \emph{pipes} for the first time!

We'll assemble our producer of bytes by using \texttt{PB.fromHandle\ hIn} --- a
producer of \texttt{ByteString}s --- and chaining it to \texttt{bsToBytes}, a
pipe that takes incoming \texttt{ByteString}s and emits their constituent,
unpacked \texttt{Word8}s one-by-one:

\begin{Shaded}
\begin{Highlighting}[]
\CommentTok{-- source: https://github.com/mstksg/inCode/tree/master/code-samples/huffman/encode.hs#L96-L97}

\OtherTok{bsToBytes ::} \DataTypeTok{Monad}\NormalTok{ m }\OtherTok{=>} \DataTypeTok{Pipe} \DataTypeTok{ByteString} \DataTypeTok{Word8}\NormalTok{ m r}
\NormalTok{bsToBytes }\FunctionTok{=}\NormalTok{ PP.mapFoldable B.unpack}
\end{Highlighting}
\end{Shaded}

Our implementation uses
\texttt{B.unpack\ ::\ ByteString\ -\textgreater{}\ {[}Word8{]}} from
\emph{pipes-bytestring}, which turns a \texttt{ByteString} into a list of its
constituent \texttt{Word8}s. We use \texttt{PP.mapFoldable}, which is sort of
like \texttt{concatMap} --- it applies the given function to every incoming
element in the stream, and emits the items in the resulting list\footnote{It
  actually works on all \texttt{Foldable}s, not just \texttt{{[}{]}}.}
one-by-one. So \texttt{bsToBytes} is a Pipe that takes in \texttt{ByteString}s
and emits each contained \texttt{Word8} one-by-one.

Then with our pipe ready, we ``run''/``use'' it, using \texttt{PP.fold}, from
the pipes Prelude. This basically runs a giant ``foldl'' all over the incoming
items of the given producer.

The fold is identical in logic to \texttt{listFreq} from a
\href{http://blog.jle.im/entry/streaming-huffman-compression-in-haskell-part-2-binary}{Part
2}:

\begin{Shaded}
\begin{Highlighting}[]
\CommentTok{-- source: https://github.com/mstksg/inCode/tree/master/code-samples/huffman/Huffman.hs#L22-L25}

\OtherTok{listFreq ::} \DataTypeTok{Ord}\NormalTok{ a }\OtherTok{=>}\NormalTok{ [a] }\OtherTok{->} \DataTypeTok{FreqTable}\NormalTok{ a}
\NormalTok{listFreq }\FunctionTok{=}\NormalTok{ foldr f M.empty}
  \KeywordTok{where}
\NormalTok{    f x m }\FunctionTok{=}\NormalTok{ M.insertWith (}\FunctionTok{+}\NormalTok{) x }\DecValTok{1}\NormalTok{ m}
\end{Highlighting}
\end{Shaded}

Except instead of folding over a list, we fold over the elements of the
producer. Note that the helper function has its arguments reversed. This whole
thing, then, will fold over all of the items produced by the given producer (all
of the \texttt{Word8}s) with our frequency-table-building.

We then use \texttt{sum} from \texttt{Data.Foldable}, which sums up all the
numbers in our frequency map. Then we use what we learned about the State monad
in
\href{http://blog.jle.im/entry/streaming-huffman-compression-in-haskell-part-1-trees}{Part
1} to build our tree (review
\href{http://blog.jle.im/entry/streaming-huffman-compression-in-haskell-part-1-trees}{Part
1} if you do not understand the declaration of \texttt{tree}). \texttt{tree} is
a \texttt{Maybe\ (PreTree\ Word8)}; we then tag on the length to our
\texttt{tree} using \texttt{fmap} and the TupleSections extension. (That is,
\texttt{(,y)} is sugar for
\texttt{(\textbackslash{}x\ -\textgreater{}\ (x,y))}).

\hypertarget{the-encoding-pipeline}{%
\subsubsection{The Encoding Pipeline}\label{the-encoding-pipeline}}

Once we have that, we can get onto the actual encoding process: the second pass.

\begin{Shaded}
\begin{Highlighting}[]
\CommentTok{-- source: https://github.com/mstksg/inCode/tree/master/code-samples/huffman/encode.hs#L78-L92}

\OtherTok{encodeFile ::}\NormalTok{ FilePath }\OtherTok{->}\NormalTok{ FilePath }\OtherTok{->} \DataTypeTok{Int} \OtherTok{->} \DataTypeTok{PreTree} \DataTypeTok{Word8} \OtherTok{->} \DataTypeTok{IO}\NormalTok{ ()}
\NormalTok{encodeFile inp out len tree }\FunctionTok{=}
\NormalTok{    withFile inp }\DataTypeTok{ReadMode}  \FunctionTok{$}\NormalTok{ \textbackslash{}hIn  }\OtherTok{->}
\NormalTok{    withFile out }\DataTypeTok{WriteMode} \FunctionTok{$}\NormalTok{ \textbackslash{}hOut }\OtherTok{->} \KeywordTok{do}
\NormalTok{      BL.hPut hOut }\FunctionTok{$}\NormalTok{ encode (len, tree)}
      \KeywordTok{let}\NormalTok{ dirsOut   }\FunctionTok{=}\NormalTok{ PB.fromHandle hIn}
                  \FunctionTok{>->}\NormalTok{ bsToBytes}
                  \FunctionTok{>->}\NormalTok{ encodeByte encTable}
\NormalTok{          bsOut     }\FunctionTok{=}\NormalTok{ view PB.pack }\FunctionTok{.}\NormalTok{ dirsBytes }\FunctionTok{$}\NormalTok{ dirsOut}
\NormalTok{          pipeline  }\FunctionTok{=}\NormalTok{ bsOut}
                  \FunctionTok{>->}\NormalTok{ PB.toHandle hOut}

\NormalTok{      runEffect pipeline}
  \KeywordTok{where}
\NormalTok{    encTable  }\FunctionTok{=}\NormalTok{ ptTable tree}
\end{Highlighting}
\end{Shaded}

First, we open our file handles for our input and output files. Then, we use
what we learned in
\href{http://blog.jle.im/entry/streaming-huffman-compression-in-haskell-part-2-binary}{Part
2} to get binary serializations of our length and our tree using
\texttt{encode}, and use \texttt{BL.hPut} to write it to our file, as the
metadata. \texttt{BL.hPut} from \texttt{Data.ByteString.Lazy} takes a file
handle and a lazy \texttt{ByteString}, and writes that \texttt{ByteString} out
to the file. We use the lazy version because \texttt{encode} gives us a lazy
\texttt{ByteString} by default.

Note that we can ``put'' \texttt{(len,\ tree)} together as a tuple instead of
putting \texttt{len} and \texttt{tree} one after the other. This is because
\texttt{(a,\ b)} has a \texttt{Binary} instance. We'll read it back in later as
a tuple, but it actually doesn't matter, because the \texttt{Binary} instance
for tuples is just putting/getting each item one after the other.

Now, we get to our actual pipes. The first ``pipeline'' is \texttt{dirsOut},
which is our stream (producer) of \texttt{Direction}s encoding the input file.
As can be read, \texttt{dirStream} is \texttt{PB.fromHandle\ hIn} (a
\texttt{ByteString} producer from the given handle) piped into our old friend
\texttt{bsToBytes} piped into \texttt{encodeByte\ encTable}, which is a pipe
taking in bytes (\texttt{Word8}), looks them up in \texttt{encTable} (the table
mapping \texttt{Word8} to \texttt{{[}Direction{]}}, which we built in
\href{http://blog.jle.im/entry/streaming-huffman-compression-in-haskell-part-2-binary}{Part
2}), and spits out the resulting \texttt{Direction}s one at a time.

\texttt{encodeByte\ encTable} is implemented ``exactly the same'' as
\texttt{bsToBytes}:

\begin{Shaded}
\begin{Highlighting}[]
\CommentTok{-- source: https://github.com/mstksg/inCode/tree/master/code-samples/huffman/encode.hs#L101-L104}

\OtherTok{encodeByte ::}\NormalTok{ (}\DataTypeTok{Ord}\NormalTok{ a, }\DataTypeTok{Monad}\NormalTok{ m)}
           \OtherTok{=>} \DataTypeTok{Map}\NormalTok{ a }\DataTypeTok{Encoding}
           \OtherTok{->} \DataTypeTok{Pipe}\NormalTok{ a }\DataTypeTok{Direction}\NormalTok{ m r}
\NormalTok{encodeByte encTable }\FunctionTok{=}\NormalTok{ PP.mapFoldable (encTable }\FunctionTok{!}\NormalTok{)}
\end{Highlighting}
\end{Shaded}

instead of using \texttt{mapFoldable} with a
\texttt{ByteString\ -\textgreater{}\ {[}Word8{]}}, we use \texttt{mapFoldable}
with a \texttt{Word8\ -\textgreater{}\ {[}Direction{]}}, which does the same
thing --- apply the function to every incoming item, and spit out the items in
the resulting list one at a time.

\texttt{(!)\ ::\ Map\ k\ v\ -\textgreater{}\ k\ -\textgreater{}\ v} is the
lookup function for \texttt{Map}s.

\hypertarget{parser}{%
\subsubsection{Parser}\label{parser}}

So now we have \texttt{dirsOut\ ::\ Producer\ Direction\ IO\ r}, which is a
producer of \texttt{Direction}s drawn from the file. It's now time to ``group
up'' the directions, using the ``producer transformer'' tactic we discussed
earlier.

\begin{Shaded}
\begin{Highlighting}[]
\CommentTok{-- source: https://github.com/mstksg/inCode/tree/master/code-samples/huffman/encode.hs#L108-L117}

\OtherTok{dirsBytes ::}\NormalTok{ (}\DataTypeTok{MonadIO}\NormalTok{ m, }\DataTypeTok{Functor}\NormalTok{ m)}
          \OtherTok{=>} \DataTypeTok{Producer} \DataTypeTok{Direction}\NormalTok{ m r}
          \OtherTok{->} \DataTypeTok{Producer} \DataTypeTok{Word8}\NormalTok{     m ()}
\NormalTok{dirsBytes p }\FunctionTok{=} \KeywordTok{do}
\NormalTok{    (result, leftovers) }\OtherTok{<-}\NormalTok{ lift }\FunctionTok{$}\NormalTok{ runStateT dirsBytesP p}
    \KeywordTok{case}\NormalTok{ result }\KeywordTok{of}
      \DataTypeTok{Just}\NormalTok{ byte }\OtherTok{->} \KeywordTok{do}
\NormalTok{        yield byte}
\NormalTok{        dirsBytes leftovers}
      \DataTypeTok{Nothing}   \OtherTok{->}\NormalTok{ return ()}
\end{Highlighting}
\end{Shaded}

\texttt{dirsBytes} turns out \texttt{Direction} producer into a \texttt{Word8}
producer by running the \emph{parser} \texttt{dirsBytesP} onto the producer, and
looping onto itself. We'll look at \texttt{dirsBytesP} later, but for now, know
that it is a parser that attempts to consume eight \texttt{Direction}s and
returns them together in a \texttt{Just\ byte} with zero padding if the stream
runs out; if the stream is already empty to start with, it returns
\texttt{Nothing}.

Remember that in \emph{pipes-parse}:

\begin{Shaded}
\begin{Highlighting}[]
\OtherTok{runStateT ::} \DataTypeTok{Parser}\NormalTok{ a m b }\OtherTok{->} \DataTypeTok{Producer}\NormalTok{ a m r }\OtherTok{->}\NormalTok{ m (b, }\DataTypeTok{Producer}\NormalTok{ a m r)}
\end{Highlighting}
\end{Shaded}

Basically, \texttt{runStateT\ parser} takes a \texttt{Producer\ a} and
``parses'' a value out of it, returning the parsed value and the
``leftover/used'' \texttt{Producer}.

In our case:

\begin{Shaded}
\begin{Highlighting}[]
\OtherTok{runStateT ::} \DataTypeTok{Parser}   \DataTypeTok{Direction} \DataTypeTok{IO}\NormalTok{ (}\DataTypeTok{Maybe} \DataTypeTok{Word8}\NormalTok{)}
          \OtherTok{->} \DataTypeTok{Producer} \DataTypeTok{Direction} \DataTypeTok{IO}\NormalTok{ r}
          \OtherTok{->} \DataTypeTok{IO}\NormalTok{ (}\DataTypeTok{Maybe} \DataTypeTok{Word8}\NormalTok{, }\DataTypeTok{Producer} \DataTypeTok{Direction} \DataTypeTok{IO}\NormalTok{ r)}
\end{Highlighting}
\end{Shaded}

So we use the \texttt{dirsBytesP} parser onto the producer we are given. If it
doesn't parse any bytes (\texttt{Nothing}), then we stop. If it does
(\texttt{Just\ byte}), then we \texttt{yield} the parsed \texttt{Word8} and then
start over again with the leftovers producer.

Let's take a look at the \texttt{dirsBytesP} parser, which parses
\texttt{Direction}s into a \texttt{Word8}:

\begin{Shaded}
\begin{Highlighting}[]
\CommentTok{-- source: https://github.com/mstksg/inCode/tree/master/code-samples/huffman/encode.hs#L123-L137}

\OtherTok{dirsBytesP ::}\NormalTok{ (}\DataTypeTok{Monad}\NormalTok{ m, }\DataTypeTok{Functor}\NormalTok{ m) }\OtherTok{=>} \DataTypeTok{Parser} \DataTypeTok{Direction}\NormalTok{ m (}\DataTypeTok{Maybe} \DataTypeTok{Word8}\NormalTok{)}
\NormalTok{dirsBytesP }\FunctionTok{=} \KeywordTok{do}
\NormalTok{    isEnd }\OtherTok{<-}\NormalTok{ isEndOfInput}
    \KeywordTok{if}\NormalTok{ isEnd}
      \KeywordTok{then}\NormalTok{ return }\DataTypeTok{Nothing}
      \KeywordTok{else} \DataTypeTok{Just} \FunctionTok{<$>}\NormalTok{ go }\DecValTok{0} \DecValTok{0}
  \KeywordTok{where}
\OtherTok{    go ::} \DataTypeTok{Monad}\NormalTok{ m }\OtherTok{=>} \DataTypeTok{Word8} \OtherTok{->} \DataTypeTok{Int} \OtherTok{->} \DataTypeTok{Parser} \DataTypeTok{Direction}\NormalTok{ m }\DataTypeTok{Word8}
\NormalTok{    go b }\DecValTok{8} \FunctionTok{=}\NormalTok{ return b}
\NormalTok{    go b i }\FunctionTok{=} \KeywordTok{do}
\NormalTok{      dir }\OtherTok{<-}\NormalTok{ draw}
      \KeywordTok{case}\NormalTok{ dir }\KeywordTok{of}
        \DataTypeTok{Just} \DataTypeTok{DLeft}  \OtherTok{->}\NormalTok{ go     b            (i }\FunctionTok{+} \DecValTok{1}\NormalTok{)}
        \DataTypeTok{Just} \DataTypeTok{DRight} \OtherTok{->}\NormalTok{ go     (setBit b i) (i }\FunctionTok{+} \DecValTok{1}\NormalTok{)}
        \DataTypeTok{Nothing}     \OtherTok{->}\NormalTok{ return b}
\end{Highlighting}
\end{Shaded}

This implementation is pretty straightforward --- ``if the producer is empty,
return \texttt{Nothing}. Otherwise, start with \texttt{00000000} and draw
\texttt{Direction}s one at a time, flipping the appropriate bit when you get a
\texttt{Right}.'' For more information on the exact functions for bitwise
operators, look into the \href{http://hackage.haskell.org/package/bits}{bits}
package, where they come from.

Note the usage of \texttt{draw}, which ``returns'' a \texttt{Nothing} if you
draw from the end of the producer, and a \texttt{Just\ x} if there is something
to draw. \texttt{draw} is special to parsers, because it lets you react on
end-of-input as a \texttt{Nothing} (as opposed to \texttt{await}). In
\texttt{go}, we loop drawing until we either get all eight bits (and return the
resulting byte) or run out of inputs (and return the byte that we have so far).

We get our direction producer by doing \texttt{dirsBytes\ dirsOut}.

\hypertarget{smart-chunker}{%
\subsubsection{Smart Chunker}\label{smart-chunker}}

And finally, we use the ``smart chunking'' provided by \emph{pipes-bytestring}
by transforming our bytes stream:

\begin{Shaded}
\begin{Highlighting}[]
\CommentTok{-- source: https://github.com/mstksg/inCode/tree/master/code-samples/huffman/encode.hs#L86-L86}

\NormalTok{          bsOut     }\FunctionTok{=}\NormalTok{ view PB.pack }\FunctionTok{.}\NormalTok{ dirsBytes }\FunctionTok{$}\NormalTok{ dirsOut}
\end{Highlighting}
\end{Shaded}

\hypertarget{all-together}{%
\subsubsection{All together}\label{all-together}}

That gives us our final \texttt{pipeline}; we lay out a series of pipes and
pipes transformers that takes our file and streamingly processes the data and
writes it into the output file.

Once we have our \texttt{pipeline}, we use \texttt{runEffect} to ``run'' it;
then\ldots{}that's it!

\hypertarget{testing-it-out}{%
\subsection{Testing it out}\label{testing-it-out}}

Cool, let's try it out with Leo Tolstoy's great classic
\href{http://www.gutenberg.org/files/2600/2600.txt}{War and Peace} from Project
Gutenberg!

\begin{Shaded}
\begin{Highlighting}[]
\NormalTok{$ }\ExtensionTok{ghc}\NormalTok{ -O2 encode.hs}
\NormalTok{$ }\ExtensionTok{./encode}\NormalTok{ warandpeace.txt warandpeace.enc}
\NormalTok{$ }\FunctionTok{du}\NormalTok{ -h warandpeace.*}
\CommentTok{# 1.8M warandpeace.enc}
\CommentTok{# 3.1M warandpeace.txt}
\end{Highlighting}
\end{Shaded}

Cool, we compressed it to 58\% of the original file size. Not bad! Using
\texttt{gzip} with default settings gives a compression of 39\%, so it's not the
best, but it's something. If we take out the encoding part of the script, we can
see that the metadata (the length and the dictionary) itself only takes 259
bytes (which is negligible) --- so 58\% is pretty much the asymptotic
compression rate.

At this point it's not as snappy (performance wise) as we'd like; a compressing
a 3.1M file is not ``super slow'' (it takes about seven seconds on my computer),
but you probably won't be compressing a gigabyte. We'll look into performance in
a later post!

\hypertarget{decoding}{%
\section{Decoding}\label{decoding}}

(Remember again that download
\href{https://github.com/mstksg/inCode/tree/master/code-samples/huffman/decode.hs}{decode.hs}
is also available online from github! Again, be sure to also grab
\href{https://github.com/mstksg/inCode/tree/master/code-samples/huffman/Huffman.hs}{Huffman.hs},
\href{https://github.com/mstksg/inCode/tree/master/code-samples/huffman/PQueue.hs}{PQueue.hs},
and
\href{https://github.com/mstksg/inCode/tree/master/code-samples/huffman/PreTree.hs}{PreTree.hs},
and
\href{https://github.com/mstksg/inCode/tree/master/code-samples/huffman/Weighted.hs}{Weighted.hs}
from posts past.)

\hypertarget{design-1}{%
\subsection{Design}\label{design-1}}

Let's try to see the plan for our decoding script, applying what we learned
before. What components do we need?

\begin{enumerate}
\def\labelenumi{\arabic{enumi}.}
\tightlist
\item
  First, a component producing decoded \texttt{Word8}s (that will be
  \texttt{view\ PB.pack}'d into a component producing decoded
  \texttt{ByteString}s with smart chunking)

  \begin{enumerate}
  \def\labelenumii{\arabic{enumii}.}
  \tightlist
  \item
    A producer that reads in \texttt{ByteString}s from a file and sends them
    downstream.
  \item
    A pipe that unpacks those \texttt{ByteString}s into \texttt{Word8}s and
    sends each one down.
  \item
    A pipe that ``unpacks'' those \texttt{Word8}s into \texttt{Direction}s and
    sends \emph{those} down.
  \item
    A pipe that traverses down the Huffman encoding tree following the incoming
    \texttt{Direction}s, and emits a decoded \texttt{Word8} every time it
    decodes a value.
  \end{enumerate}
\item
  A component consuming the incoming \texttt{ByteString}s, and writing them to
  our output file.
\end{enumerate}

\hypertarget{down-to-it-1}{%
\subsection{Down to it}\label{down-to-it-1}}

Luckily we can use most of the concepts we learned in writing the encoding
script to write the decoding script; we also have a less imports, so it's a sign
that decoding is going to be slightly simpler than encoding.

\begin{Shaded}
\begin{Highlighting}[]
\CommentTok{-- source: https://github.com/mstksg/inCode/tree/master/code-samples/huffman/decode.hs#L18-L37}

\CommentTok{-- General imports}
\KeywordTok{import} \DataTypeTok{Lens.Family2}\NormalTok{       (view)}
\KeywordTok{import} \DataTypeTok{System.Environment}\NormalTok{ (getArgs)}
\KeywordTok{import} \DataTypeTok{System.IO}\NormalTok{          (withFile, }\DataTypeTok{IOMode}\NormalTok{(..))}

\CommentTok{-- Pipes imports}
\KeywordTok{import} \DataTypeTok{Pipes}
\KeywordTok{import} \DataTypeTok{Pipes.Parse}
\KeywordTok{import} \KeywordTok{qualified} \DataTypeTok{Pipes.Binary}     \KeywordTok{as} \DataTypeTok{PB}
\KeywordTok{import} \KeywordTok{qualified} \DataTypeTok{Pipes.ByteString} \KeywordTok{as} \DataTypeTok{PB}
\KeywordTok{import} \KeywordTok{qualified} \DataTypeTok{Pipes.Prelude}    \KeywordTok{as} \DataTypeTok{PP}

\CommentTok{-- Working with Binary}
\KeywordTok{import} \DataTypeTok{Data.Bits}\NormalTok{                 (testBit)}
\KeywordTok{import} \DataTypeTok{Data.ByteString}\NormalTok{           (}\DataTypeTok{ByteString}\NormalTok{)}
\KeywordTok{import} \DataTypeTok{Data.Word}\NormalTok{                 (}\DataTypeTok{Word8}\NormalTok{)}
\KeywordTok{import} \KeywordTok{qualified} \DataTypeTok{Data.ByteString} \KeywordTok{as} \DataTypeTok{B}

\CommentTok{-- Huffman imports}
\KeywordTok{import} \DataTypeTok{PreTree}
\end{Highlighting}
\end{Shaded}

\texttt{main} should seem very familiar:

\begin{Shaded}
\begin{Highlighting}[]
\CommentTok{-- source: https://github.com/mstksg/inCode/tree/master/code-samples/huffman/decode.hs#L39-L45}

\OtherTok{main ::} \DataTypeTok{IO}\NormalTok{ ()}
\NormalTok{main }\FunctionTok{=} \KeywordTok{do}
\NormalTok{    args     }\OtherTok{<-}\NormalTok{ getArgs}
    \KeywordTok{let}\NormalTok{ (inp, out)  }\FunctionTok{=} \KeywordTok{case}\NormalTok{ args }\KeywordTok{of}
\NormalTok{                        i}\FunctionTok{:}\NormalTok{o}\FunctionTok{:}\NormalTok{_      }\OtherTok{->}\NormalTok{ (i,o)}
\NormalTok{                        _          }\OtherTok{->}\NormalTok{ error }\StringTok{"Give input and output files."}
\NormalTok{    decodeFile inp out}
\end{Highlighting}
\end{Shaded}

And now on to the juicy parts:

\begin{Shaded}
\begin{Highlighting}[]
\CommentTok{-- source: https://github.com/mstksg/inCode/tree/master/code-samples/huffman/decode.hs#L48-L69}

\OtherTok{decodeFile ::}\NormalTok{ FilePath }\OtherTok{->}\NormalTok{ FilePath }\OtherTok{->} \DataTypeTok{IO}\NormalTok{ ()}
\NormalTok{decodeFile inp out }\FunctionTok{=}
\NormalTok{    withFile inp }\DataTypeTok{ReadMode}  \FunctionTok{$}\NormalTok{ \textbackslash{}hIn  }\OtherTok{->}
\NormalTok{    withFile out }\DataTypeTok{WriteMode} \FunctionTok{$}\NormalTok{ \textbackslash{}hOut }\OtherTok{->} \KeywordTok{do}
      \KeywordTok{let}\NormalTok{ metadataPipe }\FunctionTok{=}\NormalTok{ PB.fromHandle hIn}

      \CommentTok{-- consume metapipe to read in the tree/metadata}
\NormalTok{      (metadata, decodingPipe) }\OtherTok{<-}\NormalTok{ runStateT PB.decode metadataPipe}

      \KeywordTok{case}\NormalTok{ metadata }\KeywordTok{of}
        \DataTypeTok{Left}\NormalTok{   _          }\OtherTok{->}
\NormalTok{          error }\StringTok{"Corrupt metadata."}
        \DataTypeTok{Right}\NormalTok{ (len, tree) }\OtherTok{->} \KeywordTok{do}
          \CommentTok{-- do everything with the rest}
          \KeywordTok{let}\NormalTok{ bytesOut  }\FunctionTok{=}\NormalTok{ decodingPipe }\FunctionTok{>->}\NormalTok{ bsToBytes}
                      \FunctionTok{>->}\NormalTok{ bytesToDirs  }\FunctionTok{>->}\NormalTok{ searchPT tree}
                      \FunctionTok{>->}\NormalTok{ PP.take len}
\NormalTok{              bsOut     }\FunctionTok{=}\NormalTok{ (view PB.pack) bytesOut}
\NormalTok{              pipeline  }\FunctionTok{=}\NormalTok{ bsOut}
                      \FunctionTok{>->}\NormalTok{ PB.toHandle hOut}

\NormalTok{          runEffect pipeline}
\end{Highlighting}
\end{Shaded}

\hypertarget{loading-metadata}{%
\subsubsection{Loading metadata}\label{loading-metadata}}

Loading the metadata is a snap, and it uses what we learned earlier from
\texttt{runStateT} and \texttt{Parser}s.

Here, our \texttt{Parser} is \texttt{PB.decode}, from the \emph{pipes-binary}
package (and not from \emph{binary}), and it does more or less exactly what
you'd expect: it reads in binary data from the source stream, consuming it until
it has a successful (or unsuccessful) parse, as given by the \emph{binary}
package talked about in
\href{http://blog.jle.im/entry/streaming-huffman-compression-in-haskell-part-2-binary}{Part
2}. The ``result'' is the \texttt{Either} containing the success or failure, and
the ``leftover'', consumed source stream.

In our case:

\begin{Shaded}
\begin{Highlighting}[]
\NormalTok{runStateT}
\OtherTok{  ::} \DataTypeTok{Parser}   \DataTypeTok{ByteString} \DataTypeTok{IO}\NormalTok{ (}\DataTypeTok{Either} \DataTypeTok{DecodingError}\NormalTok{ (}\DataTypeTok{Int}\NormalTok{, }\DataTypeTok{PreTree} \DataTypeTok{Word8}\NormalTok{))}
  \OtherTok{->} \DataTypeTok{Producer} \DataTypeTok{ByteString} \DataTypeTok{IO}\NormalTok{ r}
  \OtherTok{->} \DataTypeTok{IO}\NormalTok{ (}\DataTypeTok{Either} \DataTypeTok{DecodingError}\NormalTok{ (}\DataTypeTok{Int}\NormalTok{, }\DataTypeTok{PreTree} \DataTypeTok{Word8}\NormalTok{), }\DataTypeTok{Producer} \DataTypeTok{ByteString} \DataTypeTok{IO}\NormalTok{ r)}
\end{Highlighting}
\end{Shaded}

So \texttt{metadata} is \texttt{Either\ DecodingError\ (Int,\ PreTree\ Word8)}.
If we get a \texttt{Left\ e}, then we throw an error for unparseable/corrupted
metadata. If we get a \texttt{Right\ (len,\ tree)}, then we are good to go.

\hypertarget{the-decoding-pipeline}{%
\subsubsection{The Decoding Pipeline}\label{the-decoding-pipeline}}

The rest just reads like poetry!

\begin{Shaded}
\begin{Highlighting}[]
\KeywordTok{let}\NormalTok{ byteStream }\FunctionTok{=}\NormalTok{ decodingPipe }\FunctionTok{>->}\NormalTok{ bsToBytes}
             \FunctionTok{>->}\NormalTok{ bytesToDirs  }\FunctionTok{>->}\NormalTok{ searchPT tree}
             \FunctionTok{>->}\NormalTok{ PP.take len}
\end{Highlighting}
\end{Shaded}

Beautiful! \texttt{decodingPipe} is the leftover producer after the parse of the
metadata. \texttt{bsToBytes} is the same as from our encoder.
\texttt{bytesToDirs} is implemented ``exactly'' like \texttt{bsToBytes} and
\texttt{encodeByte} (from \emph{encode.hs}) --- using \texttt{PP.mapFoldable}
and a \texttt{Word8\ -\textgreater{}\ {[}Direction{]}} function.

\begin{Shaded}
\begin{Highlighting}[]
\CommentTok{-- source: https://github.com/mstksg/inCode/tree/master/code-samples/huffman/decode.hs#L96-L104}

\OtherTok{bytesToDirs ::} \DataTypeTok{Monad}\NormalTok{ m }\OtherTok{=>} \DataTypeTok{Pipe} \DataTypeTok{Word8} \DataTypeTok{Direction}\NormalTok{ m r}
\NormalTok{bytesToDirs }\FunctionTok{=}\NormalTok{ PP.mapFoldable byteToDirList}
  \KeywordTok{where}
    \CommentTok{-- Turns a byte into a list of directions}
\OtherTok{    byteToDirList ::} \DataTypeTok{Word8} \OtherTok{->}\NormalTok{ [}\DataTypeTok{Direction}\NormalTok{]}
\NormalTok{    byteToDirList b }\FunctionTok{=}\NormalTok{ map f [}\DecValTok{0}\FunctionTok{..}\DecValTok{7}\NormalTok{]}
      \KeywordTok{where}
\NormalTok{        f i }\FunctionTok{|}\NormalTok{ testBit b i }\FunctionTok{=} \DataTypeTok{DRight}
            \FunctionTok{|}\NormalTok{ otherwise   }\FunctionTok{=} \DataTypeTok{DLeft}
\end{Highlighting}
\end{Shaded}

It uses the \emph{bits} package to turn an incoming \texttt{Word8} into a list
of its constituent bits (in the form of \texttt{Direction}s), and yields each of
them in turn.

We have \texttt{searchPT\ tree}, which is a pipe turning incoming
\texttt{Direction}s into aggregate/outgoing \texttt{Word8}s by finding them on
the given \texttt{PreTree}. The implementation is a bit tricky so we're going to
go into it in more detail later.

\texttt{PP.take\ len} is new; it's from \texttt{Pipes.Prelude}, and it basically
says ``take \texttt{len} items from the source, then stop drawing.'' This is
necessary because, because of the padding of 0's we did from the encoding
script, there will be more bits in the file than are actually a part of the
encoding; using \texttt{PP.take} ensures that we don't try to read the extra
padding bits. It'll take up to \texttt{len} \texttt{Word8}s, and then stop.

And so now we have our \texttt{Word8}/byte Producer/stream!

\hypertarget{searchpt}{%
\subsubsection{searchPT}\label{searchpt}}

One could actually have written \texttt{searchPT} like this:

\begin{Shaded}
\begin{Highlighting}[]
\OtherTok{searchPT ::} \DataTypeTok{PreTree}\NormalTok{ a }\OtherTok{->} \DataTypeTok{Pipe} \DataTypeTok{Direction} \DataTypeTok{Word8}\NormalTok{ m r}
\NormalTok{searchPT pt0 }\FunctionTok{=}\NormalTok{ go pt0}
  \KeywordTok{where}
\NormalTok{    go (}\DataTypeTok{PTLeaf}\NormalTok{ x) }\FunctionTok{=} \KeywordTok{do}
\NormalTok{        yield x}
\NormalTok{        go pt0}
\NormalTok{    go (}\DataTypeTok{PTNode}\NormalTok{ pt1 pt2) }\FunctionTok{=} \KeywordTok{do}
\NormalTok{        dir }\OtherTok{<-}\NormalTok{ await}
\NormalTok{        go }\FunctionTok{$} \KeywordTok{case}\NormalTok{ dir }\KeywordTok{of}
               \DataTypeTok{DLeft}  \OtherTok{->}\NormalTok{ pt1}
               \DataTypeTok{DRight} \OtherTok{->}\NormalTok{ pt2}
\end{Highlighting}
\end{Shaded}

which looks a lot like the logic of our decoder functions from
\href{http://blog.jle.im/entry/streaming-huffman-compression-in-haskell-part-2-binary}{Part
2}.

However, we can do better. This way sort of mixes together the ``logic'' of
decoding from the yielding/continuation/recursion/pipe-ness of it all. Ideally
we'd like to be able to separate the logic. This isn't \emph{too} necessary, but
doing this will expose us to some nice \emph{pipes} idioms :)

One way we can do it is to turn \texttt{searchPT} into a
\texttt{Consumer\textquotesingle{}} (a \texttt{Consumer} with the ends not
sealed off) that consumes \texttt{Direction}s and \emph{returns} resulting
\texttt{Word8}s.

Then we use \texttt{(\textgreater{}\textasciitilde{}\ cat)}, which turns a
\texttt{Consumer\textquotesingle{}} into something that is forever consuming and
re-yielding --- in essence, it turns a \texttt{Consumer\textquotesingle{}}
returning values into a \texttt{Pipe} repeatedly yielding the returned values.

\begin{Shaded}
\begin{Highlighting}[]
\CommentTok{-- source: https://github.com/mstksg/inCode/tree/master/code-samples/huffman/decode.hs#L74-L86}

\OtherTok{searchPT ::}\NormalTok{ forall a m r}\FunctionTok{.} \DataTypeTok{Monad}\NormalTok{ m}
         \OtherTok{=>} \DataTypeTok{PreTree}\NormalTok{ a}
         \OtherTok{->} \DataTypeTok{Pipe} \DataTypeTok{Direction}\NormalTok{ a m r}
\NormalTok{searchPT t }\FunctionTok{=}\NormalTok{ searchPT' t }\FunctionTok{>~}\NormalTok{ cat}
  \KeywordTok{where}
\OtherTok{    searchPT' ::} \DataTypeTok{PreTree}\NormalTok{ a }\OtherTok{->} \DataTypeTok{Consumer'} \DataTypeTok{Direction}\NormalTok{ m a}
\NormalTok{    searchPT' (}\DataTypeTok{PTLeaf}\NormalTok{ x)       }\FunctionTok{=}
\NormalTok{        return x}
\NormalTok{    searchPT' (}\DataTypeTok{PTNode}\NormalTok{ pt1 pt2) }\FunctionTok{=} \KeywordTok{do}
\NormalTok{        dir }\OtherTok{<-}\NormalTok{ await}
\NormalTok{        searchPT' }\FunctionTok{$} \KeywordTok{case}\NormalTok{ dir }\KeywordTok{of}
                      \DataTypeTok{DLeft}  \OtherTok{->}\NormalTok{ pt1}
                      \DataTypeTok{DRight} \OtherTok{->}\NormalTok{ pt2}
\end{Highlighting}
\end{Shaded}

The logic is slightly cleaner; the gain isn't that much, but just being able to
have this separation is nice. Also, we get rid of explicit recursion. And
everybody knows that every time you can get rid of explicit recursion, you get a
big win --- in lack of potential bugs, in more concise code, and in leveraging
higher order functions. In any case, this is also a good exposure to
\texttt{(\textgreater{}\textasciitilde{})}!

\begin{center}\rule{0.5\linewidth}{\linethickness}\end{center}

\textbf{Aside}

\texttt{(\textgreater{}\textasciitilde{})} is a pretty useful thing. Basically,
when you say

\begin{Shaded}
\begin{Highlighting}[]
\NormalTok{consumer }\FunctionTok{>~}\NormalTok{ pipe}
\end{Highlighting}
\end{Shaded}

it is like saying ``\emph{Every time \texttt{pipe} \texttt{await}s, just use the
result returned by \texttt{consumer} instead}''.

We can look at \texttt{cat}:

\begin{Shaded}
\begin{Highlighting}[]
\OtherTok{cat ::} \DataTypeTok{Pipe}\NormalTok{ a a m r}
\NormalTok{cat }\FunctionTok{=}\NormalTok{ forever }\FunctionTok{$} \KeywordTok{do}
\NormalTok{        a }\OtherTok{<-}\NormalTok{ await}
\NormalTok{        yield a}
\end{Highlighting}
\end{Shaded}

Which just simply echoes/sends back down whatever it receives.

When we say:

\begin{Shaded}
\begin{Highlighting}[]
\NormalTok{consumer }\FunctionTok{>~}\NormalTok{ cat}
\end{Highlighting}
\end{Shaded}

We basically say ``every time we \texttt{await} something in \texttt{cat}, just
use \texttt{consumer}'s return value'':

\begin{Shaded}
\begin{Highlighting}[]
\NormalTok{consumer }\FunctionTok{>~}\NormalTok{ cat}
    \FunctionTok{=}\NormalTok{ forever }\FunctionTok{$} \KeywordTok{do}
\NormalTok{        a }\OtherTok{<-}\NormalTok{ consumer}
\NormalTok{        yield a}
\end{Highlighting}
\end{Shaded}

Basically, \texttt{consumer\ \textgreater{}\textasciitilde{}\ cat} repeatedly
consumes the input and yields downstream the return of the consuming.

(Remember, the value the pipe returns (the \texttt{r}) is different than the
value the pipe ``sends downstream''; the downstream values are used in
connecting with \texttt{(\textgreater{}-\textgreater{})} and the like, and the
return value is just the value that the specific thing \emph{returns} when ran,
to the thing doing the running.)

Play around with \texttt{(\textgreater{}\textasciitilde{})}-ifying different
\texttt{Pipe}s and seeing what it does to it; you might have some fun.

Why \texttt{Consumer\textquotesingle{}} and not \texttt{Consumer}? Well,
remember that all lines of the \texttt{do} block have to have the same ``yield''
type (because the Monad is \texttt{Pipe\ a\ b\ m}, so all lines have to be
\texttt{Pipe\ a\ b\ m\ r} for different \texttt{r}'s --- the \texttt{a}'s and
\texttt{b}'s (the yield type) and \texttt{m}'s have to be the same), so
\texttt{Consumer\textquotesingle{}} lets the yield type be whatever it needs to
be to match with the rest of the \texttt{do} block.

Don't worry if this is a bit complicated; you don't need to really undersatnd
this to use \emph{pipes} :)

Admittedly, my description isn't too great, so if anyone has a better one, I'd
be happy to use it here!

\begin{center}\rule{0.5\linewidth}{\linethickness}\end{center}

\hypertarget{the-rest}{%
\subsubsection{The Rest}\label{the-rest}}

And the rest is\ldots{}well, we already know it!

We use \texttt{(view\ PB.pack)\ byteStream} like last time to turn our stream of
\texttt{Word8} into a stream of \texttt{ByteString}, with ``smart chunking''.
Then we pipe that in to \texttt{PB.toHandle}, like we did last time, and have it
all flow into the output file.

We have assembled our pipeline; all we have to do now is \texttt{runEffect}, to
``run'' it. And again, that's it!

\hypertarget{testing}{%
\subsection{Testing}\label{testing}}

\begin{Shaded}
\begin{Highlighting}[]
\NormalTok{$ }\ExtensionTok{ghc}\NormalTok{ -O2 decode.hs}
\NormalTok{$ }\ExtensionTok{./decode}\NormalTok{ warandpeace.enc warandpeace.dec}
\NormalTok{$ }\ExtensionTok{md5sum}\NormalTok{ warandpeace.txt}
\CommentTok{# 3c8168e48f49784ac3c2c25d15388e96  warandpeace.txt}
\NormalTok{$ }\ExtensionTok{md5sum}\NormalTok{ warandpeace.dec}
\CommentTok{# 3c8168e48f49784ac3c2c25d15388e96  warandpeace.dec}
\end{Highlighting}
\end{Shaded}

And yup, we get an exact, lossless decompression.

Decompression is faster than compression, as you'd expect; on my computer it
takes about two seconds to decompress the 3.1M file. Still a bit slower than
we'd like, but not \emph{too} bad. Well. Maybe.

\hypertarget{conclusion}{%
\section{Conclusion}\label{conclusion}}

Hopefully from this all, you can see \emph{pipes} as a beautiful abstraction for
chaining together and transforming streaming computations in a pure,
constant-space way. I hope that looking back on it all you can see everything as
either a transformation of pipes, or a chaining of pipes.

I recommend looking more into the great \emph{pipes} documentation, joining the
\href{https://groups.google.com/forum/\#!forum/haskell-pipes}{pipes mailing
list}, and trying your own streaming data projects with \emph{pipes} to see what
you can do with it.

You should also checkout
\emph{\href{https://hackage.haskell.org/package/conduit}{conduit}} and try to
implement this streaming logic in that framework. Let me know how it turns out!

As always the great people of freenode's \#haskell are always free to answer any
questions you might have, and also of course the
\href{http://stackoverflow.com/questions/tagged/haskell}{haskell tag} on Stack
Overflow. And I'll try to address as many questions as I can in the comments!

Keep in mind that I'm still a new user of \emph{pipes} myself; if I've made any
huge mistakes in style or idiomaticness, I'm available here in the comments and
I'd appreciate any corrections y'all can offer.

So ends the ``pipes tutorial'' section of this series; tune in next time and
we'll try our best to optimize this baby! \footnote{Hopefully you aren't holding
  your breath on this one :) This next part is not scheduled any ime soon and
  might not come for a while, as I'll be pursuing some other things in the near
  future --- I apologize for any disappointment/inconvenience this may cause.}

\begin{center}\rule{0.5\linewidth}{\linethickness}\end{center}

\textbf{Bonus Round: \emph{Full Lens}}

Hey guess what! Let's try and go \emph{full lens} :)

(This section does not invalidate anything you learned already, so if you have
problems with it, it's okay :) )

Now, you might have thought, ``Hey, we used \texttt{view\ PB.pack} to turn our
\texttt{Word8} producer into a \texttt{ByteString} producer\ldots{}couldn't we
just use \texttt{view\ PB.unpack} to turn our \texttt{ByteString} producer into
a \texttt{Word8} producer in the first place???''

Yup! In fact, this takes us into a\ldots{}``pipe transformer style'' of pipes
code, as opposed to a ``pipe composition style'' of pipes code. Both ways are
considered ``idiomatic'', and it's up to you to decide what suits you more.

Basically, we don't ever need \texttt{bsToBytes}; instead of

\begin{Shaded}
\begin{Highlighting}[]
\CommentTok{-- source: https://github.com/mstksg/inCode/tree/master/code-samples/huffman/encode.hs#L65-L65}

    \KeywordTok{let}\NormalTok{ byteProducer }\FunctionTok{=}\NormalTok{ PB.fromHandle hIn }\FunctionTok{>->}\NormalTok{ bsToBytes}
\end{Highlighting}
\end{Shaded}

We can just write

\begin{Shaded}
\begin{Highlighting}[]
\KeywordTok{let}\NormalTok{ byteProducer }\FunctionTok{=}\NormalTok{ (view PB.unpack) (PB.fromHandle hIn)}
\end{Highlighting}
\end{Shaded}

Okay, one last thing.

With \emph{lens}, we not only have the ability to ``view'' the
\texttt{ByteString} producer ``as a'' \texttt{Word8} producer.

We also have the ability to \emph{modify} the \texttt{Word8} producer that we
``see''\ldots{}and \emph{put it back into} the \texttt{ByteString} producer!

That is, if I have a \texttt{ByteString} producer, I can see the \texttt{Word8}
producer, modify it, and ``stick it back into'' the \texttt{ByteString}
producer\ldots{}to basically create a new \texttt{ByteString} producer that
instead outputs our ``modified'' \texttt{Word8} producer.

It's like a fancy \texttt{fmap}. And like how \texttt{view} was how we
``unlocked'' the viewer from the lens, we use \texttt{over} to ``unlock'' the
``pull out, edit, and stick back in''.

That is, in our case,

\begin{Shaded}
\begin{Highlighting}[]
\OtherTok{over ::} \DataTypeTok{Lens'}\NormalTok{ (}\DataTypeTok{Producer} \DataTypeTok{ByteString}\NormalTok{ m r) (}\DataTypeTok{Producer} \DataTypeTok{Word8}\NormalTok{ m r)}
     \OtherTok{->}\NormalTok{ (}\DataTypeTok{Producer} \DataTypeTok{Word8}\NormalTok{ m r }\OtherTok{->} \DataTypeTok{Producer} \DataTypeTok{Word8}\NormalTok{ m r)}
     \OtherTok{->} \DataTypeTok{Producer} \DataTypeTok{ByteString}\NormalTok{ m r}
     \OtherTok{->} \DataTypeTok{Producer} \DataTypeTok{ByteString}\NormalTok{ m r}
\end{Highlighting}
\end{Shaded}

What does this mean, in practice?

That means that we can use \texttt{over}, apply a function to the \texttt{Word8}
producer, and \texttt{over} will \emph{handle the re-packing} (with the smart
chunking) for us, all in one swoop.

So, we can rewrite \texttt{bsOut}:

\begin{Shaded}
\begin{Highlighting}[]
\NormalTok{bsIn      }\FunctionTok{=}\NormalTok{ PB.fromHandle hIn}
\NormalTok{bsOut     }\FunctionTok{=}\NormalTok{ flip (over PB.unpack) bsIn }\FunctionTok{$}\NormalTok{ \textbackslash{}bytesOut }\OtherTok{->}
\NormalTok{                dirsBytes ( bytesOut}
                        \FunctionTok{>->}\NormalTok{ encodeByte encTable )}
\NormalTok{pipeline  }\FunctionTok{=}\NormalTok{ bsOut}
        \FunctionTok{>->}\NormalTok{ PB.toHandle hOut}
\end{Highlighting}
\end{Shaded}

So \texttt{over\ PB.unpack} handles the unpacking (to get \texttt{bytesOut}) and
the re-packing (after the result of \texttt{dirsBytes}) for us, in one fell
swoop.

Neat!

Okay now, good bye, for reals!

\begin{center}\rule{0.5\linewidth}{\linethickness}\end{center}

Hi, thanks for reading! You can reach me via email at
\href{mailto:justin@jle.im}{\nolinkurl{justin@jle.im}}, or at twitter at
\href{https://twitter.com/mstk}{@mstk}! This post and all others are published
under the \href{https://creativecommons.org/licenses/by-nc-nd/3.0/}{CC-BY-NC-ND
3.0} license. Corrections and edits via pull request are welcome and encouraged
at \href{https://github.com/mstksg/inCode}{the source repository}.

If you feel inclined, or this post was particularly helpful for you, why not
consider \href{https://www.patreon.com/justinle}{supporting me on Patreon}, or a
\href{bitcoin:3D7rmAYgbDnp4gp4rf22THsGt74fNucPDU}{BTC donation}? :)

\end{document}
