\documentclass[]{article}
\usepackage{lmodern}
\usepackage{amssymb,amsmath}
\usepackage{ifxetex,ifluatex}
\usepackage{fixltx2e} % provides \textsubscript
\ifnum 0\ifxetex 1\fi\ifluatex 1\fi=0 % if pdftex
  \usepackage[T1]{fontenc}
  \usepackage[utf8]{inputenc}
\else % if luatex or xelatex
  \ifxetex
    \usepackage{mathspec}
    \usepackage{xltxtra,xunicode}
  \else
    \usepackage{fontspec}
  \fi
  \defaultfontfeatures{Mapping=tex-text,Scale=MatchLowercase}
  \newcommand{\euro}{€}
\fi
% use upquote if available, for straight quotes in verbatim environments
\IfFileExists{upquote.sty}{\usepackage{upquote}}{}
% use microtype if available
\IfFileExists{microtype.sty}{\usepackage{microtype}}{}
\usepackage[margin=1in]{geometry}
\ifxetex
  \usepackage[setpagesize=false, % page size defined by xetex
              unicode=false, % unicode breaks when used with xetex
              xetex]{hyperref}
\else
  \usepackage[unicode=true]{hyperref}
\fi
\hypersetup{breaklinks=true,
            bookmarks=true,
            pdfauthor={},
            pdftitle={},
            colorlinks=true,
            citecolor=blue,
            urlcolor=blue,
            linkcolor=magenta,
            pdfborder={0 0 0}}
\urlstyle{same}  % don't use monospace font for urls
% Make links footnotes instead of hotlinks:
\renewcommand{\href}[2]{#2\footnote{\url{#1}}}
\setlength{\parindent}{0pt}
\setlength{\parskip}{6pt plus 2pt minus 1pt}
\setlength{\emergencystretch}{3em}  % prevent overfull lines
\setcounter{secnumdepth}{0}


\begin{document}

\% Intro to Machines \& Arrows (Part 1: Stream and Auto) \% Justin Le \%
February 3, 2014

\emph{Originally posted on
\textbf{\href{https://blog.jle.im/entry/intro-to-machines-arrows-part-1-stream-and.html}{in
Code}}.}

So I'm going to be running a series soon on computation and (physical)
simulations using AFRP (Arrowized Functional Reactive Programming) principles.

I consider (A)FRP to actually be a pretty game changing paradigm. It provides us
with semantics by which to \emph{compose} and build time-varying, reactive
behaviors and completely changes the way we approach any sort of
simulation/state-like project.

AFRP has its own elegant way of approaching problems, but to be able to properly
use it for simulations, we're going to have to start by learning about the
fundamental abstraction behind its \textbf{implementation}:
machines.{[}\^{}implementation{]}

(This post will assume a somewhat basic knowledge of Haskell. I'll try
explaining concepts here and there if I feel that they might not be very
commonly known. But if you have any questions, feel free to leave a comment or
stop by freenode's \#haskell on irc!)

(A short disclaimer: this article has not too much to do with the great
\href{http://hackage.haskell.org/package/machines}{machines} library by Rúnar
Bjarnason and Edward Kmett)

\section{Why FRP?}

This series doesn't directly deal with FRP, but implementing FRP is one of the
motivating factors for what we are discussing.

So why do we even bother with FRP? Why not just just program imperatively and
mutate a state? It's just as easy in a functional/pure language (with a trivial
State monad wrapper) as it is in an imperative language.

The answer is the full power of functional programming:
\textbf{\href{http://www.haskellforall.com/2012/08/the-category-design-pattern.html}{composition}}.
Anyone who has dabbled in functional programming (or who has even used Unix
pipes) has had a glimpse into the power of composition. We start with simple,
fundamental, self-contained behaviors and compose them piece-by-piece into a
complex one.

So why FRP? FRP provides for us meaningful semantics by which to \emph{compose}
time-varying and reactive behaviors, and create complex ones from smaller ones.
You reason about a self-contained behavior "wire"...and then use tools to build
complex behaviors from simple ones. You don't specify what happens when, you
specify how things \emph{should behave}, from a top-down level. And then compose
those behaviors.

So to proceed, in this post we are going to introduce Streams and (more
importantly) Autos. In the next, we will be looking at Autos as a member of the
powerful Category and Arrow typeclasses. In the final post, we'll really be
throwing in a lot of useful enhancements to the \texttt{Auto} type in hopes that
you may be prepared for anything you encounter in the real world involving it,
including the popular AFRP library
\href{http://hackage.haskell.org/package/netwire}{netwire}.

\section{Streams}

(All the code for Streams can be downloaded
\href{https://github.com/mstksg/inCode/tree/master/code-samples/machines/Stream.hs}{from
github} and tried interactively online
\href{https://www.fpcomplete.com/user/jle/machines}{at FPComplete})

Let's start with streams, one of the simpler of machines.

Streams are basically infinitely long linked lists.

\textasciitilde{}\textasciitilde{}\textasciitilde{}haskell data Stream b = SCons
(b, Stream b) \textasciitilde{}\textasciitilde{}\textasciitilde{}

(\texttt{SCons} for "Stream cons")

Compare that with the linked list data type, which is a Stream with an Ending
(\texttt{Nil}):

\textasciitilde{}\textasciitilde{}\textasciitilde{}haskell -\/- source:
https://github.com/mstksg/inCode/tree/master/code-samples/machines/Stream.hs\#L7-7
-\/- interactive: https://www.fpcomplete.com/user/jle/machines data List a =
Cons (a, List a) \textbar{} Nil
\textasciitilde{}\textasciitilde{}\textasciitilde{}

or, as is more traditionally written:

\textasciitilde{}\textasciitilde{}\textasciitilde{}haskell data {[}a{]} = (:) a
{[}a{]} \textbar{} {[}{]} \textasciitilde{}\textasciitilde{}\textasciitilde{}

It's pretty easy to build lists:

\textasciitilde{}\textasciitilde{}\textasciitilde{}haskell -\/- source:
https://github.com/mstksg/inCode/tree/master/code-samples/machines/Stream.hs\#L36-37
-\/- interactive: https://www.fpcomplete.com/user/jle/machines myList :: List
Int myList = Cons ( 1, Cons ( 2, Cons (3, Nil) ) )
\textasciitilde{}\textasciitilde{}\textasciitilde{}

which is just, in the more traditional (infix) form:

\textasciitilde{}\textasciitilde{}\textasciitilde{}haskell -\/- source:
https://github.com/mstksg/inCode/tree/master/code-samples/machines/Stream.hs\#L40-41
-\/- interactive: https://www.fpcomplete.com/user/jle/machines myList' ::
{[}Int{]} myList' = 1:(2:(3:{[}{]}))
\textasciitilde{}\textasciitilde{}\textasciitilde{}

Let's see if \texttt{myList} does what we want: (a list from 1 to 3):

\textasciitilde{}\textasciitilde{}\textasciitilde{}haskell ghci\textgreater{}
let (Cons (x,xs)) = myList ghci\textgreater{} x 1 ghci\textgreater{} :t xs xs ::
List Int ghci\textgreater{} let (Cons (y,ys)) = xs ghci\textgreater{} y 2
ghci\textgreater{} let (Cons (z,zs)) = ys ghci\textgreater{} z 3
ghci\textgreater{} zs Nil \textasciitilde{}\textasciitilde{}\textasciitilde{}

Yes! Perfect. We can "traverse" down our linked list by repeatedly pattern
matching out the "head" (the \texttt{x}, the first part of the tuple) and the
"tail" (the \texttt{xs}, the second part of the tuple).

But how are we going to build a stream...? We don't have access to \texttt{Nil},
so do we have to manually type out an infinite stream?

Let's try defining the stream \texttt{{[}1..{]}} -\/-\/- a stream that contains
every natural number starting from 1.

\textasciitilde{}\textasciitilde{}\textasciitilde{}haskell myStream' :: Stream
Int myStream' = SCons ( 1, SCons ( 2, SCons ( 3, ... ) ) )
\textasciitilde{}\textasciitilde{}\textasciitilde{}

Hm. This is going to take a while. I wonder if there's an easier way?

We can take advantage of Haskell's "lazy-by-default"-ness and leave the "rest"
of the stream as an unevaluated function call. And then we can recurse!

\textasciitilde{}\textasciitilde{}\textasciitilde{}haskell -\/- source:
https://github.com/mstksg/inCode/tree/master/code-samples/machines/Stream.hs\#L46-50
-\/- interactive: https://www.fpcomplete.com/user/jle/machines myStream ::
Stream Int myStream = streamFrom 1 where streamFrom :: Int -\textgreater{}
Stream Int streamFrom n = SCons ( n, streamFrom (n+1) )
\textasciitilde{}\textasciitilde{}\textasciitilde{}

Cool! Let's see if this \texttt{myStream} really does what we want, the same way
we tested \texttt{myList}:

\textasciitilde{}\textasciitilde{}\textasciitilde{}haskell ghci\textgreater{}
let (SCons (x, xs)) = myStream ghci\textgreater{} x 1 ghci\textgreater{} :t xs
xs :: Stream Int ghci\textgreater{} let (SCons (y, ys)) = xs ghci\textgreater{}
y 2 ghci\textgreater{} let (SCons (z, zs)) = ys ghci\textgreater{} z 3
ghci\textgreater{} let (SCons (j,js)) = zs ghci\textgreater{} j 4
ghci\textgreater{} let (SCons (k,ks)) = js ghci\textgreater{} k 5
\textasciitilde{}\textasciitilde{}\textasciitilde{}

Yes, it works perfectly! Just like in the case of List, we can "traverse" down
the stream by pattern matching out the "head" of the stream (the first part of
the tuple) and the "tail" of the stream (the second part of the tuple).

Note that we can have some fun with Haskell syntax \texttt{Stream} by adding a
record label to the first (and only) field:

\textasciitilde{}\textasciitilde{}\textasciitilde{}haskell data Stream b = SCons
\{ runStream :: (b, Stream b) \}
\textasciitilde{}\textasciitilde{}\textasciitilde{}

so that we can do fancy things like:

\textasciitilde{}\textasciitilde{}\textasciitilde{}haskell ghci\textgreater{} :t
runStream runStream :: Stream b -\textgreater{} (b, Stream b) ghci\textgreater{}
let (x, xs) = runStream myStream ghci\textgreater{} x 1
\textasciitilde{}\textasciitilde{}\textasciitilde{}

Basically, we get for free the function \texttt{runStream}, a function that
yanks the tuple out of the stream.

One minor final touch -\/-\/- because \texttt{Stream} has only one constructor
and one field, we can make it a \texttt{newtype}, which has similar usage
patterns/syntax as a \texttt{data}, but which the compiler can more easily
optimize:

\textasciitilde{}\textasciitilde{}\textasciitilde{}haskell -\/- source:
https://github.com/mstksg/inCode/tree/master/code-samples/machines/Stream.hs\#L10-10
-\/- interactive: https://www.fpcomplete.com/user/jle/machines newtype Stream b
= SCons \{ runStream :: (b, Stream b) \}
\textasciitilde{}\textasciitilde{}\textasciitilde{}

\subsubsection{Automating Traversal}

The repeated pattern matching we've been doing is kind of tedious, and it'll
only get more annoying over time, so let's make a function that can automate the
pattern matching for us really quickly so that we can test it more easily.

\texttt{streamToList} will take a Stream and perform the very straightforward
conversion into an infinite list.

\textasciitilde{}\textasciitilde{}\textasciitilde{}haskell -\/- source:
https://github.com/mstksg/inCode/tree/master/code-samples/machines/Stream.hs\#L15-16
-\/- interactive: https://www.fpcomplete.com/user/jle/machines streamToList ::
Stream b -\textgreater{} {[}b{]} streamToList (SCons (x, xs)) = x : streamToList
xs \textasciitilde{}\textasciitilde{}\textasciitilde{}

So now we can do:

\textasciitilde{}\textasciitilde{}\textasciitilde{}haskell ghci\textgreater{}
take 10 \$ streamToList myStream {[}1,2,3,4,5,6,7,8,9,10{]}
\textasciitilde{}\textasciitilde{}\textasciitilde{}

Alternatively (and for reasons which will later become clear), we can also
define \texttt{testStream}, which takes a specified amount of elements and
returns also the "resulting" stream after all of those steps, and
\texttt{testStream\_}, which is the same thing except that we throw away the
modified stream.

\textasciitilde{}\textasciitilde{}\textasciitilde{}haskell -\/- source:
https://github.com/mstksg/inCode/tree/master/code-samples/machines/Stream.hs\#L20-30
-\/- interactive: https://www.fpcomplete.com/user/jle/machines testStream ::
Stream b -\textgreater{} Int -\textgreater{} ({[}b{]}, Stream b) testStream strm
0 = ({[}{]} , strm ) testStream strm n = (y:ys, final) where (y , next ) =
runStream strm (ys, final) = testStream next (n-1)

testStream\_ :: Stream b -\textgreater{} Int -\textgreater{} {[}b{]}
testStream\_ = (fst .) . testStream
\textasciitilde{}\textasciitilde{}\textasciitilde{}

\textasciitilde{}\textasciitilde{}\textasciitilde{}haskell ghci\textgreater{}
testStream\_ myStream 10 {[}1,2,3,4,5,6,7,8,9,10{]}
\textasciitilde{}\textasciitilde{}\textasciitilde{}

\subsection{Streams are nice}

Streams are nice! If you've been using Haskell for any stretch of time, you'll
know that we use infinite lists all the time and to great usefulness for the
ends of expressiveness and abstraction.

In mathematics, streams are known as a form of
\href{http://en.wikipedia.org/wiki/Moore_machine}{Moore machines} (albeit with
potentially infinite sets of state and output values). They are machines that
basically internally progress from state to state to state to state. They just
keep on marching on...like a machine. In \texttt{myStream}, the initial state is
1. The next state is 2; the next is 3, etc. What is important is that the next
state \emph{is a function of the current state}.

This is made very apparent in our definition of \texttt{streamFrom}:

\textasciitilde{}\textasciitilde{}\textasciitilde{}haskell -\/- source:
https://github.com/mstksg/inCode/tree/master/code-samples/machines/Stream.hs\#L49-50
-\/- interactive: https://www.fpcomplete.com/user/jle/machines streamFrom :: Int
-\textgreater{} Stream Int streamFrom n = SCons ( n, streamFrom (n+1) )
\textasciitilde{}\textasciitilde{}\textasciitilde{}

The "current state" whenever we call \texttt{streamFrom\ n} is \texttt{n}...the
"next state" (the "initial state" of the "tail") is \texttt{n+1}. We could have
provided any function \texttt{::\ Int\ -\textgreater{}\ Int} there (say,
\texttt{n+2}), that would be for us our "next state" function.

So \texttt{myStream} is a Moore-like machine whose "next state" function is "the
current state plus one", and whose output is just the state itself.

\subsubsection{State your purpose!}

Did you catch that last sentence? It's a subtle point. In general, streams can
have outputs that are different than their states. As a trivial example, let's
have a stream whose state is an integer, yet whose output is a character:

\textasciitilde{}\textasciitilde{}\textasciitilde{}haskell -\/- source:
https://github.com/mstksg/inCode/tree/master/code-samples/machines/Stream.hs\#L53-57
-\/- interactive: https://www.fpcomplete.com/user/jle/machines charStream ::
Stream Char charStream = charStreamFrom 65 where charStreamFrom :: Int
-\textgreater{} Stream Char charStreamFrom n = SCons ( chr n, charStreamFrom
(n+1) ) \textasciitilde{}\textasciitilde{}\textasciitilde{}

\textasciitilde{}\textasciitilde{}\textasciitilde{}haskell ghci\textgreater{}
take 10 \$ streamToList charStream "ABCDEFGHIJ"
\textasciitilde{}\textasciitilde{}\textasciitilde{}

Wait, this is kind of weird. The type of our stream is
\texttt{Stream\ Char}...\texttt{Char} is the type of output/elements in the
stream, the "head" when we pattern match. But where is the \texttt{Int} that is
the state of our stream in the type \texttt{Stream\ Char}...?

Can we even write a function
\texttt{getState\ ::\ Stream\ b\ -\textgreater{}\ s} that works in general for
all streams?

Hm. If the state of our stream can have a type totally unrelated to the type of
the stream...that means that we probably can't even know what it type is. And
that even if we could "force" it out somehow, we would not even be able to work
with it in a type-safe way!

In fact...couldn't the state even
\emph{\href{https://github.com/mstksg/inCode/tree/master/code-samples/machines/Stream.hs\#L60-70}{vary
dynamically}} as the stream progresses?

\subsection{Continuing on}

The problem with streams, as you might have guessed, is that you can't really
affect their progress once they start. Once you start \texttt{myStream}, it'll
keep on marching on, and on, and on...you have no way to "influence" its
progression \emph{during} its march. The \emph{behavior} of our stream
\emph{can't be influenced} by the outside world in any way, once it has started.
This is a bit limiting, because we want behaviors that we can have interact with
each other.

And so, we have the natural generalization of streams (and the machine we will
be spending the most time looking at): Auto.

\section{Auto}

(All the code for Autos can be downloaded
\href{https://github.com/mstksg/inCode/tree/master/code-samples/machines/Auto.hs}{from
github} and tried interactively online
\href{https://www.fpcomplete.com/user/jle/machines}{at FPComplete})

Let's upgrade our streams, and introduce a way to affect how they progress.
Let's call it an Auto.

\textasciitilde{}\textasciitilde{}\textasciitilde{}haskell -\/- source:
https://github.com/mstksg/inCode/tree/master/code-samples/machines/Auto.hs\#L12-12
-\/- interactive: https://www.fpcomplete.com/user/jle/machines newtype Auto a b
= ACons \{ runAuto :: a -\textgreater{} (b, Auto a b) \}
\textasciitilde{}\textasciitilde{}\textasciitilde{}

Now, instead of an \texttt{SCons} containing just a tuple (a head-tails), an
\texttt{ACons} contains a \emph{function} that \emph{produces} your head-tails
tuple. Before, all of our \texttt{runStreams} produced the same tuple no matter
what. Now, our \texttt{runAuto} can produce a different tuple based on an
outside input.

This is cool!

Let's look at the type signature of Auto before we go too much further.

In \texttt{Auto\ a\ b}, \texttt{b} is (just like for a Stream) the type of your
"head" and the type of the items in your "tail". It's the type of your "stream".

\texttt{a} is the type of the "influencing input".

So now, we basically have a \texttt{Stream\ b}, except at every "step", we can
"influence" the Stream with something of type \texttt{a}.

\subsection{A Trivial Auto}

Let's look at a direct "port" of our \texttt{myStream}:

\textasciitilde{}\textasciitilde{}\textasciitilde{}haskell -\/- source:
https://github.com/mstksg/inCode/tree/master/code-samples/machines/Auto.hs\#L40-44
-\/- interactive: https://www.fpcomplete.com/user/jle/machines myStreamAuto ::
Auto a Int myStreamAuto = streamAutoFrom 1 where streamAutoFrom :: Int
-\textgreater{} Auto a Int streamAutoFrom n = ACons \$ \_ -\textgreater{} ( n,
streamAutoFrom (n+1) ) \textasciitilde{}\textasciitilde{}\textasciitilde{}

This is kind of a dumb example, but \texttt{myStreamAuto} is just the exact same
as \texttt{myStream}. It's an Auto, but it \emph{ignores its influencing input}.

Let's try it out.

\textasciitilde{}\textasciitilde{}\textasciitilde{}haskell ghci\textgreater{} :t
runAuto runAuto :: Auto a b -\textgreater{} (a -\textgreater{} (b, Auto a b))
ghci\textgreater{} let (x, xs) = runAuto myStreamAuto undefined
ghci\textgreater{} x 1 ghci\textgreater{} :t xs xs :: Auto a Int
ghci\textgreater{} let (y, ys) = runAuto xs undefined ghci\textgreater{} y 2
ghci\textgreater{} let (z, zs) = runAuto ys undefined ghci\textgreater{} z 3
\textasciitilde{}\textasciitilde{}\textasciitilde{}

Remember that we are really doing \texttt{(runAuto\ myStreamAuto)\ undefined},
but because of how Haskell associates function calls, the parentheses are
unnecessary. And hey, it kind of looks like \texttt{runAuto} is a two-parameter
function with an Auto as the first parameter and the "influence"/"input" as its
second. Which, due to the magic of currying-by-default, it basically is, in
Haskell!

\subsection{A Non-trivial Auto}

Okay, that was fun I guess. But now let's take a first look at an auto which
\emph{can} be influenced.

Let's have a stream where at every step, we can choose to "reset" the counter to
whatever integer we like.

We can do this by having the influence/input be a \texttt{Maybe\ Int}. If we
want the counter to progress normally, we pass in a \texttt{Nothing}. If we want
the counter to reset to a number \texttt{n} of our choosing, we pass in a
\texttt{Just\ n}

\textasciitilde{}\textasciitilde{}\textasciitilde{}haskell -\/- source:
https://github.com/mstksg/inCode/tree/master/code-samples/machines/Auto.hs\#L48-54
-\/- interactive: https://www.fpcomplete.com/user/jle/machines settableAuto ::
Auto (Maybe Int) Int settableAuto = counterFrom 1 where counterFrom :: Int
-\textgreater{} Auto (Maybe Int) Int counterFrom n = ACons \$
\textbackslash{}reset -\textgreater{} let c = fromMaybe n reset in ( c,
counterFrom (c + 1) ) \textasciitilde{}\textasciitilde{}\textasciitilde{}

Remember that
\texttt{fromMaybe\ ::\ a\ -\textgreater{}\ Maybe\ a\ -\textgreater{}\ a} takes a
"default" value, a Maybe value, and then returns the value inside the Maybe if
it's a \texttt{Just}, or the default value if it's a \texttt{Nothing}.

So basically, when you \texttt{runAuto} with the Auto, if you give it a
\texttt{Nothing}, it'll give you \texttt{(\ n,\ counterFrom\ (n+1)\ )}. If you
give it \texttt{Just\ m}, it'll give you \texttt{(\ m,\ counterFrom\ (m+1)\ )}.

Cool -\/-\/- let's try it out.

\textasciitilde{}\textasciitilde{}\textasciitilde{}haskell ghci\textgreater{}
let (x, xs) = runAuto settableAuto Nothing ghci\textgreater{} x 1
ghci\textgreater{} let (y, ys) = runAuto xs Nothing ghci\textgreater{} y 2
ghci\textgreater{} let (z, zs) = runAuto ys (Just 10) ghci\textgreater{} z 10
ghci\textgreater{} let (j, js) = runAuto zs Nothing ghci\textgreater{} j 11
ghci\textgreater{} let (k, ks) = runAuto js Nothing ghci\textgreater{} k 12
ghci\textgreater{} let (l, ls) = runAuto ks (Just (-1)) ghci\textgreater{} l -1
ghci\textgreater{} let (m, ms) = runAuto ls Nothing ghci\textgreater{} m 0
\textasciitilde{}\textasciitilde{}\textasciitilde{}

And there ya go.

\subsubsection{Automatic traversals for our Autos}

Again, the manual pattern matching is a little tedious so let's write us a
function to automate "progressing" down an Auto.

Like our \texttt{testStream}, \texttt{testAuto} takes an Auto. But because every
"step" needs an input, \texttt{testAuto\ auto} takes a \emph{list} that
specifies the input for every step. \texttt{testAuto} returns the resulting
collection of results, and also the modified Auto. \texttt{testAuto\_} throws
away the new Auto and just gives us the collection.

\textasciitilde{}\textasciitilde{}\textasciitilde{}haskell -\/- source:
https://github.com/mstksg/inCode/tree/master/code-samples/machines/Auto.hs\#L17-25
-\/- interactive: https://www.fpcomplete.com/user/jle/machines testAuto :: Auto
a b -\textgreater{} {[}a{]} -\textgreater{} ({[}b{]}, Auto a b) testAuto auto
{[}{]} = ({[}{]} , auto ) testAuto auto (x:xs) = (y:ys, final) where (y, next )
= runAuto auto x (ys, final) = testAuto next xs

testAuto\_ :: Auto a b -\textgreater{} {[}a{]} -\textgreater{} {[}b{]}
testAuto\_ a = fst . testAuto a
\textasciitilde{}\textasciitilde{}\textasciitilde{}

Trying it out on \texttt{settableAuto}:

\textasciitilde{}\textasciitilde{}\textasciitilde{} ghci\textgreater{}
testAuto\_ settableAuto {[} Nothing, Nothing, Just 10 , Nothing, Nothing, Just
(-1) , Nothing {]} {[}1,2,10,11,12,-1,0{]}
\textasciitilde{}\textasciitilde{}\textasciitilde{}

Alternatively, here is a fun way to test an Auto interactively, where you ask
for input at each iteration from the user -\/-\/-

\textasciitilde{}\textasciitilde{}\textasciitilde{}haskell -\/- source:
https://github.com/mstksg/inCode/tree/master/code-samples/machines/Auto.hs\#L27-32
-\/- interactive: https://www.fpcomplete.com/user/jle/machines interactAuto ::
(Read a, Show b) =\textgreater{} Auto a b -\textgreater{} IO () interactAuto a0
= do inp \textless{}- getLine let (x,a1) = runAuto a0 (read inp) print x
interactAuto a1 \textasciitilde{}\textasciitilde{}\textasciitilde{}

\textasciitilde{}\textasciitilde{}\textasciitilde{}haskell ghci\textgreater{}
interactAuto settableAuto

\begin{quote}
Nothing 1 Nothing 2 Just 10 10 Nothing 11
\textasciitilde{}\textasciitilde{}\textasciitilde{}
\end{quote}

\subsection{A Shift}

Let's shift our thinking a bit. Instead of seeing Autos as "streams you can
influence", we can think about them as "functions with state". That is, they are
functions that carry self-contained encapsulated state \emph{inside} themselves.
They are still \emph{functions} of some kind -\/-\/- you put in an "input" of
type \texttt{a} and get in an "output" (the head) of type \texttt{b}. However,
every "time" you do this, you get a different output depending on what you have
already passed in and what the internal state is. As we will see, this internal
state is completely opaque to the world. The world only has access to the
"output", the result.

(Remember that, because we're in a functional language, nothing is technically
actually really "mutable". When we say that we have a stateful function, we
really mean that every time we "call" the function, we get back an "updated"
function with the new state that behaves differently when "called").

To put it in terms of \texttt{settableAuto}:

\begin{itemize}
\tightlist
\item
  The "input" of \texttt{settableAuto} is our \texttt{Maybe\ Int} by which we
  convey or decision to reset or allow to increment by one as normal.
\item
  The "output" of \texttt{settableAuto} is the "head" of the \texttt{ACons} that
  is returned -\/-\/- the \texttt{x}, \texttt{y}, etc. It's the \texttt{Int},
  the counter.
\item
  The "state" of \texttt{settableAuto} is, in essence, the \texttt{n} of
  \texttt{counterFrom\ n}. It's the internal value by which the behavior is
  determined. The behavior of \texttt{runAuto} depends on the \texttt{n} -\/-\/-
  it either yields \texttt{n} itself and increments \texttt{n}, or ignores it.
\end{itemize}

\subsubsection{The opaque state}

It's a little tricky because the "output" and the "state" in our example
function seem to be exactly the same (just like for \texttt{myStream}), but
let's whip up a quick example where it's a little more obvious that the state
and the output are different things, and that the state is completely opaque and
encapsulated.

\textasciitilde{}\textasciitilde{}\textasciitilde{}haskell -\/- source:
https://github.com/mstksg/inCode/tree/master/code-samples/machines/Auto.hs\#L58-64
-\/- interactive: https://www.fpcomplete.com/user/jle/machines isEvenAuto ::
Auto (Maybe Int) Bool isEvenAuto = isEvenAutoFrom 1 where isEvenAutoFrom :: Int
-\textgreater{} Auto (Maybe Int) Bool isEvenAutoFrom n = ACons \$
\textbackslash{}reset -\textgreater{} let c = fromMaybe n reset in ( even c,
isEvenAutoFrom (c + 1) ) \textasciitilde{}\textasciitilde{}\textasciitilde{}

So \texttt{isEvenAuto} is the same as \texttt{settableCounterFrom}, except
instead of "yielding"/"outputting" \texttt{n}, it outputs \texttt{even\ n}
-\/-\/- \texttt{True} if \texttt{n} is even and \texttt{False} if \texttt{n} is
odd.

Here is a demonstration of its behavior -\/-\/-

\textasciitilde{}\textasciitilde{}\textasciitilde{}haskell ghci\textgreater{}
testAuto isEvenAuto {[} Nothing, Nothing, Just 10 , Nothing, Nothing, Just (-1)
, Nothing {]} {[}False,True,True,False,True{]}
\textasciitilde{}\textasciitilde{}\textasciitilde{}

Note that there is in general really no way to ever access the \texttt{n}
internally (in fact, like we said before, it is in theory possible because we
can't even know its type). It is completely sealed off from the world, except by
our explicit design. Here, we choose to only "offer" a way to "set" it using our
input.

Now it the three distinct concepts -\/-\/- the input, output, and state -\/-\/-
should be clear.

\begin{itemize}
\tightlist
\item
  The "input" again is a \texttt{Maybe\ Int} where we can choose to reset the
  march of the state.
\item
  The "output" here is now a \texttt{Bool} that says whether or not the internal
  state is even.
\item
  The "state" here is still that \texttt{n} (an \texttt{Int}), and was the same
  as in the last example. But here it is more clear that the state is
  inaccessible in general. We can only modify it in ways that the Auto
  \emph{itself} allows our "input" (in this case, a setter) to modify it. And we
  certainly can't arbitrarily "read" it.
\end{itemize}

\subsection{Autos are nice!}

It should be clear now that Autos are a more or less straightforward extension
of Streams.

In fact, you might see that every \texttt{Stream\ b} is equivalent to
\texttt{Auto\ ()\ b}, where the input is always unit.

Autos correspond loosely to the mathematical
\href{http://en.wikipedia.org/wiki/Mealy_machine}{Mealy machine}, but again with
technically potentially infinitely many possible states, input, and output
values.

Now, we have a way to model behaviors that can somehow interact with the outside
world.

\subsubsection{More on state}

As we have seen, Auto's carry all of the hidden-internal-state features of
Streams. The type of an Auto (\texttt{Auto\ a\ b}) reveals the type of the
"input" and the "ouput"...but it never reveals nor fixes the type of the
"state". The type of the state is not only unknown, by possibly dynamically
changing over the course of the Auto's progression.

What Auto offers over Stream is then a way for the outside world to access and
modify the state \emph{if the Auto wants it to}. Now, we can design Autos that,
like we have seen with \texttt{settableAuto}, we can offer limited ways to allow
the world to modify the state on our own terms.

\subsection{The Accumulator}

Let's try our hand at another Auto, but instead of looking at things as an
influencable and eternally marching stream, we're going to try to look at things
as a function with state that affects its output.

How about an Auto that "accumulates" and "sums up" all of its incoming inputs,
starting at 0? More correctly, an Auto that, given any int, "returns" the sum of
that int with all of the previous ints it has received in its lifetime.

\textasciitilde{}\textasciitilde{}\textasciitilde{}haskell -\/- source:
https://github.com/mstksg/inCode/tree/master/code-samples/machines/Auto.hs\#L67-73
-\/- interactive: https://www.fpcomplete.com/user/jle/machines summer :: Num a
=\textgreater{} Auto a a summer = sumFrom 0 where sumFrom :: Num a
=\textgreater{} a -\textgreater{} Auto a a sumFrom n = ACons \$
\textbackslash{}input -\textgreater{} let s = n + input in ( s , sumFrom s )
\textasciitilde{}\textasciitilde{}\textasciitilde{}

\textasciitilde{}\textasciitilde{}\textasciitilde{}haskell ghci\textgreater{}
let (out1, auto1) = runAuto summer 10 ghci\textgreater{} out1 10
ghci\textgreater{} let (out2, auto2) = runAuto auto1 3 ghci\textgreater{} out2
13 ghci\textgreater{} testAuto\_ auto2 {[}15,-17,6,0,-1{]} {[}28,11,17,17,16{]}
\textasciitilde{}\textasciitilde{}\textasciitilde{}

\begin{itemize}
\tightlist
\item
  The "input" is our incoming \texttt{Int} -\/-\/- 10, 3, 15, -17, etc.
\item
  The "output" is the accumulated sum/integral -\/- 10, 13, 28, 11, etc.
\item
  The "state" in this case is the accumulator, which in this case stays in sync
  with the output. But remember that this is not the case in general.
\end{itemize}

Just for kicks, let's generalize this and make an Auto version of \texttt{foldl}
(technically, more like \texttt{scanl}): give us an operator and an initial
value, and we'll "fold up" all of our inputs.

\textasciitilde{}\textasciitilde{}\textasciitilde{}haskell -\/- source:
https://github.com/mstksg/inCode/tree/master/code-samples/machines/Auto.hs\#L78-84
-\/- interactive: https://www.fpcomplete.com/user/jle/machines autoFold ::
forall a b. (b -\textgreater{} a -\textgreater{} b) -\textgreater{} b
-\textgreater{} Auto a b autoFold op init = foldFrom init where foldFrom :: b
-\textgreater{} Auto a b foldFrom x = ACons \$ \textbackslash{}input
-\textgreater{} let y = x \texttt{op} input in ( y, foldFrom y )
\textasciitilde{}\textasciitilde{}\textasciitilde{}

(the \texttt{forall} is used with the
\href{http://www.haskell.org/haskellwiki/Scoped_type_variables}{Scoped Type
Variables} extension to let us say that the \texttt{b} we mention in the type of
\texttt{foldFrom} is the same as the \texttt{b} in the type of
\texttt{autoFold}. If we leave off the type signature of \texttt{foldFrom}, this
is not necessary)

Note that \texttt{summer} then is just \texttt{autoFold\ (+)\ 0}.

You can probably imagine lots of different folds you can turn into
\texttt{autoFold}s...and indeed a lot of practical Autos are just
\texttt{autoFold}s. Here are some cute ones:

\textasciitilde{}\textasciitilde{}\textasciitilde{}haskell -\/- source:
https://github.com/mstksg/inCode/tree/master/code-samples/machines/Auto.hs\#L92-107
-\/- interactive: https://www.fpcomplete.com/user/jle/machines
accumulateIntoList :: Auto a {[}a{]} accumulateIntoList = autoFold (flip (:))
{[}{]}

productor :: Num a =\textgreater{} Auto a a productor = autoFold (*) 1

accumulateStrings :: Auto String String accumulateStrings = autoFold (++) ""

monoidAccum :: Monoid a =\textgreater{} Auto a a monoidAccum = autoFold mappend
mempty \textasciitilde{}\textasciitilde{}\textasciitilde{}

Cool, huh?

\subsubsection{Parallels with list folds}

Let's look very carefully at a comparison between the type signature of
Prelude's \texttt{foldl} and the type signature of \texttt{autoFold}:

\textasciitilde{}\textasciitilde{}\textasciitilde{}haskell foldl :: (b
-\textgreater{} a -\textgreater{} b) -\textgreater{} b -\textgreater{} ({[}a{]}
-\textgreater{} b) autoFold :: (b -\textgreater{} a -\textgreater{} b)
-\textgreater{} b -\textgreater{} Auto a b
\textasciitilde{}\textasciitilde{}\textasciitilde{}

Hm. Let's do some rearranging. Remember that in Haskell,
\texttt{(-\textgreater{})} is just an infix type operator. So we can always
rewrite \texttt{a\ -\textgreater{}\ b} as \texttt{(-\textgreater{})\ a\ b}

\textasciitilde{}\textasciitilde{}\textasciitilde{}haskell foldl :: (b
-\textgreater{} a -\textgreater{} b) -\textgreater{} b -\textgreater{} (
(-\textgreater{}) {[}a{]} b ) autoFold :: (b -\textgreater{} a -\textgreater{}
b) -\textgreater{} b -\textgreater{} ( Auto a b )
\textasciitilde{}\textasciitilde{}\textasciitilde{}

Let's get rid of some of the points, too:

\textasciitilde{}\textasciitilde{}\textasciitilde{}haskell foldl op initial ::
(-\textgreater{}) {[}a{]} b autoFold op initial :: Auto a b
\textasciitilde{}\textasciitilde{}\textasciitilde{}

So both \texttt{foldl} and \texttt{autoFold} have very similar behaviors:

Give \texttt{foldl} or \texttt{autoFold} an \emph{accumulating function} and an
\emph{initial value}, and they return \emph{a new "function thing"}.

For \texttt{foldl\ op\ initial}, this "function thing" takes a \emph{list} of
\texttt{a} values and condenses them into a \texttt{b} value.

For \texttt{autoFold\ op\ initial}, the "function thing" takes \emph{one}
\texttt{a} value and returns a \texttt{b} value based on the previous
\texttt{a}'s it has seen.

The main point here is that \texttt{autoFold} is a sort of "function" in a
way...just like the others before it. It's a...."function-like thing".

\textless{}div class="note"\textgreater{} \textbf{Aside}

Here is a quick diversion, if you're up for it. This doesn't really have too
much to do with the rest of the post, but it'll help you test your intuition a
bit with Autos.

As an exercise, compare (and contrast) these three functions of identical type
signatures:

\textasciitilde{}\textasciitilde{}\textasciitilde{}haskell map f :: {[}a{]}
-\textgreater{} {[}b{]} scanl op initial :: {[}a{]} -\textgreater{} {[}b{]}
testAuto\_ auto :: {[}a{]} -\textgreater{} {[}b{]}
\textasciitilde{}\textasciitilde{}\textasciitilde{}

(Assume that \texttt{scanl} does not include the initial accumulator...that is,
we are really talking about \texttt{drop\ 1\ .\ scanl\ op\ init})

Compare what they do conceptually. Then, for fun, try implementing some of them
in terms of the other. Which re-implementations are possible? Which ones aren't?
\textless{}/div\textgreater{}

\subsection{More Auto examples}

\href{https://github.com/mstksg/inCode/tree/master/code-samples/machines/Auto.hs\#L111-176}{I've
thrown up a few auto examples}
\href{https://www.fpcomplete.com/user/jle/machines}{that you can run online} so
you can try it out and see how different ones work. Don't get too attached to
them, because we will later be re-implementing them as compositions of smaller,
simpler building blocks.

\href{https://github.com/mstksg/inCode/tree/master/code-samples/machines/Auto.hs\#L111-122}{rollingAverage}
: \texttt{rollingAverage\ n\ ::\ Fractional\ a\ =\textgreater{}\ Auto\ a\ a}
outputs a rolling average of the last \texttt{n} values it has encountered

\textasciitilde{}\textasciitilde{}\textasciitilde{}haskell ghci\textgreater{}
testAuto\_ (rollingAverage 4) {[}2,8,4,5,1,8,3,5,1,1,8,3,5,9,2{]} {[}2.0 ,5.0
,4.67,4.75,4.5 ,4.5 ,4.25,4.25,4.25,2.5 ,3.75,3.25,4.25,6.25,4.75{]}
\textasciitilde{}\textasciitilde{}\textasciitilde{}

\href{https://github.com/mstksg/inCode/tree/master/code-samples/machines/Auto.hs\#L125-146}{onFor}
: \texttt{onFor\ p\ i\ ::\ Auto\ a\ Bool} normally outputs
\texttt{False}...except whenever the input matches the given predicate
\texttt{p\ ::\ a\ -\textgreater{}\ Bool}. Then it stays "on" (\texttt{True}) for
\texttt{i} steps.

\textasciitilde{}\textasciitilde{}\textasciitilde{}haskell ghci\textgreater{} :t
onFor even 3 onFor even 3 :: Auto Int Bool ghci\textgreater{} testAuto\_ (onFor
even 3) {[}1,1,2,1,1,1,1,4,1,6,1,1,1,1{]} {[} False, False, True , True , True ,
False, False, True , True , True , True , True , False, False {]}
\textasciitilde{}\textasciitilde{}\textasciitilde{}

\href{https://github.com/mstksg/inCode/tree/master/code-samples/machines/Auto.hs\#L149-176}{autoMap}
: \texttt{autoMap\ cap\ ::\ Auto\ (Command\ k\ v)\ (Maybe\ v)} is a neat one. It
internally holds a
\href{http://hackage.haskell.org/package/containers-0.5.4.0/docs/Data-Map.html}{Map}
(a key-value store) -\/-\/- you can give it \texttt{{[}Command{]}{[}{]}} data
types that tell it to insert, lookup, and delete values. However, it enforces a
maximum of items.

\begin{verbatim}
The main thing to note here is that you get to completely encapsulate your
"state", and allow it only to be "modified" or "viewed" under your own
terms. In OOP terms, it is like exposing only a few public methods to
modify your private state with discrimination.  If you were passed an
`autoMap` with items already inside, you would have no way to have full
"access" to the map --- you would never be able to perform general
operations (such as getting a list of all of the keys).
\end{verbatim}

\textasciitilde{}\textasciitilde{}\textasciitilde{}haskell ghci\textgreater{}
testAuto\_ (autoMap 3) \textbar{} {[} Insert "hello" 7 \textbar{} , Insert
"world" 10 \textbar{} , Insert "foo" 12 \textbar{} , Insert "bar" 15 \textbar{}
, Delete "baz" \textbar{} , Delete "world" \textbar{} , Insert "haskell" 19
\textbar{} , Lookup "world" \textbar{} , Lookup "hello" \textbar{} {]} {[} Just
7 , Just 10, Just 12 , Nothing, Nothing, Just 10 , Just 19, Nothing, Just 7 {]}
\textasciitilde{}\textasciitilde{}\textasciitilde{}

\section{"Function Things"}

Anyways, back to our main point of emphasis:

\emph{Autos are function-like things}.

They are functions...\emph{with (self-contained) state}.

Let's do an analysis for \texttt{isEvenAuto} like the one we did with
\texttt{foldAuto}. Our "input" was a \texttt{Maybe\ Int} and our "output" was a
\texttt{Bool}.

You can think of \texttt{isEvenAuto} as a "function thing" from
\texttt{Maybe\ Int} to \texttt{Bool}.

Here's another function from \texttt{Maybe\ Int} to \texttt{Bool}: (I'm going to
be using the prefix form of \texttt{(-\textgreater{})} a lot from now on)

\textasciitilde{}\textasciitilde{}\textasciitilde{}haskell -\/- source:
https://github.com/mstksg/inCode/tree/master/code-samples/machines/Auto.hs\#L181-182
maybeIsEven :: (-\textgreater{}) (Maybe Int) Bool maybeIsEven = even . fromMaybe
1 \textasciitilde{}\textasciitilde{}\textasciitilde{}

\texttt{maybeIsEven} returns \texttt{True} when value inside the \texttt{Just}
is even, or \texttt{False} if the value is odd or it's a \texttt{Nothing}.

Compare that type signature to that of our \texttt{isEvenAuto}

\textasciitilde{}\textasciitilde{}\textasciitilde{}haskell maybeIsEven ::
(-\textgreater{}) (Maybe Int) Bool isEvenAuto :: Auto (Maybe Int) Bool
\textasciitilde{}\textasciitilde{}\textasciitilde{}

\texttt{maybeIsEven} and \texttt{isEvenAuto} are \emph{both} "function-like
things". But whereas \texttt{maybeIsEven} is "memoryless" (it's the same every
time you call it), \texttt{isEvenAuto} \emph{has memory} -\/-\/- it returns a
different Boolean based on its history.

Contrast this with a Stream, which as we have seen is just an
\texttt{Auto\ ()\ b}. Streams are then "function like things" analogous to some
\texttt{(-\textgreater{})\ ()\ b}, or \texttt{()\ -\textgreater{}\ b}. We can
call functions like \texttt{()\ -\textgreater{}\ b} "constants", or "producers".
They are the same every time you call them or ask for them. Streams, however,
"return" a potentially different \texttt{b} value every time they are "asked
for". So, just like an Auto is a "function" that has memory, a Stream is like a
"constant" that has memory. A stateful generator. A "constant" that returns
something different every time you ask for it.

Anyway, you should be able to guess that, after vaguely using the phrase
"function things" several times...I'm going to surprise you all with the
revelation that the general class of these "function things" have a name! And
maybe even...a typeclass?

\section{Onward}

So far we haven't really made too convincing of an argument for the advantages
of using machines (like Auto and the related Wire). Yeah, they provide
encapsulation and a changing state...but these things come for free in most good
Object-Oriented Programming languages. So what gives?

As it turns out, as we suggested before, Autos are potentially more "composable"
than the objects of OOP. That is because, at their heart, they are just
functions. And what do functions do best (as every functional programmer knows)?
They compose! Complex object built seamlessly from simpler ones.

Now, I haven't really been able to back this up so far. We've only scratched the
surface, and we haven't even considered function-like things that can return
other function-like things. But in the next post, as we explore more the
function-like nature of these things, we will be able to witness more of the
full power of machine composition. And we'll even be able to re-implement
\emph{many} of the complex machines of this post with compositions of smaller,
simpler Autos.

\end{document}
