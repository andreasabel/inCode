\documentclass[]{article}
\usepackage{lmodern}
\usepackage{amssymb,amsmath}
\usepackage{ifxetex,ifluatex}
\usepackage{fixltx2e} % provides \textsubscript
\ifnum 0\ifxetex 1\fi\ifluatex 1\fi=0 % if pdftex
  \usepackage[T1]{fontenc}
  \usepackage[utf8]{inputenc}
\else % if luatex or xelatex
  \ifxetex
    \usepackage{mathspec}
    \usepackage{xltxtra,xunicode}
  \else
    \usepackage{fontspec}
  \fi
  \defaultfontfeatures{Mapping=tex-text,Scale=MatchLowercase}
  \newcommand{\euro}{€}
\fi
% use upquote if available, for straight quotes in verbatim environments
\IfFileExists{upquote.sty}{\usepackage{upquote}}{}
% use microtype if available
\IfFileExists{microtype.sty}{\usepackage{microtype}}{}
\usepackage[margin=1in]{geometry}
\usepackage{color}
\usepackage{fancyvrb}
\newcommand{\VerbBar}{|}
\newcommand{\VERB}{\Verb[commandchars=\\\{\}]}
\DefineVerbatimEnvironment{Highlighting}{Verbatim}{commandchars=\\\{\}}
% Add ',fontsize=\small' for more characters per line
\newenvironment{Shaded}{}{}
\newcommand{\AlertTok}[1]{\textcolor[rgb]{1.00,0.00,0.00}{\textbf{#1}}}
\newcommand{\AnnotationTok}[1]{\textcolor[rgb]{0.38,0.63,0.69}{\textbf{\textit{#1}}}}
\newcommand{\AttributeTok}[1]{\textcolor[rgb]{0.49,0.56,0.16}{#1}}
\newcommand{\BaseNTok}[1]{\textcolor[rgb]{0.25,0.63,0.44}{#1}}
\newcommand{\BuiltInTok}[1]{#1}
\newcommand{\CharTok}[1]{\textcolor[rgb]{0.25,0.44,0.63}{#1}}
\newcommand{\CommentTok}[1]{\textcolor[rgb]{0.38,0.63,0.69}{\textit{#1}}}
\newcommand{\CommentVarTok}[1]{\textcolor[rgb]{0.38,0.63,0.69}{\textbf{\textit{#1}}}}
\newcommand{\ConstantTok}[1]{\textcolor[rgb]{0.53,0.00,0.00}{#1}}
\newcommand{\ControlFlowTok}[1]{\textcolor[rgb]{0.00,0.44,0.13}{\textbf{#1}}}
\newcommand{\DataTypeTok}[1]{\textcolor[rgb]{0.56,0.13,0.00}{#1}}
\newcommand{\DecValTok}[1]{\textcolor[rgb]{0.25,0.63,0.44}{#1}}
\newcommand{\DocumentationTok}[1]{\textcolor[rgb]{0.73,0.13,0.13}{\textit{#1}}}
\newcommand{\ErrorTok}[1]{\textcolor[rgb]{1.00,0.00,0.00}{\textbf{#1}}}
\newcommand{\ExtensionTok}[1]{#1}
\newcommand{\FloatTok}[1]{\textcolor[rgb]{0.25,0.63,0.44}{#1}}
\newcommand{\FunctionTok}[1]{\textcolor[rgb]{0.02,0.16,0.49}{#1}}
\newcommand{\ImportTok}[1]{#1}
\newcommand{\InformationTok}[1]{\textcolor[rgb]{0.38,0.63,0.69}{\textbf{\textit{#1}}}}
\newcommand{\KeywordTok}[1]{\textcolor[rgb]{0.00,0.44,0.13}{\textbf{#1}}}
\newcommand{\NormalTok}[1]{#1}
\newcommand{\OperatorTok}[1]{\textcolor[rgb]{0.40,0.40,0.40}{#1}}
\newcommand{\OtherTok}[1]{\textcolor[rgb]{0.00,0.44,0.13}{#1}}
\newcommand{\PreprocessorTok}[1]{\textcolor[rgb]{0.74,0.48,0.00}{#1}}
\newcommand{\RegionMarkerTok}[1]{#1}
\newcommand{\SpecialCharTok}[1]{\textcolor[rgb]{0.25,0.44,0.63}{#1}}
\newcommand{\SpecialStringTok}[1]{\textcolor[rgb]{0.73,0.40,0.53}{#1}}
\newcommand{\StringTok}[1]{\textcolor[rgb]{0.25,0.44,0.63}{#1}}
\newcommand{\VariableTok}[1]{\textcolor[rgb]{0.10,0.09,0.49}{#1}}
\newcommand{\VerbatimStringTok}[1]{\textcolor[rgb]{0.25,0.44,0.63}{#1}}
\newcommand{\WarningTok}[1]{\textcolor[rgb]{0.38,0.63,0.69}{\textbf{\textit{#1}}}}
\ifxetex
  \usepackage[setpagesize=false, % page size defined by xetex
              unicode=false, % unicode breaks when used with xetex
              xetex]{hyperref}
\else
  \usepackage[unicode=true]{hyperref}
\fi
\hypersetup{breaklinks=true,
            bookmarks=true,
            pdfauthor={Justin Le},
            pdftitle={Applicative Regular Expressions using the Free Alternative},
            colorlinks=true,
            citecolor=blue,
            urlcolor=blue,
            linkcolor=magenta,
            pdfborder={0 0 0}}
\urlstyle{same}  % don't use monospace font for urls
% Make links footnotes instead of hotlinks:
\renewcommand{\href}[2]{#2\footnote{\url{#1}}}
\setlength{\parindent}{0pt}
\setlength{\parskip}{6pt plus 2pt minus 1pt}
\setlength{\emergencystretch}{3em}  % prevent overfull lines
\setcounter{secnumdepth}{0}

\title{Applicative Regular Expressions using the Free Alternative}
\author{Justin Le}

\begin{document}
\maketitle

\emph{Originally posted on
\textbf{\href{https://blog.jle.im/entry/free-alternative-regexp.html}{in
Code}}.}

We're going to implement applicative regular expressions and parsers (in the
style of the
\href{https://hackage.haskell.org/package/regex-applicative}{regex-applicative}
library) using free structures!

Free structures are some of my favorite tools in Haskell, and I've actually
written a few posts about them before, including
\href{https://blog.jle.im/entry/alchemical-groups.html}{this one using free
groups}, \href{https://blog.jle.im/entry/interpreters-a-la-carte-duet.html}{this
one on a free monad variation}, and
\href{https://blog.jle.im/entry/const-applicative-and-monoids.html}{this one on
a ``free'' applicative on a monoid}.

Regular expressions (and parsers) are ubiquitous in computer science and
programming, and I hope that demonstrating that they are pretty straightforward
to implement using free structures will help you see the value in free
structures without getting too bogged down in the details!

\hypertarget{regular-languages}{%
\section{Regular Languages}\label{regular-languages}}

A \emph{regular expression} is something that defines a \emph{regular language}.
\href{https://en.wikipedia.org/wiki/Regular_expression\#Formal_language_theory}{Formally},
it consists of the following primitives:

\begin{enumerate}
\def\labelenumi{\arabic{enumi}.}
\tightlist
\item
  The empty set, which always fails to match.
\item
  The empty string, which always succeeds matching the empty string.
\item
  The literal character, denoting a single matching character
\end{enumerate}

And the following operations:

\begin{enumerate}
\def\labelenumi{\arabic{enumi}.}
\tightlist
\item
  Concatenation: \texttt{RS}, sequence one after the other. A set product.
\item
  Alternation: \texttt{R\textbar{}S}, one or the other. A set union.
\item
  Kleene Star: \texttt{R*}, the repetition of \texttt{R} one or more times.
\end{enumerate}

\hypertarget{alternative}{%
\section{Alternative}\label{alternative}}

Looking at this, does this look a little familiar? It reminds me a lot of the
\texttt{Alternative} hierarchy. If a functor \texttt{f} has an
\texttt{Alternative} instance, it means that it has:

\begin{enumerate}
\def\labelenumi{\arabic{enumi}.}
\tightlist
\item
  \texttt{empty}, the failing operation
\item
  \texttt{pure\ x}, the always-succeeding operation (from the
  \texttt{Applicative} class)
\item
  \texttt{\textless{}*\textgreater{}}, the sequencing operation (from the
  \texttt{Applicative} class)
\item
  \texttt{\textless{}\textbar{}\textgreater{}}, the alternating operation
\item
  \texttt{many}, the ``one or more'' operation.
\end{enumerate}

This\ldots{}looks a lot like the construction of a regular language. The only
thing missing is the literal character primitive.

So, one way we can look at regular expressions is ``The entire
\texttt{Alternative} interface, plus a character primitive''. \emph{But!}
There's another way of looking at this, that leads us directly to free
structures.

Instead of seeing things as ``\texttt{Alternative} with a character primitive'',
we can look at it as \emph{a character primitive with a Alternative instance}.

\hypertarget{free}{%
\section{Free}\label{free}}

Let's write this out. Our character primitive will be:

\begin{Shaded}
\begin{Highlighting}[]
\KeywordTok{data} \DataTypeTok{Prim}\NormalTok{ a }\FunctionTok{=} \DataTypeTok{Prim} \DataTypeTok{Char}\NormalTok{ a}
  \KeywordTok{deriving} \DataTypeTok{Functor}
\end{Highlighting}
\end{Shaded}

Note that because we're working with functors, applicatives, alternatives, etc.,
all of our regular expressions must have an associated ``result''. The value
\texttt{Prim\ \textquotesingle{}a\textquotesingle{}\ 1\ ::\ Prim\ Int} will
represent a primitive character that, when parsed, will give a result of
\texttt{1}.

And now\ldots{}we give it an \texttt{Alternative} instance using the \emph{Free
Alternative}, from the
\emph{\href{https://hackage.haskell.org/package/free}{free}} package:

\begin{Shaded}
\begin{Highlighting}[]
\KeywordTok{import} \DataTypeTok{Control.Alternative.Free}

\KeywordTok{type} \DataTypeTok{RegExp} \FunctionTok{=} \DataTypeTok{Alt} \DataTypeTok{Prim}
\end{Highlighting}
\end{Shaded}

And that's it! That's our entire regular expression type! By giving a
\texttt{Alt} a \texttt{Functor}, we get all of the operations of
\texttt{Applicative} and \texttt{Alternative} over our base. That's because we
have \texttt{instance\ Applicative\ (Alt\ f)} and
\texttt{instance\ Alternative\ (Alt\ f)}. We now have:

\begin{enumerate}
\def\labelenumi{\arabic{enumi}.}
\tightlist
\item
  The empty set, coming from \texttt{empty} from \texttt{Alternative}
\item
  The empty string, coming from \texttt{pure} from \texttt{Applicative}
\item
  The character primitive, coming from the underlying functor \texttt{Prim} that
  we are enhancing
\item
  The concatenation operation, from \texttt{\textless{}*\textgreater{}}, from
  \texttt{Applicative}.
\item
  The alternating operation, from \texttt{\textless{}\textbar{}\textgreater{}},
  from \texttt{Alternative}.
\item
  The kleene star, from \texttt{many}, from \texttt{Alternative}.
\end{enumerate}

All of these (except for the primitive) come ``for free''!

After adding some convenient wrappers\ldots{}we're done here!

\begin{Shaded}
\begin{Highlighting}[]
\CommentTok{-- | Parse a given character as a given constant result.}
\OtherTok{charAs ::} \DataTypeTok{Char} \OtherTok{->}\NormalTok{ a }\OtherTok{->} \DataTypeTok{RegExp}\NormalTok{ a}
\NormalTok{charAs c x }\FunctionTok{=}\NormalTok{ liftAlt (}\DataTypeTok{Prim}\NormalTok{ c x)    }\CommentTok{-- liftAlt lets us use the underlying functor Prim in RegExp}

\CommentTok{-- | Parse a given character as itself.}
\OtherTok{char ::} \DataTypeTok{Char} \OtherTok{->} \DataTypeTok{RegExp} \DataTypeTok{Char}
\NormalTok{char c }\FunctionTok{=}\NormalTok{ charAs c c}

\CommentTok{-- | Parse a given string as itself.}
\OtherTok{string ::} \DataTypeTok{String} \OtherTok{->} \DataTypeTok{RegExp} \DataTypeTok{String}
\NormalTok{string }\FunctionTok{=} \FunctionTok{traverse}\NormalTok{ char        }\CommentTok{-- neat, huh}
\end{Highlighting}
\end{Shaded}

\hypertarget{examples}{%
\subsection{Examples}\label{examples}}

Let's try it out! Let's match on \texttt{(a\textbar{}b)(cd)*e} and return
\texttt{()}:

\begin{Shaded}
\begin{Highlighting}[]
\OtherTok{testRegExp_ ::} \DataTypeTok{RegExp}\NormalTok{ ()}
\NormalTok{testRegExp_ }\FunctionTok{=}\NormalTok{ void }\FunctionTok{$}\NormalTok{ (char }\CharTok{'a'} \FunctionTok{<|>}\NormalTok{ char }\CharTok{'b'}\NormalTok{)}
                  \FunctionTok{*>}\NormalTok{ many (string }\StringTok{"cd"}\NormalTok{)}
                  \FunctionTok{*>}\NormalTok{ char }\CharTok{'e'}
\end{Highlighting}
\end{Shaded}

\texttt{void} from \emph{Data.Functor} discards the results, since we only care
if it matches or not. But we use \texttt{\textless{}\textbar{}\textgreater{}}
and \texttt{*\textgreater{}} and \texttt{many} exactly how we'd expect to
concatenate and alternate things with \texttt{Applicative} and
\texttt{Alternative}.

Or maybe more interesting (but slightly more complicated), let's match on the
same one and return how many \texttt{cd}s are repeated

\begin{Shaded}
\begin{Highlighting}[]
\OtherTok{testRegExp ::} \DataTypeTok{RegExp} \DataTypeTok{Int}
\NormalTok{testRegExp }\FunctionTok{=}\NormalTok{ (char }\CharTok{'a'} \FunctionTok{<|>}\NormalTok{ char }\CharTok{'b'}\NormalTok{)}
          \FunctionTok{*>}\NormalTok{ (}\FunctionTok{length} \FunctionTok{<$>}\NormalTok{ many (string }\StringTok{"cd"}\NormalTok{))}
          \FunctionTok{<*}\NormalTok{ char }\CharTok{'e'}
\end{Highlighting}
\end{Shaded}

This one does require a little more finesse with \texttt{*\textgreater{}} and
\texttt{\textless{}*}: the arrows point towards which result to ``keep''. And
since \texttt{many\ (string\ "cd")\ ::\ RegExp\ {[}String{]}} (it returns a
list, with an item for each repetition), we can \texttt{fmap\ length} to get the
\texttt{Int} result of ``how many repetitions''.

\hypertarget{signoff}{%
\section{Signoff}\label{signoff}}

Hi, thanks for reading! You can reach me via email at
\href{mailto:justin@jle.im}{\nolinkurl{justin@jle.im}}, or at twitter at
\href{https://twitter.com/mstk}{@mstk}! This post and all others are published
under the \href{https://creativecommons.org/licenses/by-nc-nd/3.0/}{CC-BY-NC-ND
3.0} license. Corrections and edits via pull request are welcome and encouraged
at \href{https://github.com/mstksg/inCode}{the source repository}.

If you feel inclined, or this post was particularly helpful for you, why not
consider \href{https://www.patreon.com/justinle/overview}{supporting me on
Patreon}, or a \href{bitcoin:3D7rmAYgbDnp4gp4rf22THsGt74fNucPDU}{BTC donation}?
:)

\end{document}
