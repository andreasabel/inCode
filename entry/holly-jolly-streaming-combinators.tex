\documentclass[]{article}
\usepackage{lmodern}
\usepackage{amssymb,amsmath}
\usepackage{ifxetex,ifluatex}
\usepackage{fixltx2e} % provides \textsubscript
\ifnum 0\ifxetex 1\fi\ifluatex 1\fi=0 % if pdftex
  \usepackage[T1]{fontenc}
  \usepackage[utf8]{inputenc}
\else % if luatex or xelatex
  \ifxetex
    \usepackage{mathspec}
    \usepackage{xltxtra,xunicode}
  \else
    \usepackage{fontspec}
  \fi
  \defaultfontfeatures{Mapping=tex-text,Scale=MatchLowercase}
  \newcommand{\euro}{€}
\fi
% use upquote if available, for straight quotes in verbatim environments
\IfFileExists{upquote.sty}{\usepackage{upquote}}{}
% use microtype if available
\IfFileExists{microtype.sty}{\usepackage{microtype}}{}
\usepackage[margin=1in]{geometry}
\usepackage{color}
\usepackage{fancyvrb}
\newcommand{\VerbBar}{|}
\newcommand{\VERB}{\Verb[commandchars=\\\{\}]}
\DefineVerbatimEnvironment{Highlighting}{Verbatim}{commandchars=\\\{\}}
% Add ',fontsize=\small' for more characters per line
\newenvironment{Shaded}{}{}
\newcommand{\AlertTok}[1]{\textcolor[rgb]{1.00,0.00,0.00}{\textbf{#1}}}
\newcommand{\AnnotationTok}[1]{\textcolor[rgb]{0.38,0.63,0.69}{\textbf{\textit{#1}}}}
\newcommand{\AttributeTok}[1]{\textcolor[rgb]{0.49,0.56,0.16}{#1}}
\newcommand{\BaseNTok}[1]{\textcolor[rgb]{0.25,0.63,0.44}{#1}}
\newcommand{\BuiltInTok}[1]{#1}
\newcommand{\CharTok}[1]{\textcolor[rgb]{0.25,0.44,0.63}{#1}}
\newcommand{\CommentTok}[1]{\textcolor[rgb]{0.38,0.63,0.69}{\textit{#1}}}
\newcommand{\CommentVarTok}[1]{\textcolor[rgb]{0.38,0.63,0.69}{\textbf{\textit{#1}}}}
\newcommand{\ConstantTok}[1]{\textcolor[rgb]{0.53,0.00,0.00}{#1}}
\newcommand{\ControlFlowTok}[1]{\textcolor[rgb]{0.00,0.44,0.13}{\textbf{#1}}}
\newcommand{\DataTypeTok}[1]{\textcolor[rgb]{0.56,0.13,0.00}{#1}}
\newcommand{\DecValTok}[1]{\textcolor[rgb]{0.25,0.63,0.44}{#1}}
\newcommand{\DocumentationTok}[1]{\textcolor[rgb]{0.73,0.13,0.13}{\textit{#1}}}
\newcommand{\ErrorTok}[1]{\textcolor[rgb]{1.00,0.00,0.00}{\textbf{#1}}}
\newcommand{\ExtensionTok}[1]{#1}
\newcommand{\FloatTok}[1]{\textcolor[rgb]{0.25,0.63,0.44}{#1}}
\newcommand{\FunctionTok}[1]{\textcolor[rgb]{0.02,0.16,0.49}{#1}}
\newcommand{\ImportTok}[1]{#1}
\newcommand{\InformationTok}[1]{\textcolor[rgb]{0.38,0.63,0.69}{\textbf{\textit{#1}}}}
\newcommand{\KeywordTok}[1]{\textcolor[rgb]{0.00,0.44,0.13}{\textbf{#1}}}
\newcommand{\NormalTok}[1]{#1}
\newcommand{\OperatorTok}[1]{\textcolor[rgb]{0.40,0.40,0.40}{#1}}
\newcommand{\OtherTok}[1]{\textcolor[rgb]{0.00,0.44,0.13}{#1}}
\newcommand{\PreprocessorTok}[1]{\textcolor[rgb]{0.74,0.48,0.00}{#1}}
\newcommand{\RegionMarkerTok}[1]{#1}
\newcommand{\SpecialCharTok}[1]{\textcolor[rgb]{0.25,0.44,0.63}{#1}}
\newcommand{\SpecialStringTok}[1]{\textcolor[rgb]{0.73,0.40,0.53}{#1}}
\newcommand{\StringTok}[1]{\textcolor[rgb]{0.25,0.44,0.63}{#1}}
\newcommand{\VariableTok}[1]{\textcolor[rgb]{0.10,0.09,0.49}{#1}}
\newcommand{\VerbatimStringTok}[1]{\textcolor[rgb]{0.25,0.44,0.63}{#1}}
\newcommand{\WarningTok}[1]{\textcolor[rgb]{0.38,0.63,0.69}{\textbf{\textit{#1}}}}
\ifxetex
  \usepackage[setpagesize=false, % page size defined by xetex
              unicode=false, % unicode breaks when used with xetex
              xetex]{hyperref}
\else
  \usepackage[unicode=true]{hyperref}
\fi
\hypersetup{breaklinks=true,
            bookmarks=true,
            pdfauthor={Justin Le},
            pdftitle={Roll your own Holly Jolly streaming combinators with Free},
            colorlinks=true,
            citecolor=blue,
            urlcolor=blue,
            linkcolor=magenta,
            pdfborder={0 0 0}}
\urlstyle{same}  % don't use monospace font for urls
% Make links footnotes instead of hotlinks:
\renewcommand{\href}[2]{#2\footnote{\url{#1}}}
\setlength{\parindent}{0pt}
\setlength{\parskip}{6pt plus 2pt minus 1pt}
\setlength{\emergencystretch}{3em}  % prevent overfull lines
\setcounter{secnumdepth}{0}

\title{Roll your own Holly Jolly streaming combinators with Free}
\author{Justin Le}

\begin{document}
\maketitle

\emph{Originally posted on
\textbf{\href{https://blog.jle.im/entry/holly-jolly-streaming-combinators.html}{in
Code}}.}

Hi! Welcome, if you're joining us from the great
\href{https://adventofhaskell.com/}{Advent of Haskell 2020} event! Feel free to
grab a hot chocolate and sit back by the fireplace. I'm honored to be able to be
a part of the event this year; it's a great initiative and harkens back to the
age-old Haskell tradition of bite-sized Functional Programming ``advent
calendars''. I remember when I was first learning Haskell,
\href{https://ocharles.org.uk/pages/2012-12-01-24-days-of-hackage.html}{Ollie
Charles' 24 Days of Hackage series} was one of my favorite series that helped me
really get into the exciting world of Haskell and the all the doors that
functional programming can open.

All of the posts this year have been great --- they range from insightful
reflections on the nature of Haskell and programming in Haskell, or also on
specific language features. This post is going to be one of the
``project-based'' ones, where we walk through and introduce a solidly
\emph{intermediate} Haskell technique as it applies to building a useful general
toolset. I'm going to be exploring the ``functor combinator style'' where you
identify the interface you want, associate it with a common Haskell typeclass,
pick your primitives, and automatically get the ability to imbue your primitives
with the structure you need. I've talked about this previously with:

\begin{enumerate}
\def\labelenumi{\arabic{enumi}.}
\tightlist
\item
  \href{https://blog.jle.im/entry/free-alternative-regexp.html}{Applicative
  regular expressions}
\item
  \href{https://blog.jle.im/entry/functor-combinatorpedia.html}{The functor
  combinatorpedia}
\item
  \href{https://blog.jle.im/entries/series/+enhancing-functor-structures.html}{Bidirectional
  serializers}
\item
  \href{https://blog.jle.im/entry/interpreters-a-la-carte-duet.html}{Composable
  interpreters}
\end{enumerate}

and I wanted to share a recent application I have been able to use apply it with
where just \emph{thinking} about the primitives gave me almost all the
functionality I needed for a type: composable streaming combinators. This
specific application is also very applicable to integrate into any
\href{https://www.stephendiehl.com/posts/decade.html\#algebraic-effect-systems}{composable
effects system}, since it's essentially a monadic interface.

In a way, this post could also be seen as capturing the spirit of the holidays
by reminiscing about the days of yore --- looking back at one of the more
exciting times in modern Haskell's development, where competing composable
streaming libraries were at the forefront of practical innovation. The dust has
settled on that a bit, but it every time I think about composable streaming
combinators, I do get a bit nostalgic :)

This post is written for an \emph{intermediate} Haskell audience, and will
assume you have a familiarity with monads and monadic interfaces, and also a
little bit of experience with monad transformers. Note --- there are many ways
to arrive at the same result, but this post is more of a demonstration of a
certain style and approach that has benefited my greatly in the past.

All of the code in this page
\href{https://github.com/mstksg/inCode/tree/master/code-samples/misc/streaming-combinators-free.hs}{can
be found online at github}!

\hypertarget{an-effectful-picture}{%
\section{An effectful picture}\label{an-effectful-picture}}

The goal here is to make a system of composable pipes that are ``pull-based'',
so we can process data as it is read in from IO only as we need it, and never do
more work than we need to do up-front or leak memory when we stop using it.

So, the way I usually approach things like these is: ``dress for the interface
you want, not the one you have.'' It involves:

\begin{enumerate}
\def\labelenumi{\arabic{enumi}.}
\tightlist
\item
  Thinking of the \texttt{m\ a} you want and how you would want to combine
  it/use it.
\item
  Express the primitive actions of that thing
\item
  Use some sort of free structure or effects system to enhance that primitive
  with the interface you are looking for.
\end{enumerate}

So, let's imagine our type!

\begin{Shaded}
\begin{Highlighting}[]
\KeywordTok{type} \DataTypeTok{Pipe}\NormalTok{ i o m a }\OtherTok{=} \OperatorTok{...}
\end{Highlighting}
\end{Shaded}

where a \texttt{Pipe\ i\ o\ m\ a} represents a pipe component where:

\begin{itemize}
\tightlist
\item
  \texttt{i}: the type of the input the pipe expects from upstream
\item
  \texttt{o}: the type of the output the pipe will be yielding upstream
\item
  \texttt{m}: the monad that the underlying actions live in
\item
  \texttt{a}: the overall result of the pipe once it has terminated.
\end{itemize}

One nice thing about this setup is that by picking different values for the type
parameters, we can already get a nice classification for interesting subtypes:

\begin{enumerate}
\def\labelenumi{\arabic{enumi}.}
\item
  If \texttt{i} is \texttt{()} (or universally quantified\footnote{``Universally
    quantified'' here means that the pipe's type is left fully polymorphic (with
    no constraints) over \texttt{i}, the input.}) --- a
  \texttt{Pipe\ ()\ o\ m\ a} --- it means that the pipe doesn't ever expect any
  sort of information upstream, and so can be considered a ``source'' that keeps
  on churning out values.
\item
  If \texttt{o} is \texttt{Void} (or universally quantified) --- a
  \texttt{Pipe\ i\ Void\ m\ a} --- it means that the pipe will never yield
  anything downstream, because \texttt{Void} has no inhabitants that could
  possibly be yielded.

\begin{Shaded}
\begin{Highlighting}[]
\KeywordTok{data} \DataTypeTok{Void}
\end{Highlighting}
\end{Shaded}

  This means that it acts like a ``sink'' that will keep on eating \texttt{i}
  values without ever outputting anything downstream.
\item
  If \texttt{i} is \texttt{()} and \texttt{o} is \texttt{Void} (or they are both
  universally quantified), then the pipe doesn't expect any sort of information
  upstream, and also won't ever yield anything downstream\ldots{} a
  \texttt{Pipe\ ()\ Void\ m\ a} is just an \texttt{m\ a}! In the biz, we often
  call this an ``effect''.
\item
  If \texttt{a} is \texttt{Void} (or universally quantified) --- a
  \texttt{Pipe\ i\ o\ m\ Void} --- it means that the pipe will never terminate,
  since \texttt{Void} has no inhabitants that could it could possibly produce
  upon termination.
\end{enumerate}

To me, I think it embodies a lot of the nice principles about the ``algebra'' of
types that can be used to reason with inputs and outputs. Plus, it allows us to
unify sources, sinks, and non-terminating pipes all in one type!

Now let's think of the interface we want. We want to be able to:

\begin{Shaded}
\begin{Highlighting}[]
\CommentTok{{-}{-} | Yield a value \textasciigrave{}o\textasciigrave{} downstream}
\OtherTok{yield ::}\NormalTok{ o }\OtherTok{{-}>} \DataTypeTok{Pipe}\NormalTok{ i o m ()}

\CommentTok{{-}{-} | Await a value \textasciigrave{}i\textasciigrave{} upstream}
\OtherTok{await ::} \DataTypeTok{Pipe}\NormalTok{ i o m (}\DataTypeTok{Maybe}\NormalTok{ i)}

\CommentTok{{-}{-} | Terminate immediately with a result value}
\FunctionTok{return}\OtherTok{ ::}\NormalTok{ a }\OtherTok{{-}>} \DataTypeTok{Pipe}\NormalTok{ i o m a}

\CommentTok{{-}{-} | Sequence pipes one{-}after{-}another:}
\CommentTok{{-}{-} "do this until it terminates, then that one next"}
\OtherTok{(>>) ::} \DataTypeTok{Pipe}\NormalTok{ i o m a }\OtherTok{{-}>} \DataTypeTok{Pipe}\NormalTok{ i o m b }\OtherTok{{-}>} \DataTypeTok{Pipe}\NormalTok{ i o m b}

\CommentTok{{-}{-} | In fact let\textquotesingle{}s just make it a full fledged monad, why not?  We\textquotesingle{}re designing}
\NormalTok{our dream interface here}\OperatorTok{.}
\OtherTok{(>>=) ::} \DataTypeTok{Pipe}\NormalTok{ i o m a }\OtherTok{{-}>}\NormalTok{ (a }\OtherTok{{-}>} \DataTypeTok{Pipe}\NormalTok{ i o m b) }\OtherTok{{-}>} \DataTypeTok{Pipe}\NormalTok{ i o m b}

\CommentTok{{-}{-} | A pipe that simply does action in the underlying monad and terminates with}
\CommentTok{{-}{-} the result}
\OtherTok{lift ::}\NormalTok{ m a }\OtherTok{{-}>} \DataTypeTok{Pipe}\NormalTok{ i o m a}

\CommentTok{{-}{-} | Compose pipes, linking the output of one to the input of the other}
\OtherTok{(.|) ::} \DataTypeTok{Pipe}\NormalTok{ i j m a }\OtherTok{{-}>} \DataTypeTok{Pipe}\NormalTok{ j o m b }\OtherTok{{-}>} \DataTypeTok{Pipe}\NormalTok{ i o m b}

\CommentTok{{-}{-} | Finally: run it all on a pipe expecting no input and never yielding:}
\OtherTok{runPipe ::} \DataTypeTok{Pipe}\NormalTok{ () }\DataTypeTok{Void}\NormalTok{ m a }\OtherTok{{-}>}\NormalTok{ m a}
\end{Highlighting}
\end{Shaded}

This looks like a complicated list\ldots but actually most of these come from
ubiquitous Haskell typeclasses like \texttt{Monad} and \texttt{Applicative}.
We'll see how this comes into play later, when we learn how to get these
instances for our types for free. This makes the actual ``work'' we have to do
very small.

So, these are going to be implementing ``conduit-style'' streaming combinators,
where streaming actions are monadic, and monadic sequencing represents ``do this
after this one is done.'' Because of this property, they work well as
\emph{pull-based} pipes, yields will block until a corresponding await can
accept what is yielded.

\hypertarget{dress-for-the-interface-you-want}{%
\subsection{Dress for the interface you
want}\label{dress-for-the-interface-you-want}}

``Dress for the interface you want, not the one you have'', they all told me. So
let's pretend we already implemented this interface\ldots what could we do with
it?

Well, can write simple sources like ``yield the contents from a file
line-by-line'':

\begin{Shaded}
\begin{Highlighting}[]
\CommentTok{{-}{-} source: https://github.com/mstksg/inCode/tree/master/code{-}samples/misc/streaming{-}combinators{-}free.hs\#L67{-}L74}

\OtherTok{sourceHandle ::} \DataTypeTok{Handle} \OtherTok{{-}>} \DataTypeTok{Pipe}\NormalTok{ i }\DataTypeTok{String} \DataTypeTok{IO}\NormalTok{ ()}
\NormalTok{sourceHandle handle }\OtherTok{=} \KeywordTok{do}
\NormalTok{    res }\OtherTok{<{-}}\NormalTok{ lift }\OperatorTok{$}\NormalTok{ tryJust (guard }\OperatorTok{.}\NormalTok{ isEOFError) (hGetLine handle)}
    \KeywordTok{case}\NormalTok{ res }\KeywordTok{of}
      \DataTypeTok{Left}\NormalTok{  \_   }\OtherTok{{-}>} \FunctionTok{return}\NormalTok{ ()}
      \DataTypeTok{Right}\NormalTok{ out }\OtherTok{{-}>} \KeywordTok{do}
\NormalTok{        yield out}
\NormalTok{        sourceHandle handle}
\end{Highlighting}
\end{Shaded}

Note that because the \texttt{i} is universally quantified, it means that we
know that \texttt{sourceFile} never ever awaits or touches any input: it's
purely a source.

We can even write a simple sink, like ``await and print the results to stdout as
they come'':

\begin{Shaded}
\begin{Highlighting}[]
\CommentTok{{-}{-} source: https://github.com/mstksg/inCode/tree/master/code{-}samples/misc/streaming{-}combinators{-}free.hs\#L76{-}L83}

\OtherTok{sinkStdout ::} \DataTypeTok{Pipe} \DataTypeTok{String}\NormalTok{ o }\DataTypeTok{IO}\NormalTok{ ()}
\NormalTok{sinkStdout }\OtherTok{=} \KeywordTok{do}
\NormalTok{    inp }\OtherTok{<{-}}\NormalTok{ await}
    \KeywordTok{case}\NormalTok{ inp }\KeywordTok{of}
      \DataTypeTok{Nothing} \OtherTok{{-}>} \FunctionTok{pure}\NormalTok{ ()}
      \DataTypeTok{Just}\NormalTok{ x  }\OtherTok{{-}>} \KeywordTok{do}
\NormalTok{        lift }\OperatorTok{$} \FunctionTok{putStrLn}\NormalTok{ x}
\NormalTok{        sinkStdout}
\end{Highlighting}
\end{Shaded}

And maybe we can write a pipe that takes input strings and converts them to all
capital letters and re-yields them:

\begin{Shaded}
\begin{Highlighting}[]
\CommentTok{{-}{-} source: https://github.com/mstksg/inCode/tree/master/code{-}samples/misc/streaming{-}combinators{-}free.hs\#L85{-}L92}

\OtherTok{toUpperPipe ::} \DataTypeTok{Monad}\NormalTok{ m }\OtherTok{=>} \DataTypeTok{Pipe} \DataTypeTok{String} \DataTypeTok{String}\NormalTok{ m ()}
\NormalTok{toUpperPipe }\OtherTok{=} \KeywordTok{do}
\NormalTok{    inp }\OtherTok{<{-}}\NormalTok{ await}
    \KeywordTok{case}\NormalTok{ inp }\KeywordTok{of}
      \DataTypeTok{Nothing} \OtherTok{{-}>} \FunctionTok{pure}\NormalTok{ ()}
      \DataTypeTok{Just}\NormalTok{ x  }\OtherTok{{-}>} \KeywordTok{do}
\NormalTok{        yield (}\FunctionTok{map} \FunctionTok{toUpper}\NormalTok{ x)}
\NormalTok{        toUpperPipe}
\end{Highlighting}
\end{Shaded}

And we can maybe write a pipe that stops as soon as it reads the line
\texttt{STOP}.

\begin{Shaded}
\begin{Highlighting}[]
\CommentTok{{-}{-} source: https://github.com/mstksg/inCode/tree/master/code{-}samples/misc/streaming{-}combinators{-}free.hs\#L94{-}L103}

\OtherTok{untilSTOP ::} \DataTypeTok{Monad}\NormalTok{ m }\OtherTok{=>} \DataTypeTok{Pipe} \DataTypeTok{String} \DataTypeTok{String}\NormalTok{ m ()}
\NormalTok{untilSTOP }\OtherTok{=} \KeywordTok{do}
\NormalTok{    inp }\OtherTok{<{-}}\NormalTok{ await}
    \KeywordTok{case}\NormalTok{ inp }\KeywordTok{of}
      \DataTypeTok{Nothing} \OtherTok{{-}>} \FunctionTok{pure}\NormalTok{ ()}
      \DataTypeTok{Just}\NormalTok{ x}
        \OperatorTok{|}\NormalTok{ x }\OperatorTok{==} \StringTok{"STOP"} \OtherTok{{-}>} \FunctionTok{pure}\NormalTok{ ()}
        \OperatorTok{|} \FunctionTok{otherwise}   \OtherTok{{-}>} \KeywordTok{do}
\NormalTok{            yield x}
\NormalTok{            untilSTOP}
\end{Highlighting}
\end{Shaded}

\texttt{untilSTOP} is really sort of the crux of what makes these streaming
systems useful: we only pull items from the file as we need it, and
\texttt{untilSTOP} will stop pulling anything as soon as we hit \texttt{STOP},
so no IO will happen anymore if the upstream sink does IO.

\hypertarget{our-ideal-program}{%
\subsection{Our Ideal Program}\label{our-ideal-program}}

Now ideally, we'd want to write a program that lets us compose the above pipes
to read from a file and output its contents to stdout, until it sees a STOP
line:

\begin{Shaded}
\begin{Highlighting}[]
\CommentTok{{-}{-} source: https://github.com/mstksg/inCode/tree/master/code{-}samples/misc/streaming{-}combinators{-}free.hs\#L105{-}L110}

\OtherTok{samplePipe ::} \DataTypeTok{Handle} \OtherTok{{-}>} \DataTypeTok{Pipe}\NormalTok{ i o }\DataTypeTok{IO}\NormalTok{ ()}
\NormalTok{samplePipe handle }\OtherTok{=}
\NormalTok{       sourceHandle handle}
    \OperatorTok{.|}\NormalTok{ untilSTOP}
    \OperatorTok{.|}\NormalTok{ toUpperPipe}
    \OperatorTok{.|}\NormalTok{ sinkStdout}
\end{Highlighting}
\end{Shaded}

\hypertarget{implementing-the-interface-you-want}{%
\section{Implementing the interface you
want}\label{implementing-the-interface-you-want}}

Step 2 of our plan was to identify the primitive actions we want. Looking at our
interface, it seems like the few things that let us ``create'' a \texttt{Pipe}
from scratch (instead of combining existing ones) are:

\begin{Shaded}
\begin{Highlighting}[]
\OtherTok{yield  ::}\NormalTok{ o }\OtherTok{{-}>} \DataTypeTok{Pipe}\NormalTok{ i o m ()}
\OtherTok{await  ::} \DataTypeTok{Pipe}\NormalTok{ i o m (}\DataTypeTok{Maybe}\NormalTok{ i)}
\OtherTok{lift   ::}\NormalTok{ m a }\OtherTok{{-}>} \DataTypeTok{Pipe}\NormalTok{ i o m a}
\FunctionTok{return}\OtherTok{ ::}\NormalTok{ a   }\OtherTok{{-}>} \DataTypeTok{Pipe}\NormalTok{ i o m a}
\end{Highlighting}
\end{Shaded}

However, we can note that \texttt{lift} and \texttt{return} can be gained just
from having a \texttt{Monad} and \texttt{MonadTrans} instance. So let's assume
we have those instances.

\begin{Shaded}
\begin{Highlighting}[]
\KeywordTok{class} \DataTypeTok{Monad}\NormalTok{ m }\KeywordTok{where}
\OtherTok{    return ::}\NormalTok{ a }\OtherTok{{-}>}\NormalTok{ m a}

\KeywordTok{class} \DataTypeTok{MonadTrans}\NormalTok{ p }\KeywordTok{where}
\OtherTok{    lift ::}\NormalTok{ m a }\OtherTok{{-}>}\NormalTok{ p m a}
\end{Highlighting}
\end{Shaded}

The functor combinator plan is to identify your primitives, and let free
structures give you the instances (in our case, \texttt{Monad} and
\texttt{MonadTrans}) you need for them.

So this means we only need two primitives: \texttt{yield} and \texttt{await}.
Then we just throw them into some machinery that gives us a free \texttt{Monad}
and \texttt{MonadTrans} structure, and we're golden :)

In the style of the \emph{\href{https://hackage.haskell.org/package/free}{free}}
library, we'd write base functions to get an ADT that describes the primitive
actions:

\begin{Shaded}
\begin{Highlighting}[]
\CommentTok{{-}{-} source: https://github.com/mstksg/inCode/tree/master/code{-}samples/misc/streaming{-}combinators{-}free.hs\#L21{-}L24}

\KeywordTok{data} \DataTypeTok{PipeF}\NormalTok{ i o a }\OtherTok{=}
    \DataTypeTok{YieldF}\NormalTok{ o a}
  \OperatorTok{|} \DataTypeTok{AwaitF}\NormalTok{ (}\DataTypeTok{Maybe}\NormalTok{ i }\OtherTok{{-}>}\NormalTok{ a)}
    \KeywordTok{deriving} \DataTypeTok{Functor}
\end{Highlighting}
\end{Shaded}

The general structure of the base functor style is to represent each primitive
as a constructor: include any inputs, and then a continuation on what to do if
you had the result.

For example:

\begin{enumerate}
\def\labelenumi{\arabic{enumi}.}
\tightlist
\item
  For \texttt{YieldF}, you need an \texttt{o} to be able to yield. The second
  field should really be the continuation \texttt{()\ -\textgreater{}\ a}, since
  the result is \texttt{()}, but that's equivalent to \texttt{a} in Haskell.
\item
  For \texttt{AwaitF}, you don't need any parameters to await, but the
  continuation is \texttt{Maybe\ i\ -\textgreater{}\ a} since you need to
  specify how to handle the \texttt{Maybe\ i} result.
\end{enumerate}

(This is specifically the structure that
\emph{\href{https://hackage.haskell.org/package/free}{free}} expects, but this
principle can be ported to any algebraic effects system.)

And now\ldots we're done! We can use the \texttt{FreeT} type from
\emph{\href{https://hackage.haskell.org/package/free/docs/Control-Monad-Trans-Free.html}{Control.Monad.Trans.Free}},
and now we have our pipe interface, with a \texttt{Monad} and
\texttt{MonadTrans} instance!

\begin{Shaded}
\begin{Highlighting}[]
\KeywordTok{type} \DataTypeTok{Pipe}\NormalTok{ i o }\OtherTok{=} \DataTypeTok{FreeT}\NormalTok{ (}\DataTypeTok{PipeF}\NormalTok{ i o)}
\end{Highlighting}
\end{Shaded}

This takes our base functor and imbues it with a full \texttt{Monad} and
\texttt{MonadTrans} instance:

\begin{Shaded}
\begin{Highlighting}[]
\OtherTok{lift ::}\NormalTok{ m a }\OtherTok{{-}>} \DataTypeTok{FreeT}\NormalTok{ (}\DataTypeTok{PipeF}\NormalTok{ i o) m a}
\OtherTok{lift ::}\NormalTok{ m a }\OtherTok{{-}>} \DataTypeTok{Pipe}\NormalTok{ i o m a}

\FunctionTok{return}\OtherTok{ ::}\NormalTok{ a }\OtherTok{{-}>} \DataTypeTok{FreeT}\NormalTok{ (}\DataTypeTok{PipeF}\NormalTok{ i o) m a}
\FunctionTok{return}\OtherTok{ ::}\NormalTok{ a }\OtherTok{{-}>} \DataTypeTok{Pipe}\NormalTok{ i o m a}

\OtherTok{(>>)  ::} \DataTypeTok{Pipe}\NormalTok{ i o m a }\OtherTok{{-}>} \DataTypeTok{Pipe}\NormalTok{ i o m b }\OtherTok{{-}>} \DataTypeTok{Pipe}\NormalTok{ i o m b}
\OtherTok{(>>=) ::} \DataTypeTok{Pipe}\NormalTok{ i o m a }\OtherTok{{-}>}\NormalTok{ (a }\OtherTok{{-}>} \DataTypeTok{Pipe}\NormalTok{ i o m b) }\OtherTok{{-}>} \DataTypeTok{Pipe}\NormalTok{ i o m b}
\end{Highlighting}
\end{Shaded}

That's the essence of the free structure: it \emph{adds} to our base functor
(\texttt{PipeF}) exactly the structure it needs to be able to implement the
instances it is free on. And it's all free as in beer! :D

Now we just need our functions to lift our primitives to \texttt{Pipe}, using
\texttt{liftF\ ::\ f\ a\ -\textgreater{}\ FreeT\ f\ m\ a}:

\begin{Shaded}
\begin{Highlighting}[]
\CommentTok{{-}{-} source: https://github.com/mstksg/inCode/tree/master/code{-}samples/misc/streaming{-}combinators{-}free.hs\#L28{-}L32}

\OtherTok{yield ::} \DataTypeTok{Monad}\NormalTok{ m }\OtherTok{=>}\NormalTok{ o }\OtherTok{{-}>} \DataTypeTok{Pipe}\NormalTok{ i o m ()}
\NormalTok{yield x }\OtherTok{=}\NormalTok{ liftF }\OperatorTok{$} \DataTypeTok{YieldF}\NormalTok{ x ()}

\OtherTok{await ::} \DataTypeTok{Monad}\NormalTok{ m }\OtherTok{=>} \DataTypeTok{Pipe}\NormalTok{ i o m (}\DataTypeTok{Maybe}\NormalTok{ i)}
\NormalTok{await }\OtherTok{=}\NormalTok{ liftF }\OperatorTok{$} \DataTypeTok{AwaitF} \FunctionTok{id}
\end{Highlighting}
\end{Shaded}

(these things you can usually just fill in using type tetris, filling in values
with typed holes into they typecheck).

Note that all of the individual pipes we had planned work as-is!

\begin{Shaded}
\begin{Highlighting}[]
\CommentTok{{-}{-} source: https://github.com/mstksg/inCode/tree/master/code{-}samples/misc/streaming{-}combinators{-}free.hs\#L67{-}L103}

\OtherTok{sourceHandle ::} \DataTypeTok{Handle} \OtherTok{{-}>} \DataTypeTok{Pipe}\NormalTok{ i }\DataTypeTok{String} \DataTypeTok{IO}\NormalTok{ ()}
\NormalTok{sourceHandle handle }\OtherTok{=} \KeywordTok{do}
\NormalTok{    res }\OtherTok{<{-}}\NormalTok{ lift }\OperatorTok{$}\NormalTok{ tryJust (guard }\OperatorTok{.}\NormalTok{ isEOFError) (hGetLine handle)}
    \KeywordTok{case}\NormalTok{ res }\KeywordTok{of}
      \DataTypeTok{Left}\NormalTok{  \_   }\OtherTok{{-}>} \FunctionTok{return}\NormalTok{ ()}
      \DataTypeTok{Right}\NormalTok{ out }\OtherTok{{-}>} \KeywordTok{do}
\NormalTok{        yield out}
\NormalTok{        sourceHandle handle}

\OtherTok{sinkStdout ::} \DataTypeTok{Pipe} \DataTypeTok{String}\NormalTok{ o }\DataTypeTok{IO}\NormalTok{ ()}
\NormalTok{sinkStdout }\OtherTok{=} \KeywordTok{do}
\NormalTok{    inp }\OtherTok{<{-}}\NormalTok{ await}
    \KeywordTok{case}\NormalTok{ inp }\KeywordTok{of}
      \DataTypeTok{Nothing} \OtherTok{{-}>} \FunctionTok{pure}\NormalTok{ ()}
      \DataTypeTok{Just}\NormalTok{ x  }\OtherTok{{-}>} \KeywordTok{do}
\NormalTok{        lift }\OperatorTok{$} \FunctionTok{putStrLn}\NormalTok{ x}
\NormalTok{        sinkStdout}

\OtherTok{toUpperPipe ::} \DataTypeTok{Monad}\NormalTok{ m }\OtherTok{=>} \DataTypeTok{Pipe} \DataTypeTok{String} \DataTypeTok{String}\NormalTok{ m ()}
\NormalTok{toUpperPipe }\OtherTok{=} \KeywordTok{do}
\NormalTok{    inp }\OtherTok{<{-}}\NormalTok{ await}
    \KeywordTok{case}\NormalTok{ inp }\KeywordTok{of}
      \DataTypeTok{Nothing} \OtherTok{{-}>} \FunctionTok{pure}\NormalTok{ ()}
      \DataTypeTok{Just}\NormalTok{ x  }\OtherTok{{-}>} \KeywordTok{do}
\NormalTok{        yield (}\FunctionTok{map} \FunctionTok{toUpper}\NormalTok{ x)}
\NormalTok{        toUpperPipe}

\OtherTok{untilSTOP ::} \DataTypeTok{Monad}\NormalTok{ m }\OtherTok{=>} \DataTypeTok{Pipe} \DataTypeTok{String} \DataTypeTok{String}\NormalTok{ m ()}
\NormalTok{untilSTOP }\OtherTok{=} \KeywordTok{do}
\NormalTok{    inp }\OtherTok{<{-}}\NormalTok{ await}
    \KeywordTok{case}\NormalTok{ inp }\KeywordTok{of}
      \DataTypeTok{Nothing} \OtherTok{{-}>} \FunctionTok{pure}\NormalTok{ ()}
      \DataTypeTok{Just}\NormalTok{ x}
        \OperatorTok{|}\NormalTok{ x }\OperatorTok{==} \StringTok{"STOP"} \OtherTok{{-}>} \FunctionTok{pure}\NormalTok{ ()}
        \OperatorTok{|} \FunctionTok{otherwise}   \OtherTok{{-}>} \KeywordTok{do}
\NormalTok{            yield x}
\NormalTok{            untilSTOP}
\end{Highlighting}
\end{Shaded}

That's because using \texttt{FreeT}, we imbue the structure required to do
monadic chaining (do notation) and MonadTrans (\texttt{lift}) for free!

To ``run'' our pipes, we can use \texttt{FreeT}'s ``interpreter'' function. This
follows the same pattern as for many free structures: specify how to handle each
individual base functor constructor, and it then gives you a handler to handle
the entire thing.

\begin{Shaded}
\begin{Highlighting}[]
\NormalTok{iterT}
\OtherTok{    ::}\NormalTok{ (}\DataTypeTok{PipeF}\NormalTok{ i o (m a) }\OtherTok{{-}>}\NormalTok{ m a)  }\CommentTok{{-}{-} \^{} given a way to handle each base functor constructor ...}
    \OtherTok{{-}>} \DataTypeTok{Pipe}\NormalTok{ i o m a }\OtherTok{{-}>}\NormalTok{ m a       }\CommentTok{{-}{-} \^{} here\textquotesingle{}s a way to handle the whole thing}
\end{Highlighting}
\end{Shaded}

So let's write our base functor handler. Remember that we established earlier we
can only ``run'' a \texttt{Pipe\ ()\ Void\ m\ a}: that is, pipes where
\texttt{await} can always be fed with no information (\texttt{()}) and no
\texttt{yield} is ever called (because you cannot yield with \texttt{Void}, a
type with no inhabitants). We can directly translate this to how we handle each
constructor:

\begin{Shaded}
\begin{Highlighting}[]
\CommentTok{{-}{-} source: https://github.com/mstksg/inCode/tree/master/code{-}samples/misc/streaming{-}combinators{-}free.hs\#L59{-}L62}

\OtherTok{handlePipeF ::} \DataTypeTok{PipeF}\NormalTok{ () }\DataTypeTok{Void}\NormalTok{ (m a) }\OtherTok{{-}>}\NormalTok{ m a}
\NormalTok{handlePipeF }\OtherTok{=}\NormalTok{ \textbackslash{}}\KeywordTok{case}
    \DataTypeTok{YieldF}\NormalTok{ o \_ }\OtherTok{{-}>}\NormalTok{ absurd o}
    \DataTypeTok{AwaitF}\NormalTok{ f   }\OtherTok{{-}>}\NormalTok{ f (}\DataTypeTok{Just}\NormalTok{ ())}
\end{Highlighting}
\end{Shaded}

And so we get our full \texttt{runPipe}:

\begin{Shaded}
\begin{Highlighting}[]
\CommentTok{{-}{-} source: https://github.com/mstksg/inCode/tree/master/code{-}samples/misc/streaming{-}combinators{-}free.hs\#L64{-}L65}

\OtherTok{runPipe ::} \DataTypeTok{Monad}\NormalTok{ m }\OtherTok{=>} \DataTypeTok{Pipe}\NormalTok{ () }\DataTypeTok{Void}\NormalTok{ m a }\OtherTok{{-}>}\NormalTok{ m a}
\NormalTok{runPipe }\OtherTok{=}\NormalTok{ iterT handlePipeF}
\end{Highlighting}
\end{Shaded}

I think this process exemplifies most of the major beats when working with free
structures:

\begin{enumerate}
\def\labelenumi{\arabic{enumi}.}
\tightlist
\item
  Define the base functor
\item
  Allow the free structure to imbue the proper structure over your base functor
\item
  Write your interpreter to interpret the constructors of your base functor, and
  the free structure will give you a way to interpret the entire structure.
\end{enumerate}

\hypertarget{chaining}{%
\subsection{Chaining}\label{chaining}}

If you look at the list of all the things we wanted, we're still missing one
thing: pipe composition/input-output chaining. That's because it isn't a
primitive operation (like yield or await), and it wasn't given to us for free by
our free structure (\texttt{FreeT}, which gave us monadic composition and monad
transformer ability). So with how we have currently written it, there isn't any
way of getting around writing \texttt{(.\textbar{})} manually. So let's roll up
our sleeves and do the (admittedly minimal amount of) dirty work.

Let's think about the semantics of our pipe chaining. We want to never do more
work than we need to do, so we'll be ``pull-based'': for
\texttt{f\ .\textbar{}\ g}, try running \texttt{g} as much as possible until it
awaits anything from \texttt{f}. Only then do we try doing \texttt{f}.

To implement this, we're going to have to dig in a little bit to the
implementation/structure of \texttt{FreeT}:

\begin{Shaded}
\begin{Highlighting}[]
\KeywordTok{newtype} \DataTypeTok{FreeT}\NormalTok{ f m a }\OtherTok{=} \DataTypeTok{FreeT}
\NormalTok{    \{}\OtherTok{ runFreeT ::}\NormalTok{ m (}\DataTypeTok{FreeF}\NormalTok{ f a (}\DataTypeTok{FreeT}\NormalTok{ f m a)) \}}

\KeywordTok{data} \DataTypeTok{FreeF}\NormalTok{ f a b}
      \DataTypeTok{Pure}\NormalTok{ a}
    \OperatorTok{|} \DataTypeTok{Free}\NormalTok{ (f b)}
\end{Highlighting}
\end{Shaded}

If this looks a little complicated, don't worry: on the face of it, it can be a
bit intimidating. And why is there a second internal data type?

Well, you can think of \texttt{FreeF\ f\ a\ b} as being a fancy version of
\texttt{Either\ a\ (f\ b)}. And the implementation of \texttt{FreeT} is saying
that \texttt{FreeT\ f\ m\ a} is \emph{an m-action} that produces
\texttt{Either\ a\ (FreeT\ f\ m\ a)}. So for example, \texttt{FreeT\ f\ IO\ a}
is an IO action that produces \emph{either} the \texttt{a} (we're done, end
here!) or a \texttt{f\ (FreeT\ f\ m\ a))} (we have to handle an \texttt{f}
here!)

\begin{Shaded}
\begin{Highlighting}[]
\KeywordTok{newtype} \DataTypeTok{FreeT}\NormalTok{ f m a }\OtherTok{=} \DataTypeTok{FreeT}
\NormalTok{    \{}\OtherTok{ runFreeT ::}\NormalTok{ m (}\DataTypeTok{Either}\NormalTok{ a (f (}\DataTypeTok{FreeT}\NormalTok{ f m a))) \}}
\end{Highlighting}
\end{Shaded}

At the top level, \texttt{FreeT} is an action in the underlying monad (just like
\texttt{MaybeT}, \texttt{ExceptT}, \texttt{StateT}, etc.). Let's take that into
account and write our implementation (with a hefty bit of help from the
typechecker and typed holes)! Remember our plan: for \texttt{f\ .\textbar{}\ g},
\emph{start unrolling \texttt{g}} until it needs anything, and then ask
\texttt{f} when it does.

\begin{Shaded}
\begin{Highlighting}[]
\NormalTok{(}\OperatorTok{.|}\NormalTok{)}
\OtherTok{    ::} \DataTypeTok{Monad}\NormalTok{ m}
    \OtherTok{=>} \DataTypeTok{Pipe}\NormalTok{ a b m x         }\CommentTok{{-}{-} \^{} pipe from a {-}> b}
    \OtherTok{{-}>} \DataTypeTok{Pipe}\NormalTok{ b c m y         }\CommentTok{{-}{-} \^{} pipe from b {-}> c}
    \OtherTok{{-}>} \DataTypeTok{Pipe}\NormalTok{ a c m y         }\CommentTok{{-}{-} \^{} pipe from a {-}> c}
\NormalTok{pf }\OperatorTok{.|}\NormalTok{ pg }\OtherTok{=} \KeywordTok{do}
\NormalTok{    gRes }\OtherTok{<{-}}\NormalTok{ lift }\OperatorTok{$}\NormalTok{ runFreeT pg          }\CommentTok{{-}{-} 1}
    \KeywordTok{case}\NormalTok{ gRes }\KeywordTok{of}
      \DataTypeTok{Pure}\NormalTok{ x            }\OtherTok{{-}>} \FunctionTok{pure}\NormalTok{ x       }\CommentTok{{-}{-} 2}
      \DataTypeTok{Free}\NormalTok{ (}\DataTypeTok{YieldF}\NormalTok{ o x) }\OtherTok{{-}>} \KeywordTok{do}           \CommentTok{{-}{-} 3}
\NormalTok{        yield o}
\NormalTok{        pf }\OtherTok{\textasciigrave{}comp\textasciigrave{}}\NormalTok{ x}
      \DataTypeTok{Free}\NormalTok{ (}\DataTypeTok{AwaitF}\NormalTok{ g  ) }\OtherTok{{-}>} \KeywordTok{do}           \CommentTok{{-}{-} 4}
\NormalTok{        fRes }\OtherTok{<{-}}\NormalTok{ lift }\OperatorTok{$}\NormalTok{ runFreeT pf}
        \KeywordTok{case}\NormalTok{ fRes }\KeywordTok{of}
          \DataTypeTok{Pure}\NormalTok{ \_            }\OtherTok{{-}>} \FunctionTok{pure}\NormalTok{ () }\OtherTok{\textasciigrave{}comp\textasciigrave{}}\NormalTok{ g }\DataTypeTok{Nothing}     \CommentTok{{-}{-} 5}
          \DataTypeTok{Free}\NormalTok{ (}\DataTypeTok{YieldF}\NormalTok{ o y) }\OtherTok{{-}>}\NormalTok{ y       }\OtherTok{\textasciigrave{}comp\textasciigrave{}}\NormalTok{ g (}\DataTypeTok{Just}\NormalTok{ o)    }\CommentTok{{-}{-} 6}
          \DataTypeTok{Free}\NormalTok{ (}\DataTypeTok{AwaitF}\NormalTok{ f  ) }\OtherTok{{-}>} \KeywordTok{do}                           \CommentTok{{-}{-} 7}
\NormalTok{            i }\OtherTok{<{-}}\NormalTok{ await}
\NormalTok{            f i }\OtherTok{\textasciigrave{}comp\textasciigrave{}} \DataTypeTok{FreeT}\NormalTok{ (}\FunctionTok{pure}\NormalTok{ gRes)}
\end{Highlighting}
\end{Shaded}

Here are some numbered notes and comments:

\begin{enumerate}
\def\labelenumi{\arabic{enumi}.}
\tightlist
\item
  Start unrolling the downstream pipe \texttt{pg}, in the underlying monad
  \texttt{m}!
\item
  If \texttt{pg} produced \texttt{Pure\ x}, it means we're done pulling
  anything. So just quit out with \texttt{pure\ x}.
\item
  If \texttt{pg} produced \texttt{Free\ (YieldF\ o\ x)}, it means it's yielding
  an \texttt{o} and continuing on with \texttt{x}. So let's just yield that
  \texttt{o} and move on to the composition of \texttt{pf} with the next pipe
  \texttt{x}.
\item
  If \texttt{pg} produced \texttt{Free\ (AwaitF\ g)}, now things get
  interesting. We need to unroll \texttt{pf} until it yields some
  \texttt{Maybe\ b}, and feed that to
  \texttt{g\ ::\ Maybe\ b\ -\textgreater{}\ Pipe\ b\ c\ m\ y}.
\item
  If \texttt{pf} produced \texttt{Pure\ y}, that means it was done! So
  \texttt{g} gets a \texttt{Nothing}, and we move from there. Note we have to
  compose with a dummy pipe \texttt{pure\ ()} to make the types match up
  properly.
\item
  If \texttt{pf} produced \texttt{YieldF\ o\ y}, then we have found our match!
  So give \texttt{g\ (Just\ o)}, and now we recursively compose the next pipe
  (\texttt{y}) with the that \texttt{g} gave us.
\item
  If \texttt{pf} produced \texttt{AwaitF\ f}, then we're in a bind, aren't we?
  We now have two layers waiting for something further upstream. So, we return a
  new pipe that awaits it and feeds it to \texttt{f} when it gets it, give it to
  \texttt{f}, and then compose \texttt{f\ i\ ::\ Pipe\ a\ b\ m\ x} with
  \texttt{pg}'s result (wrapping up \texttt{gRes} back into a
  \texttt{FreeT}/\texttt{Pipe} so the types match up).
\end{enumerate}

Admittedly (!) this is the ``ugly'' part of this derivation: sometimes we just
can't get everything for free. But getting the Monad, Applicative, Functor,
MonadTrans, etc. instances is probably nice enough to justify this inconvenience
:) And who knows, there might be a free structure that I don't know about that
gives us all of these \emph{plus} piping for free.

\hypertarget{putting-it-all-together}{%
\subsection{Putting it all together}\label{putting-it-all-together}}

Let's see if it runs!

\begin{Shaded}
\begin{Highlighting}[]
\CommentTok{{-}{-} source: https://github.com/mstksg/inCode/tree/master/code{-}samples/misc/streaming{-}combinators{-}free.hs\#L105{-}L110}

\OtherTok{samplePipe ::} \DataTypeTok{Handle} \OtherTok{{-}>} \DataTypeTok{Pipe}\NormalTok{ i o }\DataTypeTok{IO}\NormalTok{ ()}
\NormalTok{samplePipe handle }\OtherTok{=}
\NormalTok{       sourceHandle handle}
    \OperatorTok{.|}\NormalTok{ untilSTOP}
    \OperatorTok{.|}\NormalTok{ toUpperPipe}
    \OperatorTok{.|}\NormalTok{ sinkStdout}
\end{Highlighting}
\end{Shaded}

\begin{verbatim}
$ cat testpipefile.txt
hello
world
STOP
okay
goodbye
\end{verbatim}

\begin{Shaded}
\begin{Highlighting}[]
\NormalTok{ghci}\OperatorTok{>}\NormalTok{ withFile }\StringTok{"testpipefile.txt"} \DataTypeTok{ReadMode} \OperatorTok{$}\NormalTok{ \textbackslash{}handle }\OtherTok{{-}>}
\NormalTok{        runPipe (samplePipe handle)}
\CommentTok{{-}{-} HELLO}
\CommentTok{{-}{-} WORLD}
\end{Highlighting}
\end{Shaded}

Smooth as silk :D

\hypertarget{takeways}{%
\section{Takeways}\label{takeways}}

Most of this was thought up when I needed\footnote{This came about when I was
  developing my numerical
  \emph{\href{https://hackage.haskell.org/package/emd}{emd}} library.} a tool
that was \emph{sort of} like conduit, \emph{sort of} like pipes, \emph{sort of}
like the other libraries\ldots and I thought I had to read up on the theory of
pipes and iteratees and trampolines and fancy pants math stuff to be able to
make anything useful in this space. I remember being very discouraged when I
read about this stuff as a wee new Haskeller, because the techniques seemed so
foreign and out of the range of my normal Haskell experience.

However, with a level head, I just thought --- ``ok, I just need a monad (trans)
with two primitive actions: await, and yield. Why don't I just make an await and
yield and get automatic \texttt{Monad} and \texttt{MonadTrans} instances with
the appropriate free structure?''

As we can see\ldots this works just fine! We only needed to implement one extra
thing (\texttt{.\textbar{}}) to get the interface of our dreams.

The next time you need to make some monad that might seem exotic, try this out
and see if it works for you :)

Happy holidays, and merry Christmas!

\hypertarget{exercises}{%
\section{Exercises}\label{exercises}}

Click on the links in the corner of the text boxes for solutions! (or just check
out
\href{https://github.com/mstksg/inCode/tree/master/code-samples/misc/streaming-combinators-free.hs}{the
source file})

\begin{enumerate}
\def\labelenumi{\arabic{enumi}.}
\item
  An \texttt{Pipe\ i\ o\ m\ a} ``takes'' \texttt{i} and ``produces'' \texttt{o},
  so it should make sense to make pre-map and post-map functions:

\begin{Shaded}
\begin{Highlighting}[]
\CommentTok{{-}{-} source: https://github.com/mstksg/inCode/tree/master/code{-}samples/misc/streaming{-}combinators{-}free.hs\#L125{-}L128}

\OtherTok{postMap ::} \DataTypeTok{Monad}\NormalTok{ m }\OtherTok{=>}\NormalTok{ (o }\OtherTok{{-}>}\NormalTok{ o\textquotesingle{}) }\OtherTok{{-}>} \DataTypeTok{Pipe}\NormalTok{ i o m a }\OtherTok{{-}>} \DataTypeTok{Pipe}\NormalTok{ i o\textquotesingle{} m a}

\OtherTok{preMap ::} \DataTypeTok{Monad}\NormalTok{ m }\OtherTok{=>}\NormalTok{ (i\textquotesingle{} }\OtherTok{{-}>}\NormalTok{ i) }\OtherTok{{-}>} \DataTypeTok{Pipe}\NormalTok{ i o m a }\OtherTok{{-}>} \DataTypeTok{Pipe}\NormalTok{ i\textquotesingle{} o m a}
\end{Highlighting}
\end{Shaded}

  That pre-maps all inputs the pipe would receive, and post-maps all of the
  values it yields.

  Hint: This actually is made a lot simpler to write with the handy
  \texttt{transFreeT} combinator, which lets you swap out/change the base
  functor:

\begin{Shaded}
\begin{Highlighting}[]
\NormalTok{transFreeT}
\OtherTok{    ::}\NormalTok{ (}\KeywordTok{forall}\NormalTok{ a}\OperatorTok{.}\NormalTok{ f a }\OtherTok{{-}>}\NormalTok{ g a)     }\CommentTok{{-}{-} \^{} polymorphic function to edit the base functor}
    \OtherTok{{-}>} \DataTypeTok{FreeT}\NormalTok{ f m b}
    \OtherTok{{-}>} \DataTypeTok{FreeT}\NormalTok{ g m b}

\NormalTok{transFreeT}
\OtherTok{    ::}\NormalTok{ (}\KeywordTok{forall}\NormalTok{ a}\OperatorTok{.} \DataTypeTok{PipeF}\NormalTok{ i o a }\OtherTok{{-}>} \DataTypeTok{PipeF}\NormalTok{ i\textquotesingle{} o\textquotesingle{} a)  }\CommentTok{{-}{-} \^{} polymorphic function to edit the base functor}
    \OtherTok{{-}>} \DataTypeTok{Pipe}\NormalTok{ i  o  m a}
    \OtherTok{{-}>} \DataTypeTok{Pipe}\NormalTok{ i\textquotesingle{} o\textquotesingle{} m a}
\end{Highlighting}
\end{Shaded}

  We could then write pre-map and post-map function on \texttt{PipeF} and
  translate them to \texttt{Pipe} using \texttt{transFreeT}:

\begin{Shaded}
\begin{Highlighting}[]
\CommentTok{{-}{-} source: https://github.com/mstksg/inCode/tree/master/code{-}samples/misc/streaming{-}combinators{-}free.hs\#L117{-}L129}

\OtherTok{postMapF ::}\NormalTok{ (o }\OtherTok{{-}>}\NormalTok{ o\textquotesingle{}) }\OtherTok{{-}>} \DataTypeTok{PipeF}\NormalTok{ i o a }\OtherTok{{-}>} \DataTypeTok{PipeF}\NormalTok{ i o\textquotesingle{} a}

\OtherTok{preMapF ::}\NormalTok{ (i\textquotesingle{} }\OtherTok{{-}>}\NormalTok{ i) }\OtherTok{{-}>} \DataTypeTok{PipeF}\NormalTok{ i o a }\OtherTok{{-}>} \DataTypeTok{PipeF}\NormalTok{ i\textquotesingle{} o a}

\OtherTok{postMap ::} \DataTypeTok{Monad}\NormalTok{ m }\OtherTok{=>}\NormalTok{ (o }\OtherTok{{-}>}\NormalTok{ o\textquotesingle{}) }\OtherTok{{-}>} \DataTypeTok{Pipe}\NormalTok{ i o m a }\OtherTok{{-}>} \DataTypeTok{Pipe}\NormalTok{ i o\textquotesingle{} m a}
\NormalTok{postMap f }\OtherTok{=}\NormalTok{ transFreeT (postMapF f)}

\OtherTok{preMap ::} \DataTypeTok{Monad}\NormalTok{ m }\OtherTok{=>}\NormalTok{ (i\textquotesingle{} }\OtherTok{{-}>}\NormalTok{ i) }\OtherTok{{-}>} \DataTypeTok{Pipe}\NormalTok{ i o m a }\OtherTok{{-}>} \DataTypeTok{Pipe}\NormalTok{ i\textquotesingle{} o m a}
\NormalTok{preMap f }\OtherTok{=}\NormalTok{ transFreeT (preMapF f)}
\end{Highlighting}
\end{Shaded}
\item
  Let's say we modified our \texttt{PipeF} slightly to take another parameter
  \texttt{u}, the result type of the upstream pipe.

\begin{Shaded}
\begin{Highlighting}[]
\KeywordTok{data} \DataTypeTok{PipeF}\NormalTok{ i o u a }\OtherTok{=}
    \DataTypeTok{YieldF}\NormalTok{ o a}
  \OperatorTok{|} \DataTypeTok{AwaitF}\NormalTok{ (}\DataTypeTok{Either}\NormalTok{ u i }\OtherTok{{-}>}\NormalTok{ a)}

\KeywordTok{type} \DataTypeTok{Pipe}\NormalTok{ i o u }\OtherTok{=} \DataTypeTok{FreeT}\NormalTok{ (}\DataTypeTok{PipeF}\NormalTok{ i o u)}

\OtherTok{await ::} \DataTypeTok{Pipe}\NormalTok{ i o m (}\DataTypeTok{Either}\NormalTok{ u i)}
\NormalTok{await }\OtherTok{=}\NormalTok{ liftF }\OperatorTok{$} \DataTypeTok{AwaitF} \FunctionTok{id}
\end{Highlighting}
\end{Shaded}

  So now \texttt{await} would be fed \texttt{i} things yielded from upstream,
  but sometimes you'd get a \texttt{Left} indicating that the upstream pipe has
  terminated.

  What would be the implications if \texttt{u} is \texttt{Void}?

\begin{Shaded}
\begin{Highlighting}[]
\KeywordTok{type} \DataTypeTok{CertainPipe}\NormalTok{ i o }\OtherTok{=} \DataTypeTok{Pipe}\NormalTok{ i o }\DataTypeTok{Void}
\end{Highlighting}
\end{Shaded}

  What could you do in a \texttt{CertainPipe\ i\ o\ m\ a} that you couldn't
  normally do with our \texttt{Pipe\ i\ o\ m\ a}?
\item
  We mentioned earlier that a ``source'' could have type

\begin{Shaded}
\begin{Highlighting}[]
\KeywordTok{type} \DataTypeTok{Source} \OtherTok{=} \DataTypeTok{Pipe}\NormalTok{ ()}
\end{Highlighting}
\end{Shaded}

  And a \texttt{Source\ o\ m\ a} would be something that keeps on pumping out
  \texttt{o}s as much as we need, without requiring any upstream input.

  This is actually the essential behavior of the (true) list monad transformer,
  as esposed by the
  \emph{\href{https://hackage.haskell.org/package/list-transformer}{list-transformer}}
  package.

  In that package, \texttt{ListT} is defined as:

\begin{Shaded}
\begin{Highlighting}[]
\KeywordTok{newtype} \DataTypeTok{ListT}\NormalTok{ m a }\OtherTok{=} \DataTypeTok{ListT}\NormalTok{ \{}\OtherTok{ next ::}\NormalTok{ m (}\DataTypeTok{Step}\NormalTok{ m a) \}}

\KeywordTok{data} \DataTypeTok{Step}\NormalTok{ m a }\OtherTok{=} \DataTypeTok{Cons}\NormalTok{ a (}\DataTypeTok{ListT}\NormalTok{ m a) }\OperatorTok{|} \DataTypeTok{Nil}
\end{Highlighting}
\end{Shaded}

  And it's a type that can yield out new \texttt{a}s on-demand, until exhausted.

  In fact, \texttt{Source\ o\ m\ ()} is equivalent to \texttt{ListT\ m\ o}.
  Write the functions to convert between them! :D

\begin{Shaded}
\begin{Highlighting}[]
\CommentTok{{-}{-} source: https://github.com/mstksg/inCode/tree/master/code{-}samples/misc/streaming{-}combinators{-}free.hs\#L133{-}L141}

\OtherTok{toListT ::} \DataTypeTok{Monad}\NormalTok{ m }\OtherTok{=>} \DataTypeTok{Pipe}\NormalTok{ () o m a }\OtherTok{{-}>} \DataTypeTok{L.ListT}\NormalTok{ m o}

\OtherTok{fromListT ::} \DataTypeTok{Monad}\NormalTok{ m }\OtherTok{=>} \DataTypeTok{L.ListT}\NormalTok{ m o }\OtherTok{{-}>} \DataTypeTok{Pipe}\NormalTok{ i o m ()}
\end{Highlighting}
\end{Shaded}

  Unfortunately we cannot use \texttt{iterT} because the last type parameter of
  each is different. But manual pattern matching (like how we wrote
  \texttt{(.\textbar{})}) isn't too bad!

  The semantics of \texttt{ListT} api is that
  \texttt{x\ \textless{}\textbar{}\textgreater{}\ y} will ``do'' (and emit the
  result) \texttt{x} before moving on to what \texttt{y} would emit. And
  \texttt{empty} is the \texttt{ListT} that signals it is done producing.
  \texttt{\textless{}\textbar{}\textgreater{}} and \texttt{pure} and
  \texttt{empty} for \texttt{ListT} are roughly analogous to
  \texttt{\textgreater{}\textgreater{}} and \texttt{yield} and \texttt{return}
  for \texttt{Source}, respectively.
\end{enumerate}

\hypertarget{special-thanks}{%
\section{Special Thanks}\label{special-thanks}}

I am very humbled to be supported by an amazing community, who make it possible
for me to devote time to researching and writing these posts. Very special
thanks to my supporter at the ``Amazing'' level on
\href{https://www.patreon.com/justinle/overview}{patreon}, Josh Vera! :)

\hypertarget{signoff}{%
\section{Signoff}\label{signoff}}

Hi, thanks for reading! You can reach me via email at
\href{mailto:justin@jle.im}{\nolinkurl{justin@jle.im}}, or at twitter at
\href{https://twitter.com/mstk}{@mstk}! This post and all others are published
under the \href{https://creativecommons.org/licenses/by-nc-nd/3.0/}{CC-BY-NC-ND
3.0} license. Corrections and edits via pull request are welcome and encouraged
at \href{https://github.com/mstksg/inCode}{the source repository}.

If you feel inclined, or this post was particularly helpful for you, why not
consider \href{https://www.patreon.com/justinle/overview}{supporting me on
Patreon}, or a \href{bitcoin:3D7rmAYgbDnp4gp4rf22THsGt74fNucPDU}{BTC donation}?
:)

\end{document}
