\documentclass[]{article}
\usepackage{lmodern}
\usepackage{amssymb,amsmath}
\usepackage{ifxetex,ifluatex}
\usepackage{fixltx2e} % provides \textsubscript
\ifnum 0\ifxetex 1\fi\ifluatex 1\fi=0 % if pdftex
  \usepackage[T1]{fontenc}
  \usepackage[utf8]{inputenc}
\else % if luatex or xelatex
  \ifxetex
    \usepackage{mathspec}
    \usepackage{xltxtra,xunicode}
  \else
    \usepackage{fontspec}
  \fi
  \defaultfontfeatures{Mapping=tex-text,Scale=MatchLowercase}
  \newcommand{\euro}{€}
\fi
% use upquote if available, for straight quotes in verbatim environments
\IfFileExists{upquote.sty}{\usepackage{upquote}}{}
% use microtype if available
\IfFileExists{microtype.sty}{\usepackage{microtype}}{}
\usepackage[margin=1in]{geometry}
\usepackage{color}
\usepackage{fancyvrb}
\newcommand{\VerbBar}{|}
\newcommand{\VERB}{\Verb[commandchars=\\\{\}]}
\DefineVerbatimEnvironment{Highlighting}{Verbatim}{commandchars=\\\{\}}
% Add ',fontsize=\small' for more characters per line
\newenvironment{Shaded}{}{}
\newcommand{\AlertTok}[1]{\textcolor[rgb]{1.00,0.00,0.00}{\textbf{#1}}}
\newcommand{\AnnotationTok}[1]{\textcolor[rgb]{0.38,0.63,0.69}{\textbf{\textit{#1}}}}
\newcommand{\AttributeTok}[1]{\textcolor[rgb]{0.49,0.56,0.16}{#1}}
\newcommand{\BaseNTok}[1]{\textcolor[rgb]{0.25,0.63,0.44}{#1}}
\newcommand{\BuiltInTok}[1]{#1}
\newcommand{\CharTok}[1]{\textcolor[rgb]{0.25,0.44,0.63}{#1}}
\newcommand{\CommentTok}[1]{\textcolor[rgb]{0.38,0.63,0.69}{\textit{#1}}}
\newcommand{\CommentVarTok}[1]{\textcolor[rgb]{0.38,0.63,0.69}{\textbf{\textit{#1}}}}
\newcommand{\ConstantTok}[1]{\textcolor[rgb]{0.53,0.00,0.00}{#1}}
\newcommand{\ControlFlowTok}[1]{\textcolor[rgb]{0.00,0.44,0.13}{\textbf{#1}}}
\newcommand{\DataTypeTok}[1]{\textcolor[rgb]{0.56,0.13,0.00}{#1}}
\newcommand{\DecValTok}[1]{\textcolor[rgb]{0.25,0.63,0.44}{#1}}
\newcommand{\DocumentationTok}[1]{\textcolor[rgb]{0.73,0.13,0.13}{\textit{#1}}}
\newcommand{\ErrorTok}[1]{\textcolor[rgb]{1.00,0.00,0.00}{\textbf{#1}}}
\newcommand{\ExtensionTok}[1]{#1}
\newcommand{\FloatTok}[1]{\textcolor[rgb]{0.25,0.63,0.44}{#1}}
\newcommand{\FunctionTok}[1]{\textcolor[rgb]{0.02,0.16,0.49}{#1}}
\newcommand{\ImportTok}[1]{#1}
\newcommand{\InformationTok}[1]{\textcolor[rgb]{0.38,0.63,0.69}{\textbf{\textit{#1}}}}
\newcommand{\KeywordTok}[1]{\textcolor[rgb]{0.00,0.44,0.13}{\textbf{#1}}}
\newcommand{\NormalTok}[1]{#1}
\newcommand{\OperatorTok}[1]{\textcolor[rgb]{0.40,0.40,0.40}{#1}}
\newcommand{\OtherTok}[1]{\textcolor[rgb]{0.00,0.44,0.13}{#1}}
\newcommand{\PreprocessorTok}[1]{\textcolor[rgb]{0.74,0.48,0.00}{#1}}
\newcommand{\RegionMarkerTok}[1]{#1}
\newcommand{\SpecialCharTok}[1]{\textcolor[rgb]{0.25,0.44,0.63}{#1}}
\newcommand{\SpecialStringTok}[1]{\textcolor[rgb]{0.73,0.40,0.53}{#1}}
\newcommand{\StringTok}[1]{\textcolor[rgb]{0.25,0.44,0.63}{#1}}
\newcommand{\VariableTok}[1]{\textcolor[rgb]{0.10,0.09,0.49}{#1}}
\newcommand{\VerbatimStringTok}[1]{\textcolor[rgb]{0.25,0.44,0.63}{#1}}
\newcommand{\WarningTok}[1]{\textcolor[rgb]{0.38,0.63,0.69}{\textbf{\textit{#1}}}}
\ifxetex
  \usepackage[setpagesize=false, % page size defined by xetex
              unicode=false, % unicode breaks when used with xetex
              xetex]{hyperref}
\else
  \usepackage[unicode=true]{hyperref}
\fi
\hypersetup{breaklinks=true,
            bookmarks=true,
            pdfauthor={Justin Le},
            pdftitle={Enhancing Functor Structures Step-By-Step (Part 1)},
            colorlinks=true,
            citecolor=blue,
            urlcolor=blue,
            linkcolor=magenta,
            pdfborder={0 0 0}}
\urlstyle{same}  % don't use monospace font for urls
% Make links footnotes instead of hotlinks:
\renewcommand{\href}[2]{#2\footnote{\url{#1}}}
\setlength{\parindent}{0pt}
\setlength{\parskip}{6pt plus 2pt minus 1pt}
\setlength{\emergencystretch}{3em}  % prevent overfull lines
\setcounter{secnumdepth}{0}

\title{Enhancing Functor Structures Step-By-Step (Part 1)}
\author{Justin Le}

\begin{document}
\maketitle

\emph{Originally posted on
\textbf{\href{https://blog.jle.im/entry/enhancing-functor-structures-step-by-step-1.html}{in
Code}}.}

A style of Haskell programming that I've been pretty excited about with over the
past year or so is something that I can maybe call a ``functor structure''
designed pattern. This is the interest that culminated in my
\href{https://blog.jle.im/entry/functor-combinatorpedia.html}{Functor
Combinatorpedia} post last year and the
\href{https://hackage.haskell.org/package/functor-combinators}{functor-combinators}
library. In the blog post I called this style the ``functor combinator'' style
because it involved building these functor structures out of small, simple
pieces. This is just one way to build functor structures --- you could always
just make them directly from scratch, but they can get rather messy and it's
often times easier to use pre-made structures that have all the helper functions
already defined for you. But I never really talked about \emph{how} you would
build them from scratch, and also never really explored the more exotic types of
lowercase-f functors in Hask --- contravariant functors and invariant functors.

In this post we're going to be exploring these different types of functor
structures step-by-step, by starting with a simple useful structure and
enhancing it piece by piece. This process reflects a lot of the way I personally
work through these things --- I normally don't get the whole powerful structure
all the way; instead I incrementally add things as I see how things fit
together.

We're going build the tools to describe a \emph{json schema}, in the form of an
algebraic data type -- sums and products. We'll start off just building things
we can use to \emph{describe} the schema (by printing out documentation), and by
the end of the journey we'll also be able to parse and generate values with our
schema.

This series is designed for an intermediate Haskeller with familiarity in things
like monadic parser combinators.

\hypertarget{the-schema}{%
\section{The Schema}\label{the-schema}}

Let's start with the simplest level of describing our schema: a plain ol' AST
describing the possibilities our schema can take.

\begin{Shaded}
\begin{Highlighting}[]
\CommentTok{{-}{-} source: https://github.com/mstksg/inCode/tree/master/code{-}samples/functor{-}structures/doc.hs\#L10{-}L32}

\KeywordTok{data} \DataTypeTok{Schema} \OtherTok{=}
      \DataTypeTok{RecordType}\NormalTok{  [}\DataTypeTok{Field}\NormalTok{]}
    \OperatorTok{|} \DataTypeTok{SumType}\NormalTok{     [}\DataTypeTok{Choice}\NormalTok{]}
    \OperatorTok{|} \DataTypeTok{SchemaLeaf}  \DataTypeTok{Primitive}
  \KeywordTok{deriving} \DataTypeTok{Show}

\KeywordTok{data} \DataTypeTok{Choice} \OtherTok{=} \DataTypeTok{Choice}
\NormalTok{    \{}\OtherTok{ choiceName  ::} \DataTypeTok{String}
\NormalTok{    ,}\OtherTok{ choiceValue ::} \DataTypeTok{Schema}
\NormalTok{    \}}
  \KeywordTok{deriving} \DataTypeTok{Show}

\KeywordTok{data} \DataTypeTok{Field} \OtherTok{=} \DataTypeTok{Field}
\NormalTok{    \{}\OtherTok{ fieldName  ::} \DataTypeTok{String}
\NormalTok{    ,}\OtherTok{ fieldValue ::} \DataTypeTok{Schema}
\NormalTok{    \}}
  \KeywordTok{deriving} \DataTypeTok{Show}

\KeywordTok{data} \DataTypeTok{Primitive} \OtherTok{=}
      \DataTypeTok{PString}
    \OperatorTok{|} \DataTypeTok{PNumber}
    \OperatorTok{|} \DataTypeTok{PBool}
  \KeywordTok{deriving} \DataTypeTok{Show}
\end{Highlighting}
\end{Shaded}

Our schema will either represent a record of many different fields, a sum of
many different options, or a primitive value. If it's a sum type, it'll be
described by a list of \texttt{Choice}, which describes each branch. If it's a
record type, it'll be described by a list of \texttt{Field}, which describes
each field. If it's a primitive type, it'll a \texttt{Primitive}, which is
either a string, number, or boolean.

Our end goal is to be able to write a schema for a type like

\begin{Shaded}
\begin{Highlighting}[]
\CommentTok{{-}{-} source: https://github.com/mstksg/inCode/tree/master/code{-}samples/functor{-}structures/doc.hs\#L34{-}L37}

\KeywordTok{data} \DataTypeTok{Customer} \OtherTok{=}
      \DataTypeTok{CPerson}\NormalTok{   \{}\OtherTok{ cpName ::} \DataTypeTok{String}\NormalTok{,}\OtherTok{ cpAge ::} \DataTypeTok{Int}\NormalTok{ \}}
    \OperatorTok{|} \DataTypeTok{CBusiness}\NormalTok{ \{}\OtherTok{ cbEmployees ::} \DataTypeTok{Int}\NormalTok{ \}}
  \KeywordTok{deriving} \DataTypeTok{Show}
\end{Highlighting}
\end{Shaded}

and be able to represent documenting, parsing, and printing it all within
\texttt{Schema}. For our basic \texttt{Schema} above, this looks like:

\begin{Shaded}
\begin{Highlighting}[]
\CommentTok{{-}{-} source: https://github.com/mstksg/inCode/tree/master/code{-}samples/functor{-}structures/doc.hs\#L39{-}L53}

\OtherTok{customerSchema ::} \DataTypeTok{Schema}
\NormalTok{customerSchema }\OtherTok{=} \DataTypeTok{SumType}
\NormalTok{    [ }\DataTypeTok{Choice}
\NormalTok{        \{ choiceName  }\OtherTok{=} \StringTok{"Person"}
\NormalTok{        , choiceValue }\OtherTok{=} \DataTypeTok{RecordType}
\NormalTok{            [ }\DataTypeTok{Field}\NormalTok{ \{ fieldName }\OtherTok{=} \StringTok{"Name"}\NormalTok{, fieldValue }\OtherTok{=} \DataTypeTok{SchemaLeaf} \DataTypeTok{PString}\NormalTok{ \}}
\NormalTok{            , }\DataTypeTok{Field}\NormalTok{ \{ fieldName }\OtherTok{=} \StringTok{"Age"}\NormalTok{ , fieldValue }\OtherTok{=} \DataTypeTok{SchemaLeaf} \DataTypeTok{PNumber}\NormalTok{ \}}
\NormalTok{            ]}
\NormalTok{        \}}
\NormalTok{    , }\DataTypeTok{Choice}
\NormalTok{        \{ choiceName  }\OtherTok{=} \StringTok{"Business"}
\NormalTok{        , choiceValue }\OtherTok{=} \DataTypeTok{RecordType}
\NormalTok{            [ }\DataTypeTok{Field}\NormalTok{ \{ fieldName }\OtherTok{=} \StringTok{"Employees"}\NormalTok{, fieldValue }\OtherTok{=} \DataTypeTok{SchemaLeaf} \DataTypeTok{PNumber}\NormalTok{ \} ]}
\NormalTok{        \}}
\NormalTok{    ]}
\end{Highlighting}
\end{Shaded}

And a value like

\begin{Shaded}
\begin{Highlighting}[]
\DataTypeTok{PCustomer}\NormalTok{ \{ cpName }\OtherTok{=} \StringTok{"Sam"}\NormalTok{, cpAge }\OtherTok{=} \DecValTok{40}\NormalTok{ \}}
\end{Highlighting}
\end{Shaded}

might be represented by a json value like

\begin{Shaded}
\begin{Highlighting}[]
\FunctionTok{\{} \DataTypeTok{"tag"}\FunctionTok{:} \StringTok{"Customer"}\FunctionTok{,} 
  \DataTypeTok{"contents"}\FunctionTok{:}
    \FunctionTok{\{} \DataTypeTok{"Name"}\FunctionTok{:} \StringTok{"Sam"}
    \FunctionTok{,} \DataTypeTok{"Age"}\FunctionTok{:} \FloatTok{40.0}
    \FunctionTok{\}}
\FunctionTok{\}}
\end{Highlighting}
\end{Shaded}

\hypertarget{documentation}{%
\section{Documentation}\label{documentation}}

Using our schema type, let's make a documentation generator. It'll take a
\texttt{Schema} and nicely formatted documentation describing what the schema
itself.

To make our lives easier, we'll be using the
\emph{\href{https://hackage.haskell.org/package/prettyprinter}{prettyprinter}}
library, which will handle indentation, horizontal and vertical concatenation,
and other printing concerns for us.

The structure of this will match the structure of our future schema processors
--- you'll find they all look like this in some form or another!

\begin{Shaded}
\begin{Highlighting}[]
\CommentTok{{-}{-} source: https://github.com/mstksg/inCode/tree/master/code{-}samples/functor{-}structures/doc.hs\#L55{-}L62}

\NormalTok{schemaDoc}
\OtherTok{    ::} \DataTypeTok{String}       \CommentTok{{-}{-} \^{} name}
    \OtherTok{{-}>} \DataTypeTok{Schema}       \CommentTok{{-}{-} \^{} schema}
    \OtherTok{{-}>} \DataTypeTok{PP.Doc}\NormalTok{ x}
\NormalTok{schemaDoc title }\OtherTok{=}\NormalTok{ \textbackslash{}}\KeywordTok{case}
    \DataTypeTok{RecordType}\NormalTok{ fs }\OtherTok{{-}>}\NormalTok{ recordDoc title fs}
    \DataTypeTok{SumType}\NormalTok{ cs    }\OtherTok{{-}>}\NormalTok{ sumDoc title cs}
    \DataTypeTok{SchemaLeaf}\NormalTok{ p  }\OtherTok{{-}>}\NormalTok{ leafDoc title p}
\end{Highlighting}
\end{Shaded}

So \texttt{schemaDoc} will take the name of our type and a schema, and generate
a \texttt{PP.Doc\ x}, the type of a text document in the \emph{prettyprinter}
library.\footnote{I'm using \texttt{x} as the name of the type variable (instead
  of something more traditional like \texttt{a}) to indicate that it isn't meant
  to be referenced or used anywhere in any consistent way. Just remember it
  doesn't mean anything special syntactically!} Nothing too fancy here other
than simply matching on what type of schema we're dealing with, and dispatching
a different function for each type.

\begin{Shaded}
\begin{Highlighting}[]
\CommentTok{{-}{-} source: https://github.com/mstksg/inCode/tree/master/code{-}samples/functor{-}structures/doc.hs\#L64{-}L72}

\OtherTok{recordDoc ::} \DataTypeTok{String} \OtherTok{{-}>}\NormalTok{ [}\DataTypeTok{Field}\NormalTok{] }\OtherTok{{-}>} \DataTypeTok{PP.Doc}\NormalTok{ x}
\NormalTok{recordDoc title fs }\OtherTok{=}\NormalTok{ PP.vsep [}
\NormalTok{      PP.pretty (}\StringTok{"\{"} \OperatorTok{<>}\NormalTok{ title }\OperatorTok{<>} \StringTok{"\}"}\NormalTok{)}
\NormalTok{    , PP.indent }\DecValTok{2} \OperatorTok{.}\NormalTok{ PP.vsep }\OperatorTok{$}
        \FunctionTok{map}\NormalTok{ (\textbackslash{}fld }\OtherTok{{-}>} \StringTok{"*"} \OperatorTok{PP.<+>}\NormalTok{ PP.indent }\DecValTok{2}\NormalTok{ (fieldDoc fld)) fs}
\NormalTok{    ]}
  \KeywordTok{where}
\OtherTok{    fieldDoc ::} \DataTypeTok{Field} \OtherTok{{-}>} \DataTypeTok{PP.Doc}\NormalTok{ x}
\NormalTok{    fieldDoc }\DataTypeTok{Field}\NormalTok{\{}\OperatorTok{..}\NormalTok{\} }\OtherTok{=}\NormalTok{ schemaDoc fieldName fieldValue}
\end{Highlighting}
\end{Shaded}

First, the \texttt{RecordType} handler uses \texttt{PP.vsep}, which takes a list
of docs and concatenates them together vertically. The first doc is the title
around curly braces, and the second doc is a \texttt{*}'d list of
\texttt{fieldDoc} for each \texttt{fs}, concatenated vertically and indented two
spaces.

\texttt{fieldDoc} itself is then going to recursively call \texttt{schemaDoc}
with the field name as the title and the field value as the sub-schema.

\begin{Shaded}
\begin{Highlighting}[]
\CommentTok{{-}{-} source: https://github.com/mstksg/inCode/tree/master/code{-}samples/functor{-}structures/doc.hs\#L74{-}L83}

\OtherTok{sumDoc ::} \DataTypeTok{String} \OtherTok{{-}>}\NormalTok{ [}\DataTypeTok{Choice}\NormalTok{] }\OtherTok{{-}>} \DataTypeTok{PP.Doc}\NormalTok{ x}
\NormalTok{sumDoc title cs }\OtherTok{=}\NormalTok{ PP.vsep [}
\NormalTok{      PP.pretty (}\StringTok{"("} \OperatorTok{<>}\NormalTok{ title }\OperatorTok{<>} \StringTok{")"}\NormalTok{)}
\NormalTok{    , }\StringTok{"Choice of:"}
\NormalTok{    , PP.indent }\DecValTok{2} \OperatorTok{.}\NormalTok{ PP.vsep }\OperatorTok{$}
        \FunctionTok{map}\NormalTok{ choiceDoc cs}
\NormalTok{    ]}
  \KeywordTok{where}
\OtherTok{    choiceDoc ::} \DataTypeTok{Choice} \OtherTok{{-}>} \DataTypeTok{PP.Doc}\NormalTok{ x}
\NormalTok{    choiceDoc }\DataTypeTok{Choice}\NormalTok{\{}\OperatorTok{..}\NormalTok{\} }\OtherTok{=}\NormalTok{ schemaDoc choiceName choiceValue}
\end{Highlighting}
\end{Shaded}

The \texttt{SumType} handler is going to be doing more or less the same thing,
except surrounding the title in \texttt{(} \texttt{)} parentheses. There's
another line \texttt{Choice\ of:}, followed by the list of each choice,
vertically concatenated and indented twice.

\begin{Shaded}
\begin{Highlighting}[]
\CommentTok{{-}{-} source: https://github.com/mstksg/inCode/tree/master/code{-}samples/functor{-}structures/doc.hs\#L85{-}L93}

\OtherTok{leafDoc ::} \DataTypeTok{String} \OtherTok{{-}>} \DataTypeTok{Primitive} \OtherTok{{-}>} \DataTypeTok{PP.Doc}\NormalTok{ x}
\NormalTok{leafDoc title p }\OtherTok{=}\NormalTok{ PP.pretty (title }\OperatorTok{<>} \StringTok{":"}\NormalTok{)}
           \OperatorTok{PP.<+>}\NormalTok{ primDoc p}
  \KeywordTok{where}
\OtherTok{    primDoc ::} \DataTypeTok{Primitive} \OtherTok{{-}>} \DataTypeTok{PP.Doc}\NormalTok{ x}
\NormalTok{    primDoc }\OtherTok{=}\NormalTok{ \textbackslash{}}\KeywordTok{case}
      \DataTypeTok{PString} \OtherTok{{-}>} \StringTok{"string"}
      \DataTypeTok{PNumber} \OtherTok{{-}>} \StringTok{"number"}
      \DataTypeTok{PBool}   \OtherTok{{-}>} \StringTok{"bool"}
\end{Highlighting}
\end{Shaded}

Finally, the \texttt{leafDoc} handler prints out the schema of a single list: it
gives the name of the leaf and then either \texttt{"string"}, \texttt{"number"},
or \texttt{"bool"}.

Hopefully that wasn't too bad! There were a lot of moving parts because we have
a recursive data type, but in the end hopefully each specific branch was
self-contained enough to understand on their own. We can test it out on
\texttt{customerSchema}, taking advantage of the fact that \texttt{PP.Doc}'s
\texttt{Show} instance will render the document:

\begin{verbatim}
ghci> schemaDoc "Customer" customerSchema
(Customer)
Choice of:
  {Person}
    *   Name: string
    *   Age: number
  {Business}
    *   Employees: number
\end{verbatim}

It works!

\hypertarget{parsing-with-covariance}{%
\section{Parsing with Covariance}\label{parsing-with-covariance}}

Now, let's talk about using our \texttt{Schema} type to define a json parser.
We're going to rewrite \texttt{Schema} to take a type parameter to represent the
type we want to parse into. A \texttt{Schema\ a} will be a schema that can be
used to generate documentation \emph{and} describe a parser of \texttt{a}s. In
the end, we want \texttt{customerSchema\ ::\ Schema\ Customer}, and a function
like

\begin{Shaded}
\begin{Highlighting}[]
\OtherTok{schemaParser ::} \DataTypeTok{Schema}\NormalTok{ a }\OtherTok{{-}>} \DataTypeTok{Parse} \DataTypeTok{ErrType}\NormalTok{ a}
\end{Highlighting}
\end{Shaded}

to generate a json parser of \texttt{a}s. We'll be using the json parser type
\texttt{Parse\ err\ a} from
\emph{\href{https://hackage.haskell.org/package/aeson-better-errors}{aeson-better-errors}}
(not because of the better errors, but just because it's closer to an actual
incremental/stateful parser than other alternatives out there), which can be run
with
\texttt{parse\ ::\ Parse\ err\ a\ -\textgreater{}\ ByteString\ -\textgreater{}\ Either\ (ParseError\ err)\ a}.
So our final interface will look like:

\begin{Shaded}
\begin{Highlighting}[]
\CommentTok{{-}{-} source: https://github.com/mstksg/inCode/tree/master/code{-}samples/functor{-}structures/parse.hs\#L142{-}L143}

\OtherTok{parseSchema ::} \DataTypeTok{Schema}\NormalTok{ a }\OtherTok{{-}>} \DataTypeTok{ByteString} \OtherTok{{-}>} \DataTypeTok{Either}\NormalTok{ (}\DataTypeTok{A.ParseError} \DataTypeTok{ErrType}\NormalTok{) a}
\NormalTok{parseSchema sc }\OtherTok{=}\NormalTok{ A.parse (schemaParser sc)}
\end{Highlighting}
\end{Shaded}

To do this, we now have to include information on ``how to parse an \texttt{a}''
in our schema. Remember that our actual json format is pretty fixed/structured,
so we don't need to address things like how to encode sum types within our
\texttt{Schema} type itself. The only variation that our \texttt{Schema} can
express can be isolated to one place: the \texttt{Primitive} type.

\begin{Shaded}
\begin{Highlighting}[]
\CommentTok{{-}{-} source: https://github.com/mstksg/inCode/tree/master/code{-}samples/functor{-}structures/parse.hs\#L43{-}L56}

\KeywordTok{data} \DataTypeTok{Primitive}\NormalTok{ a }\OtherTok{=}
      \DataTypeTok{PString}\NormalTok{ (}\DataTypeTok{String} \OtherTok{{-}>} \DataTypeTok{Maybe}\NormalTok{ a)}
    \OperatorTok{|} \DataTypeTok{PNumber}\NormalTok{ (}\DataTypeTok{Scientific} \OtherTok{{-}>} \DataTypeTok{Maybe}\NormalTok{ a)}
    \OperatorTok{|} \DataTypeTok{PBool}\NormalTok{   (}\DataTypeTok{Bool} \OtherTok{{-}>} \DataTypeTok{Maybe}\NormalTok{ a)}
  \KeywordTok{deriving} \DataTypeTok{Functor}

\OtherTok{pString ::} \DataTypeTok{Primitive} \DataTypeTok{String}
\NormalTok{pString }\OtherTok{=} \DataTypeTok{PString} \DataTypeTok{Just}

\OtherTok{pInt ::} \DataTypeTok{Primitive} \DataTypeTok{Int}
\NormalTok{pInt }\OtherTok{=} \DataTypeTok{PNumber}\NormalTok{ toBoundedInteger}

\OtherTok{pBool ::} \DataTypeTok{Primitive} \DataTypeTok{Bool}
\NormalTok{pBool }\OtherTok{=} \DataTypeTok{PBool} \DataTypeTok{Just}
\end{Highlighting}
\end{Shaded}

And that's it! Different data types will parse these primitive values in
different ways, but other than that, everything is sort of already
pre-determined. Remember, our \texttt{Schema} type is only meant to express the
different ways our rigidly defined structure can vary.

\hypertarget{finding-ap}{%
\subsection{Finding Ap}\label{finding-ap}}

However, this small change (and adding the type parameter) leaves in a
predicament. What should \texttt{Schema} look like?

At first glance, we might think we could just write

\begin{Shaded}
\begin{Highlighting}[]
\KeywordTok{data} \DataTypeTok{Schema}\NormalTok{ a }\OtherTok{=}
      \DataTypeTok{RecordType}\NormalTok{  [}\DataTypeTok{Field}\NormalTok{ a]}
    \OperatorTok{|} \DataTypeTok{SumType}\NormalTok{     [}\DataTypeTok{Choice}\NormalTok{ a]}
    \OperatorTok{|} \DataTypeTok{SchemaLeaf}\NormalTok{  (}\DataTypeTok{Primitive}\NormalTok{ a)}
  \KeywordTok{deriving} \DataTypeTok{Functor}

\CommentTok{{-}{-} source: https://github.com/mstksg/inCode/tree/master/code{-}samples/functor{-}structures/parse.hs\#L31{-}L41}

\KeywordTok{data} \DataTypeTok{Field}\NormalTok{ a }\OtherTok{=} \DataTypeTok{Field}
\NormalTok{    \{}\OtherTok{ fieldName  ::} \DataTypeTok{String}
\NormalTok{    ,}\OtherTok{ fieldValue ::} \DataTypeTok{Schema}\NormalTok{ a}
\NormalTok{    \}}
  \KeywordTok{deriving} \DataTypeTok{Functor}

\KeywordTok{data} \DataTypeTok{Choice}\NormalTok{ a }\OtherTok{=} \DataTypeTok{Choice}
\NormalTok{    \{}\OtherTok{ choiceName  ::} \DataTypeTok{String}
\NormalTok{    ,}\OtherTok{ choiceValue ::} \DataTypeTok{Schema}\NormalTok{ a}
\NormalTok{    \}}
  \KeywordTok{deriving} \DataTypeTok{Functor}
\end{Highlighting}
\end{Shaded}

But there's a problem here: \texttt{RecordType} is a combination of
\texttt{Field}s, but\ldots each \texttt{Field} is of a different type! For
example, in our \texttt{Customer} example, the \texttt{Person} branch has two
fields: \texttt{Name} and \texttt{Age}. Our name schema would look like
\texttt{nameField\ ::\ Field\ String}, and our age schema would look like
\texttt{ageField\ ::\ Field\ Int}\ldots and so you can't really put that into a
list like \texttt{{[}Field\ a{]}} since they each have different types. And
further more, we want a final \texttt{Customer} (in our
\texttt{Schema\ Customer}), a type which is different from both \texttt{String}
and \texttt{Int}.

What we need is a way to express heterogeneous collection/sequence of
\texttt{Field\ a}, coupled with a way of ``combining'' all of them to create an
aggregate value of a final type. A type that says ``use a bunch of
\texttt{Field} of \texttt{x}s of different types to generate a final
\texttt{a}''.

There are a couple of ways to arrive at this mystery type. One way is to
recognize ``combine a bunch of \texttt{f\ x}s of different types to create an
\texttt{f\ b}'' is essentially the MO of the \emph{Applicative} abstraction, and
so essentially we want to give \texttt{Field} some sort of \texttt{Applicative}
structure. And so we can reach for ``the type that gives something an
\texttt{Applicative} structure'', the
\href{https://hackage.haskell.org/package/free/docs/Control-Applicative-Free.html}{free
applicative}.

Another way is to think about it as an enhancement along a functor combinator
described in the
\href{https://blog.jle.im/entry/functor-combinatorpedia.html}{functor
combinatorpedia}. Here we know we want to enhance \texttt{Field} in a specific
way, so we can scan the list of functor combinators until there is one that we
need. And scrolling down, we see:

\begin{quote}
\href{https://blog.jle.im/entry/functor-combinatorpedia.html\#ap-ap1}{\textbf{Ap
/ Ap1 }}

\textbf{Origin}:
\emph{\href{https://hackage.haskell.org/package/free/docs/Control-Applicative-Free.html}{Control.Applicative.Free}}
/
\emph{\href{https://hackage.haskell.org/package/functor-combinators/docs/Data-Functor-Apply-Free.html}{Data.Functor.Apply.Free}}

\textbf{Enhancement}: The ability to provide multiple \texttt{f}s that the
interpreter \emph{must} consume \emph{all} of. (\ldots)

While \texttt{ListF} may be considered ``multiple options \emph{offered}'',
\texttt{Ap} can be considered ``multiple actions all \emph{required}''. The
interpreter must consume/interpret \emph{all} of the multiple \texttt{f}s in
order to interpret an \texttt{Ap}.

Note that ordering is not enforced: while the consumer must handle each
\texttt{f} eventually, they are free to handle it in whatever order they desire.
In fact, they could even all be handled in parallel. See \texttt{Free} for a
version where ordering is enforced.

\ldots{}

Because this has an \texttt{Applicative} instance, you can use
\texttt{(\textless{}*\textgreater{})\ ::\ Ap\ f\ (a\ -\textgreater{}\ b)\ -\textgreater{}\ Ap\ f\ a\ -\textgreater{}\ Ap\ f\ b}
to sequence multiple \texttt{Ap\ f}s together, and
\texttt{pure\ ::\ a\ -\textgreater{}\ Ap\ f\ a} to produce a ``no-op''
\texttt{Ap} without any \texttt{f}s.
\end{quote}

That sounds like it matches to me! In order to parse a \texttt{RecordType}, we
need to parse \emph{every} \texttt{Field}. It doesn't make any sense to skip one
field or the other: they all need to be processed and parsed. This sounds like
just the thing we need.

The description here also gives a clue for what we might want to use for
\texttt{SumType} (\texttt{ListF} sounds like a good companion for the behavior
we want sum type parsers to have)

Another way to come to this conclusion is to think about it in terms of
tensoring functors: ``how do we want to tensor together \emph{two}
\texttt{Field}s''?

Well, earlier we said that we want to combine fields of different types, while
also providing a way of combining the two types together to create a final
aggregate type. We can express this in Haskell with something like:

\begin{Shaded}
\begin{Highlighting}[]
\KeywordTok{data} \DataTypeTok{TwoFields} \OtherTok{=} \KeywordTok{forall}\NormalTok{ x y}\OperatorTok{.} \DataTypeTok{TwoFields}\NormalTok{ (}\DataTypeTok{Field}\NormalTok{ x) (}\DataTypeTok{Field}\NormalTok{ y) (x }\OtherTok{{-}>}\NormalTok{ y }\OtherTok{{-}>}\NormalTok{ a)}
\end{Highlighting}
\end{Shaded}

In existential syntax, this says that \texttt{TwoFields\ a} consists of
\texttt{Field\ x} and \texttt{Field\ y} of a ``hidden'' \texttt{x} and
\texttt{y}, as well as a function to combine the \texttt{x} and \texttt{y} to
make an \texttt{a}.

We can look up what type of tensor this is in the functor combinatorpedia.
Scrolling down, we see:

\begin{quote}
\href{https://blog.jle.im/entry/functor-combinatorpedia.html\#day}{\textbf{Day}}

\textbf{Origin}:
\emph{\href{https://hackage.haskell.org/package/kan-extensions/docs/Data-Functor-Day.html}{Data.Functor.Day}}

\textbf{Mixing Strategy}: ``Both, together forever'': provide values from
\emph{both} functors, and the user \emph{must} also \emph{use} both.

\ldots{}

Unlike for \texttt{:*:}, you always have to interpret \emph{both} functor values
in order to interpret a \texttt{Day}. It's a ``full mixing''.

The mechanism for this is interesting in and of itself. Looking at the
definition of the data type:

\begin{Shaded}
\begin{Highlighting}[]
\KeywordTok{data} \DataTypeTok{Day}\NormalTok{ f g a }\OtherTok{=} \KeywordTok{forall}\NormalTok{ x y}\OperatorTok{.} \DataTypeTok{Day}\NormalTok{ (f x) (g y) (x }\OtherTok{{-}>}\NormalTok{ y }\OtherTok{{-}>}\NormalTok{ a)}
\end{Highlighting}
\end{Shaded}

We see that because \texttt{x} and \texttt{y} are ``hidden'' from the external
world, we can't directly use them without applying the ``joining'' function
\texttt{x\ -\textgreater{}\ y\ -\textgreater{}\ a}. Due to how existential types
work, we can't get anything out of it that ``contains'' \texttt{x} or
\texttt{y}. Because of this, \emph{using} the joining function requires
\emph{both} \texttt{f\ x} and \texttt{g\ y}. If we only use \texttt{f\ x}, we
can only get, at best,\texttt{f\ (y\ -\textgreater{}\ a)}; if we only use
\texttt{g\ y}, we can only get, at best, \texttt{g\ (x\ -\textgreater{}\ a)}. In
order to fully eliminate \emph{both} existential variables, we need to get the
\texttt{x} and \texttt{y} from \emph{both} \texttt{f\ x} and \texttt{g\ y}, as
if the two values held separate halves of the key.
\end{quote}

It seems as if our \texttt{TwoFields} is exactly \texttt{Day\ Field\ Field}, so
we're on the right track.

Reading further on in the \texttt{Day} section, we see:

\begin{quote}
\textbf{List type}: (\ldots) \texttt{Ap\ f\ a} is a bunch of \texttt{f\ x}s
\texttt{Day}d with each other.
\end{quote}

All three different ways you might have arrived at the conclusion of using
\texttt{Ap}!

\hypertarget{using-ap}{%
\subsection{Using Ap}\label{using-ap}}

With this let's write our final \texttt{Schema} type:

\begin{Shaded}
\begin{Highlighting}[]
\CommentTok{{-}{-} source: https://github.com/mstksg/inCode/tree/master/code{-}samples/functor{-}structures/parse.hs\#L25{-}L47}

\KeywordTok{data} \DataTypeTok{Schema}\NormalTok{ a }\OtherTok{=}
      \DataTypeTok{RecordType}\NormalTok{  (}\DataTypeTok{Ap} \DataTypeTok{Field}\NormalTok{ a)}
    \OperatorTok{|} \DataTypeTok{SumType}\NormalTok{     (}\DataTypeTok{ListF} \DataTypeTok{Choice}\NormalTok{ a)}
    \OperatorTok{|} \DataTypeTok{SchemaLeaf}\NormalTok{  (}\DataTypeTok{Primitive}\NormalTok{ a)}
  \KeywordTok{deriving} \DataTypeTok{Functor}

\KeywordTok{data} \DataTypeTok{Choice}\NormalTok{ a }\OtherTok{=} \DataTypeTok{Choice}
\NormalTok{    \{}\OtherTok{ choiceName  ::} \DataTypeTok{String}
\NormalTok{    ,}\OtherTok{ choiceValue ::} \DataTypeTok{Schema}\NormalTok{ a}
\NormalTok{    \}}
  \KeywordTok{deriving} \DataTypeTok{Functor}

\KeywordTok{data} \DataTypeTok{Field}\NormalTok{ a }\OtherTok{=} \DataTypeTok{Field}
\NormalTok{    \{}\OtherTok{ fieldName  ::} \DataTypeTok{String}
\NormalTok{    ,}\OtherTok{ fieldValue ::} \DataTypeTok{Schema}\NormalTok{ a}
\NormalTok{    \}}
  \KeywordTok{deriving} \DataTypeTok{Functor}

\KeywordTok{data} \DataTypeTok{Primitive}\NormalTok{ a }\OtherTok{=}
      \DataTypeTok{PString}\NormalTok{ (}\DataTypeTok{String} \OtherTok{{-}>} \DataTypeTok{Maybe}\NormalTok{ a)}
    \OperatorTok{|} \DataTypeTok{PNumber}\NormalTok{ (}\DataTypeTok{Scientific} \OtherTok{{-}>} \DataTypeTok{Maybe}\NormalTok{ a)}
    \OperatorTok{|} \DataTypeTok{PBool}\NormalTok{   (}\DataTypeTok{Bool} \OtherTok{{-}>} \DataTypeTok{Maybe}\NormalTok{ a)}
  \KeywordTok{deriving} \DataTypeTok{Functor}
\end{Highlighting}
\end{Shaded}

\hypertarget{signoff}{%
\section{Signoff}\label{signoff}}

Hi, thanks for reading! You can reach me via email at
\href{mailto:justin@jle.im}{\nolinkurl{justin@jle.im}}, or at twitter at
\href{https://twitter.com/mstk}{@mstk}! This post and all others are published
under the \href{https://creativecommons.org/licenses/by-nc-nd/3.0/}{CC-BY-NC-ND
3.0} license. Corrections and edits via pull request are welcome and encouraged
at \href{https://github.com/mstksg/inCode}{the source repository}.

If you feel inclined, or this post was particularly helpful for you, why not
consider \href{https://www.patreon.com/justinle/overview}{supporting me on
Patreon}, or a \href{bitcoin:3D7rmAYgbDnp4gp4rf22THsGt74fNucPDU}{BTC donation}?
:)

\end{document}
