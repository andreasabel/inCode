\documentclass[]{article}
\usepackage{lmodern}
\usepackage{amssymb,amsmath}
\usepackage{ifxetex,ifluatex}
\usepackage{fixltx2e} % provides \textsubscript
\ifnum 0\ifxetex 1\fi\ifluatex 1\fi=0 % if pdftex
  \usepackage[T1]{fontenc}
  \usepackage[utf8]{inputenc}
\else % if luatex or xelatex
  \ifxetex
    \usepackage{mathspec}
    \usepackage{xltxtra,xunicode}
  \else
    \usepackage{fontspec}
  \fi
  \defaultfontfeatures{Mapping=tex-text,Scale=MatchLowercase}
  \newcommand{\euro}{€}
\fi
% use upquote if available, for straight quotes in verbatim environments
\IfFileExists{upquote.sty}{\usepackage{upquote}}{}
% use microtype if available
\IfFileExists{microtype.sty}{\usepackage{microtype}}{}
\usepackage[margin=1in]{geometry}
\usepackage{color}
\usepackage{fancyvrb}
\newcommand{\VerbBar}{|}
\newcommand{\VERB}{\Verb[commandchars=\\\{\}]}
\DefineVerbatimEnvironment{Highlighting}{Verbatim}{commandchars=\\\{\}}
% Add ',fontsize=\small' for more characters per line
\newenvironment{Shaded}{}{}
\newcommand{\KeywordTok}[1]{\textcolor[rgb]{0.00,0.44,0.13}{\textbf{#1}}}
\newcommand{\DataTypeTok}[1]{\textcolor[rgb]{0.56,0.13,0.00}{#1}}
\newcommand{\DecValTok}[1]{\textcolor[rgb]{0.25,0.63,0.44}{#1}}
\newcommand{\BaseNTok}[1]{\textcolor[rgb]{0.25,0.63,0.44}{#1}}
\newcommand{\FloatTok}[1]{\textcolor[rgb]{0.25,0.63,0.44}{#1}}
\newcommand{\ConstantTok}[1]{\textcolor[rgb]{0.53,0.00,0.00}{#1}}
\newcommand{\CharTok}[1]{\textcolor[rgb]{0.25,0.44,0.63}{#1}}
\newcommand{\SpecialCharTok}[1]{\textcolor[rgb]{0.25,0.44,0.63}{#1}}
\newcommand{\StringTok}[1]{\textcolor[rgb]{0.25,0.44,0.63}{#1}}
\newcommand{\VerbatimStringTok}[1]{\textcolor[rgb]{0.25,0.44,0.63}{#1}}
\newcommand{\SpecialStringTok}[1]{\textcolor[rgb]{0.73,0.40,0.53}{#1}}
\newcommand{\ImportTok}[1]{#1}
\newcommand{\CommentTok}[1]{\textcolor[rgb]{0.38,0.63,0.69}{\textit{#1}}}
\newcommand{\DocumentationTok}[1]{\textcolor[rgb]{0.73,0.13,0.13}{\textit{#1}}}
\newcommand{\AnnotationTok}[1]{\textcolor[rgb]{0.38,0.63,0.69}{\textbf{\textit{#1}}}}
\newcommand{\CommentVarTok}[1]{\textcolor[rgb]{0.38,0.63,0.69}{\textbf{\textit{#1}}}}
\newcommand{\OtherTok}[1]{\textcolor[rgb]{0.00,0.44,0.13}{#1}}
\newcommand{\FunctionTok}[1]{\textcolor[rgb]{0.02,0.16,0.49}{#1}}
\newcommand{\VariableTok}[1]{\textcolor[rgb]{0.10,0.09,0.49}{#1}}
\newcommand{\ControlFlowTok}[1]{\textcolor[rgb]{0.00,0.44,0.13}{\textbf{#1}}}
\newcommand{\OperatorTok}[1]{\textcolor[rgb]{0.40,0.40,0.40}{#1}}
\newcommand{\BuiltInTok}[1]{#1}
\newcommand{\ExtensionTok}[1]{#1}
\newcommand{\PreprocessorTok}[1]{\textcolor[rgb]{0.74,0.48,0.00}{#1}}
\newcommand{\AttributeTok}[1]{\textcolor[rgb]{0.49,0.56,0.16}{#1}}
\newcommand{\RegionMarkerTok}[1]{#1}
\newcommand{\InformationTok}[1]{\textcolor[rgb]{0.38,0.63,0.69}{\textbf{\textit{#1}}}}
\newcommand{\WarningTok}[1]{\textcolor[rgb]{0.38,0.63,0.69}{\textbf{\textit{#1}}}}
\newcommand{\AlertTok}[1]{\textcolor[rgb]{1.00,0.00,0.00}{\textbf{#1}}}
\newcommand{\ErrorTok}[1]{\textcolor[rgb]{1.00,0.00,0.00}{\textbf{#1}}}
\newcommand{\NormalTok}[1]{#1}
\ifxetex
  \usepackage[setpagesize=false, % page size defined by xetex
              unicode=false, % unicode breaks when used with xetex
              xetex]{hyperref}
\else
  \usepackage[unicode=true]{hyperref}
\fi
\hypersetup{breaklinks=true,
            bookmarks=true,
            pdfauthor={Justin Le},
            pdftitle={Introduction to Singletons (Part 2)},
            colorlinks=true,
            citecolor=blue,
            urlcolor=blue,
            linkcolor=magenta,
            pdfborder={0 0 0}}
\urlstyle{same}  % don't use monospace font for urls
% Make links footnotes instead of hotlinks:
\renewcommand{\href}[2]{#2\footnote{\url{#1}}}
\setlength{\parindent}{0pt}
\setlength{\parskip}{6pt plus 2pt minus 1pt}
\setlength{\emergencystretch}{3em}  % prevent overfull lines
\setcounter{secnumdepth}{0}

\title{Introduction to Singletons (Part 2)}
\author{Justin Le}

\begin{document}
\maketitle

\emph{Originally posted on
\textbf{\href{https://blog.jle.im/entry/introduction-to-singletons-2.html}{in
Code}}.}

\section{Ditching the Phantom}\label{ditching-the-phantom}

Now, sometimes we don't actually care about the state of the door in our type,
and we don't \emph{want} the state of the door in its type. Our
\texttt{lockAnyDoor} function earlier was an example.

We have a couple of options here --- first, we can create a new type
\texttt{SomeDoor}, that doesn't have the opened/closed status in its type, but
rather as a runtime value:

\begin{Shaded}
\begin{Highlighting}[]
\KeywordTok{data} \DataTypeTok{SomeDoor} \FunctionTok{=} \DataTypeTok{MkSomeDoor}
\NormalTok{    \{}\OtherTok{ someDoorState    ::} \DataTypeTok{DoorState}
\NormalTok{    ,}\OtherTok{ someDoorMaterial ::} \DataTypeTok{String}
\NormalTok{    \}}

\CommentTok{-- or, in GADT syntax}
\KeywordTok{data} \DataTypeTok{SomeDoor}\OtherTok{ ::} \DataTypeTok{Type} \KeywordTok{where}
    \DataTypeTok{MkSomeDoor} \OtherTok{::}
\NormalTok{      \{}\OtherTok{ someDoorState    ::} \DataTypeTok{DoorState}
\NormalTok{      ,}\OtherTok{ someDoorMaterial ::} \DataTypeTok{String}
\NormalTok{      \} }\OtherTok{->} \DataTypeTok{SomeDoor}
\end{Highlighting}
\end{Shaded}

We could have actually been using this type the entire time, if we didn't care
about type safety. In the real world and in real applications, we might have
actually written \texttt{SomeDoor} \emph{before} we ever thought about
\texttt{Door} with a phantom type. It's definitely the more typical ``standard''
Haskell thing.

It's possible to ``construct'' this from our original typed \texttt{Door}, using
a smart constructor/conversion function:

\begin{Shaded}
\begin{Highlighting}[]
\OtherTok{fromDoor ::} \DataTypeTok{SingDS}\NormalTok{ s }\OtherTok{->} \DataTypeTok{Door}\NormalTok{ s }\OtherTok{->} \DataTypeTok{SomeDoor}
\NormalTok{fromDoor }\DataTypeTok{SOpened}\NormalTok{ (}\DataTypeTok{UnsafeMkDoor}\NormalTok{ m) }\FunctionTok{=} \DataTypeTok{MkSomeDoor} \DataTypeTok{Opened}\NormalTok{ m}
\NormalTok{formDoor }\DataTypeTok{SClosed}\NormalTok{ (}\DataTypeTok{UnsafeMkDoor}\NormalTok{ m) }\FunctionTok{=} \DataTypeTok{MkSomeDoor} \DataTypeTok{Closed}\NormalTok{ m}
\NormalTok{formDoor }\DataTypeTok{SLocked}\NormalTok{ (}\DataTypeTok{UnsafeMkDoor}\NormalTok{ m) }\FunctionTok{=} \DataTypeTok{MkSomeDoor} \DataTypeTok{Locked}\NormalTok{ m}
\end{Highlighting}
\end{Shaded}

\subsection{SomeDoor to Door}\label{somedoor-to-door}

Now, \texttt{SomeDoor} is great. But because it's a completely different type,
we potentially have to write the same function for both \texttt{Door} and
\texttt{SomeDoor}, because they have different implementations. For example:

\begin{Shaded}
\begin{Highlighting}[]
\OtherTok{closeSomeOpenedDoor ::} \DataTypeTok{SomeDoor} \OtherTok{->} \DataTypeTok{Maybe} \DataTypeTok{SomeDoor}
\NormalTok{closeSomeOpenedDoor (}\DataTypeTok{MkSomeDoor} \DataTypeTok{Opened}\NormalTok{ m) }\FunctionTok{=} \DataTypeTok{Just}\NormalTok{ (}\DataTypeTok{MkSomeDoor} \DataTypeTok{Closed}\NormalTok{ m)}
\NormalTok{closeSomeOpenedDoor (}\DataTypeTok{MkSomeDoor} \DataTypeTok{Closed}\NormalTok{ m) }\FunctionTok{=} \DataTypeTok{Nothing}
\NormalTok{closeSomeOpenedDoor (}\DataTypeTok{MkSomeDoor} \DataTypeTok{Locked}\NormalTok{ m) }\FunctionTok{=} \DataTypeTok{Nothing}
\end{Highlighting}
\end{Shaded}

Wouldn't it be nice if we can \emph{re-use} our original \texttt{closeDoor}?
This is a toy example, and in real life, closing a door might have some
complicated runtime logic, and it'd be annoying to have to \emph{re-implement}
it for both \texttt{SomeDoor} and \texttt{Door}.

\subsubsection{Converting into an
existential}\label{converting-into-an-existential}

One thing we can do is write a function to convert a \texttt{SomeDoor} into a
\texttt{Door}, so we can re-use our original \texttt{closeDoor}. We'd convert
our \texttt{SomeDoor} into a \texttt{Door} to re-use our
\texttt{closeDoor\ ::\ Door\ \textquotesingle{}Opened\ -\textgreater{}\ Door\ \textquotesingle{}Closed}
on it if possible!

However, going from \texttt{SomeDoor} to \texttt{Door\ s} is slightly trickier
in Haskell than going the other way around. One trick we often use is a
CPS-style existential type.

The essential concept is that normal Haskell type variables are universally
qualified, meaning that the \emph{caller} can pick how to instantiate
\texttt{s}. However, we want a function where the \emph{function} can pick the
\texttt{s}, and the caller must handle whatever \texttt{s} is given by the
function:

\begin{Shaded}
\begin{Highlighting}[]
\OtherTok{withSomeDoor ::} \DataTypeTok{SomeDoor} \OtherTok{->}\NormalTok{ (forall s}\FunctionTok{.} \DataTypeTok{SingDS}\NormalTok{ s }\OtherTok{->} \DataTypeTok{Door}\NormalTok{ s }\OtherTok{->}\NormalTok{ r) }\OtherTok{->}\NormalTok{ r}
\NormalTok{withSomeDoor (}\DataTypeTok{MkSomeDoor} \DataTypeTok{Opened}\NormalTok{ m) f }\FunctionTok{=}\NormalTok{ f }\DataTypeTok{SOpened}\NormalTok{ (}\DataTypeTok{UnsafeMkDoor}\NormalTok{ m)}
\NormalTok{withSomeDoor (}\DataTypeTok{MkSomeDoor} \DataTypeTok{Closed}\NormalTok{ m) f }\FunctionTok{=}\NormalTok{ f }\DataTypeTok{SClosed}\NormalTok{ (}\DataTypeTok{UnsafeMkDoor}\NormalTok{ m)}
\NormalTok{withSomeDoor (}\DataTypeTok{MkSomeDoor} \DataTypeTok{Locked}\NormalTok{ m) f }\FunctionTok{=}\NormalTok{ f }\DataTypeTok{SLocked}\NormalTok{ (}\DataTypeTok{UnsafeMkDoor}\NormalTok{ m)}
\end{Highlighting}
\end{Shaded}

Notice the funky CPS-like type signature of \texttt{withSomeDoor}. To use
\texttt{withSomeDoor} and access the \texttt{Door}, you have to pass in a
function to handle \emph{any possible \texttt{s}}. And, as you can see, the
function passed in might be given an \texttt{SOpened}, an \texttt{SClosed}, or
an \texttt{SLocked}. It has to be able to handle all three!

Here, we call \texttt{s} \emph{existentially quantified}. The
\texttt{withSomeDoor} function gets to pick which \texttt{s} to give \texttt{f}.
So, the \texttt{s} type variable is directly chosen by the \emph{function}, and
not by the caller.

So we can implement \texttt{closeSomeOpenedDoor} (and even a
\texttt{lockAnySomeDoor}) using this conversion function:

\begin{Shaded}
\begin{Highlighting}[]
\OtherTok{closeSomeOpenedDoor ::} \DataTypeTok{SomeDoor} \OtherTok{->} \DataTypeTok{Maybe}\NormalTok{ (}\DataTypeTok{Door} \CharTok{'Closed)}
\NormalTok{closeSomeOpenedDoor sd }\FunctionTok{=}\NormalTok{ withSomeDoor sd }\FunctionTok{$}\NormalTok{ \textbackslash{}}\KeywordTok{case}
    \DataTypeTok{SOpened} \OtherTok{->}\NormalTok{ \textbackslash{}d }\OtherTok{->} \DataTypeTok{Just}\NormalTok{ (closeDoor d)}
    \DataTypeTok{SClosed} \OtherTok{->}\NormalTok{ \textbackslash{}_ }\OtherTok{->} \DataTypeTok{Nothing}
    \DataTypeTok{SLocked} \OtherTok{->}\NormalTok{ \textbackslash{}_ }\OtherTok{->} \DataTypeTok{Nothing}

\OtherTok{lockAnySomeDoor ::} \DataTypeTok{SomeDoor} \OtherTok{->} \DataTypeTok{Door} \CharTok{'Locked}
\NormalTok{lockAnySomeDoor sd }\FunctionTok{=}\NormalTok{ withSomeDoor sd }\FunctionTok{$}\NormalTok{ \textbackslash{}s d }\OtherTok{->}
\NormalTok{    lockAnyDoor s d}
\end{Highlighting}
\end{Shaded}

\subsubsection{The Existential Datatype}\label{the-existential-datatype}

However, there's another path we can take. With the power of singletons, we can
actually implement \texttt{SomeDoor} \emph{in terms of} \texttt{Door}, using an
\textbf{existential data type}:

\begin{Shaded}
\begin{Highlighting}[]
\CommentTok{-- using existential constructor syntax}
\KeywordTok{data} \DataTypeTok{SomeDoor} \FunctionTok{=}\NormalTok{ forall s}\FunctionTok{.} \DataTypeTok{MkSomeDoor}\NormalTok{ (}\DataTypeTok{SingDS}\NormalTok{ s) (}\DataTypeTok{Door}\NormalTok{ s)}

\CommentTok{-- or, using GADT syntax (preferred)}
\CommentTok{-- source: https://github.com/mstksg/inCode/tree/master/code-samples/singletons/Door.hs#L78-82}
\KeywordTok{data} \DataTypeTok{SomeDoor}\OtherTok{ ::} \DataTypeTok{Type} \KeywordTok{where}
    \DataTypeTok{MkSomeDoor} \OtherTok{::}
\NormalTok{      \{}\OtherTok{ someDoorState ::} \DataTypeTok{SingDS}\NormalTok{ s}
\NormalTok{      ,}\OtherTok{ someDoorDoor  ::} \DataTypeTok{Door}\NormalTok{ s}
\NormalTok{      \} }\OtherTok{->} \DataTypeTok{SomeDoor}
\end{Highlighting}
\end{Shaded}

\texttt{MkSomeDoor} is a constructor for an existential data type, meaning that
the data type ``hides'' a type variable \texttt{s}.

Hopefully you can see the similarities between our original \texttt{SomeDoor}
and this one.

\begin{Shaded}
\begin{Highlighting}[]
\CommentTok{-- Original type}
\KeywordTok{data} \DataTypeTok{SomeDoor} \KeywordTok{where}
    \DataTypeTok{MkSomeDoor}\OtherTok{ ::} \DataTypeTok{DoorState} \OtherTok{->} \DataTypeTok{String} \OtherTok{->} \DataTypeTok{SomeDoor}
\CommentTok{-- New existential type}
\KeywordTok{data} \DataTypeTok{SomeDoor} \KeywordTok{where}
    \DataTypeTok{MkSomeDoor}\OtherTok{ ::} \DataTypeTok{SingDS}\NormalTok{ s  }\OtherTok{->} \DataTypeTok{Door}\NormalTok{ s }\OtherTok{->} \DataTypeTok{SomeDoor}
\end{Highlighting}
\end{Shaded}

The key differences are:

\begin{itemize}
\tightlist
\item
  Our first \texttt{SomeDoor} contains a \texttt{DoorState}, and this new
  \texttt{SomeDoor} contains a \texttt{SingDS} (a \emph{singleton} for the
  \texttt{DoorState}):
\item
  Our first \texttt{SomeDoor} contains essentially a re-implementation of the
  \texttt{Door} type, but the new \texttt{SomeDoor} contains an actual
  \texttt{Door}, so we can re-use functions on \texttt{Door}s.
\end{itemize}

In Haskell, existential data types are pretty nice, syntactically, to work with.
For a comparison, let's re-implement our previous functions with our new data
type:

\begin{Shaded}
\begin{Highlighting}[]
\CommentTok{-- source: https://github.com/mstksg/inCode/tree/master/code-samples/singletons/Door.hs#L84-91}
\OtherTok{closeSomeOpenedDoor ::} \DataTypeTok{SomeDoor} \OtherTok{->} \DataTypeTok{Maybe}\NormalTok{ (}\DataTypeTok{Door} \CharTok{'Closed)}
\NormalTok{closeSomeOpenedDoor }\FunctionTok{=}\NormalTok{ \textbackslash{}}\KeywordTok{case}
    \DataTypeTok{MkSomeDoor} \DataTypeTok{SOpened}\NormalTok{ d }\OtherTok{->} \DataTypeTok{Just}\NormalTok{ (closeDoor d)}
    \DataTypeTok{MkSomeDoor} \DataTypeTok{SClosed}\NormalTok{ d }\OtherTok{->} \DataTypeTok{Nothing}
    \DataTypeTok{MkSomeDoor} \DataTypeTok{SLocked}\NormalTok{ _ }\OtherTok{->} \DataTypeTok{Nothing}

\OtherTok{lockAnySomeDoor ::} \DataTypeTok{SomeDoor} \OtherTok{->} \DataTypeTok{Door} \CharTok{'Locked}
\NormalTok{lockAnySomeDoor (}\DataTypeTok{MkSomeDoor}\NormalTok{ s d) }\FunctionTok{=}\NormalTok{ lockAnyDoor s d}
\end{Highlighting}
\end{Shaded}

Much more convenient, because \emph{we already have a \texttt{Door}!} And we
don't have to re-implement one like we did for our original \texttt{SomeDoor} --
all of our original code works directly!

It's important to remember that our original separate-implementation
\texttt{SomeDoor} is, functionally, identical to the new code-reusing
\texttt{Door}. The reason why they are the same is that \emph{having an
existentially quantified singleton is the same as having a value of the
corresponding type.} Having an existentially quantified \texttt{SingDS\ s} is
\emph{the same as} having a value of type \texttt{DoorState}.

If they're identical, why use a \texttt{SingDS} or the new \texttt{SomeDoor} at
all? One main reason (besides allowing code-reuse) is that \emph{using the
singleton lets us recover the type}. Essentially, a \texttt{SingDS\ s} not only
contains whether it is Opened/Closed/Locked\ldots{}it contains it in a way that
GHC can use to \emph{bring it all back} to the type level.

Basically, \texttt{SingDS} allows us to re-use our original \texttt{Door\ s}
implementation, because we store both the \texttt{Door}\ldots{}\emph{and} the
\texttt{s} at the type level. You should read it as storing \texttt{s} and
\texttt{Door\ s}, together, at runtime. It also lets GHC \emph{check} our
implementations, to help ensure that they are correct, because you maintain the
\texttt{s} at the type level.

\subsubsection{Some Lingo}\label{some-lingo}

In the language of dependently typed programming, we call \texttt{SomeDoor} a
\textbf{dependent sum}, because you can imagine it basically as:

\begin{Shaded}
\begin{Highlighting}[]
\KeywordTok{data} \DataTypeTok{SomeDoor} \FunctionTok{=} \DataTypeTok{SDOpened}\NormalTok{ (}\DataTypeTok{Door} \CharTok{'Opened)}
              \FunctionTok{|} \DataTypeTok{SDClosed}\NormalTok{ (}\DataTypeTok{Door} \CharTok{'Closed)}
              \FunctionTok{|} \DataTypeTok{SDLocked}\NormalTok{ (}\DataTypeTok{Door} \CharTok{'Locked)}
\end{Highlighting}
\end{Shaded}

A three-way sum between a \texttt{Door\ \textquotesingle{}Opened}, a
\texttt{Door\ \textquotesingle{}Closed}, and a
\texttt{Door\ \textquotesingle{}Locked}, essentially. If you have a
\texttt{SomeDoor}, it's \emph{either} an opened door, a closed door, or a locked
door. Try looking at this new \texttt{SomeDoor} until you realize that this type
is the same type as the previous \texttt{SomeDoor}!

You might also see \texttt{SomeDoor} called a \textbf{dependent pair}, because
it's basically an existentially quantified tuple of the type (the \texttt{s},
witnessed by the \texttt{SingDS\ s}) with a value (the \texttt{Door\ s}).

\subsection{Types at Runtime}\label{types-at-runtime}

With this last tool, we finally have enough to build a function to ``make'' a
door with the status unknown until runtime:

\begin{Shaded}
\begin{Highlighting}[]
\CommentTok{-- source: https://github.com/mstksg/inCode/tree/master/code-samples/singletons/Door.hs#L93-97}
\OtherTok{mkSomeDoor ::} \DataTypeTok{DoorState} \OtherTok{->} \DataTypeTok{String} \OtherTok{->} \DataTypeTok{SomeDoor}
\NormalTok{mkSomeDoor }\FunctionTok{=}\NormalTok{ \textbackslash{}}\KeywordTok{case}
    \DataTypeTok{Opened} \OtherTok{->} \DataTypeTok{MkSomeDoor} \DataTypeTok{SOpened} \FunctionTok{.}\NormalTok{ mkDoor }\DataTypeTok{SOpened}
    \DataTypeTok{Closed} \OtherTok{->} \DataTypeTok{MkSomeDoor} \DataTypeTok{SClosed} \FunctionTok{.}\NormalTok{ mkDoor }\DataTypeTok{SClosed}
    \DataTypeTok{Locked} \OtherTok{->} \DataTypeTok{MkSomeDoor} \DataTypeTok{SLocked} \FunctionTok{.}\NormalTok{ mkDoor }\DataTypeTok{SLocked}
\end{Highlighting}
\end{Shaded}

\begin{Shaded}
\begin{Highlighting}[]
\NormalTok{ghci}\FunctionTok{>} \KeywordTok{let}\NormalTok{ mySomeDoor }\FunctionTok{=}\NormalTok{ mkSomeDoor }\DataTypeTok{Opened} \StringTok{"Birch"}
\NormalTok{ghci}\FunctionTok{>} \FunctionTok{:}\NormalTok{t mySomeDoor}
\DataTypeTok{SomeDoor}
\NormalTok{ghci}\FunctionTok{>}\NormalTok{ putStrLn }\FunctionTok{$} \KeywordTok{case}\NormalTok{ mySomeDoor }\KeywordTok{of}
        \DataTypeTok{MkSomeDoor} \DataTypeTok{SOpened}\NormalTok{ _ }\OtherTok{->} \StringTok{"mySomeDoor was opened!"}
        \DataTypeTok{MkSomeDoor} \DataTypeTok{SClosed}\NormalTok{ _ }\OtherTok{->} \StringTok{"mySomeDoor was closed!"}
        \DataTypeTok{MkSomeDoor} \DataTypeTok{SLocked}\NormalTok{ _ }\OtherTok{->} \StringTok{"mySomeDoor was locked!"}
\NormalTok{mySomeDoor was opened}\FunctionTok{!}
\end{Highlighting}
\end{Shaded}

Using \texttt{mkSomeDoor}, we can truly pass in a \texttt{DoorState} that we
generate at runtime (from IO, or a user prompt, or a configuration file, maybe),
and create a \texttt{Door} based on it.

Take \emph{that}, type erasure! :D

We could even directly return a \texttt{Door} with an existentially quantified
door status in CPS style:

\begin{Shaded}
\begin{Highlighting}[]
\OtherTok{withDoor ::} \DataTypeTok{DoorState} \OtherTok{->} \DataTypeTok{String} \OtherTok{->}\NormalTok{ (forall s}\FunctionTok{.} \DataTypeTok{SingDS}\NormalTok{ s }\OtherTok{->} \DataTypeTok{Door}\NormalTok{ s }\OtherTok{->}\NormalTok{ r) }\OtherTok{->}\NormalTok{ r}
\NormalTok{withDoor s m f }\FunctionTok{=} \KeywordTok{case}\NormalTok{ s }\KeywordTok{of}
    \DataTypeTok{Opened} \OtherTok{->}\NormalTok{ f }\DataTypeTok{SOpened}\NormalTok{ (}\DataTypeTok{UnsafeMkDoor}\NormalTok{ m)}
    \DataTypeTok{Closed} \OtherTok{->}\NormalTok{ f }\DataTypeTok{SClosed}\NormalTok{ (}\DataTypeTok{UnsafeMkDoor}\NormalTok{ m)}
    \DataTypeTok{Locked} \OtherTok{->}\NormalTok{ f }\DataTypeTok{SLocked}\NormalTok{ (}\DataTypeTok{UnsafeMkDoor}\NormalTok{ m)}
\end{Highlighting}
\end{Shaded}

\begin{Shaded}
\begin{Highlighting}[]
\NormalTok{ghci}\FunctionTok{>}\NormalTok{ withDoor }\DataTypeTok{Opened} \StringTok{"Birch"} \FunctionTok{$}\NormalTok{ \textbackslash{}s d }\OtherTok{->} \KeywordTok{case}\NormalTok{ s }\KeywordTok{of}
         \DataTypeTok{SOpened} \OtherTok{->} \StringTok{"Opened door!"}
         \DataTypeTok{SClosed} \OtherTok{->} \StringTok{"Closed door!"}
         \DataTypeTok{SLocked} \OtherTok{->} \StringTok{"Locked door!"}
\DataTypeTok{Opened}\NormalTok{ door}\FunctionTok{!}
\end{Highlighting}
\end{Shaded}

This allows us to \emph{truly} directly generate a \texttt{Door\ s} with an
\texttt{s} that can vary at runtime.

\subsubsection{Reification}\label{reification}

The general pattern we are exploiting here is called \textbf{reification} --
we're taking a dynamic run-time value, and lifting it to the type level as a
type (here, the type variable \texttt{s}). You can think of reification as the
opposite of reflection, and imagine the two as being the ``gateway'' between the
type-safe and unsafe world. In the dynamic world of a \texttt{DoorState} value,
you have no type safety. You live in the world of \texttt{SomeDoor},
\texttt{closeSomeOpenedDoor}, \texttt{lockAnySomeDoor}, etc. But, you can
\emph{reify} your \texttt{DoorState} value to a \emph{type}, and enter the
type-safe world of \texttt{Door\ s}, \texttt{closeDoor}, \texttt{lockDoor}, and
\texttt{lockAnyDoor}.

It might be more meaningful then to write a direct reification function for our
\texttt{DoorState}, in CPS style. Then, we can actually write our
\texttt{withDoor} in terms of it!

\begin{Shaded}
\begin{Highlighting}[]
\CommentTok{-- source: https://github.com/mstksg/inCode/tree/master/code-samples/singletons/Door.hs#L99-107}
\OtherTok{withDoorState ::} \DataTypeTok{DoorState} \OtherTok{->}\NormalTok{ (forall s}\FunctionTok{.} \DataTypeTok{SingDS}\NormalTok{ s }\OtherTok{->}\NormalTok{ r) }\OtherTok{->}\NormalTok{ r}
\NormalTok{withDoorState }\FunctionTok{=}\NormalTok{ \textbackslash{}}\KeywordTok{case}
    \DataTypeTok{Opened} \OtherTok{->}\NormalTok{ \textbackslash{}f }\OtherTok{->}\NormalTok{ f }\DataTypeTok{SOpened}
    \DataTypeTok{Closed} \OtherTok{->}\NormalTok{ \textbackslash{}f }\OtherTok{->}\NormalTok{ f }\DataTypeTok{SClosed}
    \DataTypeTok{Locked} \OtherTok{->}\NormalTok{ \textbackslash{}f }\OtherTok{->}\NormalTok{ f }\DataTypeTok{SLocked}

\OtherTok{withDoor ::} \DataTypeTok{DoorState} \OtherTok{->} \DataTypeTok{String} \OtherTok{->}\NormalTok{ (forall s}\FunctionTok{.} \DataTypeTok{SingDS}\NormalTok{ s }\OtherTok{->} \DataTypeTok{Door}\NormalTok{ s }\OtherTok{->}\NormalTok{ r) }\OtherTok{->}\NormalTok{ r}
\NormalTok{withDoor s m f }\FunctionTok{=}\NormalTok{ withDoorState s }\FunctionTok{$}\NormalTok{ \textbackslash{}sds }\OtherTok{->}
\NormalTok{                    f sds (}\DataTypeTok{UnsafeMkDoor}\NormalTok{ m)}
\end{Highlighting}
\end{Shaded}

\section{Sing}\label{sing}

\begin{itemize}
\item
  \texttt{toSing\ ::\ DoorState\ -\textgreater{}\ SomeSing\ DoorState} takes us
  from values to their (existentially quantified) singletons

\begin{Shaded}
\begin{Highlighting}[]
\NormalTok{ghci}\FunctionTok{>} \KeywordTok{let}\NormalTok{ s }\FunctionTok{=}\NormalTok{ toSing }\DataTypeTok{Opened}
\NormalTok{ghci}\FunctionTok{>} \FunctionTok{:}\NormalTok{t s}
\OtherTok{s ::} \DataTypeTok{SomeSing} \DataTypeTok{DoorState}
\NormalTok{ghci}\FunctionTok{>}\NormalTok{ putStrLn }\FunctionTok{$} \KeywordTok{case}\NormalTok{ s }\KeywordTok{of}
        \DataTypeTok{SomeSing} \DataTypeTok{SOpened} \OtherTok{->} \StringTok{"Opened."}
        \DataTypeTok{SomeSing} \DataTypeTok{SClosed} \OtherTok{->} \StringTok{"SClosed."}
        \DataTypeTok{SomeSing} \DataTypeTok{SLocked} \OtherTok{->} \StringTok{"SLocked."}
\StringTok{"Opened."}
\end{Highlighting}
\end{Shaded}

  \texttt{SomeSing} is like \texttt{SomeDoor} in that it is an existentially
  quantified singleton:

\begin{Shaded}
\begin{Highlighting}[]
\KeywordTok{data} \DataTypeTok{SomeSing} \DataTypeTok{DoorState}\OtherTok{ ::} \DataTypeTok{Type} \KeywordTok{where}
    \DataTypeTok{SomeSing}\OtherTok{ ::} \DataTypeTok{Sing}\NormalTok{ s }\OtherTok{->} \DataTypeTok{SomeSing} \DataTypeTok{DoorState}

\CommentTok{-- or, more accurately, since `SomeSing` is polykinded}
\KeywordTok{data} \DataTypeTok{SomeSing}\OtherTok{ ::}\NormalTok{ k }\OtherTok{->} \DataTypeTok{Type} \KeywordTok{where}
    \DataTypeTok{SomeSing}\OtherTok{ ::} \DataTypeTok{Sing}\NormalTok{ (}\OtherTok{a ::}\NormalTok{ k) }\OtherTok{->} \DataTypeTok{SomeSing}\NormalTok{ k}
\end{Highlighting}
\end{Shaded}
\end{itemize}

\begin{enumerate}
\def\labelenumi{\arabic{enumi}.}
\setcounter{enumi}{2}
\item
  Implement \texttt{withSomeDoor} for the existentially quantified
  \texttt{SomeDoor} type.

\begin{Shaded}
\begin{Highlighting}[]
\CommentTok{-- source: https://github.com/mstksg/inCode/tree/master/code-samples/singletons/DoorSingletons.hs#L61-103}
\KeywordTok{data} \DataTypeTok{SomeDoor}\OtherTok{ ::} \DataTypeTok{Type} \KeywordTok{where}
    \DataTypeTok{MkSomeDoor}\OtherTok{ ::} \DataTypeTok{Sing}\NormalTok{ s }\OtherTok{->} \DataTypeTok{Door}\NormalTok{ s }\OtherTok{->} \DataTypeTok{SomeDoor}

\OtherTok{withSomeDoor ::} \DataTypeTok{SomeDoor} \OtherTok{->}\NormalTok{ (forall s}\FunctionTok{.} \DataTypeTok{Sing}\NormalTok{ s }\OtherTok{->} \DataTypeTok{Door}\NormalTok{ s }\OtherTok{->}\NormalTok{ r) }\OtherTok{->}\NormalTok{ r}
\end{Highlighting}
\end{Shaded}
\item
  Implement \texttt{openAnySomeDoor}, which should work like
  \texttt{lockAnySomeDoor}, just wrapping an application of \texttt{openAnyDoor}
  inside a \texttt{SomeDoor}.

\begin{Shaded}
\begin{Highlighting}[]
\CommentTok{-- source: https://github.com/mstksg/inCode/tree/master/code-samples/singletons/DoorSingletons.hs#L99-99}
\OtherTok{openAnySomeDoor ::} \DataTypeTok{Int} \OtherTok{->} \DataTypeTok{SomeDoor} \OtherTok{->} \DataTypeTok{Maybe}\NormalTok{ (}\DataTypeTok{Door} \CharTok{'Opened)}
\end{Highlighting}
\end{Shaded}

  You \textbf{shouild not} use \texttt{UnsafeMkDoor} directly.

  Note that because we wrote \texttt{openAnyDoor} in ``implicit style'', we
  might have to convert between \texttt{SingI\ s\ =\textgreater{}} and
  \texttt{Sing\ s\ -\textgreater{}} style, using \texttt{withSingI}.
\end{enumerate}

\end{document}
