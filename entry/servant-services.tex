\documentclass[]{article}
\usepackage{lmodern}
\usepackage{amssymb,amsmath}
\usepackage{ifxetex,ifluatex}
\usepackage{fixltx2e} % provides \textsubscript
\ifnum 0\ifxetex 1\fi\ifluatex 1\fi=0 % if pdftex
  \usepackage[T1]{fontenc}
  \usepackage[utf8]{inputenc}
\else % if luatex or xelatex
  \ifxetex
    \usepackage{mathspec}
    \usepackage{xltxtra,xunicode}
  \else
    \usepackage{fontspec}
  \fi
  \defaultfontfeatures{Mapping=tex-text,Scale=MatchLowercase}
  \newcommand{\euro}{€}
\fi
% use upquote if available, for straight quotes in verbatim environments
\IfFileExists{upquote.sty}{\usepackage{upquote}}{}
% use microtype if available
\IfFileExists{microtype.sty}{\usepackage{microtype}}{}
\usepackage[margin=1in]{geometry}
\ifxetex
  \usepackage[setpagesize=false, % page size defined by xetex
              unicode=false, % unicode breaks when used with xetex
              xetex]{hyperref}
\else
  \usepackage[unicode=true]{hyperref}
\fi
\hypersetup{breaklinks=true,
            bookmarks=true,
            pdfauthor={Justin Le},
            pdftitle={Setting up a dead-simple TCP/IP service using servant},
            colorlinks=true,
            citecolor=blue,
            urlcolor=blue,
            linkcolor=magenta,
            pdfborder={0 0 0}}
\urlstyle{same}  % don't use monospace font for urls
% Make links footnotes instead of hotlinks:
\renewcommand{\href}[2]{#2\footnote{\url{#1}}}
\setlength{\parindent}{0pt}
\setlength{\parskip}{6pt plus 2pt minus 1pt}
\setlength{\emergencystretch}{3em}  % prevent overfull lines
\setcounter{secnumdepth}{0}

\title{Setting up a dead-simple TCP/IP service using servant}
\author{Justin Le}

\begin{document}
\maketitle

\emph{Originally posted on
\textbf{\href{https://blog.jle.im/entry/servant-services.html}{in Code}}.}

In my time I've written a lot of throwaway binary TCP/IP services (servers and
services you can interact with over an internet connection, through command line
interface or GUI). For me, this involves designing a protocol from scratch every
time with varying levels of hand-rolled authentication and error detection (Send
this byte for this command, this byte for this other command, etc.). Once I
design the protocol, I then have to write both the client and the server ---
something I usually do from scratch over the raw TCP streams.

This process was fun (and informative) the first few times I did it, but
spinning it up from scratch again every time discouraged me from doing it very
often. However, thankfully, with the
\emph{\href{https://hackage.haskell.org/package/servant}{servant}} haskell
library, writing a TCP server/client pair for a TCP service becomes dead-simple
--- the barrier for creating one fades away that designing/writing a service
becomes a tool that I reach for immediately in a lot of cases without second
thought.

\emph{servant} is usually advertised as a tool for writing web servers and web
applications, but it's easily adapted to write non-web things as well
(especially with the help of
\emph{\href{https://hackage.haskell.org/package/servant-cli}{servant-cli}}).
Let's dive in and write a simple TCP/IP service (a todo list manager) to see how
straightforward the process is!

\hypertarget{signoff}{%
\section{Signoff}\label{signoff}}

Hi, thanks for reading! You can reach me via email at
\href{mailto:justin@jle.im}{\nolinkurl{justin@jle.im}}, or at twitter at
\href{https://twitter.com/mstk}{@mstk}! This post and all others are published
under the \href{https://creativecommons.org/licenses/by-nc-nd/3.0/}{CC-BY-NC-ND
3.0} license. Corrections and edits via pull request are welcome and encouraged
at \href{https://github.com/mstksg/inCode}{the source repository}.

If you feel inclined, or this post was particularly helpful for you, why not
consider \href{https://www.patreon.com/justinle/overview}{supporting me on
Patreon}, or a \href{bitcoin:3D7rmAYgbDnp4gp4rf22THsGt74fNucPDU}{BTC donation}?
:)

\end{document}
