\documentclass[]{article}
\usepackage{lmodern}
\usepackage{amssymb,amsmath}
\usepackage{ifxetex,ifluatex}
\usepackage{fixltx2e} % provides \textsubscript
\ifnum 0\ifxetex 1\fi\ifluatex 1\fi=0 % if pdftex
  \usepackage[T1]{fontenc}
  \usepackage[utf8]{inputenc}
\else % if luatex or xelatex
  \ifxetex
    \usepackage{mathspec}
    \usepackage{xltxtra,xunicode}
  \else
    \usepackage{fontspec}
  \fi
  \defaultfontfeatures{Mapping=tex-text,Scale=MatchLowercase}
  \newcommand{\euro}{€}
\fi
% use upquote if available, for straight quotes in verbatim environments
\IfFileExists{upquote.sty}{\usepackage{upquote}}{}
% use microtype if available
\IfFileExists{microtype.sty}{\usepackage{microtype}}{}
\usepackage[margin=1in]{geometry}
\usepackage{color}
\usepackage{fancyvrb}
\newcommand{\VerbBar}{|}
\newcommand{\VERB}{\Verb[commandchars=\\\{\}]}
\DefineVerbatimEnvironment{Highlighting}{Verbatim}{commandchars=\\\{\}}
% Add ',fontsize=\small' for more characters per line
\newenvironment{Shaded}{}{}
\newcommand{\AlertTok}[1]{\textcolor[rgb]{1.00,0.00,0.00}{\textbf{#1}}}
\newcommand{\AnnotationTok}[1]{\textcolor[rgb]{0.38,0.63,0.69}{\textbf{\textit{#1}}}}
\newcommand{\AttributeTok}[1]{\textcolor[rgb]{0.49,0.56,0.16}{#1}}
\newcommand{\BaseNTok}[1]{\textcolor[rgb]{0.25,0.63,0.44}{#1}}
\newcommand{\BuiltInTok}[1]{#1}
\newcommand{\CharTok}[1]{\textcolor[rgb]{0.25,0.44,0.63}{#1}}
\newcommand{\CommentTok}[1]{\textcolor[rgb]{0.38,0.63,0.69}{\textit{#1}}}
\newcommand{\CommentVarTok}[1]{\textcolor[rgb]{0.38,0.63,0.69}{\textbf{\textit{#1}}}}
\newcommand{\ConstantTok}[1]{\textcolor[rgb]{0.53,0.00,0.00}{#1}}
\newcommand{\ControlFlowTok}[1]{\textcolor[rgb]{0.00,0.44,0.13}{\textbf{#1}}}
\newcommand{\DataTypeTok}[1]{\textcolor[rgb]{0.56,0.13,0.00}{#1}}
\newcommand{\DecValTok}[1]{\textcolor[rgb]{0.25,0.63,0.44}{#1}}
\newcommand{\DocumentationTok}[1]{\textcolor[rgb]{0.73,0.13,0.13}{\textit{#1}}}
\newcommand{\ErrorTok}[1]{\textcolor[rgb]{1.00,0.00,0.00}{\textbf{#1}}}
\newcommand{\ExtensionTok}[1]{#1}
\newcommand{\FloatTok}[1]{\textcolor[rgb]{0.25,0.63,0.44}{#1}}
\newcommand{\FunctionTok}[1]{\textcolor[rgb]{0.02,0.16,0.49}{#1}}
\newcommand{\ImportTok}[1]{#1}
\newcommand{\InformationTok}[1]{\textcolor[rgb]{0.38,0.63,0.69}{\textbf{\textit{#1}}}}
\newcommand{\KeywordTok}[1]{\textcolor[rgb]{0.00,0.44,0.13}{\textbf{#1}}}
\newcommand{\NormalTok}[1]{#1}
\newcommand{\OperatorTok}[1]{\textcolor[rgb]{0.40,0.40,0.40}{#1}}
\newcommand{\OtherTok}[1]{\textcolor[rgb]{0.00,0.44,0.13}{#1}}
\newcommand{\PreprocessorTok}[1]{\textcolor[rgb]{0.74,0.48,0.00}{#1}}
\newcommand{\RegionMarkerTok}[1]{#1}
\newcommand{\SpecialCharTok}[1]{\textcolor[rgb]{0.25,0.44,0.63}{#1}}
\newcommand{\SpecialStringTok}[1]{\textcolor[rgb]{0.73,0.40,0.53}{#1}}
\newcommand{\StringTok}[1]{\textcolor[rgb]{0.25,0.44,0.63}{#1}}
\newcommand{\VariableTok}[1]{\textcolor[rgb]{0.10,0.09,0.49}{#1}}
\newcommand{\VerbatimStringTok}[1]{\textcolor[rgb]{0.25,0.44,0.63}{#1}}
\newcommand{\WarningTok}[1]{\textcolor[rgb]{0.38,0.63,0.69}{\textbf{\textit{#1}}}}
\usepackage{graphicx}
\makeatletter
\def\maxwidth{\ifdim\Gin@nat@width>\linewidth\linewidth\else\Gin@nat@width\fi}
\def\maxheight{\ifdim\Gin@nat@height>\textheight\textheight\else\Gin@nat@height\fi}
\makeatother
% Scale images if necessary, so that they will not overflow the page
% margins by default, and it is still possible to overwrite the defaults
% using explicit options in \includegraphics[width, height, ...]{}
\setkeys{Gin}{width=\maxwidth,height=\maxheight,keepaspectratio}
\ifxetex
  \usepackage[setpagesize=false, % page size defined by xetex
              unicode=false, % unicode breaks when used with xetex
              xetex]{hyperref}
\else
  \usepackage[unicode=true]{hyperref}
\fi
\hypersetup{breaklinks=true,
            bookmarks=true,
            pdfauthor={Justin Le},
            pdftitle={Degenerate Hyper-Dimensional Game of Life},
            colorlinks=true,
            citecolor=blue,
            urlcolor=blue,
            linkcolor=magenta,
            pdfborder={0 0 0}}
\urlstyle{same}  % don't use monospace font for urls
% Make links footnotes instead of hotlinks:
\renewcommand{\href}[2]{#2\footnote{\url{#1}}}
\setlength{\parindent}{0pt}
\setlength{\parskip}{6pt plus 2pt minus 1pt}
\setlength{\emergencystretch}{3em}  % prevent overfull lines
\setcounter{secnumdepth}{0}

\title{Degenerate Hyper-Dimensional Game of Life}
\author{Justin Le}

\begin{document}
\maketitle

\emph{Originally posted on
\textbf{\href{https://blog.jle.im/entry/degenerate-hyper-dimensional-game-of-life.html}{in
Code}}.}

tldr: Over the course of a month, we were able to successive new mathematical
properties of a ``degenerate'' hyper-dimensional game of life" to take a ``10
dimensions may just barely be possible on a supercomputer'' to ``10 dimensions
is easy enough to be run on any modern browser, and 40 dimensions can be reached
with a compiled language''. Includes interactive visualizations and simulations!

This is a story about breaking the degenerate hyper-dimensional game of life by
exploratory visualizations and math! Let's travel back in time: t'was the night
before Thursday, December 17, 2020, The release of
\href{https://adventofcode.com/2020/day/17}{``Conway Cubes''}, day 17 of the
``Advent of Code'' (fun little coding puzzles building up to Christmas). One
part about Advent of Code I've always found especially fun is that, because the
problems are so self-contained and tidy, they are often \emph{open-ended} in the
interesting ways you can solve them or expand them.

On the surface, Day 17 seemed to essentially be a straightforward extension of
\href{https://en.wikipedia.org/wiki/Conway\%27s_Game_of_Life}{Conway's Game Of
Life} (``GoL''). GoL is a simulation played out on a 2d grid, where cells are
``on'' and ``off'', and at each step of the simulation, the on/off cells spread
and propagate in fascinating ways based on the state of their neighbors.

The twist of the Advent of Code puzzle is it asks what would happen if we played
out the rules of GoL in 3d, and then 4d! The ``starting conditions'' are a 8x8
2D grid picked out for each participant, and the puzzle solution is the number
of live cells after six steps. My personal starting conditions were:

\begin{verbatim}
#####..#
#..###.#
###.....
.#.#.#..
##.#..#.
######..
.##..###
###.####
\end{verbatim}

I submitted my answer with a direct implementation (scoring the 66th spot on the
leader board for that day)\ldots and that was that for the ``competitive'' part.
But the real fun always starts after! When discussing with some friends, we
started talking about the trade-offs of different implementations and realized
that the extra dimensionality was no joke\ldots as you upped the number of
dimensions, the number of points you have to consider grow as
\includegraphics{https://latex.codecogs.com/png.latex?O\%28\%282t\%2B6\%29\%5Ed\%29},
and the number of neighbors of each point to check grows as
\includegraphics{https://latex.codecogs.com/png.latex?O\%283\%5Ed\%29}. So for
4D it's definitely possible to solve naively\ldots but anything higher is going
to strain. My naive solution on 6D took three minutes, and 7D in a reasonable
amount of time (612,220,032 points with 2,186 neighbors each) seemed
\emph{impossible} on commercial consumer hardware because of the sheer number of
points in 7D space. But I thought\ldots what if a breakthrough in optimization
was possible? I set my goal as 10D (3,570,467,226,624 points with 59,048
neighbors each), not knowing if it was possible.

And soon\ldots a breakthrough did come! Someone brought up that if we look at
the 3d version, we see there's actually a \emph{mirror symmetry}! That is,
because everything starts off on the xy plane, with z=0, the resulting
progression must be symmetrical on both sides (positive and negative z).

\begin{figure}
\centering
\includegraphics{/img/entries/advent-gol/life3d.gif}
\caption{d=3 animation by
\href{https://www.reddit.com/r/adventofcode/comments/kfa3nr/2020_day_17_godot_cubes_i_think_i_went_a_bit_too/}{u/ZuBsPaCe}}
\end{figure}

In the end that means we only have to simulate one of the
``halves''/``quadrants'' of the higher-dimensional space, since all
``quadrants'' are identical! This saves down the number of points by a factor of
two for each extra dimension
(\includegraphics{https://latex.codecogs.com/png.latex?O\%282\%5E\%7Bd-2\%7D\%29}).
My 7D implementation completed in 6 minutes! 8D still hung forever, though.

Well, it didn't get us to d=10\ldots but this discovery completely changed how
we saw this puzzle. With one breakthrough down, we began to believe that there
would be more just around the corner, made possible by our problem's special
degeneracy (that is, that we start on a 2d slice).

Such a dream (as posed in
\href{https://www.reddit.com/r/adventofcode/comments/kfb6zx/day_17_getting_to_t6_at_for_higher_spoilerss/}{this
reddit thread I started}) turned into a month-long quest of breakthrough after
breakthrough, exploiting different aspects of this degeneracy! It was a long,
harrowing journey full of sudden twists and turns and bursts of excitement when
new innovations came. And in the end, the hopeful question ``What if d=10 was
possible?'' turned into ``d=10 in 100ms, d=40 in eight minutes.'' I even got
d=10 fast enough to run on easily any modern browser --- this post includes
those simulations! Furthermore, the whole journey became an adventure in the
power of visualization combined with abstract thinking.

So, let's take a deep dive --- deeper than you probably ever expected to dive
into any particular degenerate starting conditions of a hyper-dimensional game
of life :D

There will be python code samples here and there, but just for context, my
actual solvers I developed along the way were written in Haskell, and all of the
solving logic embedded in this post was written in Purescript and compiled to
Javascript.

\hypertarget{starting-off}{%
\section{Starting Off}\label{starting-off}}

First of all, let's meet our friend for the rest of this journey. In the drawer
below, you can draw (with your mouse) the 8x8 grid you want to simulate for the
rest of this post. As you draw, the rest of the visualizations will update to
use this as their initial conditions.

\leavevmode\hypertarget{golDrawer}{}%
Please enable Javascript

And for fun, here's a 2D vanilla game of life implementation (for six time
steps) to test out your creation. I recommend trying out some of the
\href{https://en.wikipedia.org/wiki/Conway\%27s_Game_of_Life\#Examples_of_patterns}{interesting
well-known patterns}!

\leavevmode\hypertarget{gol2D}{}%
Please enable Javascript

Now that that's there, let's start at the beginning: what's the naive, baseline
solution?

A reasonable initial thought would be:

\begin{enumerate}
\def\labelenumi{\arabic{enumi}.}
\tightlist
\item
  Keep a 2D (or 3D, or 4D, etc.) array of booleans.
\item
  At each step:

  \begin{enumerate}
  \def\labelenumii{\alph{enumii}.}
  \tightlist
  \item
    Make a fresh copy of the entire space
    (\includegraphics{https://latex.codecogs.com/png.latex?O\%28n\%5Ed\%29}).
  \item
    Loop over each item in your array
    (\includegraphics{https://latex.codecogs.com/png.latex?O\%28n\%5Ed\%29}).
    Count all of the neighbors
    (\includegraphics{https://latex.codecogs.com/png.latex?O\%283\%5Ed\%29})
    that are \texttt{true} (``alive''), and write to the new array based on the
    rules table of GoL (2 or 3 neighbors for a live cell stays alive, 3
    neighbors for a dead cell turns alive).
  \end{enumerate}
\item
  You have a new array! Loop again six times.
\end{enumerate}

Sounds reasonable enough! And this does work for the 2D case pretty well (like
in the \href{https://adventofcode.com/2020/day/11}{Day 11 puzzle}). However,
there are some clear issues when moving into higher dimensions. The size of your
array grows exponentially on your dimension, and so does the number of neighbors
you'd have to check. And the
\href{https://en.wikipedia.org/wiki/Curse_of_dimensionality}{curse of
dimensionality} assures us that more and more of that array would become wasted
as the proportion of ``on'' points shrinks to zero for higher dimensions.

Oh, but what's that? The percentage of ``on'' points shrinks to zero for higher
dimensions? That actually sounds like something we can use to our advantage!
The\ldots{}\emph{blessing of dimensionality}, I daresay? Because we know the
vast majority of our points will be ``off'', there's another method.

\begin{enumerate}
\def\labelenumi{\arabic{enumi}.}
\tightlist
\item
  Keep a \emph{set} of points that are ``on''.
\item
  At each step:

  \begin{enumerate}
  \def\labelenumii{\alph{enumii}.}
  \item
    Initialize a dynamic map (key-value store) of points to integers. This map
    associates each point to the number of live neighbors it has.
  \item
    For each step, iterate over each of your ``on'' points, expand all of their
    neighbors \includegraphics{https://latex.codecogs.com/png.latex?n_i}
    (\includegraphics{https://latex.codecogs.com/png.latex?\%28O\%283\%5Ed\%29\%29}),
    and increment the value associated with
    \includegraphics{https://latex.codecogs.com/png.latex?n_i} in your dynamic
    map.

    For example, if the point \texttt{\textless{}2,3\textgreater{}} is in your
    set of live points, you would add increment the map's values at keys
    \texttt{\textless{}1,2\textgreater{}},
    \texttt{\textless{}2,2\textgreater{}},
    \texttt{\textless{}3,2\textgreater{}}, etc.: all 8 neighbors of
    \texttt{\textless{}2,3\textgreater{}}.
  \item
    Collect your new set of on points: keep all of the keys in your dynamic map
    corresponding to live points if their integers are 2 or 3, and keep all of
    the keys in your dynamic map corresponding to dead points if their integers
    are 3.
  \end{enumerate}
\item
  You have a new set! Loop again six times!
\end{enumerate}

I discuss this algorithm much more deeply with actual code in
\href{https://github.com/mstksg/advent-of-code-2020/blob/master/reflections-out/day17.md}{my
solutions write-up in my Advent of Code reflections journal}.

This method nets us a huge advantage because we now only have to loop over the
number of items that we know are alive! Any points far away from our set of
alive points can be properly ignored. This narrows down our huge iteration
space, and the benefits compound with every dimension due to the blessing of
dimensionality!

The nice thing about this method is that it's easy enough to generalize to any
dimension: instead of, say, keeping \texttt{{[}x,y{]}} in your set for 2D, just
keep \texttt{{[}x,y,z{]}} for 3D, or any length array of coordinates. One minor
trick you need to think through is generating all
\includegraphics{https://latex.codecogs.com/png.latex?3\%5Ed-1} neighbors, but
but that's going to come down to a d-ary
\href{https://observablehq.com/@d3/d3-cross}{cartesian product} of
\texttt{{[}-1,0,1{]}} to itself.

Here's a version of the set-based implementation, using a nice trick I learned
from \href{https://twitter.com/phaazon_}{phaazon} to get the right neighbors by
doing a cartesian product against \texttt{{[}0,-1,1{]}}, which leaves the first
item as the \texttt{\textless{}0,0\textgreater{}} ``original point'' we want to
exclude:

\begin{Shaded}
\begin{Highlighting}[]
\ImportTok{from}\NormalTok{ itertools }\ImportTok{import}\NormalTok{ islice, product}
\ImportTok{from}\NormalTok{ collections }\ImportTok{import}\NormalTok{ Counter}

\KeywordTok{def}\NormalTok{ mk\_neighbs(point):}
    \CommentTok{"""Return neighboring points, each equally weighted}

\CommentTok{    (1,2)}
\CommentTok{    => [(1, 1), (1, 3), (0, 2), (0, 1), (0, 3), (2, 2), (2, 1), (2, 3)]}
\CommentTok{    """}
\NormalTok{    gen }\OperatorTok{=}\NormalTok{ product(}\OperatorTok{*}\NormalTok{[[x, x}\DecValTok{{-}1}\NormalTok{, x}\OperatorTok{+}\DecValTok{1}\NormalTok{] }\ControlFlowTok{for}\NormalTok{ x }\KeywordTok{in}\NormalTok{ point])}
    \CommentTok{\# skip the first item, the original point}
    \BuiltInTok{next}\NormalTok{(gen)}
    \ControlFlowTok{return}\NormalTok{ gen}

\KeywordTok{def}\NormalTok{ step(pts):}
    \CommentTok{"""Takes a set of points (tuples) and steps them in the simulation}
\CommentTok{    """}
\NormalTok{    neighbs }\OperatorTok{=}\NormalTok{ Counter()}
    \ControlFlowTok{for}\NormalTok{ point }\KeywordTok{in}\NormalTok{ pts:}
\NormalTok{        neighbs }\OperatorTok{+=}\NormalTok{ Counter(mk\_neighbs(point))}

    \KeywordTok{def}\NormalTok{ validate(point, ncount):}
        \ControlFlowTok{if}\NormalTok{ point }\KeywordTok{in}\NormalTok{ pts:}
            \ControlFlowTok{return}\NormalTok{ ncount }\OperatorTok{==} \DecValTok{2} \KeywordTok{or}\NormalTok{ ncount }\OperatorTok{==} \DecValTok{3}
        \ControlFlowTok{else}\NormalTok{:}
            \ControlFlowTok{return}\NormalTok{ ncount }\OperatorTok{==} \DecValTok{3}

    \ControlFlowTok{return}\NormalTok{ [p }\ControlFlowTok{for}\NormalTok{ p, n }\KeywordTok{in}\NormalTok{ neighbs.items() }\ControlFlowTok{if}\NormalTok{ validate(p, n)]}
\end{Highlighting}
\end{Shaded}

\hypertarget{three-dimensions}{%
\section{Three Dimensions}\label{three-dimensions}}

Let's see how this looks for the 3D case! To make things easier to see, we can
render things in ``slices'' in 3D space: each grid represents a slice at a
different z level (ie, the z=0 square represents all squares
\includegraphics{https://latex.codecogs.com/png.latex?\%3Cx\%2Cy\%2C0\%3E}).
Press ``Play'' to have the simulation cycle through 6 time steps!

\leavevmode\hypertarget{gol3D}{}%
Please enable Javascript

In ``reality'', each of those 13 slices above are stacked on top of each other
in 3D space. You'll see that most initial conditions will spread out from the
center z=0 point, which means they are actually spreading ``up and down'' the z
axis.

If you mouse over (or tap) any individual tiny
\texttt{\textless{}x,y\textgreater{}} cell, you'll see the all of the 26
(\includegraphics{https://latex.codecogs.com/png.latex?3\%5Ed-1})
\texttt{\textless{}x,y,z\textgreater{}} 3D neighbors of the point you're
hovering over highlighted in blue --- these 26 points form a 3D cube around your
mouse once everything is stacked correctly. You can use this cube to help see
how the simulation progresses. If your mouse is hovering over a live cell, and
there are 2 or 3 live cells highlighted in your cube, it'll stay alive in the
next time step. If your mouse is hovering over a dead cell and there are exactly
3 live cells highlighted in your cube, it will come alive in the next step.

\hypertarget{axis-reflection-symmetry}{%
\subsection{Axis Reflection Symmetry}\label{axis-reflection-symmetry}}

Try playing around with different initial conditions to see how they evolve! See
any patterns?

Well, the yellow highlight might have given given things away, but\ldots note
that the entire thing has reflection symmetry across z=0! z=1 is always the same
as z=-1, z=2 is always the same as z=-2, etc. Fundamentally, this is because our
starting solution has z-plane symmetry: the initial 2D slice is symmetric with
reflections across z, because z=0. This is the first ``degeneracy'' that this
blog post's title is referring to. The negative and positive directions are
interchangeable! This is reflected in the yellow highlight on hover: when you
mouse-over a z square, its corresponding reflected twin is highlighted, and will
always be identical.

This means that we actually only need to simulate \emph{positive} z's\ldots and
for our final answer we just ``un-reflect'' to get the total number.

This is exactly what freenode IRC user sim642 noticed late into the night of
December 16th:

\begin{quote}
I wanted to ask this before but forgot: did anyone try to take advantage of the
symmetry, e.g.~in z axis in part 1? Should halve the amount of calculations you
have to do.

Only some extra work at the end to differentiate z=0 and z\textgreater0
positions to know which to count twice And in part 2 I feel like you could also
exploit the symmetry in w axis simultaneously
\end{quote}

Wow, let's do this! Apparently the picture is slightly more complicated than
simply halving the points. We also need to change how to distribute neighbors.
That's because, once we commit to only keeping the positive z's, some cells need
to be double-counted as neighbors. In particular, any \texttt{z=0} cell would
previously had a neighbor at both \texttt{z=-1} and \texttt{z=1}\ldots but now
if we only keep the positive z's, it would have \texttt{z=1} as a neighbor
\emph{twice}.

The following interactive demo lets you explore what this looks like:

\leavevmode\hypertarget{golSyms3DForward}{}%
Please enable Javascript

Each square represents an entire ``slice'' of z. When you mouse-over or tap a
z-cell, its z-neighbors are highlighted with how many times that neighbor has to
be counted, and the green bar tells you from what direction that neighborship
arose from. For example, mousing over z=3, z=2 and z=4 get highlighted with the
values ``1'' because they are neighbors of 3, on the left and right side
(respectively). Note that one neat property for all squares (except for z=6,
which goes off the scale) is that the ``total'' higher-dimensional neighbors is
always 2
(\includegraphics{https://latex.codecogs.com/png.latex?3\%5E\%28d-2\%29-1}),
it's just that where those neighbors fall is re-arranged slightly.

The tricky square is now z=0: if you mouse-over it, you'll see that it has a
single neighbor z=1 that is counted twice, as a neighbor from both the left and
right side.

We can compute the above diagram by expanding z=0 to its neighbors (z=-1, and
z=1), applying the absolute value function, and seeing how points double-up.
This gives us the ``forward neighbors'', and we can directly use it for the
original ``keep the full array'' GoL implementation.

However, for the ``keep active points and expand their neighbors'' GoL
implementation\ldots we have to find the opposite. Remember that to build our
``neighbors map'' (the map of points to how many active neighbors they have), we
have each cell ``pro-actively'' add its contributions to all of its neighbors.
\texttt{\textless{}1,2,3\textgreater{}} is a neighbor to
\texttt{\textless{}1,3,4\textgreater{}} once, so when we expand
\texttt{\textless{}1,2,3\textgreater{}} we would increment the value in the map
at \texttt{\textless{}1,3,4\textgreater{}} by 1 because
\texttt{\textless{}1,2,3\textgreater{}} is a neighbor of
\texttt{\textless{}1,3,4\textgreater{}} once.

Now, how do we count \texttt{\textless{}1,3,1\textgreater{}} expanding into
\texttt{\textless{}1,3,0\textgreater{}}? Well, normally,
\texttt{\textless{}1,3,1\textgreater{}} is a neighbor of
\texttt{\textless{}1,3,0\textgreater{}} once. However, if we only keep the
normalized z values, \texttt{\textless{}1,3,1\textgreater{}} is a neighbor of
\texttt{\textless{}1,3,0\textgreater{}}\ldots twice! To compute the total
neighbor count of \texttt{\textless{}1,3,0\textgreater{}}, we have to count the
contribution from \texttt{\textless{}1,3,1\textgreater{}} twice (once for
\texttt{\textless{}1,3,1\textgreater{}} and once for
\texttt{\textless{}1,3,-1\textgreater{}}, which was normalized away).

That means we have to follow the rules in the previous demo \emph{backwards},
like:

\leavevmode\hypertarget{golSyms3DReverse}{}%
Please enable Javascript

These are the ``reverse neighbors'': how much times a given point counts as a
neighbor for its surrounding points. Here, mousing over z=1 shows that it counts
as a neighbor for z=0 twice, from both the left and the right. It also counts as
a neighbor for z=2 once (from the left side).

We can account for this by hard-coding the rules into our step algorithm: if our
z goes from \texttt{1} to \texttt{0}, increment its value twice in the neighbor
map. Otherwise, simply increment by 1 as normal.

This rule is relatively easy to implement, and as a result we now halved our
total number of points we need to keep and check for 3D! It's also simple enough
to generalize (just do the \texttt{1\ -\textgreater{}\ 0} check for every
``higher dimension'' and double its contribution for each
\texttt{1\ -\textgreater{}\ 0} transition is seen)\ldots and that means we
reduce the number of 4D points we need to track by a factor of four, the number
of 5D points by a factor of eight, the number of 6D points by a factor of
16\ldots{} now our total points to check only grows as
\includegraphics{https://latex.codecogs.com/png.latex?O\%28n\%5Ed\%20\%2F\%202\%5E\%7Bd-2\%7D\%29}
instead of
\includegraphics{https://latex.codecogs.com/png.latex?O\%28n\%5Ed\%29}!

This discovery late in the Tuesday night of the 16th was what inspired us to
believe and dream that more breakthroughs might be possible to bring things down
even further.

And those breakthroughs soon came\ldots{}

\hypertarget{four-dimensions}{%
\section{Four Dimensions}\label{four-dimensions}}

Let's look at how the 4 dimensions works! We can visualize this by taking
``z-w'' slices at different x-y planes as well. The labels in these boxes are
the \texttt{\textless{}z,w\textgreater{}} of each slice. The very center is
\texttt{\textless{}z,w\textgreater{}\ =\ \textless{}0,0\textgreater{}} the row
in the middle from the top is \texttt{w=0}, and the column in the very middle
from the left is \texttt{z=0}. It's basically taking the 3D visualization above
and expanding it in an extra dimension. Press ``Play'' to run your initial
conditions!

\leavevmode\hypertarget{gol4D}{}%
Please enable Javascript

We get something interesting as well: most initial conditions will spread out
from the center
\texttt{\textless{}z,w\textgreater{}\ =\ \textless{}0,0\textgreater{}} point
radially, spreading outwards into positive and negative z and w. Mouse-over or
tap any individual tiny \texttt{\textless{}x.y\textgreater{}} cell and you'll
see each of its 80
(\includegraphics{https://latex.codecogs.com/png.latex?3\%5Ed-1})
\texttt{\textless{}x,y,z,w\textgreater{}} 4D neighbors highlighted in blue,
forming a little 3x3x3 ``tesseract'' (4D cube, or hypercube). Like in the 3D
case, you can use this little hypercube to track how the simulation progresses:
if your mouse if hovering over a live cell with 2 or 3 live cells in its
hypercube, it'll stay alive in the next step, if it's hovering over a dead cell
with 3 live cells in its hypercube, it'll come alive in the next step.

\hypertarget{diagonal-reflection-symmetry}{%
\subsection{Diagonal Reflection Symmetry}\label{diagonal-reflection-symmetry}}

Play around and explore how simulations evolve! You will notice that the axis
reflection symmetry is still preserved, but four ways (the slice at
\texttt{\textless{}z,w\textgreater{}\ =\ \textless{}3,4\textgreater{}} is always
going to be identical to the slice at \texttt{\textless{}-3,4\textgreater{}},
\texttt{\textless{}3,-4\textgreater{}}, and
\texttt{\textless{}-3,-4\textgreater{}}). These are reflected in the ``deep
yellow'' highlights above when you mouse over a zw square. (Ignore the lighter
yellow highlights for now!)

And now, for the next big breakthrough. I think this situation shows the power
of visualization well, because this exact visualization was what reddit user
\emph{u/cetttbycett} was looking at when
\href{https://www.reddit.com/r/adventofcode/comments/kfjhwh/year_2020_day_17_part_2_using_symmetry_in_4d_space/}{they
made this post} late Thursday the 17th/early Friday the 18th\ldots and
everything changed \emph{forever}.

\begin{quote}
I noticed that the expansion of active cubes for part 2 is symmetric with
respect to two hyperplanes in 4d space: These hyperplanes can be described by w
= 0 and w-z = 0.

Using these symmetries could make the code nearly eight times as fast.I was
wondering if anyone tried that.
\end{quote}

What \emph{u/cetttbycettt} saw is what you can see now in the demo above: it's
all of the \emph{light yellow} highlighted squares when you mouse-over. In
addition to the z=0 and w=0 lines (the two lines down the middle, up-down and
left-right), we also have another line of symmetry: z=w and w=z, the diagonal
lines!

That's right, a zw slice at
\texttt{\textless{}z,w\textgreater{}=\textless{}3,4\textgreater{}} is
\emph{identical} to the one at \texttt{\textless{}4,3\textgreater{}}, and so
also \texttt{\textless{}-3,4\textgreater{}},
\texttt{\textless{}3,-4\textgreater{}}, \texttt{\textless{}-3,-4\textgreater{}},
\texttt{\textless{}-4,3\textgreater{}}, \texttt{\textless{}4,-3\textgreater{}},
and \texttt{\textless{}-4,-3\textgreater{}}! Each slice is potentially repeated
\emph{eight} times! The exceptions are the points on the lines of symmetry
themselves, which are each repeated four times, and also
\texttt{\textless{}z,w\textgreater{}=\textless{}0,0\textgreater{}}, which is in
its own class.

So, our first breakthrough meant that we only have to simulate \emph{positive}
coordinates (a single quadrant)\ldots our next breakthrough means that we only
have to simulate coordinates on a single ``wedge'' half-quadrant\ldots and then
duplicate those eight times at the end.

Arbitrarily, let's say we only simulate the north-by-northeast wedge, because
it's easy to normalize/compact all points onto that wedge: they're the points
\texttt{\textless{}z,w\textgreater{}} where both components are positive and in
non-decreasing order. So, \texttt{\textless{}4,-3\textgreater{}} gets
``normalized'' to \texttt{\textless{}3,4\textgreater{}}.

Okay, so we found a new symmetry\ldots but we ran into the same issue as before.
How do we propagate neighbors? To help us, see what's going on, let's look at
the map of neighbors between different \texttt{\textless{}z,w\textgreater{}}
squares, for the single zw wedge we are simulating.

\leavevmode\hypertarget{golSyms4DForward}{}%
Please enable Javascript

These are the ``forward neighbors''; we can compute them by expanding a point to
its neighbors, and then normalizing our points and seeing how they double (or
quadruple) up.

\begin{Shaded}
\begin{Highlighting}[]
\KeywordTok{def}\NormalTok{ normalize(point):}
    \CommentTok{"""Normalize a point by sorting the absolute values}

\CommentTok{    (2, {-}1)}
\CommentTok{    => (1, 2)}
\CommentTok{    """}
    \ControlFlowTok{return} \BuiltInTok{tuple}\NormalTok{(}\BuiltInTok{sorted}\NormalTok{([}\BuiltInTok{abs}\NormalTok{(x) }\ControlFlowTok{for}\NormalTok{ x }\KeywordTok{in}\NormalTok{ point]))}

\KeywordTok{def}\NormalTok{ forward\_neighbs(point):}
    \CommentTok{"""Generate the forward neighbors of a point}

\CommentTok{    (0, 1)}
\CommentTok{    => \{(0, 1): 2, (1, 2): 2, (1, 1): 2, (0, 0): 1, (0, 2): 1\}}
\CommentTok{    """}
    \ControlFlowTok{return}\NormalTok{ Counter([normalize(neighb) }\ControlFlowTok{for}\NormalTok{ neighb }\KeywordTok{in}\NormalTok{ mk\_neighbs(point)])}
\end{Highlighting}
\end{Shaded}

For example, mouse over
\texttt{\textless{}z,w\textgreater{}=\textless{}3,3\textgreater{}} and see it
has eight total higher-dimensional neighbors (like all points should, though
this visualization leaves out points at w\textgreater6). It's \emph{supposed} to
have a neighbor at \texttt{\textless{}4,3\textgreater{}}, but that gets
reflected back onto \texttt{\textless{}3,4\textgreater{}} during our
normalization process, so you see that the point
\texttt{\textless{}3,3\textgreater{}} has a neighbor at
\texttt{\textless{}3,4\textgreater{}} ``double-counted''. The green squares (in
the north and west positions) at \texttt{\textless{}3,4\textgreater{}} when you
hover over \texttt{\textless{}3,3\textgreater{}} show that
\texttt{\textless{}3,4\textgreater{}} is a neighbor of
\texttt{\textless{}3,3\textgreater{}} both to its north and to its west.

Also, we have something really odd show up for the first time. Mouse over a
point like \texttt{\textless{}z,w\textgreater{}=\textless{}2,3\textgreater{}}
and see that it has a neighbor in\ldots itself? What's going on here? Well, it
is \emph{supposed} to have a neighbor at \texttt{\textless{}3,2\textgreater{}}
but that gets normalized/reflected back onto
\texttt{\textless{}2,3\textgreater{}} --- it reflects onto itself! The green
square in the Southeast means that \texttt{\textless{}2,3\textgreater{}}'s
southeast neighbor is\ldots itself!

The ``forward neighbors'' are useful for understanding what's going on, but to
actually run our simulation we again need to find the ``reverse neighbors'':
from a given point A, how many times is that point a neighbor of another point
B?

We can compute this in brute-force using a cache: iterate over each point,
expand all its neighbors
\includegraphics{https://latex.codecogs.com/png.latex?a_i}, normalize that
neighbor, and then set
\includegraphics{https://latex.codecogs.com/png.latex?a_i} in the cache to the
multiplicity after normalization.

\begin{Shaded}
\begin{Highlighting}[]
\KeywordTok{def}\NormalTok{ reverse\_neighbs\_table(t\_max):}
    \CommentTok{"""Tabulate the reverse neighbors of all zw slices reachable before t\_max}
\CommentTok{    """}
\NormalTok{    weights }\OperatorTok{=}\NormalTok{ \{\}}

    \ControlFlowTok{for}\NormalTok{ i }\KeywordTok{in} \BuiltInTok{range}\NormalTok{(t\_max):}
        \ControlFlowTok{for}\NormalTok{ j }\KeywordTok{in} \BuiltInTok{range}\NormalTok{(i, t\_max):}
            \ControlFlowTok{for}\NormalTok{ neighb, ncount }\KeywordTok{in}\NormalTok{ forward\_neighbs((i, j)).items():}
                \ControlFlowTok{if}\NormalTok{ neighb }\KeywordTok{in}\NormalTok{ weights:}
\NormalTok{                    weights[neighb][(i, j)] }\OperatorTok{=}\NormalTok{ ncount}
                \ControlFlowTok{else}\NormalTok{:}
\NormalTok{                    weights[neighb] }\OperatorTok{=}\NormalTok{ \{(i, j): ncount\}}

    \ControlFlowTok{return}\NormalTok{ weights}
\end{Highlighting}
\end{Shaded}

This seems pretty expensive and wasteful, so we'd like to maybe find a formula
to be able to do this using mathematical operations. So, let's explore!

\leavevmode\hypertarget{golSyms4DReverse}{}%
Please enable Javascript

After exploring this interactively, we can maybe think of some rules we can
apply.

\begin{enumerate}
\def\labelenumi{\arabic{enumi}.}
\tightlist
\item
  If we have a point \texttt{\textless{}z,z\textgreater{}} directly on the z=w
  diagonal, just use its five normal left/up neighbors with weight 1 each.
\item
  If we have a point \texttt{\textless{}z,z+1\textgreater{}} on the ``inner-er''
  diagonal, use its five normal left/up neighbors with weight 1, but its south
  and west neighbors have weight 2, and the point reflects onto \emph{itself}
  with weight 1.
\item
  If we're on \texttt{z=1} and we move into \texttt{z=0}, double that count
  (phew, the same rule as in the 3D case earlier)
\item
  If we're on w=1 and we move into w=0, double that count (same as before)
\item
  And\ldots I guess \texttt{\textless{}0,1\textgreater{}} reflects onto itself
  \emph{twice}? I guess that technically falls under a combination of rule 2 and
  rule 4, but we don't directly observe the motion into w=0 before it gets
  reflected so it has to be special-cased.
\end{enumerate}

Okay, those rules are \emph{sliiightly} more complicated than our 3D rules (``if
we go from z=1 to z=0, double-count it'')\ldots but they're at least mechanical
enough to code in, even if not beautiful. You can probably foresee that it might
be tough to generalize, but\ldots we'll tackle that when we get there :)

For now, we have a super-fast implementation of 4D GoL with our special
degeneracy! The runtime gets reduced by a factor of 8!

Now, onward to 5D!

\hypertarget{breaking-through}{%
\section{Breaking Through}\label{breaking-through}}

By stepping into looking at 5D, we've stepped into a brand new territory ---
we're now past what the original question was asking about, and into simply
exploring a personal curiosity for fun. No longer are we ``super-optimizing''
the puzzle --- we're now warping the original challenge to levels it was never
designed to handle.

It's difficult to visualize how things look in 5 dimensions, so this is where it
gets a little tricky to make any progress, mentally. The first thing we need to
figure out is how exactly we can generalize the ``z=w'' symmetry from 4D to be
able to take advantage of it in 5D\ldots and hopefully in a way that can
generalize to arbitrary dimensions. Along the way we'd also like to get rid of
our hacky 4D neighbor multiplicity rules and get something a little cleaner.

I struggled with for a while without making too much headway\ldots but on the
morning of Friday, December 18th, arguably one of the biggest revelations of the
entire journey was dropped by Michal Marsalek on u/cetttbycettt's reddit thread.
It was a big deal, because not only did it allow us to generalize our symmetries
to higher dimensions, but it also \emph{proved} a specific degeneracy that
allowed 10D simulation to be definitely 100\% \emph{solvable}.

\hypertarget{permutation-symmetry}{%
\subsection{Permutation Symmetry}\label{permutation-symmetry}}

Here was Michal's
\href{https://www.reddit.com/r/adventofcode/comments/kfjhwh/year_2020_day_17_part_2_using_symmetry_in_4d_space/gg9vr6m/}{historic
post}:

\begin{quote}
Yes, all the higher dimensions are interchangeable, there's nothing that
distinquishes them. That is, if there's an active cell at position (x,y,
a,b,c,d,e,f,g) then, there's also one at (x,y, c,d,g,e,f,a) and at all other
permutations, of coordinates a-g). That is the number of cells that one need to
track can be reduced by factor of
\includegraphics{https://latex.codecogs.com/png.latex?\%28d-2\%29\%21\%20\%5Ctimes\%202\%5E\%7Bd-2\%7D}
(at least if time goes to infinity).

\ldots we can use symmetries coming from permutations, to only track cells where
\includegraphics{https://latex.codecogs.com/png.latex?\%7Cx_0\%7C\%20\%3C\%2013\%2C\%5C\%2C\%20\%7Cx_1\%7C\%20\%3C\%2013\%2C\%5C\%2C\%200\%20\%5Cleq\%20x_2\%20\%5Cleq\%20x_3\%20\%5Cleq\%5C\%2C\%5Cldots\%5C\%2C\%20\%5Cleq\%20x_\%7Bd-1\%7D\%20\%5Cleq\%20t}.
There's
\includegraphics{https://latex.codecogs.com/png.latex?20\%5E2\%20\%5Ctimes\%20\%5Csum_\%7Bk\%3D0\%7D\%5E\%7Bt\%7D\%20\%7B\%20\%7Bd-3\%2Bk\%7D\%20\%5Cchoose\%20\%7Bk\%7D\%20\%7D}
such cells.
\end{quote}

\emph{(equations slightly modified)}

And boy was this exciting to read. First of all, it gave a way to generalize the
z=w symmetry: it's just permutation symmetry for all higher-dimensional
coordinates! But the big kicker here: See that last formula? Let's look at it
more closely, using
\includegraphics{https://latex.codecogs.com/png.latex?\%5Chat\%7Bd\%7D} to
represent \includegraphics{https://latex.codecogs.com/png.latex?d-2}, the number
of higher dimensions:

{[} 20\^{}2 \textbackslash times \textbackslash sum\_\{k=0\}\^{}\{t\} \{
\{\textbackslash hat\{d\}-1+k\}\textbackslash choose\{k\}
\}{]}(https://latex.codecogs.com/png.latex?\%0A20\%5E2\%20\%5Ctimes\%20\%5Csum\_\%7Bk\%3D0\%7D\%5E\%7Bt\%7D\%20\%7B\%20\%7B\%5Chat\%7Bd\%7D-1\%2Bk\%7D\%5Cchoose\%7Bk\%7D\%20\%7D\%0A
" 20\^{}2 \times \sum\_\{k=0\}\^{}\{t\} \{ \{\hat{d}-1+k\}\choose{k} \} ")

That sum has only the amount of terms fixed with the maximum timestamp! That
means we only ever have 6 terms to expand, no matter how high the dimensions are
--- at 10D and even 100D! Furthermore, we can simplify the above using
properties of the binomial distribution to get

{[} 20\^{}2 \textbackslash times \{
\{\textbackslash hat\{d\}+6\}\textbackslash choose\{6\}
\}{]}(https://latex.codecogs.com/png.latex?\%0A20\%5E2\%20\%5Ctimes\%20\%7B\%20\%7B\%5Chat\%7Bd\%7D\%2B6\%7D\%5Cchoose\%7B6\%7D\%20\%7D\%0A
" 20\^{}2 \times { {\hat{d}+6}\choose{6} } ")

This binomial coefficient is actually polynomial on
\includegraphics{https://latex.codecogs.com/png.latex?\%5Chat\%7Bd\%7D} --- it's
\includegraphics{https://latex.codecogs.com/png.latex?\%5Cfrac\%7B1\%7D\%7B6\%21\%7D\%20\%5Cprod_\%7Bk\%3D1\%7D\%5E6\%20\%28\%5Chat\%7Bd\%7D\%2Bk\%29}
--- a sixth degree polynomial (leading term
\includegraphics{https://latex.codecogs.com/png.latex?\%5Cfrac\%7B1\%7D\%7B6\%21\%7D\%20\%5Chat\%7Bd\%7D\%5E6}),
in fact. This means that we have turned the number of points we potentially need
to track from exponential
(\includegraphics{https://latex.codecogs.com/png.latex?O\%2813\%5E\%7B\%5Chat\%7Bd\%7D\%7D\%29})
to slightly smaller exponential
(\includegraphics{https://latex.codecogs.com/png.latex?O\%286\%5E\%7B\%5Chat\%7Bd\%7D\%7D\%29})
to now \emph{polynomial}
\includegraphics{https://latex.codecogs.com/png.latex?O\%28\%5Chat\%7Bd\%7D\%5E6\%29}!

So, not only did we figure out a way to generalize/compute our symmetries, we
also now know that this method lets us keep our point set \emph{polynomial} on
the dimension, instead of exponential.

To put a concrete number for context, for that dream of d=10, here are only
\includegraphics{https://latex.codecogs.com/png.latex?\%7B\%20\%7B8\%2B6\%7D\%20\%5Cchoose\%206\%20\%7D},
or 3003 potential unique \texttt{\textless{}z,w,...\textgreater{}} points, once
you factor out symmetries! The number went down from
\includegraphics{https://latex.codecogs.com/png.latex?13\%5E8} (815,730,721)
potential unique \texttt{\textless{}z,w,...\textgreater{}} points to
\includegraphics{https://latex.codecogs.com/png.latex?6\%5E8} (1,679,616)
potential unique points with positive/negative symmetry to just 3003 with
permutation symmetry.\footnote{For dramatic effect, I've omitted the fact that
  while there are only 3003 possible higher-dimensional points, there are
  \includegraphics{https://latex.codecogs.com/png.latex?20\%5E2\%20\%5Ctimes\%203003}
  actual unique points possible factoring in the 20x20 x-y grid. Still, it's a
  pretty big improvement over the original situation
  (\includegraphics{https://latex.codecogs.com/png.latex?20\%5E2\%20\%5Ctimes\%20815730721}).}
Furthermore, because of the blessing of dimensionality mentioned earlier, we can
expect more and more of those to be empty as we increase our dimensions.

And in a flash, 10D didn't feel like a dream anymore. It felt like an
inevitability. And now, it was a race to see who could get there first.

\hypertarget{the-race-to-10d}{%
\subsection{The Race to 10D}\label{the-race-to-10d}}

Unfortunately, the exact record of who reached and posted 10D first is a bit
lost to history due to reddit's editing records. A few people maintained and
updated their posts to prevent clutter, but the record and time stamp of when
they first hit 10D is lost. If any of them happens to read this and can more
accurately verify their times, I'd be happy to update!

For me, I'm sure I was not the first one, but in my chat logs I chimed into
freenode's \texttt{\#\#adventofcode-spoilers} channel in excitement in the wee
morning hours (PST) Saturday December 19th:

\begin{verbatim}
2020-12-19 02:32:42 | jle`    d=10 in 9m58s
2020-12-19 02:33:05 | jle`    hooray my goal :)
2020-12-19 02:33:08 | jle`    time to sleep now
2020-12-19 02:33:12 | xerox_  goodnight
2020-12-19 02:33:35 | jle`    xerox_: thanks :)
\end{verbatim}

Pure joy! :D

\href{https://www.reddit.com/r/adventofcode/comments/kfb6zx/day_17_getting_to_t6_at_for_higher_spoilerss/ggaaqsy/}{Peter
Tseng} made a post on Thursday night with times, but I can't remember if it
incorporated all the symmetries or originally included d=10.
\href{https://www.reddit.com/r/adventofcode/comments/kfb6zx/day_17_getting_to_t6_at_for_higher_spoilerss/ggsx9e9/}{Michal
Marsalek} was apparently able to implement the idea that they originally
proposed by the following Wednesday (December 23rd) in Nim to blow everyone's
time out of the water: 3.0 seconds!

At that point, it was pretty unbelievable to me that what started out as a dream
goal that we couldn't have completed on a supercomputer had, through successive
revelations and insights building on each other one by one, could now be done in
3 seconds.

But hey, I promised 100ms in the introduction, and a fast d=40, right?

With our goal completed, it was now time to dig in a little deeper and see how
far this baby could go.

\hypertarget{diving-deeper-terminology}{%
\subsection{Diving Deeper: Terminology}\label{diving-deeper-terminology}}

Before we go any further, let's take a break to clarify and introduce some
terminology we'll be using for the rest of this post.

\begin{itemize}
\item
  I've been using the word \textbf{slice} to talk about a 2D grid representing a
  single higher-dimensional \texttt{\textless{}z,w...\textgreater{}} coordinate
  --- they're the 13 grids in the 3D simulation and the 169 grids in the 4D
  simulation.
\item
  I've also been using \textbf{cell} to refer to an exact specific
  \texttt{\textless{}x,y,z,w,..\textgreater{}} spot --- they are the tiny
  squares inside each grid in the simulations above.
\item
  I'll start using the word
  \textbf{\href{https://www.youtube.com/watch?v=Dp8sYTlLQRY}{coset}} to refer
  the set of all of the duplicates of an \texttt{\textless{}x,y\textgreater{}}
  across all permutations and negations of
  \texttt{\textless{}z,w,q,..\textgreater{}}, since they all behave the same
  (they are either all on or all off together). So
  \texttt{\textless{}x,y,1,2\textgreater{}},
  \texttt{\textless{}x,y,2,1\textgreater{}},
  \texttt{\textless{}x,y,-1,2\textgreater{}},
  \texttt{\textless{}x,y,1,-2\textgreater{}},
  \texttt{\textless{}x,y,-1,-2\textgreater{}},
  \texttt{\textless{}x,y,-2,1\textgreater{}},
  \texttt{\textless{}x,y,2,-1\textgreater{}}, and
  \texttt{\textless{}x,y,-2,-1\textgreater{}} are all a part of the same coset,
  represented by the normalized form \texttt{\textless{}x,y,1,2\textgreater{}}.
  Now, during our simulation, we only need to simulate one member from each
  coset, because every member is identically present or not present. For the
  sake of implementation, we simulate the arbitrary \emph{normalized} (positive
  and sorted) member only. Because of this, we'll sometimes refer to the
  normalized item and the coset it represents as the same thing.
\item
  I'll also start using \textbf{slice coset} to talk about the set of all
  \texttt{\textless{}z,w,...\textgreater{}} slices) across its permutations and
  negations. The slices at z-w coordinates of
  \texttt{\textless{}1,2\textgreater{}}, \texttt{\textless{}2,1\textgreater{}},
  \texttt{\textless{}-1,2\textgreater{}},
  \texttt{\textless{}1,-2\textgreater{}},
  \texttt{\textless{}-1,-2\textgreater{}},
  \texttt{\textless{}-2,1\textgreater{}},
  \texttt{\textless{}2,-1\textgreater{}}, and
  \texttt{\textless{}-2,-1\textgreater{}} are all a part of the same coset,
  represented by the normalized form \texttt{\textless{}1,2\textgreater{}}. All
  of the slices at each of those zw coordinates will always be identical, so we
  can talk the state of a single slice at \texttt{\textless{}1,2\textgreater{}}
  as representing the state of its entire coset.

  Slice cosets are what are being highlighted on mouseovers for the 3D and 4D
  simulations. They are also what the big squares represent for the forward and
  backward neighbor demos of each: each slice stands in for their entire slice
  coset, and we show the amount of times each normalized slice coset element is
  a neighbor of the other.
\end{itemize}

\hypertarget{tackling-the-neighbor-problem}{%
\section{Tackling the Neighbor Problem}\label{tackling-the-neighbor-problem}}

My initial d=10 time clocked in at just under 10 minutes initially, but as early
as next Wednesday we knew that a sub-5 second time was possible. So where was
the gap?

Well, I didn't really know what to do about the neighbor multiplicity problem. I
was still brute-forcing by way of forward neighbors + normalizing (as in the
sample 4D python code snippet earlier). The naive brute-force method requires
computing \emph{all}
\includegraphics{https://latex.codecogs.com/png.latex?3\%5E\%7B\%20\%7B\%5Chat\%7Bd\%7D\%7D\%20\%7D\%20-\%201}
higher-dimensional neighbors\ldots so even though the number of points I'd have
to track grows polynomially, I still had that pesky exponential factor in
building my neighbor map. And at high dimensions, that exponential factor
dominates over everything.

So put on your hard hats and working boots \ldots{} we're going to dive deep
into the world of hyper-dimensional symmetries!

\hypertarget{five-dimensions}{%
\subsection{Five Dimensions}\label{five-dimensions}}

First, let's start visualizing how things look like in 5 dimensions, now that we
know what our slice coset/representative structure looks like. Partially to help
us gain an intuition for some of what's going on, and also partially to show
that intuition at the individual component level can only get so far.

It's a bit difficult to duplicate the same forward/reverse neighbor demos for 4D
as we had for 4D, so here's a different representation. Here is a demo of all of
the \texttt{\textless{}z,w,q\textgreater{}} slice cosets (the wedge of
normalized points we track for our implementation) and both their forward and
reverse neighbor weights of each other (computable using the method we used for
4D). The \texttt{q} axis is represented as stacked zw sections from left to
right.

\leavevmode\hypertarget{golSyms5D}{}%
Please enable Javascript

As you mouse-over a slice coset representative (a single square), all of its
neighbors will be highlighted, including reflections. The red dot on the left is
the ``forward'' neighbor (how many times that other slice is a neighbor of the
hovered slice) and the blue dot on the left is the ``reverse'' neighbor (how
many times the hovered slice is a neighbor of the other slice). For example, if
you hover over
\texttt{\textless{}z,w,q\textgreater{}=\textless{}1,3,4\textgreater{}}, you can
see that \texttt{\textless{}0,3,4\textgreater{}} is its neighbor twice, and
\texttt{\textless{}1,3,4\textgreater{}} is
\texttt{\textless{}0,3,4\textgreater{}}'s neighbor four times. These four times
come from the non-normalized reflections of
\texttt{\textless{}1,3,4\textgreater{}} at
\texttt{\textless{}1,3,4\textgreater{}},
\texttt{\textless{}1,4,3\textgreater{}},
\texttt{\textless{}-1,3,4\textgreater{}}, and
\texttt{\textless{}-1,4,3\textgreater{}}. Some squares are also neighbors to
themselves (like \texttt{\textless{}1,4,5\textgreater{}}, which reflects off of
the top edge at \texttt{\textless{}1,5,4\textgreater{}}) and some are not (like
\texttt{\textless{}1,3,5\textgreater{}}).
\href{https://www.youtube.com/watch?v=rSfebOXSBOE}{Mind bottling}!

At least one pattern we can see clearly is that if your points are 4 or lower,
the sum of all the red dots (the forward neighbors) is
\includegraphics{https://latex.codecogs.com/png.latex?3\%5E3-1} = 26, just like
how the sum of forward neighbors for interior points in 3D is
\includegraphics{https://latex.codecogs.com/png.latex?3\%5E2-1\%3D8}, and for 2D
is \includegraphics{https://latex.codecogs.com/png.latex?3\%5E2-1\%20\%3D\%202}.

Another very important pattern is that ``is a neighbor'' seems to be reversible:
the set of all \emph{forward} neighbors of a point is the same as all
\emph{reverse} neighbors of a point --- the only difference is the
multiplicities! But, wherever you see a red dot, you will also always see a blue
dot. No single-dot squares.

Anyway, you can explore this a little bit and try to come up with a set of
ad-hoc rules like we did for 4D\ldots but I think we've reached the limits of
how far that method can go. We can generate these values simply enough using the
expand-normalize-tabulate method we did for 4D, but there should be a way to
compute these weights \emph{directly}, in a clean fashion that doesn't require
branching special cases and patterns. It's clear that we are limited until we
can find this method.

\hypertarget{go-with-the-flow}{%
\subsection{Go with the Flow}\label{go-with-the-flow}}

One way to arrive at the key insight is to look at what the ``valid'' normalized
coordinates are, and find a better way to encode them in a way that is always
valid \emph{by construction}.

In our case, what do all our normalized
\texttt{\textless{}z,w,...\textgreater{}} coordinates look like? Well, they are
always non-decreasing, and always are less than the current timestep. Keeping
t=6 as our goal still, this means that valid coordinates in 10D are strings of
eight numbers, like \texttt{0,1,1,1,3,5,5,6}, or \texttt{0,0,3,4,4,4,6,6}, or
\texttt{1,1,2,3,3,4,5,5}.\footnote{It's also interesting to note that above 9D
  (where there are 7 higher-dimensional coordinates), there is always at least
  one duplicated number. Although I don't really know a way to explicitly
  exploit that fact even now, it does mean that there's a qualitative difference
  between 9D and below and 10D and above: anything above 9D is\ldots especially
  degenerate.}

Working directly with is not quite the best choice. For example, let's say we
want to compute a neighbor of \texttt{0,1,1,1,3,5,5,6}. Well, We can imagine
that the very first \texttt{1} moves to be a \texttt{2}, resulting in
\texttt{0,2,1,1,3,5,5,6}. However, we're now in un-normalized territory\ldots we
have to re-sort it to turn it into \texttt{0,1,1,2,3,5,5,6}. It's just not
something we can directly manipulate with simple rules and still stay in the
valid state space without complicated restrictions or rules.

If you stare at many different sample points (slice cosets, to be precise) for a
while, you might start to build an internal model in your head\ldots these
points are really all just consecutive runs of 1s, 2s, 3s, etc., at different
lengths. For example, you start seeing \texttt{0,1,1,1,3,5,5,6} as ``one 0,
three 1s, one 3, two 5s, one 6''. You also start seeing \texttt{0,0,3,4,4,4,6,6}
in your head as ``two 0s, one 3, three 4s, two 6s''.

You build this mental model in your head that helps you understand/process the
common structure between all of these points. And now, maybe we can turn that
model into an encoding. What if we encoded each higher-dimensional coordinate as
``number of each position seen?'' For example, we can encode
\texttt{0,1,1,1,3,5,5,6} as \texttt{1-3-0-1-0-2-1}: the first slot represents
how many 0s we have the second how many 1s, the next how many 2s, the next how
many 3s, etc. We can encode \texttt{0,0,3,4,4,4,6,6} as \texttt{2-0-0-1-3-0-2}
and \texttt{1,1,2,3,3,4,5,5} as \texttt{0-2-1-2-1-2-0}. All valid 10D points
encode to a vector of \emph{six} numbers (at t=6), which is valid as long as all
of the components sum to 8!
(\includegraphics{https://latex.codecogs.com/png.latex?10-2})

Similarly for, say, 20D's higher-dimensional coordinates: they're all vectors of
six numbers, valid as long as all components sum to 18!

And now, a ``valid transition'' becomes extremely easy to encode. You just
represent a valid transition as an amount ``flowing'' from one of those bins to
another. For example, turning a \texttt{1} into a \texttt{2} in
\texttt{1-3-0-1-0-2-1} turns it into \texttt{1-2-1-1-0-2-1}. We took one of the
three 1s and turned them into a single 2. In this method, we don't have to do
any normalization because this ``flowing'' operation automatically preserves the
sum-to-a-fixed-number invariant!

The only tricky mathy thing we need to remember is that we have to ask ``how
many ways'' we could move from three 1s and zero 2s to two 1s and one 2. In our
case, there are three ways: we could move the first 1, the second 1, or the
third 1. From \texttt{0,1,1,1,3,5,5,6}, we could have done
\texttt{0,2,1,1,3,5,5,6}, \texttt{0,1,2,1,3,5,5,6}, or \texttt{0,1,1,2,3,5,5,6},
and all of those would count as a possible way to transition from
\texttt{1-3-0-1-0-2-1} to \texttt{1-2-1-1-0-2-1}.

In general, here's a way to compute it: we start out with a single bin of 3, and
then we end up with a bin of 2 and a bin of 1. The number of transitions there
is
\includegraphics{https://latex.codecogs.com/png.latex?3\%21\%20\%2F\%20\%282\%21\%201\%21\%29}:
the number of ways we can arrange our original three elements into a bin of 2
and 1.

That's basically it! We can walk bin-by-bin, assembling a new vector from an old
one, by looking at the different bin-to-bin flows step-by-step. To find the
multiplicities, in the case of forward neighbors, we can start out with the
total ways to rearrange the total components we have (the
\includegraphics{https://latex.codecogs.com/png.latex?3\%21} we had earlier),
and at each step divide by the resulting bin sizes we end up with (the
\includegraphics{https://latex.codecogs.com/png.latex?2\%21\%201\%21} we had
earlier).

One final note: we do have to treat transitions from \texttt{0} to \texttt{1}
slightly differently, because some of them could have been transitions from 0 to
-1. For example, if we had \texttt{2-0-0-0} into \texttt{0-2-0-0}, you could
have had two 0s both turn into 1s, or you could have had one 0 turn into a 1 and
one turn into a -1 (which get reflected as 0 to 1 once you normalize), or you
could have had both 0s turn into -1s. All in the end this factors to a
multiplication of \includegraphics{https://latex.codecogs.com/png.latex?2\%5En},
\includegraphics{https://latex.codecogs.com/png.latex?n} being the number of
0-to-1 transitions, at the end.

Because of the special care taken for 0 to 1 transitions, it's more convenient
to move ``backwards'', from the highest component to the 0 component, so that
your options at the 0 component are already pre-determined for you by the
choices you have already made.

Alright, enough words, let's look at this in action! Here is a \emph{tree}
describing all the ways you can flow from bin to bin! As an example, let's look
the 6D case of ways each point is a neighbor of \texttt{0,2,2,3}
(\texttt{1-0-2-1}), which you can pick from the drop-down.

\leavevmode\hypertarget{golTreeForward}{}%
Please enable Javascript

As you can see, each ``branch'' in three (flowing from left to right) is a
different way to fill in the bin. At each node, the upper vector is the
``source'' vector, and the lower vector is the ``target'' vector we build
step-by-step. And bin-by-bin, we begin to move components from our source vector
into our target vector. The branches in the tree reflects different ways we can
commit a bin in our target vector. For example, at the very first split, we can
either pick our final vector to be \texttt{?-?-?-?-0} (leaving that 3 bin alone)
or \texttt{?-?-?-?-1} (swiping a component from that 3 bin in the source
vector). The number to the right of the node represents how we modify our
weights according to the choices we make according to the logic above. And all
other nodes on the far right are the end products: the actual neighbors, along
with their multiplicities.

If you mouse-over or tap a node, it'll highlight the trace from the beginning to
the node you are highlighting, so you can see all of the choices made, as well
as all the modifications made to our running multiplicity counter at each step.
It'll also show the ``regular'' representation, ie
\texttt{\textless{}{[}2,2{]},2,4\textgreater{}}, which means that that node has
already commited to having \texttt{\textless{}?,?,2,4\textgreater{}} in the
target vector, but still has two 2s in the source vector to pull in and
distribute.

One final thing we need to keep track of is to not count a point transitioning
to itself if it results from no actual internal changes. This can be done by
checking if each of our bin choices involved exactly no inter-bin flows.

Phew! That's a bit of a mathematical doozy, huh? But trust me when I say it's
easier to understand if you try out a few different points from the drop-down
menu and trace out the different possible paths, and how the multiplicities are
affected. After a few examples in different dimensions, it might start to make
sense. Try looking at the lower dimensions too to see if they match up with what
we figured out before.

\hypertarget{reverse-flow}{%
\subsection{Reverse Flow}\label{reverse-flow}}

That's great and all, but we're really here for the reverse neighbor weights,
right? Because that's what we actually need to compute the simulation.

Luckily, as we noted before in the 5D case, ``is a neighbor'' is a reversible
relationship: If a point is a forward neighbor, it is also a reverse neighbor.
This means that the branching structure for forward and reverse neighbor trees
are all the same. The only difference is the multiplicities.

The tweak here is that we go backwards: We start at 1, and then multiply by the
possibilities of getting the final total arrangement in the bin you have. For
example, moving from \texttt{0-4-1} to \texttt{0-2-3}, on that final bin,
represents an increase in multiplicity of
\includegraphics{https://latex.codecogs.com/png.latex?3\%21\%20\%2F\%20\%282\%21\%201\%21\%29}:
\includegraphics{https://latex.codecogs.com/png.latex?3\%21} for the final
count, and a \includegraphics{https://latex.codecogs.com/png.latex?2\%21} from
the ways the two 1 bin components could move into the 2 bin, and a
\includegraphics{https://latex.codecogs.com/png.latex?1\%21} from the way that
the single 2-bin component stays in place.

\leavevmode\hypertarget{golTreeReverse}{}%
Please enable Javascript

And that's it! We have tackled the reverse neighbor weights problem with some
branching bin flows and combinatorics!{[}\^{}honesty{]}

\hypertarget{stacks-on-stacks-visualizting-arbitrary-dimensions}{%
\section{Stacks On Stacks: Visualizting Arbitrary
Dimensions}\label{stacks-on-stacks-visualizting-arbitrary-dimensions}}

\leavevmode\hypertarget{golFlat}{}%
Please enable Javascript

\hypertarget{signoff}{%
\section{Signoff}\label{signoff}}

Hi, thanks for reading! You can reach me via email at
\href{mailto:justin@jle.im}{\nolinkurl{justin@jle.im}}, or at twitter at
\href{https://twitter.com/mstk}{@mstk}! This post and all others are published
under the \href{https://creativecommons.org/licenses/by-nc-nd/3.0/}{CC-BY-NC-ND
3.0} license. Corrections and edits via pull request are welcome and encouraged
at \href{https://github.com/mstksg/inCode}{the source repository}.

If you feel inclined, or this post was particularly helpful for you, why not
consider \href{https://www.patreon.com/justinle/overview}{supporting me on
Patreon}, or a \href{bitcoin:3D7rmAYgbDnp4gp4rf22THsGt74fNucPDU}{BTC donation}?
:)

\end{document}
