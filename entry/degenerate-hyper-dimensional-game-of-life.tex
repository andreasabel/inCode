\documentclass[]{article}
\usepackage{lmodern}
\usepackage{amssymb,amsmath}
\usepackage{ifxetex,ifluatex}
\usepackage{fixltx2e} % provides \textsubscript
\ifnum 0\ifxetex 1\fi\ifluatex 1\fi=0 % if pdftex
  \usepackage[T1]{fontenc}
  \usepackage[utf8]{inputenc}
\else % if luatex or xelatex
  \ifxetex
    \usepackage{mathspec}
    \usepackage{xltxtra,xunicode}
  \else
    \usepackage{fontspec}
  \fi
  \defaultfontfeatures{Mapping=tex-text,Scale=MatchLowercase}
  \newcommand{\euro}{€}
\fi
% use upquote if available, for straight quotes in verbatim environments
\IfFileExists{upquote.sty}{\usepackage{upquote}}{}
% use microtype if available
\IfFileExists{microtype.sty}{\usepackage{microtype}}{}
\usepackage[margin=1in]{geometry}
\usepackage{graphicx}
\makeatletter
\def\maxwidth{\ifdim\Gin@nat@width>\linewidth\linewidth\else\Gin@nat@width\fi}
\def\maxheight{\ifdim\Gin@nat@height>\textheight\textheight\else\Gin@nat@height\fi}
\makeatother
% Scale images if necessary, so that they will not overflow the page
% margins by default, and it is still possible to overwrite the defaults
% using explicit options in \includegraphics[width, height, ...]{}
\setkeys{Gin}{width=\maxwidth,height=\maxheight,keepaspectratio}
\ifxetex
  \usepackage[setpagesize=false, % page size defined by xetex
              unicode=false, % unicode breaks when used with xetex
              xetex]{hyperref}
\else
  \usepackage[unicode=true]{hyperref}
\fi
\hypersetup{breaklinks=true,
            bookmarks=true,
            pdfauthor={Justin Le},
            pdftitle={Degenerate Hyper-Dimensional Game of Life},
            colorlinks=true,
            citecolor=blue,
            urlcolor=blue,
            linkcolor=magenta,
            pdfborder={0 0 0}}
\urlstyle{same}  % don't use monospace font for urls
% Make links footnotes instead of hotlinks:
\renewcommand{\href}[2]{#2\footnote{\url{#1}}}
\setlength{\parindent}{0pt}
\setlength{\parskip}{6pt plus 2pt minus 1pt}
\setlength{\emergencystretch}{3em}  % prevent overfull lines
\setcounter{secnumdepth}{0}

\title{Degenerate Hyper-Dimensional Game of Life}
\author{Justin Le}

\begin{document}
\maketitle

\emph{Originally posted on
\textbf{\href{https://blog.jle.im/entry/degenerate-hyper-dimensional-game-of-life.html}{in
Code}}.}

tldr: Over the course of a month, we were able to successive new mathematical
properties of a ``degenerate'' hyper-dimensional game of life" to take a ``10
dimensions may just barely be possible on a supercomputer'' to ``10 dimensions
is easy enough to be run on any modern browser, and 40 dimensions can be reached
with a compiled language''. Includes interactive visualizations and simulations!

This is a story about breaking the degenerate hyper-dimensional game of life by
exploratory visualizations and math! Let's travel back in time: t'was the night
before December 17, 2020, The release of
\href{https://adventofcode.com/2020/day/17}{``Conway Cubes''}, day 17 of the
``Advent of Code'' (fun little coding puzzles building up to Christmas). One
part about Advent of Code I've always found especially fun is that, because the
problems are so self-contained and tidy, they are often \emph{open-ended} in the
interesting ways you can solve them or expand them.

On the surface, Day 17 seemed to essentially be a straightforward extension of
\href{https://en.wikipedia.org/wiki/Conway\%27s_Game_of_Life}{Conway's Game Of
Life} (``GoL''). GoL is a simulation played out on a 2d grid, where cells are
``on'' and ``off'', and at each step of the simulation, the on/off cells spread
and propagate in fascinating ways based on the state of their neighbors.

The twist of the Advent of Code puzzle is it asks what would happen if we played
out the rules of GoL in 3d, and then 4d! The ``starting conditions'' are a 8x8
2D grid picked out for each participant, and the puzzle solution is the number
of live cells after six steps. My personal starting conditions were:

\begin{verbatim}
#####..#
#..###.#
###.....
.#.#.#..
##.#..#.
######..
.##..###
###.####
\end{verbatim}

I submitted my answer with a direct implementation (scoring the 66th spot on the
leader board for that day)\ldots and that was that for the ``competitive'' part.
But the real fun always starts after! When discussing with some friends, we
started talking about the trade-offs of different implementations and realized
that the extra dimensionality was no joke\ldots as you upped the number of
dimensions, the number of points you have to consider grow as
\includegraphics{https://latex.codecogs.com/png.latex?O\%28\%282t\%2B6\%29\%5Ed\%29},
and the number of neighbors of each point to check grows as
\includegraphics{https://latex.codecogs.com/png.latex?O\%283\%5Ed\%29}. So for
4D it's definitely possible to solve naively\ldots but anything higher is going
to strain. My naive solution on 6D took three minutes, and 7D in a reasonable
amount of time (612,220,032 points with 2,186 neighbors each) seemed
\emph{impossible} on commercial consumer hardware because of the sheer number of
points in 7D space. But I thought\ldots what if a breakthrough in optimization
was possible? I set my goal as 10D (3,570,467,226,624 points with 59,048
neighbors each), not knowing if it was possible.

And soon\ldots a breakthrough did come! Someone brought up that if we look at
the 3d version, we see there's actually a \emph{mirror symmetry}! That is,
because everything starts off on the xy plane, with z=0, the resulting
progression must be symmetrical on both sides (positive and negative z).

\begin{figure}
\centering
\includegraphics{/img/entries/advent-gol/life3d.gif}
\caption{d=3 animation by
\href{https://www.reddit.com/r/adventofcode/comments/kfa3nr/2020_day_17_godot_cubes_i_think_i_went_a_bit_too/}{u/ZuBsPaCe}}
\end{figure}

In the end that means we only have to simulate one of the
``halves''/``quadrants'' of the higher-dimensional space, since all
``quadrants'' are identical! This saves down the number of points by a factor of
two for each extra dimension
(\includegraphics{https://latex.codecogs.com/png.latex?O\%282\%5E\%7Bd-2\%7D\%29}).
My 7D implementation completed in 6 minutes! 8D still hung forever, though.

Well, it didn't get us to d=10\ldots but this discovery completely changed how
we saw this puzzle. With one breakthrough down, we began to believe that there
would be more just around the corner, made possible by our problem's special
degeneracy (that is, that we start on a 2d slice).

Such a dream (as posed in
\href{https://www.reddit.com/r/adventofcode/comments/kfb6zx/day_17_getting_to_t6_at_for_higher_spoilerss/}{this
reddit thread I started}) turned into a month-long quest of breakthrough after
breakthrough, exploiting different aspects of this degeneracy! It was a long,
harrowing journey full of sudden twists and turns and bursts of excitement when
new innovations came. And in the end, the hopeful question ``What if d=10 was
possible?'' turned into ``d=10 in 100ms, d=40 in eight minutes.'' I even got
d=10 fast enough to run on easily any modern browser --- this post includes
those simulations! Furthermore, the whole journey became an adventure in the
power of visualization combined with abstract thinking.

So, let's take a deep dive --- deeper than you probably ever expected to dive
into any particular degenerate starting conditions of a hyper-dimensional game
of life :D

\hypertarget{the-baseline}{%
\section{The Baseline}\label{the-baseline}}

First of all, let's meet our friend for the rest of this journey. In the drawer
below, you can draw (with your mouse) the 8x8 grid you want to simulate for the
rest of this post. As you draw, the rest of the visualizations will update to
use this as their initial conditions.

\leavevmode\hypertarget{golDrawer}{}%
Please enable Javascript

And for fun, here's a 2D vanilla game of life implementation (for six time
steps) to test out your creation. I recommend trying out some of the
\href{https://en.wikipedia.org/wiki/Conway\%27s_Game_of_Life\#Examples_of_patterns}{interesting
well-known patterns}!

\leavevmode\hypertarget{gol2D}{}%
Please enable Javascript

Now that that's there, let's start at the beginning: what's the naive, baseline
solution?

A reasonable initial thought would be:

\begin{enumerate}
\def\labelenumi{\arabic{enumi}.}
\tightlist
\item
  Keep a 2D (or 3D, or 4D, etc.) array of booleans.
\item
  At each step:

  \begin{enumerate}
  \def\labelenumii{\alph{enumii}.}
  \tightlist
  \item
    Make a fresh copy of the entire space
    (\includegraphics{https://latex.codecogs.com/png.latex?O\%28n\%5Ed\%29}).
  \item
    Loop over each item in your array
    (\includegraphics{https://latex.codecogs.com/png.latex?O\%28n\%5Ed\%29}).
    Count all of the neighbors
    (\includegraphics{https://latex.codecogs.com/png.latex?O\%283\%5Ed\%29})
    that are \texttt{true} (``alive''), and write to the new array based on the
    rules table of GoL (2 or 3 neighbors for a live cell stays alive, 3
    neighbors for a dead cell turns alive).
  \end{enumerate}
\item
  You have a new array! Loop again six times.
\end{enumerate}

Sounds reasonable enough! And this does work for the 2D case pretty well (like
in the \href{https://adventofcode.com/2020/day/11}{Day 11 puzzle}). However,
there are some clear issues when moving into higher dimensions. The size of your
array grows exponentially on your dimension, and so does the number of neighbors
you'd have to check. And the
\href{https://en.wikipedia.org/wiki/Curse_of_dimensionality}{curse of
dimensionality} assures us that more and more of that array would become wasted
as the proportion of ``on'' points shrinks to zero for higher dimensions.

Oh, but what's that? The percentage of ``on'' points shrinks to zero for higher
dimensions? That actually sounds like something we can use to our advantage!
The\ldots{}\emph{blessing of dimensionality}, I daresay? Because we know the
vast majority of our points will be ``off'', there's another method.

\begin{enumerate}
\def\labelenumi{\arabic{enumi}.}
\tightlist
\item
  Keep a \emph{set} of points that are ``on''.
\item
  At each step:

  \begin{enumerate}
  \def\labelenumii{\alph{enumii}.}
  \item
    Initialize a dynamic map (key-value store) of points to integers (this will
    record the number of live neighbors of each point).
  \item
    For each step, iterate over each of your ``on'' points, expand all of their
    neighbors \includegraphics{https://latex.codecogs.com/png.latex?n_i}
    (\includegraphics{https://latex.codecogs.com/png.latex?\%28O\%283\%5Ed\%29\%29}),
    and increment the value associated with
    \includegraphics{https://latex.codecogs.com/png.latex?n_i} in your dynamic
    map.

    For example, if the point \texttt{{[}2,3{]}} is in your set of live points,
    you would add increment the map's values at keys \texttt{{[}1,2{]}},
    \texttt{{[}2,2{]}}, \texttt{{[}3,2{]}}, etc.: all 8 neighbors of
    \texttt{{[}2,3{]}}.
  \item
    Collect your new set of on points: keep all of the keys in your dynamic map
    corresponding to live points if their integers are 2 or 3, and keep all of
    the keys in your dynamic map corresponding to dead points if their integers
    are 3.
  \end{enumerate}
\item
  You have a new set! Loop again six times!
\end{enumerate}

I discuss this algorithm much more deeply with actual code in
\href{https://github.com/mstksg/advent-of-code-2020/blob/master/reflections-out/day17.md}{my
solutions write-up in my Advent of Code reflections journal}.

This method nets us a huge advantage because we now only have to loop over the
number of items that we know are alive! Any points far away from our set of
alive points can be properly ignored. This narrows down our huge iteration
space, and the benefits compound with every dimension due to the blessing of
dimensionality!\footnote{There is a small tweak (brought to our attention by
  \href{https://www.reddit.com/r/adventofcode/comments/kfb6zx/day_17_getting_to_t6_at_for_higher_spoilerss/ghmllf8}{Peter
  Tseng}) that people often add to this to avoid the costly check of the
  original set in step 2c: when you iterate over each point, normally you'd
  increment the eight neighbors' map values by 1. Instead, you can increment the
  eight neighbors' map values by \emph{2}, and then increment the point itself
  by 1. Then in the final integer under each key, \texttt{n\ /\ 2} or
  \texttt{n\ \textgreater{}\textgreater{}\ 1} gives you the number of neighbors
  and \texttt{n\ \%\ 2} (modulo) gives you whether or not that cell was alive.}

The nice thing about this method is that it's easy enough to generalize to any
dimension: instead of, say, keeping \texttt{{[}x,y{]}} in your set, just keep
\texttt{{[}x,y,z{]}}, or any length array of coordinates. One minor trick you
need to think through is generating all
\includegraphics{https://latex.codecogs.com/png.latex?3\%5Ed-1} neighbors, but
but that's going to come down to a d-ary
\href{https://observablehq.com/@d3/d3-cross}{cartesian product} of
\texttt{{[}-1,0,1{]}} to itself.\footnote{A cute trick (that I forgot who I
  heard it from first) with this is that if you cartesian-product
  \texttt{{[}0,-1,1{]}} to itself d times, the first item will be
  \texttt{{[}0,0,0..{]}}! This integrates extremely well with the other bit
  shifty tweak: to generate all of the contributions for d=3, cartesian product
  \texttt{{[}0,-1,1{]}} to itself three times, increment the key of the first
  resulting item by 1, and increment the key of the rest of them by 2.}

We can visualize this in 3D, but it might be nice to render this as a collection
of ``slices'' in 3D space. Each square represents a slice at a different Z
level: the middle one is z=0, the ones to the left and right are z=-1 and z=1,
etc.

\leavevmode\hypertarget{gol3D}{}%
Please enable Javascript

WIP

\leavevmode\hypertarget{gol4D}{}%
Please enable Javascript

\leavevmode\hypertarget{golFlat}{}%
Please enable Javascript

\leavevmode\hypertarget{golSyms3DForward}{}%
Please enable Javascript

\leavevmode\hypertarget{golSyms3DReverse}{}%
Please enable Javascript

\leavevmode\hypertarget{golSyms4DForward}{}%
Please enable Javascript

\leavevmode\hypertarget{golSyms4DReverse}{}%
Please enable Javascript

\leavevmode\hypertarget{golTreeForward}{}%
Please enable Javascript

\leavevmode\hypertarget{golTreeReverse}{}%
Please enable Javascript

\leavevmode\hypertarget{golSyms5D}{}%
Please enable Javascript

\begin{verbatim}
sim642  I wanted to ask this before but forgot: did anyone try to take advantage of the symmetry, e.g. in z axis in part 1?
sim642  Should halve the amount of calculations you have to do
sim642  Only some extra work at the end to differentiate z=0 and z>0 positions to know which to count twice
sim642  And in part 2 I feel like you could also exploit the symmetry in w axis simultaneously
\end{verbatim}

\hypertarget{signoff}{%
\section{Signoff}\label{signoff}}

Hi, thanks for reading! You can reach me via email at
\href{mailto:justin@jle.im}{\nolinkurl{justin@jle.im}}, or at twitter at
\href{https://twitter.com/mstk}{@mstk}! This post and all others are published
under the \href{https://creativecommons.org/licenses/by-nc-nd/3.0/}{CC-BY-NC-ND
3.0} license. Corrections and edits via pull request are welcome and encouraged
at \href{https://github.com/mstksg/inCode}{the source repository}.

If you feel inclined, or this post was particularly helpful for you, why not
consider \href{https://www.patreon.com/justinle/overview}{supporting me on
Patreon}, or a \href{bitcoin:3D7rmAYgbDnp4gp4rf22THsGt74fNucPDU}{BTC donation}?
:)

\end{document}
