\documentclass[]{article}
\usepackage{lmodern}
\usepackage{amssymb,amsmath}
\usepackage{ifxetex,ifluatex}
\usepackage{fixltx2e} % provides \textsubscript
\ifnum 0\ifxetex 1\fi\ifluatex 1\fi=0 % if pdftex
  \usepackage[T1]{fontenc}
  \usepackage[utf8]{inputenc}
\else % if luatex or xelatex
  \ifxetex
    \usepackage{mathspec}
    \usepackage{xltxtra,xunicode}
  \else
    \usepackage{fontspec}
  \fi
  \defaultfontfeatures{Mapping=tex-text,Scale=MatchLowercase}
  \newcommand{\euro}{€}
\fi
% use upquote if available, for straight quotes in verbatim environments
\IfFileExists{upquote.sty}{\usepackage{upquote}}{}
% use microtype if available
\IfFileExists{microtype.sty}{\usepackage{microtype}}{}
\usepackage[margin=1in]{geometry}
\usepackage{color}
\usepackage{fancyvrb}
\newcommand{\VerbBar}{|}
\newcommand{\VERB}{\Verb[commandchars=\\\{\}]}
\DefineVerbatimEnvironment{Highlighting}{Verbatim}{commandchars=\\\{\}}
% Add ',fontsize=\small' for more characters per line
\newenvironment{Shaded}{}{}
\newcommand{\AlertTok}[1]{\textcolor[rgb]{1.00,0.00,0.00}{\textbf{#1}}}
\newcommand{\AnnotationTok}[1]{\textcolor[rgb]{0.38,0.63,0.69}{\textbf{\textit{#1}}}}
\newcommand{\AttributeTok}[1]{\textcolor[rgb]{0.49,0.56,0.16}{#1}}
\newcommand{\BaseNTok}[1]{\textcolor[rgb]{0.25,0.63,0.44}{#1}}
\newcommand{\BuiltInTok}[1]{#1}
\newcommand{\CharTok}[1]{\textcolor[rgb]{0.25,0.44,0.63}{#1}}
\newcommand{\CommentTok}[1]{\textcolor[rgb]{0.38,0.63,0.69}{\textit{#1}}}
\newcommand{\CommentVarTok}[1]{\textcolor[rgb]{0.38,0.63,0.69}{\textbf{\textit{#1}}}}
\newcommand{\ConstantTok}[1]{\textcolor[rgb]{0.53,0.00,0.00}{#1}}
\newcommand{\ControlFlowTok}[1]{\textcolor[rgb]{0.00,0.44,0.13}{\textbf{#1}}}
\newcommand{\DataTypeTok}[1]{\textcolor[rgb]{0.56,0.13,0.00}{#1}}
\newcommand{\DecValTok}[1]{\textcolor[rgb]{0.25,0.63,0.44}{#1}}
\newcommand{\DocumentationTok}[1]{\textcolor[rgb]{0.73,0.13,0.13}{\textit{#1}}}
\newcommand{\ErrorTok}[1]{\textcolor[rgb]{1.00,0.00,0.00}{\textbf{#1}}}
\newcommand{\ExtensionTok}[1]{#1}
\newcommand{\FloatTok}[1]{\textcolor[rgb]{0.25,0.63,0.44}{#1}}
\newcommand{\FunctionTok}[1]{\textcolor[rgb]{0.02,0.16,0.49}{#1}}
\newcommand{\ImportTok}[1]{#1}
\newcommand{\InformationTok}[1]{\textcolor[rgb]{0.38,0.63,0.69}{\textbf{\textit{#1}}}}
\newcommand{\KeywordTok}[1]{\textcolor[rgb]{0.00,0.44,0.13}{\textbf{#1}}}
\newcommand{\NormalTok}[1]{#1}
\newcommand{\OperatorTok}[1]{\textcolor[rgb]{0.40,0.40,0.40}{#1}}
\newcommand{\OtherTok}[1]{\textcolor[rgb]{0.00,0.44,0.13}{#1}}
\newcommand{\PreprocessorTok}[1]{\textcolor[rgb]{0.74,0.48,0.00}{#1}}
\newcommand{\RegionMarkerTok}[1]{#1}
\newcommand{\SpecialCharTok}[1]{\textcolor[rgb]{0.25,0.44,0.63}{#1}}
\newcommand{\SpecialStringTok}[1]{\textcolor[rgb]{0.73,0.40,0.53}{#1}}
\newcommand{\StringTok}[1]{\textcolor[rgb]{0.25,0.44,0.63}{#1}}
\newcommand{\VariableTok}[1]{\textcolor[rgb]{0.10,0.09,0.49}{#1}}
\newcommand{\VerbatimStringTok}[1]{\textcolor[rgb]{0.25,0.44,0.63}{#1}}
\newcommand{\WarningTok}[1]{\textcolor[rgb]{0.38,0.63,0.69}{\textbf{\textit{#1}}}}
\usepackage{graphicx}
\makeatletter
\def\maxwidth{\ifdim\Gin@nat@width>\linewidth\linewidth\else\Gin@nat@width\fi}
\def\maxheight{\ifdim\Gin@nat@height>\textheight\textheight\else\Gin@nat@height\fi}
\makeatother
% Scale images if necessary, so that they will not overflow the page
% margins by default, and it is still possible to overwrite the defaults
% using explicit options in \includegraphics[width, height, ...]{}
\setkeys{Gin}{width=\maxwidth,height=\maxheight,keepaspectratio}
\ifxetex
  \usepackage[setpagesize=false, % page size defined by xetex
              unicode=false, % unicode breaks when used with xetex
              xetex]{hyperref}
\else
  \usepackage[unicode=true]{hyperref}
\fi
\hypersetup{breaklinks=true,
            bookmarks=true,
            pdfauthor={Justin Le},
            pdftitle={Breaking a Degenerate Hyper-Dimensional Game of Life},
            colorlinks=true,
            citecolor=blue,
            urlcolor=blue,
            linkcolor=magenta,
            pdfborder={0 0 0}}
\urlstyle{same}  % don't use monospace font for urls
% Make links footnotes instead of hotlinks:
\renewcommand{\href}[2]{#2\footnote{\url{#1}}}
\setlength{\parindent}{0pt}
\setlength{\parskip}{6pt plus 2pt minus 1pt}
\setlength{\emergencystretch}{3em}  % prevent overfull lines
\setcounter{secnumdepth}{0}

\title{Breaking a Degenerate Hyper-Dimensional Game of Life}
\author{Justin Le}

\begin{document}
\maketitle

\emph{Originally posted on
\textbf{\href{https://blog.jle.im/entry/degenerate-hyper-dimensional-game-of-life.html}{in
Code}}.}

tldr: Demonstrated with interactive visualizations and simulations --- over the
course of a month, we were able to successive new mathematical properties of a
``degenerate'' hyper-dimensional game of life" to take a ``7 dimensions may just
barely be possible on a supercomputer, could we ever reach 10 dimensions?'' to
``10 dimensions is easy enough to be run on any modern browser, and 40
dimensions can be reached with a compiled language''.

This is a story about breaking the degenerate hyper-dimensional game of life by
interactive exploratory visualizations and math!

T'was the night before Thursday, December 17, 2020, the release of
\href{https://adventofcode.com/2020/day/17}{``Conway Cubes''} (Day 17 of
\href{https://adventofcode.com/2020}{Advent of Code 2020}). Advent of Code is a
sequence of fun little themed coding puzzles building up to Christmas. I always
enjoyed these puzzles because they are so self-contained and tidy that they are
often \emph{open-ended} in the interesting ways you can solve them or expand
them (which I've written
\href{https://blog.jle.im/entries/series/+advent-of-code.html}{many blog posts
on}).

On the surface, Day 17 seemed to be a straightforward extension of
\href{https://en.wikipedia.org/wiki/Conway\%27s_Game_of_Life}{Conway's Game Of
Life} (``GoL''). GoL is a simulation played out on a 2D grid, where cells are
``on'' and ``off'', and at each step of the simulation the states spread and
propagate in interesting ways based on the state of their neighbors (a 2D
cellular automata). The twist of the Advent of Code puzzle is it asks what would
happen if we played out the rules of GoL in 3D instead, and then 4D!

I submitted my solution on my assigned puzzle input with a naive implementation
(scoring the 66th spots on the leaderboards for that day), concluding the
``competitive'' part. But the real fun always starts after! When discussing with
some friends, we started talking about the trade-offs of different
implementations and realized that the extra dimensionality was no joke: as you
upped the number of dimensions, the number of points you have to consider grow
exponentially, and so does the number of neighbors at each point to check. 4D
can be solved naively, but anything higher is going to be strained. My naive
solution on 6D took three minutes, and 7D in a reasonable amount of time
(requiring as much as 612,220,032 points with 2,186 neighbors each) seemed
\emph{impossible} on commercial consumer hardware because of the sheer number of
points in 7D space. But I thought\ldots what if a breakthrough in optimization
was possible? I set an (arbitrary) personal goal of reaching 10D
(3,570,467,226,624 points with 59,048 neighbors each), not knowing if it would
ever be possible.

And soon\ldots a breakthrough did come! Someone brought up that if we look at
the 3d version, we see there's actually a \emph{mirror symmetry}! Because
everything starts off on the xy plane, with z=0, the resulting progression must
be symmetrical on both sides (positive and negative z).

\begin{figure}
\centering
\includegraphics{/img/entries/advent-gol/life3d.gif}
\caption{3D GoL animation demonstrating mirror symmetry by
\href{https://www.reddit.com/r/adventofcode/comments/kfa3nr/2020_day_17_godot_cubes_i_think_i_went_a_bit_too/}{u/ZuBsPaCe}}
\end{figure}

This meant that we only have to simulate the \emph{positive} points (since the
negative points are identical). This saves down the number of points by a factor
of two for each extra dimension! Unfortunately, this wouldn't quite get us to
10D, but the discovery completely changed how we saw this puzzle. With one
breakthrough down, we began to believe that there would be more just around the
corner, made possible by our problem's special 2D-slice starting degeneracy.

Such a dream (as posed in
\href{https://www.reddit.com/r/adventofcode/comments/kfb6zx/day_17_getting_to_t6_at_for_higher_spoilerss/}{this
reddit thread I started}) turned into a month-long ride of breakthrough after
breakthrough, exploiting different aspects of this degeneracy. It was a month
full of sudden twists and turns and bursts of excitement whenever new
innovations came. And in the end, the hopeful question ``7D is barely in reach;
what if 10D was possible?'' turned into ``10D in 100ms, 40D in eight
minutes\ldots can we do 50D quickly?'' This post even includes simulations to
prove that we got 10D fast enough to run on easily on any modern browser. The
whole journey became an adventure in the power of visualization combined with
abstract thinking.

So, let's take a deep dive --- deeper than you probably ever expected to dive
into any particular degenerate starting conditions of a hyper-dimensional game
of life :D

There will be python code samples here and there, but just for context, my
actual solvers I developed along the way were
\href{https://github.com/mstksg/advent-of-code-2020/blob/master/src/AOC/Challenge/Day17.hs}{written
in Haskell}, and all of the solving logic embedded in this post was written in
Purescript
(\href{https://github.com/mstksg/inCode/blob/master/app-purescript/Gol.purs}{online
here}) and compiled to Javascript.

\hypertarget{table-of-figures}{%
\section{Table of Figures}\label{table-of-figures}}

For reference, the interactive elements in this post are:

\begin{enumerate}
\def\labelenumi{\arabic{enumi}.}
\tightlist
\item
  {[}Initial condition drawer{]}{[}\#golDrawer{]}
\item
  {[}2D Game of Life{]}{[}\#gol2D{]}
\item
  {[}3D Game of Life{]}{[}\#gol3D{]}
\item
  {[}3D Forward Neighbor Multiplicities{]}{[}\#golSyms3DForward{]} / {[}3D
  Reverse Neighbor Multiplicities{]}{[}\#golSyms3DReverse{]}
\item
  {[}5D Game of Life{]}{[}\#gol5D{]}
\item
  {[}4D Forward Neighbor Multiplicities{]}{[}\#golSyms4DForward{]} / {[}4D
  Reverse Neighbor Multiplicities{]}{[}\#golSyms4DReverse{]}
\item
  {[}5D Neighbor Multiplicities{]}{[}\#golSyms5D{]}
\item
  {[}General Neighbor Multiplicity Algorithm{]}{[}\#golTree{]}
\item
  {[}N-D Game of Life{]}{[}\#golFlat{]}
\end{enumerate}

\hypertarget{starting-off}{%
\section{Starting Off}\label{starting-off}}

First of all, let's meet our friend for the rest of this journey. In the drawer
below, you can \emph{draw} (with your mouse) the 8x8 grid you want to simulate
for the rest of this post. As you draw, the rest of the visualizations will
update to use this as their initial conditions, so feel free to jump back and
forth as you're reading to customize and change what you want to simulate.

\leavevmode\hypertarget{golDrawer}{}%
\textbf{Element 1:} Initial Condition Drawer

\leavevmode\hypertarget{golDrawerCont}{}%
Please enable Javascript

Here are some sample fun ones you can try out (click to load):

\begin{itemize}
\tightlist
\item
  The
  \href{https://blog.jle.im/entry/degenerate-hyper-dimensional-game-of-life.html?points=____._▝▖_._▀▘_.____}{classic
  glider}, a default if only for how iconic it is.
\item
  The
  \href{https://blog.jle.im/entry/degenerate-hyper-dimensional-game-of-life.html?points=____._▞▖_._▝__.____}{tub},
  which is a ``still-life'' in 2D, but explodes into a twinkling frenzy during
  the 4D animation.
\item
  The
  \href{https://blog.jle.im/entry/degenerate-hyper-dimensional-game-of-life.html?points=████.████.████.████}{full
  block}, which dies out in 2D but produces very appealing patterns in 3D and
  4D. A
  \href{https://blog.jle.im/entry/degenerate-hyper-dimensional-game-of-life.html?points=▛▀▀▜.▌▛▜▐.▌▙▟▐.▙▄▄▟}{bulls-eye}
  also yields interesting ``geometric'' patterns at higher dimensions. A
  \href{https://blog.jle.im/entry/degenerate-hyper-dimensional-game-of-life.html?points=████.▄▗▖▄.▀▝▘▀.████}{broken
  bar code} also yields explosively intricate alternating behavior at higher
  dimensions.
\item
  The
  \href{https://blog.jle.im/entry/degenerate-hyper-dimensional-game-of-life.html?points=██▟█.▜███.███▙.█▛██}{spiral
  galaxy} with rotational symmetry, which produces rich spiral galaxy patterns
  up until 6D, but then all of a sudden becomes blocky and bland at 7D and
  above.
\item
  My
  \href{https://blog.jle.im/entry/degenerate-hyper-dimensional-game-of-life.html?points=▛▜▙▐.▜▚▗_.█▟▄▘.▟▌▟█}{own
  personal assigned puzzle input}, to see what the typical input looks like that
  people had to run on December 17th.
\end{itemize}

I recommend trying out some of the
\href{https://en.wikipedia.org/wiki/Conway\%27s_Game_of_Life\#Examples_of_patterns}{other
interesting well-known patterns} from 2D GoL, as well! If you find any
interesting ones, I would be excited to hear about them!

For fun, here's a 2D vanilla game of life implementation (for six time steps) to
test out your creation. Note that some starting conditions will putter out in
2D, but expand forever in 3D+ due to the abundance of neighbors.

\leavevmode\hypertarget{gol2D}{}%
\textbf{Element 2:} 2D Game of Life

\leavevmode\hypertarget{gol2DCont}{}%
Please enable Javascript

Now that that's there, let's start at the beginning: what's the naive, baseline
solution?

A reasonable initial thought would be:

\begin{enumerate}
\def\labelenumi{\arabic{enumi}.}
\tightlist
\item
  Keep a 2D (or 3D, or 4D, etc.) array of booleans.
\item
  At each step:

  \begin{enumerate}
  \def\labelenumii{\alph{enumii}.}
  \tightlist
  \item
    Make a fresh copy of the entire space
    (\includegraphics{https://latex.codecogs.com/png.latex?O\%28n\%5Ed\%29}).
  \item
    Loop over each item in your array
    (\includegraphics{https://latex.codecogs.com/png.latex?O\%28n\%5Ed\%29}).
    Count all of the neighbors
    (\includegraphics{https://latex.codecogs.com/png.latex?O\%283\%5Ed\%29})
    that are \texttt{true} (``alive''), and write to the new array based on the
    rules table of GoL (2 or 3 neighbors for a live cell stays alive, 3
    neighbors for a dead cell turns alive).
  \end{enumerate}
\item
  You have a new array! Loop again six times.
\end{enumerate}

Sounds reasonable enough! This does work for the 2D case pretty well (like in
the \href{https://adventofcode.com/2020/day/11}{Day 11 puzzle}). However, there
are some clear issues when moving into higher dimensions. The size of your array
grows exponentially on your dimension, and so does the number of neighbors you'd
have to check. And the
\href{https://en.wikipedia.org/wiki/Curse_of_dimensionality}{curse of
dimensionality} assures us that more and more of that array would become wasted
as the proportion of ``on'' points shrinks to zero for higher dimensions.

Oh, but what's that? The percentage of ``on'' points shrinks to zero for higher
dimensions? That actually sounds like something we can use to our
\emph{advantage}! The \emph{blessing of dimensionality}, I daresay? Because we
know the vast majority of our points will be ``off'', there's another approach:

\begin{enumerate}
\def\labelenumi{\arabic{enumi}.}
\tightlist
\item
  Keep a \emph{set} of points that are ``on''.
\item
  At each step:

  \begin{enumerate}
  \def\labelenumii{\alph{enumii}.}
  \item
    Initialize a dynamic map (key-value store, like a python dict or Counter) of
    points to integers. This map associates each point to the number of live
    neighbors it has.
  \item
    For each step, iterate over each of your ``on'' points, expand all of their
    neighbors \includegraphics{https://latex.codecogs.com/png.latex?n_i}
    (\includegraphics{https://latex.codecogs.com/png.latex?\%28O\%283\%5Ed\%29\%29}),
    and increment the value associated with
    \includegraphics{https://latex.codecogs.com/png.latex?n_i} in your dynamic
    map.

    For example, if the point \texttt{\textless{}2,3\textgreater{}} is in your
    set of live points, you would add increment the map's values at keys
    \texttt{\textless{}1,2\textgreater{}},
    \texttt{\textless{}2,2\textgreater{}},
    \texttt{\textless{}3,2\textgreater{}}, etc.: all 8 neighbors of
    \texttt{\textless{}2,3\textgreater{}}.
  \item
    Collect your new set of ``on'' points: keep all of the keys in your dynamic
    map corresponding to live points if their integers are 2 or 3, and keep all
    of the keys in your dynamic map corresponding to dead points if their
    integers are 3.
  \end{enumerate}
\item
  You have a new set! Loop again six times!
\end{enumerate}

(I discuss this algorithm much more deeply with actual code in
\href{https://github.com/mstksg/advent-of-code-2020/blob/master/reflections-out/day17.md}{my
solutions write-up in my Advent of Code reflections journal})

This method nets us a huge advantage because we now only have to loop over the
number of items that we know are alive! Any points away from our set of alive
points can be properly ignored. This narrows down our huge iteration space, and
the benefits compound with every dimension due to the blessing of
dimensionality.\footnote{And\ldots there's actually a neat optimization we can
  use (brought to our attention by
  \href{https://www.reddit.com/r/adventofcode/comments/kfb6zx/day_17_getting_to_t6_at_for_higher_spoilerss/ghmllf8}{Peter
  Tseng}) to avoid the check of the original set in step 2c above: when you
  iterate over each point, increment the eight neighbors' map values by
  \emph{2}, and then increment the point itself by 1. Then in the final integer
  under each key, \texttt{n\ /\ 2} or
  \texttt{n\ \textgreater{}\textgreater{}\ 1} gives you the number of neighbors
  and \texttt{n\ \%\ 2} (modulus) gives you whether or not that cell was alive.}

Note that both methods can be straightforwardly generalized to any dimension; in
this second method, this just means a different sized tuple/vector in your set
of alive points (\texttt{{[}x,y{]}} vs.~\texttt{{[}x,y,z{]}}). One extra
concern, though, is that you need to think through generating all
\includegraphics{https://latex.codecogs.com/png.latex?3\%5Ed-1} neighbors:
that's going to come down to a d-ary
\href{https://observablehq.com/@d3/d3-cross}{cartesian product} of
\texttt{{[}-1,0,1{]}} to itself.

Here's a python implementation of the set-based method, using a nice trick I
learned from \href{https://twitter.com/phaazon_}{phaazon} and
\href{https://github.com/glguy}{glguy} to get the right neighbors by doing a
cartesian product against \texttt{{[}0,-1,1{]}} instead of
\texttt{{[}-1,0,1{]}}, which leaves the first item as the
\texttt{\textless{}0,0\textgreater{}} ``original point'' we want to exclude.

\begin{Shaded}
\begin{Highlighting}[]
\ImportTok{from}\NormalTok{ itertools }\ImportTok{import}\NormalTok{ islice, product}
\ImportTok{from}\NormalTok{ collections }\ImportTok{import}\NormalTok{ Counter}

\KeywordTok{def}\NormalTok{ mk\_neighbs(point):}
    \CommentTok{"""Return neighboring points, with the original point first}

\CommentTok{    (1, 2)}
\CommentTok{    => [(1, 2), (1, 1), (1, 3), (0, 2), (0, 1), (0, 3), (2, 2), (2, 1), (2, 3)]}
\CommentTok{    """}
    \ControlFlowTok{return} \BuiltInTok{list}\NormalTok{(product(}\OperatorTok{*}\NormalTok{[[x, x}\DecValTok{{-}1}\NormalTok{, x}\OperatorTok{+}\DecValTok{1}\NormalTok{] }\ControlFlowTok{for}\NormalTok{ x }\KeywordTok{in}\NormalTok{ point]))}

\KeywordTok{def}\NormalTok{ step\_naive(pts):}
    \CommentTok{"""Takes a set of points (tuples) and steps them in the simulation}
\CommentTok{    """}
\NormalTok{    neighbs }\OperatorTok{=}\NormalTok{ Counter()}
    \ControlFlowTok{for}\NormalTok{ point }\KeywordTok{in}\NormalTok{ pts:}
        \CommentTok{\# skip the first item, the original point}
\NormalTok{        neighbs }\OperatorTok{+=}\NormalTok{ Counter(mk\_neighbs(point)[}\DecValTok{1}\NormalTok{:])}

    \KeywordTok{def}\NormalTok{ validate(point, ncount):}
        \ControlFlowTok{if}\NormalTok{ point }\KeywordTok{in}\NormalTok{ pts:}
            \ControlFlowTok{return}\NormalTok{ ncount }\OperatorTok{==} \DecValTok{2} \KeywordTok{or}\NormalTok{ ncount }\OperatorTok{==} \DecValTok{3}
        \ControlFlowTok{else}\NormalTok{:}
            \ControlFlowTok{return}\NormalTok{ ncount }\OperatorTok{==} \DecValTok{3}

    \ControlFlowTok{return} \BuiltInTok{frozenset}\NormalTok{(p }\ControlFlowTok{for}\NormalTok{ p, n }\KeywordTok{in}\NormalTok{ neighbs.items() }\ControlFlowTok{if}\NormalTok{ validate(p, n))}
\end{Highlighting}
\end{Shaded}

\hypertarget{three-dimensions}{%
\section{Three Dimensions}\label{three-dimensions}}

Let's see how this looks for the 3D case! To make things easier to see, we can
render things in ``slices'' in 3D space: each grid represents a slice at a
different z level (ie, the z=0 square represents all squares
\texttt{\textless{}x,y,0\textgreater{}}, the z=1 square represents all squares
\texttt{\textless{}x,y,1\textgreater{}}, etc.). Press ``Play'' to have the
simulation cycle through 6 time steps!

\leavevmode\hypertarget{gol3D}{}%
\textbf{Element 3:} 3D Game of Life

\leavevmode\hypertarget{gol3DCont}{}%
Please enable Javascript

(Some ones I suggest trying out are the flowery
\href{https://blog.jle.im/entry/degenerate-hyper-dimensional-game-of-life.html?points=██▟█.▜███.███▙.█▛██}{spiral
galaxy} pattern and patterns with a single reflection symmetry, like the
\href{https://blog.jle.im/entry/degenerate-hyper-dimensional-game-of-life.html?points=████.▄▗▖▄.▀▝▘▀.████}{broken
bar code}; double symmetry like
\href{https://blog.jle.im/entry/degenerate-hyper-dimensional-game-of-life.html?points=▛▀▀▜.▌▛▜▐.▌▙▟▐.▙▄▄▟}{bulls-eye}
look nice too!)

In reality, each of those 13 slices above are stacked on top of each other in 3D
space. You'll see that your live cells spread out from the center z=0 slice,
which means they are actually spreading ``up and down'' the z axis.

If you mouse over (or tap) any individual tiny
\texttt{\textless{}x,y\textgreater{}} cell, you'll see the all of the 26
(\includegraphics{https://latex.codecogs.com/png.latex?3\%5Ed-1})
\texttt{\textless{}x,y,z\textgreater{}} 3D neighbors of the point you're
hovering over highlighted in blue --- these 26 points form a 3D cube around your
mouse through the stacked slices. You can use this cube to help see how the
simulation progresses. If your mouse is hovering over a live cell, and there are
2 or 3 live cells highlighted in your cube, it'll stay alive in the next time
step. If your mouse is hovering over a dead cell and there are exactly 3 live
cells highlighted in your cube, it will come alive in the next step.

\hypertarget{axis-reflection-symmetry}{%
\subsection{Axis Reflection Symmetry}\label{axis-reflection-symmetry}}

Try playing around with different initial conditions to see how they evolve! See
any patterns?

Freenode IRC user sim642 noticed something late into the night of December 16th:

\begin{quote}
I wanted to ask this before but forgot: did anyone try to take advantage of the
symmetry, e.g.~in z axis in part 1? Should halve the amount of calculations you
have to do.

Only some extra work at the end to differentiate z=0 and z\textgreater0
positions to know which to count twice And in part 2 I feel like you could also
exploit the symmetry in w axis simultaneously
\end{quote}

You might have seen this too: the entire thing has reflection symmetry across
z=0! z=1 is always the same as z=-1, z=2 is always the same as z=-2, etc.
Fundamentally, this is because our starting solution has a \emph{z-axis
symmetry}: the initial 2D slice is symmetric with reflections across z, because
z=0 for all of those points. This is the first ``degeneracy'' that this blog
post's title is referring to: the negative and positive directions are
interchangeable! This is reflected in the yellow highlight on hover: when you
mouse-over a z square, its corresponding reflected twin is highlighted, and will
always be identical.

This means that we actually only need to simulate \emph{positive} z's\ldots and
for our final answer we just ``un-reflect'' to get the total number.

Let's do this! Apparently the picture is slightly more complicated than simply
halving the points. We also need to change how to distribute neighbors. That's
because, once we commit to only keeping the positive z's, some cells need to be
double-counted as neighbors. In particular, any \texttt{z=0} cell would
previously had a neighbor at both \texttt{z=-1} and \texttt{z=1}\ldots but now
if we only keep the positive z's, it would have \texttt{z=1} as a neighbor
\emph{twice}.

The following interactive element lets you explore what this looks like:

\leavevmode\hypertarget{golSyms3DForward}{}%
\textbf{Element 4a:} 3D Forward Neighbor Multiplicities

\leavevmode\hypertarget{golSyms3DForwardCont}{}%
Please enable Javascript

Each square represents an entire ``slice'' of z. When you mouse-over or tap a
z-cell, its z-neighbors are highlighted with how many times that neighbor has to
be counted, and the green bar tells you from what direction that neighborship
arose from. For example, mousing over z=3, you will see z=2 and z=4 get
highlighted with the values ``1'' because they are neighbors of 3, on the left
and right side (respectively). Note that one neat property for all squares
(except for z=6, which goes off the scale) is that the ``total''
higher-dimensional neighbors is always 2
(\includegraphics{https://latex.codecogs.com/png.latex?3\%5E\%28d-2\%29-1});
\emph{where} those neighbors fall is just re-arranged slightly.

The tricky square is now z=0: if you mouse-over it, you'll see that it has a
single neighbor z=1 that is counted \emph{twice}, as a neighbor from both the
left and right side.

We can compute the above diagram by expanding any z to its neighbors (z-1, and
z+1), applying the absolute value function, and seeing how points double-up.
This gives us the \textbf{forward neighbors}, and we can directly use it for the
original ``keep the full array'' GoL implementation method.

However, for the ``keep active points and expand their neighbors'' GoL
implementation\ldots we have to find the opposite of this. Remember that to
build our ``neighbors map'' (the map of points to how many active neighbors they
have), we have each cell ``pro-actively'' add its contributions to all of its
neighbors. \texttt{\textless{}1,2,3\textgreater{}} is a neighbor to
\texttt{\textless{}1,3,4\textgreater{}} once, so when we expand
\texttt{\textless{}1,2,3\textgreater{}} we would increment the value in the map
at \texttt{\textless{}1,3,4\textgreater{}} by 1 because
\texttt{\textless{}1,2,3\textgreater{}} is a neighbor of
\texttt{\textless{}1,3,4\textgreater{}} once.

So the question becomes: how do we count \texttt{\textless{}1,3,1\textgreater{}}
expanding into \texttt{\textless{}1,3,0\textgreater{}}? Well, normally,
\texttt{\textless{}1,3,1\textgreater{}} is a neighbor of
\texttt{\textless{}1,3,0\textgreater{}} once. However, if we only keep the
normalized z values, \texttt{\textless{}1,3,1\textgreater{}} is a neighbor of
\texttt{\textless{}1,3,0\textgreater{}}\ldots twice! To compute the total
neighbor count of \texttt{\textless{}1,3,0\textgreater{}}, we have to count the
contribution from \texttt{\textless{}1,3,1\textgreater{}} twice (once for
\texttt{\textless{}1,3,1\textgreater{}} and once for
\texttt{\textless{}1,3,-1\textgreater{}}, which was normalized away).

That means we have to follow the original rules, but \emph{backwards}, like:

\leavevmode\hypertarget{golSyms3DReverse}{}%
\textbf{Element 4b:} 3D Reverse Neighbor Multiplicities

\leavevmode\hypertarget{golSyms3DReverseCont}{}%
Please enable Javascript

These are the \textbf{reverse neighbors}: how many times a given point counts as
a neighbor for its surrounding points. Here, mousing over z=1 shows that it
counts as a neighbor for z=0 twice, from both the left and the right. It also
counts as a neighbor for z=2 once (from the left side).

We can account for this by hard-coding the rules into our step algorithm: if our
z goes from \texttt{1} to \texttt{0}, increment its value twice in the neighbor
map. Otherwise, simply increment by 1 as normal.

This rule is relatively straightforward to implement, and as a result we now
halved our total number of points we need to keep and check for 3D! There's also
a nice way to generalize to arbitrary dimensions: for every
\texttt{1\ -\textgreater{}\ 0} transition in a higher dimension, multiply by
two. That means we reduce the number of 4D points we need to track by a factor
of four, the number of 5D points by a factor of eight, the number of 6D points
by a factor of 16\ldots{} now our total points to check only grows as
\includegraphics{https://latex.codecogs.com/png.latex?O\%28n\%5Ed\%20\%2F\%202\%5E\%7Bd-2\%7D\%29}
instead of
\includegraphics{https://latex.codecogs.com/png.latex?O\%28n\%5Ed\%29}!

Here is a python implementation of this generalization:

\begin{Shaded}
\begin{Highlighting}[]
\KeywordTok{def}\NormalTok{ axis\_weight(source,target):}
    \CommentTok{"""Retuns how many times the given source{-}>target transition should be}
\CommentTok{    counted in target\textquotesingle{}s neighbor count}
\CommentTok{    """}
    \CommentTok{\# ignore x,y}
\NormalTok{    higher\_source }\OperatorTok{=}\NormalTok{ source[}\DecValTok{2}\NormalTok{:]}
\NormalTok{    higher\_target }\OperatorTok{=}\NormalTok{ target[}\DecValTok{2}\NormalTok{:]}
    \ControlFlowTok{return} \DecValTok{2}\OperatorTok{**}\BuiltInTok{sum}\NormalTok{(}\OperatorTok{*}\NormalTok{[}\DecValTok{1} \ControlFlowTok{for}\NormalTok{ i,j }\KeywordTok{in} \BuiltInTok{zip}\NormalTok{(higher\_source, higher\_target) }\ControlFlowTok{if}\NormalTok{ i }\OperatorTok{==} \DecValTok{1} \KeywordTok{and}\NormalTok{ j }\OperatorTok{==} \DecValTok{0}\NormalTok{])}

\KeywordTok{def}\NormalTok{ mk\_positive\_neighbs(point):}
    \CommentTok{"""mk\_neighbs, but only with positive higher dimensional points}
\CommentTok{    """}
    \CommentTok{\# this is a very wasteful implementation, for demonstrative purposes}
    \ControlFlowTok{return}\NormalTok{ [ngb }\ControlFlowTok{for}\NormalTok{ ngb }\KeywordTok{in}\NormalTok{ mk\_neighbs(point) }\ControlFlowTok{if} \BuiltInTok{all}\NormalTok{ (i }\OperatorTok{>=} \DecValTok{0} \ControlFlowTok{for}\NormalTok{ i }\KeywordTok{in}\NormalTok{ ngb[}\DecValTok{2}\NormalTok{:])]}

\KeywordTok{def}\NormalTok{ step\_axis(pts):}
    \CommentTok{"""Takes a set of points (tuples) and steps them in the simulation}
\CommentTok{    according to axis{-}reflection symmetry.}
\CommentTok{    """}
\NormalTok{    neighbs }\OperatorTok{=}\NormalTok{ Counter()}
    \ControlFlowTok{for}\NormalTok{ point }\KeywordTok{in}\NormalTok{ pts:}
\NormalTok{        neighbs }\OperatorTok{+=}\NormalTok{ Counter(\{ ngb: axis\_weight(point,ngb) }\ControlFlowTok{for}\NormalTok{ ngb }\KeywordTok{in}\NormalTok{ mk\_positive\_neighbs(point)[}\DecValTok{1}\NormalTok{:]\})}

    \KeywordTok{def}\NormalTok{ validate(point, ncount):}
        \ControlFlowTok{if}\NormalTok{ point }\KeywordTok{in}\NormalTok{ pts:}
            \ControlFlowTok{return}\NormalTok{ ncount }\OperatorTok{==} \DecValTok{2} \KeywordTok{or}\NormalTok{ ncount }\OperatorTok{==} \DecValTok{3}
        \ControlFlowTok{else}\NormalTok{:}
            \ControlFlowTok{return}\NormalTok{ ncount }\OperatorTok{==} \DecValTok{3}

    \ControlFlowTok{return} \BuiltInTok{frozenset}\NormalTok{(p }\ControlFlowTok{for}\NormalTok{ p, n }\KeywordTok{in}\NormalTok{ neighbs.items() }\ControlFlowTok{if}\NormalTok{ validate(p, n))}
\end{Highlighting}
\end{Shaded}

This discovery late in the Tuesday night of the 16th was what inspired us to
believe and dream that more breakthroughs might be possible to bring things down
even further.

And those breakthroughs soon came!

\hypertarget{four-dimensions}{%
\section{Four Dimensions}\label{four-dimensions}}

Let's look at how 4D game works! We can visualize this by taking ``z-w'' slices
at different x-y planes. The labels in the following boxes are the
\texttt{\textless{}z,w\textgreater{}} of each slice. The very center is
\texttt{\textless{}z,w\textgreater{}\ =\ \textless{}0,0\textgreater{}} the row
in the middle from the top is \texttt{w=0}, and the column in the very middle
from the left is \texttt{z=0}. It's basically taking the 3D visualization above
and expanding it in an extra dimension. Press ``Play'' to run your initial
conditions!

\leavevmode\hypertarget{gol4D}{}%
\textbf{Element 5:} 4D Game of Life

\leavevmode\hypertarget{gol4DCont}{}%
Please enable Javascript

(Some patterns I recommend trying are the patterns with double reflection
symmetry like
\href{https://blog.jle.im/entry/degenerate-hyper-dimensional-game-of-life.html?points=▛▀▀▜.▌▛▜▐.▌▙▟▐.▙▄▄▟}{bulls-eye}
and
\href{https://blog.jle.im/entry/degenerate-hyper-dimensional-game-of-life.html?points=████.████.████.████}{full
block}, rotational symmetry like
\href{https://blog.jle.im/entry/degenerate-hyper-dimensional-game-of-life.html?points=██▟█.▜███.███▙.█▛██}{spiral
galaxy}, and single-reflection symmetries like
\href{https://blog.jle.im/entry/degenerate-hyper-dimensional-game-of-life.html?points=████.▄▗▖▄.▀▝▘▀.████}{broken
bar code}).

Most initial conditions will spread out from the center
\texttt{\textless{}z,w\textgreater{}\ =\ \textless{}0,0\textgreater{}} point
radially, spreading outwards into positive and negative z and w. Mouse-over or
tap any individual tiny \texttt{\textless{}x.y\textgreater{}} cell and you'll
see each of its 80
(\includegraphics{https://latex.codecogs.com/png.latex?3\%5Ed-1})
\texttt{\textless{}x,y,z,w\textgreater{}} 4D neighbors highlighted in blue,
forming a little 3x3x3 ``tesseract'' (4D cube, or hypercube). Like in the 3D
case, you can use this little hypercube to track how the simulation progresses:
if your mouse if hovering over a live cell with 2 or 3 live cells in its
hypercube, it'll stay alive in the next step, if it's hovering over a dead cell
with 3 live cells in its hypercube, it'll come alive in the next step.

\hypertarget{diagonal-reflection-symmetry}{%
\subsection{Diagonal Reflection Symmetry}\label{diagonal-reflection-symmetry}}

Play around and explore how simulations evolve! You will notice that the axis
reflection symmetry is still preserved, as expected, but four ways (the slice at
\texttt{\textless{}z,w\textgreater{}\ =\ \textless{}3,4\textgreater{}} is always
going to be identical to the slice at \texttt{\textless{}-3,4\textgreater{}},
\texttt{\textless{}3,-4\textgreater{}}, and
\texttt{\textless{}-3,-4\textgreater{}}). These are reflected in the ``deep
yellow'' highlights above when you mouse over a zw square. (Ignore the lighter
yellow highlights for now!)

And now, for the next big breakthrough: I think this situation shows the power
of visualization well, because this exact visualization was what reddit user
\emph{u/cetttbycett} was looking at when
\href{https://www.reddit.com/r/adventofcode/comments/kfjhwh/year_2020_day_17_part_2_using_symmetry_in_4d_space/}{they
made this post} late Thursday the 17th/early Friday the 18th\ldots and
everything changed \emph{forever}.

\begin{quote}
I noticed that the expansion of active cubes for part 2 is symmetric with
respect to two hyperplanes in 4d space: These hyperplanes can be described by w
= 0 and w-z = 0.

Using these symmetries could make the code nearly eight times as fast.I was
wondering if anyone tried that.
\end{quote}

What \emph{u/cetttbycettt} saw is what you can see now in the simulation above:
it's all of the \emph{light yellow} highlighted squares when you mouse-over
showing even \emph{more} identical slices. In addition to the z=0 and w=0 lines
(the two lines down the middle, up-down and left-right), we also have another
line of symmetry: z=w and w=z, the diagonal lines!

That's right, a zw slice at
\texttt{\textless{}z,w\textgreater{}=\textless{}3,4\textgreater{}} is
\emph{identical} to the one at \texttt{\textless{}4,3\textgreater{}}, and so
also \texttt{\textless{}-3,4\textgreater{}},
\texttt{\textless{}3,-4\textgreater{}}, \texttt{\textless{}-3,-4\textgreater{}},
\texttt{\textless{}-4,3\textgreater{}}, \texttt{\textless{}4,-3\textgreater{}},
and \texttt{\textless{}-4,-3\textgreater{}}! Each slice is potentially repeated
\emph{eight} times! The exceptions are the points on the lines of symmetry
themselves, which are each repeated four times, and also
\texttt{\textless{}z,w\textgreater{}=\textless{}0,0\textgreater{}}, which is in
its own class.

So, our first breakthrough meant that we only have to simulate \emph{positive}
coordinates (a single quadrant)\ldots our next breakthrough means that we only
have to simulate coordinates on a single ``wedge'' half-quadrant\ldots and then
duplicate those eight times at the end.

Arbitrarily, let's say we only simulate the north-by-northeast wedge, because
it's easy to normalize/compact all points onto that wedge: you just need to
absolute-value all the components and sort them. So,
\texttt{\textless{}4,-3\textgreater{}} gets ``normalized'' to
\texttt{\textless{}3,4\textgreater{}}.

We found a new symmetry now, but we run into the same issue as before: How do we
propagate neighbors? To help us see what's going on, let's look at the map of
neighbors between different \texttt{\textless{}z,w\textgreater{}} squares, for
the single zw wedge we are simulating.

\leavevmode\hypertarget{golSyms4DForward}{}%
\textbf{Element 6a:} 4D Forward Neighbor Multiplicities

\leavevmode\hypertarget{golSyms4DForwardCont}{}%
Please enable Javascript

These are the \emph{forward neighbors}; we can compute them by expanding a point
to its neighbors, and then normalizing our points and seeing how they double (or
quadruple) up.

\begin{Shaded}
\begin{Highlighting}[]
\KeywordTok{def}\NormalTok{ normalize(point):}
    \CommentTok{"""Normalize a point by sorting the absolute values}

\CommentTok{    (2, {-}1)}
\CommentTok{    => (1, 2)}
\CommentTok{    """}
    \ControlFlowTok{return} \BuiltInTok{tuple}\NormalTok{(}\BuiltInTok{sorted}\NormalTok{([}\BuiltInTok{abs}\NormalTok{(x) }\ControlFlowTok{for}\NormalTok{ x }\KeywordTok{in}\NormalTok{ point]))}

\KeywordTok{def}\NormalTok{ forward\_neighbs(point):}
    \CommentTok{"""Generate the higher{-}dimensional forward neighbors of a point}

\CommentTok{    (0, 1)}
\CommentTok{    => \{(0, 1): 2, (1, 2): 2, (1, 1): 2, (0, 0): 1, (0, 2): 1\}}
\CommentTok{    """}
    \ControlFlowTok{return}\NormalTok{ Counter([normalize(neighb) }\ControlFlowTok{for}\NormalTok{ neighb }\KeywordTok{in}\NormalTok{ mk\_neighbs(point)[}\DecValTok{1}\NormalTok{:]])}
\end{Highlighting}
\end{Shaded}

For example, mouse over
\texttt{\textless{}z,w\textgreater{}=\textless{}3,3\textgreater{}} and see it
has eight total higher-dimensional neighbors (like all points should, though
this visualization leaves out points at w\textgreater6). It's \emph{supposed} to
have a neighbor at \texttt{\textless{}4,3\textgreater{}}, but that gets
reflected back onto \texttt{\textless{}3,4\textgreater{}} during our
normalization process, so you see that the point
\texttt{\textless{}3,3\textgreater{}} has a neighbor at
\texttt{\textless{}3,4\textgreater{}} ``double-counted''. The green squares (in
the north and west positions) at \texttt{\textless{}3,4\textgreater{}} when you
hover over \texttt{\textless{}3,3\textgreater{}} show that
\texttt{\textless{}3,4\textgreater{}} is a neighbor of
\texttt{\textless{}3,3\textgreater{}} both to its north and to its west.

Also, we have something really odd show up for the first time. Mouse over a
point like \texttt{\textless{}z,w\textgreater{}=\textless{}2,3\textgreater{}}
and see that it has a neighbor in\ldots itself? What's going on here? Well, it
is \emph{supposed} to have a neighbor at \texttt{\textless{}3,2\textgreater{}}
but that gets normalized/reflected back onto
\texttt{\textless{}2,3\textgreater{}} --- it reflects onto itself! The green
square in the Southeast means that \texttt{\textless{}2,3\textgreater{}}'s
southeast neighbor is\ldots itself!

The forward neighbors are useful for understanding what's going on, but to
actually run our simulation we again need to find the \emph{reverse neighbors}:
from a given point A, how many times is that point a neighbor of another point
B?

We can compute this in brute-force using a cache: iterate over each point,
expand all its neighbors
\includegraphics{https://latex.codecogs.com/png.latex?a_i}, normalize that
neighbor, and then set
\includegraphics{https://latex.codecogs.com/png.latex?a_i} in the cache to the
multiplicity after normalization.

\begin{Shaded}
\begin{Highlighting}[]
\KeywordTok{def}\NormalTok{ reverse\_neighbs\_table(t\_max):}
    \CommentTok{"""Tabulate the reverse neighbors of all zw slices reachable before t\_max}
\CommentTok{    """}
\NormalTok{    weights }\OperatorTok{=}\NormalTok{ \{\}}

    \ControlFlowTok{for}\NormalTok{ i }\KeywordTok{in} \BuiltInTok{range}\NormalTok{(t\_max):}
        \ControlFlowTok{for}\NormalTok{ j }\KeywordTok{in} \BuiltInTok{range}\NormalTok{(i, t\_max):}
            \ControlFlowTok{for}\NormalTok{ neighb, ncount }\KeywordTok{in}\NormalTok{ forward\_neighbs((i, j)).items():}
                \ControlFlowTok{if}\NormalTok{ neighb }\KeywordTok{in}\NormalTok{ weights:}
\NormalTok{                    weights[neighb][(i, j)] }\OperatorTok{=}\NormalTok{ ncount}
                \ControlFlowTok{else}\NormalTok{:}
\NormalTok{                    weights[neighb] }\OperatorTok{=}\NormalTok{ \{(i, j): ncount\}}

    \ControlFlowTok{return}\NormalTok{ weights}
\end{Highlighting}
\end{Shaded}

This seems pretty expensive and wasteful, so we'd like to maybe find a formula
to be able to do this using mathematical operations. So, let's explore!

\leavevmode\hypertarget{golSyms4DReverse}{}%
\textbf{Element 6b:} 4D Reverse Neighbor Multiplicities

\leavevmode\hypertarget{golSyms4DReverseCont}{}%
Please enable Javascript

After exploring this interactively, we can maybe think of some rules we can
apply.

\begin{enumerate}
\def\labelenumi{\arabic{enumi}.}
\tightlist
\item
  If we have a point \texttt{\textless{}z,z\textgreater{}} directly on the z=w
  diagonal, just use its five normal left/up neighbors with weight 1 each.
\item
  If we have a point \texttt{\textless{}z,z+1\textgreater{}} on the ``inner-er''
  diagonal, use its five normal left/up neighbors with weight 1, but its south
  and west neighbors have weight 2, and the point reflects onto \emph{itself}
  with weight 1.
\item
  If we're on \texttt{z=1} and we move into \texttt{z=0}, double that count
  (phew, the same rule as in the 3D case earlier)
\item
  If we're on w=1 and we move into w=0, double that count (same as before)
\item
  And\ldots I guess \texttt{\textless{}0,1\textgreater{}} reflects onto itself
  \emph{twice}? I guess that technically falls under a combination of rule 2 and
  rule 4, but we don't directly observe the motion into w=0 before it gets
  reflected so it has to be special-cased.
\end{enumerate}

Okay, those rules are \emph{sliiightly} more complicated than our 3D rules (``if
we go from z=1 to z=0, double-count it'')\ldots but they're at least mechanical
enough to code in, even if not beautiful. You can probably foresee that it might
be tough to generalize, but\ldots we'll tackle that when we get there :)

For now, we have a super-fast implementation of 4D GoL with our special
degeneracy! The runtime gets reduced by a factor of 8!

For clarity, here's an example implementation of how we can do this
higher-dimensional wrangling:

\begin{Shaded}
\begin{Highlighting}[]
\KeywordTok{def}\NormalTok{ reverse\_neighbs(point):}
    \CommentTok{"""Return normalized higher{-}dimensional points, with their reverse}
\CommentTok{    multiplicities}

\CommentTok{    (0, 1)}
\CommentTok{    => \{(0, 0): 4, (0, 1): 2, (1, 1): 2, (0, 2): 1, (1, 2): 1\}}
\CommentTok{    """}
    \CommentTok{\# implementation elided}
    \CommentTok{\# one possibility is to lookup into reverse\_neighbs\_table(t\_max)[point]}
    \ControlFlowTok{return}\NormalTok{ \{\}}

\KeywordTok{def}\NormalTok{ step\_with\_weights(pts):}
\NormalTok{    neighbs }\OperatorTok{=}\NormalTok{ Counter()}
    \ControlFlowTok{for}\NormalTok{ point }\KeywordTok{in}\NormalTok{ pts:}
        \CommentTok{\# 2d component, <x,y>}
\NormalTok{        pt\_2d }\OperatorTok{=}\NormalTok{ point[:}\DecValTok{2}\NormalTok{]}
        \CommentTok{\# higher{-}dimension components, <z,w,...>}
\NormalTok{        pt\_nd }\OperatorTok{=}\NormalTok{ point[}\DecValTok{2}\NormalTok{:]}

        \CommentTok{\# insert neighbors in the same 2d slice, not including itself}
\NormalTok{        neighbs }\OperatorTok{+=}\NormalTok{ Counter([ngb }\OperatorTok{+}\NormalTok{ pt\_nd }\ControlFlowTok{for}\NormalTok{ ngb }\KeywordTok{in}\NormalTok{ mk\_neighbs(pt\_2d)[}\DecValTok{1}\NormalTok{:]])}
        \CommentTok{\# insert neighbors in the neighboring 2d slices}
\NormalTok{        neighbs }\OperatorTok{+=}\NormalTok{ Counter(\{(ngb\_2 }\OperatorTok{+}\NormalTok{ ngb\_n): wt}
                                \ControlFlowTok{for}\NormalTok{ ngb\_n, wt }\KeywordTok{in}\NormalTok{ reverse\_neighbs(pt\_nd)}
                                \ControlFlowTok{for}\NormalTok{ ngb\_2 }\KeywordTok{in}\NormalTok{ mk\_neighbs(pt\_2d)}
\NormalTok{                          \})}

    \KeywordTok{def}\NormalTok{ validate(point, ncount):}
        \ControlFlowTok{if}\NormalTok{ point }\KeywordTok{in}\NormalTok{ pts:}
            \ControlFlowTok{return}\NormalTok{ ncount }\OperatorTok{==} \DecValTok{2} \KeywordTok{or}\NormalTok{ ncount }\OperatorTok{==} \DecValTok{3}
        \ControlFlowTok{else}\NormalTok{:}
            \ControlFlowTok{return}\NormalTok{ ncount }\OperatorTok{==} \DecValTok{3}

    \ControlFlowTok{return} \BuiltInTok{frozenset}\NormalTok{(p }\ControlFlowTok{for}\NormalTok{ p, n }\KeywordTok{in}\NormalTok{ neighbs.items() }\ControlFlowTok{if}\NormalTok{ validate(p, n))}
\end{Highlighting}
\end{Shaded}

Now, onward to 5D!

\hypertarget{breaking-through}{%
\section{Breaking Through}\label{breaking-through}}

By stepping into 5D, we've moved into a brand new territory --- we're now past
what the original question was asking about, and into simply exploring a
personal curiosity for fun. No longer are we ``super-optimizing'' the puzzle ---
we're now warping the original challenge to levels it was never designed to
handle.

It's difficult to visualize how things look in 5 dimensions, so this is where it
gets a little tricky to make any progress, mentally. The first thing we need to
figure out is how exactly we can generalize the ``z=w'' symmetry from 4D to be
able to take advantage of it in 5D\ldots and hopefully in a way that can
generalize to arbitrary dimensions. Along the way we'd also like to get rid of
our hacky 4D neighbor multiplicity rules and get something a little cleaner.

I struggled with for a while without making too much headway\ldots but on the
morning of Friday, December 18th, arguably one of the biggest revelations of the
entire journey was dropped by Michal Marsalek on u/cetttbycettt's reddit thread.
It was a big deal, because not only did it allow us to generalize our symmetries
to higher dimensions, but it also \emph{proved} a specific degeneracy that
allowed 10D simulation to be definitely 100\% \emph{solvable}.

\hypertarget{permutation-symmetry}{%
\subsection{Permutation Symmetry}\label{permutation-symmetry}}

Here was Michal's
\href{https://www.reddit.com/r/adventofcode/comments/kfjhwh/year_2020_day_17_part_2_using_symmetry_in_4d_space/gg9vr6m/}{historic
post}:

\begin{quote}
Yes, all the higher dimensions are interchangeable, there's nothing that
distinquishes them. That is, if there's an active cell at position (x,y,
a,b,c,d,e,f,g) then, there's also one at (x,y, c,d,g,e,f,a) and at all other
permutations, of coordinates a-g). That is the number of cells that one need to
track can be reduced by factor of
\includegraphics{https://latex.codecogs.com/png.latex?\%28d-2\%29\%21\%20\%5Ctimes\%202\%5E\%7Bd-2\%7D}
(at least if time goes to infinity).

\ldots we can use symmetries coming from permutations, to only track cells where
\includegraphics{https://latex.codecogs.com/png.latex?\%7Cx_0\%7C\%20\%3C\%2013\%2C\%5C\%2C\%20\%7Cx_1\%7C\%20\%3C\%2013\%2C\%5C\%2C\%200\%20\%5Cleq\%20x_2\%20\%5Cleq\%20x_3\%20\%5Cleq\%5C\%2C\%5Cldots\%5C\%2C\%20\%5Cleq\%20x_\%7Bd-1\%7D\%20\%5Cleq\%20t}.
There's
\includegraphics{https://latex.codecogs.com/png.latex?20\%5E2\%20\%5Ctimes\%20\%5Csum_\%7Bk\%3D0\%7D\%5E\%7Bt\%7D\%20\%7B\%20\%7Bd-3\%2Bk\%7D\%20\%5Cchoose\%20\%7Bk\%7D\%20\%7D}
such cells.
\end{quote}

\emph{(equations slightly modified)}

And boy was this exciting to read. First of all, it gave a way to generalize the
z=w symmetry: it's just
\href{https://en.wikipedia.org/wiki/Permutation}{permutation symmetry} for all
higher-dimensional coordinates! But the big kicker here: See that last formula?
Let's look at it more closely, using
\includegraphics{https://latex.codecogs.com/png.latex?\%5Chat\%7Bd\%7D} to
represent \includegraphics{https://latex.codecogs.com/png.latex?d-2}, the number
of higher dimensions:

{[} 20\^{}2 \textbackslash times \textbackslash sum\_\{k=0\}\^{}\{t\} \{
\{\textbackslash hat\{d\}-1+k\}\textbackslash choose\{k\}
\}{]}(https://latex.codecogs.com/png.latex?\%0A20\%5E2\%20\%5Ctimes\%20\%5Csum\_\%7Bk\%3D0\%7D\%5E\%7Bt\%7D\%20\%7B\%20\%7B\%5Chat\%7Bd\%7D-1\%2Bk\%7D\%5Cchoose\%7Bk\%7D\%20\%7D\%0A
" 20\^{}2 \times \sum\_\{k=0\}\^{}\{t\} \{ \{\hat{d}-1+k\}\choose{k} \} ")

(That notation is the
\href{https://en.wikipedia.org/wiki/Binomial_coefficient}{binomial coefficient},
if you aren't familiar with it.) Note that the summation has a \emph{fixed
number of terms} (with respect to dimension)! That means we only ever have 6
terms to expand, no matter how high the dimensions are --- at 10D and even 100D!
Furthermore, we can simplify the above using properties of binomial coefficients
to get

{[} 20\^{}2 \textbackslash times \{
\{\textbackslash hat\{d\}+6\}\textbackslash choose\{6\}
\}{]}(https://latex.codecogs.com/png.latex?\%0A20\%5E2\%20\%5Ctimes\%20\%7B\%20\%7B\%5Chat\%7Bd\%7D\%2B6\%7D\%5Cchoose\%7B6\%7D\%20\%7D\%0A
" 20\^{}2 \times { {\hat{d}+6}\choose{6} } ")

This binomial coefficient is actually polynomial on
\includegraphics{https://latex.codecogs.com/png.latex?\%5Chat\%7Bd\%7D} --- it's
\includegraphics{https://latex.codecogs.com/png.latex?\%5Cfrac\%7B1\%7D\%7B6\%21\%7D\%20\%5Cprod_\%7Bk\%3D1\%7D\%5E6\%20\%28\%5Chat\%7Bd\%7D\%2Bk\%29}
--- a sixth degree polynomial (leading term
\includegraphics{https://latex.codecogs.com/png.latex?\%5Cfrac\%7B1\%7D\%7B6\%21\%7D\%20\%5Chat\%7Bd\%7D\%5E6}),
in fact. This means that we have turned the number of points we potentially need
to track from exponential
(\includegraphics{https://latex.codecogs.com/png.latex?O\%2813\%5E\%7B\%5Chat\%7Bd\%7D\%7D\%29})
to slightly smaller exponential
(\includegraphics{https://latex.codecogs.com/png.latex?O\%286\%5E\%7B\%5Chat\%7Bd\%7D\%7D\%29})
to now \emph{polynomial}
\includegraphics{https://latex.codecogs.com/png.latex?O\%28\%5Chat\%7Bd\%7D\%5E6\%29}!

So, not only did we figure out a way to generalize/compute our symmetries, we
also now know that this method lets us keep our point set \emph{polynomial} on
the dimension, instead of exponential.

To put a concrete number for context, for that dream of 10D, here are only
\includegraphics{https://latex.codecogs.com/png.latex?\%7B\%20\%7B8\%2B6\%7D\%20\%5Cchoose\%206\%20\%7D},
or 3003 potential unique \texttt{\textless{}z,w,...\textgreater{}} points, once
you factor out symmetries! The number went down from
\includegraphics{https://latex.codecogs.com/png.latex?13\%5E8} (815,730,721)
potential unique \texttt{\textless{}z,w,...\textgreater{}} points to
\includegraphics{https://latex.codecogs.com/png.latex?6\%5E8} (1,679,616)
potential unique points with positive/negative symmetry to just 3003 with
permutation symmetry.\footnote{For dramatic effect, I've omitted the fact that
  while there are only 3003 possible higher-dimensional points, there are
  \includegraphics{https://latex.codecogs.com/png.latex?20\%5E2\%20\%5Ctimes\%203003}
  actual unique points possible factoring in the 20x20 x-y grid. Still, it's a
  pretty big improvement over the original situation
  (\includegraphics{https://latex.codecogs.com/png.latex?20\%5E2\%20\%5Ctimes\%20815730721}).}
Furthermore, because of the blessing of dimensionality mentioned earlier, we can
expect more and more of those to be empty as we increase our dimensions.

And in a flash, 10D didn't feel like a dream anymore; it felt like an
inevitability. And now, it was just a race to see who could get there first.

\hypertarget{the-race-to-10d}{%
\subsection{The Race to 10D}\label{the-race-to-10d}}

Unfortunately, the exact record of who reached and posted 10D first is a bit
lost to history due to reddit's editing records (not that ``first'' is
necessarily a meaningful title to hold; there's no prize, and everyone is
working at their own pace). A few people maintained and updated their posts to
prevent clutter, but the record and time stamp of when they first posted 10D is
lost. If any of them happens to read this and can more accurately verify their
times, I'd be happy to update!

For me, I'm sure I was not the first one, but in my chat logs I see that I
chimed into freenode's \texttt{\#\#adventofcode-spoilers} channel in excitement
in the wee morning hours (PST) Saturday December 19th:

\begin{verbatim}
2020-12-19 02:32:42   jle`    | d=10 in 9m58s
2020-12-19 02:33:05   jle`    | hooray my goal :)
2020-12-19 02:33:08   jle`    | time to sleep now
2020-12-19 02:33:12   xerox_  | goodnight
2020-12-19 02:33:35   jle`    | xerox_: thanks :)
\end{verbatim}

Pure joy! :D

\href{https://www.reddit.com/r/adventofcode/comments/kfb6zx/day_17_getting_to_t6_at_for_higher_spoilerss/ggaaqsy/}{Peter
Tseng} made a post on \emph{Thursday} night with times, but I can't remember if
it incorporated all the symmetries or originally included 10D,
\href{https://www.reddit.com/r/adventofcode/comments/kfb6zx/day_17_getting_to_t6_at_for_higher_spoilerss/ggsx9e9/}{Michal
Marsalek} was able to implement the idea that they originally proposed by the
following Wednesday (December 23rd) in Nim to blow everyone's time out of the
water: 3 seconds flat!

At that point, it was pretty unbelievable to me that what started out as a dream
goal that we couldn't have reasonably completed on a supercomputer had, through
successive revelations and insights building on each other one by one, could now
be done in 3 seconds.

But hey, I promised 100ms in the introduction, and a way to reach 40D, right?

With our original goal completed, it was now time to dig in a little deeper and
see how far this baby could go.

\hypertarget{diving-deeper-terminology}{%
\subsection{Diving Deeper: Terminology}\label{diving-deeper-terminology}}

Before we go any further, let's take a break to clarify and introduce some
terminology we'll be using for the rest of this post.

\begin{itemize}
\item
  I've been using the word \textbf{slice} to talk about a 2D grid representing a
  single higher-dimensional \texttt{\textless{}z,w...\textgreater{}} coordinate
  --- they're the 13 grids in {[}the 3D simulation{]}{[}\#gol3D{]} and the 169
  grids in {[}the 4D simulation{]}{[}\#gol4D{]}.
\item
  I've also been using \textbf{cell} to refer to an exact specific
  \texttt{\textless{}x,y,z,w,..\textgreater{}} spot --- they are the tiny
  squares inside each grid in the simulations above.
\item
  I'll start using the word
  \textbf{\href{https://www.youtube.com/watch?v=Dp8sYTlLQRY}{coset}} to refer
  the set of all of the duplicates of an \texttt{\textless{}x,y\textgreater{}}
  across all permutations and negations of
  \texttt{\textless{}z,w,q,..\textgreater{}}, since they all behave the same
  (they are either all on or all off together). So
  \texttt{\textless{}x,y,1,2\textgreater{}},
  \texttt{\textless{}x,y,2,1\textgreater{}},
  \texttt{\textless{}x,y,-1,2\textgreater{}},
  \texttt{\textless{}x,y,1,-2\textgreater{}},
  \texttt{\textless{}x,y,-1,-2\textgreater{}},
  \texttt{\textless{}x,y,-2,1\textgreater{}},
  \texttt{\textless{}x,y,2,-1\textgreater{}}, and
  \texttt{\textless{}x,y,-2,-1\textgreater{}} are all a part of the same coset,
  represented by the normalized form \texttt{\textless{}x,y,1,2\textgreater{}}.
  Now, during our simulation, we only need to simulate one member from each
  coset, because every member is identically present or not present. For the
  sake of implementation, we simulate the arbitrary \emph{normalized} (positive
  and sorted) member only. Because of this, we'll sometimes refer to the
  normalized item and the coset it represents as the same thing.
\item
  I'll also start using \textbf{slice coset} to talk about the set of all
  \texttt{\textless{}z,w,...\textgreater{}} slices) across its permutations and
  negations. The slices at z-w coordinates of
  \texttt{\textless{}1,2\textgreater{}}, \texttt{\textless{}2,1\textgreater{}},
  \texttt{\textless{}-1,2\textgreater{}},
  \texttt{\textless{}1,-2\textgreater{}},
  \texttt{\textless{}-1,-2\textgreater{}},
  \texttt{\textless{}-2,1\textgreater{}},
  \texttt{\textless{}2,-1\textgreater{}}, and
  \texttt{\textless{}-2,-1\textgreater{}} are all a part of the same coset,
  represented by the normalized form \texttt{\textless{}1,2\textgreater{}}. All
  of the slices at each of those zw coordinates will always be identical, so we
  can talk the state of a single slice at \texttt{\textless{}1,2\textgreater{}}
  as representing the state of its entire coset.

  Slice cosets are what are being highlighted on mouseovers for the 3D and 4D
  simulations. They are also what the big squares represent for the {[}3D
  Forward Neighbors{]}{[}\#golSyms3DForward{]}, the {[}3D Reverse
  Neighbors{]}{[}\#golSyms3DReverse{]}, the {[}4D Reverse
  Neighbors{]}{[}\#golSyms4DForward{]}, and the {[}4D Reverse
  neighbors{]}{[}\#golSyms4DReverse{]} elements: each slice stands in for their
  entire slice coset, and we show the amount of times each normalized slice
  coset element is a neighbor of the other.
\end{itemize}

\hypertarget{tackling-the-neighbor-problem}{%
\section{Tackling the Neighbor Problem}\label{tackling-the-neighbor-problem}}

My initial d=10 time clocked in at just under 10 minutes initially, but as early
as next Wednesday we knew that a sub-5 second time was possible. So where was
the gap?

Well, I didn't really know what to do about the neighbor multiplicity problem. I
was still brute-forcing by way of forward neighbors + normalizing (as in the
sample 4D python code snippet earlier). The naive brute-force method requires
computing \emph{all}
\includegraphics{https://latex.codecogs.com/png.latex?3\%5E\%7B\%20\%7B\%5Chat\%7Bd\%7D\%7D\%20\%7D\%20-\%201}
higher-dimensional neighbors. So, even though the number of points I'd have to
track grows polynomially, I still had that pesky exponential factor in building
my neighbor map. And at high dimensions, that exponential factor dominates over
everything.

So put on your hard hats and working boots \ldots{} we're going to dive deep
into the world of hyper-dimensional symmetries!

\hypertarget{five-dimensions}{%
\subsection{Five Dimensions}\label{five-dimensions}}

First, let's start visualizing how things look like in 5 dimensions, now that we
know what our slice coset/representative structure looks like. Partially to help
us gain an intuition for some of what's going on, and also partially to show
that intuition at the individual component level can only get so far.

It's a bit difficult to duplicate the same {[}forward
neighbor{]}{[}\#golSyms4DForward{]}/{[}reverse
neighbor{]}{[}\#golSyms4DReverse{]} interactive elements as we had for 4D, so
here's a different representation. Here is an interactive element of all of the
\texttt{\textless{}z,w,q\textgreater{}} slice cosets (the wedge of normalized
points we track for our implementation) and both their forward and reverse
neighbor weights of each other (computable using the method we used for 4D). The
\texttt{q} axis is represented as stacked zw sections from left to right.

\leavevmode\hypertarget{golSyms5D}{}%
\textbf{Element 7:} 5D Neighbor Multiplicities

\leavevmode\hypertarget{golSyms5DCont}{}%
Please enable Javascript

As you mouse-over a slice coset representative (a single square), all of its
neighbors will be highlighted, including reflections. The red dot on the left is
the ``forward'' neighbor multiplicity (how many times that other slice is a
neighbor of the hovered slice) and the blue dot on the left is the ``reverse''
neighbor multiplicity (how many times the hovered slice is a neighbor of the
other slice). For example, if you hover over
\texttt{\textless{}z,w,q\textgreater{}=\textless{}1,3,4\textgreater{}}, you can
see that \texttt{\textless{}0,3,4\textgreater{}} is its neighbor twice, and
\texttt{\textless{}1,3,4\textgreater{}} is
\texttt{\textless{}0,3,4\textgreater{}}'s neighbor four times. These four times
come from the non-normalized reflections of
\texttt{\textless{}1,3,4\textgreater{}} at
\texttt{\textless{}1,3,4\textgreater{}},
\texttt{\textless{}1,4,3\textgreater{}},
\texttt{\textless{}-1,3,4\textgreater{}}, and
\texttt{\textless{}-1,4,3\textgreater{}}. Some squares are also neighbors to
themselves (like \texttt{\textless{}1,4,5\textgreater{}}, which reflects off of
the top edge at \texttt{\textless{}1,5,4\textgreater{}}) and some are not (like
\texttt{\textless{}1,3,5\textgreater{}}).
\href{https://www.youtube.com/watch?v=rSfebOXSBOE}{Mind bottling}!

At least one pattern we can see clearly is that if you are at a point where each
component is 4 or lower (so it doesn't run off the edge of our table), the sum
of all the red dots (the forward neighbors) is
\includegraphics{https://latex.codecogs.com/png.latex?3\%5E3-1} = 26, just like
how the sum of forward neighbors for interior points in 3D is
\includegraphics{https://latex.codecogs.com/png.latex?3\%5E2-1} = 8, and for 2D
is \includegraphics{https://latex.codecogs.com/png.latex?3\%5E1-1} = 2.

Another very important pattern is that ``is a neighbor'' seems to be reversible:
the set of all \emph{forward} neighbors of a point is the same as all
\emph{reverse} neighbors of a point --- the only difference is the
multiplicities! That is, wherever you see a red dot, you will also always see a
blue dot. No single-dot squares.

Anyway, you can explore this a little bit and try to come up with a set of
ad-hoc rules like we did for 4D, but I think we've reached the limits of how far
that method can go. We can generate these values simply enough using the
expand-normalize-tabulate method we did for 4D, but it's pretty inefficient, and
there should be a way to compute these weights \emph{directly} in a clean
fashion that doesn't require hard-coding special cases and patterns. It's clear
that we are limited until we can find this method.

\hypertarget{go-with-the-flow}{%
\subsection{Go with the Flow}\label{go-with-the-flow}}

What do all our valid normalized \texttt{\textless{}z,w,...\textgreater{}}
coordinates look like? Well, they are always non-decreasing, and always are less
than or equal to the current timestep. Keeping t=6 as our goal still, this means
that valid coordinates in 10D are strings of eight numbers, like
\texttt{0,1,1,1,3,5,5,6}, or \texttt{0,0,3,4,4,4,6,6}, or
\texttt{1,1,2,3,3,4,5,5}.\footnote{It's also interesting to note that above 9D
  (where there are 7 higher-dimensional coordinates), there is always at least
  one duplicated number. Although I don't really know a way to explicitly
  exploit that fact even now, it does mean that there's a qualitative difference
  between 9D and below and 10D and above: anything above 9D is\ldots especially
  degenerate.}

We run into problems working with this format, though. For example, if we're
computing a neighbor of \texttt{0,1,1,1,3,5,5,6}, we can imagine that the very
first \texttt{1} could move to be a \texttt{2}, resulting in
\texttt{0,2,1,1,3,5,5,6}. However, we're now in \emph{un-normalized}
territory\ldots we have to re-sort it to turn it into \texttt{0,1,1,2,3,5,5,6}.
This encoding isn't something we can directly manipulate in a nice way.

Because of how everything is always non-decreasing, what if we instead encoded
each higher-dimensional coordinate as ``count of each value seen?'' For example,
we can encode \texttt{0,1,1,1,3,5,5,6} as \texttt{1-3-0-1-0-2-1}: the first slot
represents how many 0s we have, the second how many 1s, the next how many 2s,
the next how many 3s, etc. We can encode \texttt{0,0,3,4,4,4,6,6} as
\texttt{2-0-0-1-3-0-2} and \texttt{1,1,2,3,3,4,5,5} as \texttt{0-2-1-2-1-2-0}.
The \emph{sum} of the components gives you the total number of higher dimensions
(ie, 10D vectors sum to 8)

And now, a ``valid transition'' becomes easy to enforce: it's an amount
``flowing'' from one of those bins to another. For example, turning a \texttt{1}
into a \texttt{2} in \texttt{1-3-0-1-0-2-1} turns it into
\texttt{1-2-1-1-0-2-1}. We took one of the three 1s and turned them into a
single 2. This ``flowing'' operation automatically gives us a valid coordinate
without any renormalizing necessary! This gives us an algorithm to compute
neighbors: we can walk bin-to-bin, ``flowing'' components from our origin vector
to our new vector.

We now have a way to compute neighbors without requiring renormalization! We no
longer have to try all
\includegraphics{https://latex.codecogs.com/png.latex?3\%5Ed-1} (exponential)
candidates and re-normalize: we can now only iterate through the ones we care
about.

However, the tricky math is in computing the multiplicities. Interestingly
enough, in this case, the \emph{reverse} direction is actually easier to
conceptualize than the forward direction. Good for us, because that's the
direction we actually need!

If we start at \texttt{0-2-1-3} (\texttt{1,1,2,3,3,3}) and ``flow'' to, say,
\texttt{0-0-5-0} (\texttt{2,2,2,2,2}) and dump all our bins into 2. How many
ways could this flow happen? The answer happens to be the
\href{https://en.wikipedia.org/wiki/Multinomial_theorem}{multinomial
coefficient}
\includegraphics{https://latex.codecogs.com/png.latex?5\%20\%5Cchoose\%20\%7B2\%2C1\%2C3\%7D}
(or
\includegraphics{https://latex.codecogs.com/png.latex?5\%21\%20\%2F\%20\%282\%21\%201\%21\%203\%21\%29}):
there are \includegraphics{https://latex.codecogs.com/png.latex?5\%21} ways to
end up with 5 in the bin, but that \texttt{5} came from contributions of
\texttt{2+1+3} from either side, and so we divide by the ways we could pick from
those contributing bins (2!, 1!, and 3!).

Finally, we have to treat multiplicities for transitions from 0 to 1 slightly
differently, because they can arise either a 0 to 1 transition or a 0 to -1
transition. This comes out to a multiplication of
\includegraphics{https://latex.codecogs.com/png.latex?2\%5En} at the end (n
being the amount of 0-to-1 flow). Because of this special care, it's actually
more convenient to fill in bin-by-bin ``backwards'', from the 6 slot to the 5
slot to the 4 slot, etc., because your options at the 0 component are already
pre-determined for you by the choices you have already made. It keeps the tree a
more manageable shape.

Alright, enough words, let's look at this in action! The following element shows
the \emph{tree} describing all the ways you can flow from bin to bin. As an
example, let's look the 6D case of ways each point is a neighbor of
\texttt{0,2,2,3} (\texttt{1-0-2-1}), which you can pick from the drop-down.

\leavevmode\hypertarget{golTree}{}%
\textbf{Element 8:} General Neighbor Multiplicity Algorithm

\leavevmode\hypertarget{golTreeCont}{}%
Please enable Javascript

As you can see, each ``branch'' in the tree (reading from left to right) is a
different way to fill in a given bin, from right to left. At each node, the
displayed upper vector is the ``source'' vector, and the lower vector is the
``target'' vector we build bin-by-bin. Bin-by-bin, we begin to move components
from our source vector into our target vector. The branches in the tree reflects
different ways we can commit a bin in our target vector. For example, at the
very first split, we can either pick our final vector to be \texttt{?-?-?-?-0}
(leaving that 3 bin alone) or \texttt{?-?-?-?-1} (swiping a component from that
3 bin in the source vector). The operation shown to the right of the node
represents how we modify our weights according to the choices we make according
to the logic above. The nodes on the far right also show the end products: the
actual neighbors, along with their multiplicities.

If you mouse-over or tap a node, it'll highlight the trace from the beginning to
the node you are highlighting, so you can see all of the choices made, as well
as all the operations applied to our running multiplicity counter at each step.
It'll also show the contributions from the left, center, and right of the
current bin being picked (the
\includegraphics{https://latex.codecogs.com/png.latex?2\%2B1\%2B3} in the
example above), and also the ``regular'' vector representation. For example,
\texttt{\textless{}{[}2,2{]},2,4\textgreater{}} means that that node has already
committed to having \texttt{\textless{}?,?,2,4\textgreater{}} in the target
vector, but still has two 2s in the source vector to pull in and distribute.

One final thing we need to keep track of is to not count a point transitioning
to itself if it results from no actual internal changes. This can be done by
checking if each of our bin choices involved exactly no inter-bin flows (they
were all of the form \texttt{0+x+0}).

Phew! That's a bit of a mathematical doozy, huh? But trust me when I say it's
easier to understand if play around with the interactive element and follow
along the traces. After a few examples in different dimensions, it might start
to make sense. Try looking at the lower dimensions too to see if they match up
with what we figured out before.

You can also flip the switch on the element to compute reverse and forward
neighbors. Luckily, as we noted before, if a point is a forward neighbor, it is
also a reverse neighbor. This means that the branching structure for forward and
reverse neighbor trees are exactly the same; the only difference is how the
multiplicities are calculated. In this case, the forward direction is just the
original calculation reversed! The diagram shows how the multiplicities are
accumulated; feel free to try to work out exactly how this works as an exercise!

And with that, we have tackled the reverse neighbor weights problem with some
branching bin flows and combinatorics!

\hypertarget{stacks-on-stacks-visualizing-arbitrary-dimensions}{%
\section{Stacks On Stacks: Visualizing Arbitrary
Dimensions}\label{stacks-on-stacks-visualizing-arbitrary-dimensions}}

You might have noticed that ever since our 4D record, we haven't had a new
visualization of simulation, despite now having higher dimensions in our grasp.
Why not?

Well, there's the question of \emph{how} you might even visualize this. You can
``zoom out'' and take higher-dimensional slices of our 4D visualization and
repeat this ad nauseum, but that doesn't really add anything or give any insight
as to what's really going on.

I believe that this is what caused us to all collectively get ``stuck'' together
around 20 dimensions. The rush of the revelations one after within a single week
pushed us into trying many different things. I had a couple of dead-end forays
into pre-cacheing and had a lot of code (that I was ecstatic to be able to later
delete) working with an sqlite3 database.\footnote{One thing I did discover
  (that I won't spend too much time on here) is a way to
  \href{https://www.reddit.com/r/adventofcode/comments/kfb6zx/day_17_getting_to_t6_at_for_higher_spoilerss/gim68l0/}{index
  into an enumeration} of all of the slice cosets (that is, all the normalized
  higher-dimensional coordinates). I no longer store
  \texttt{\textless{}z,w,...\textgreater{}} points as vectors, but rather as a
  single integer representing their index in that enumeration, which is easier
  to access and store. I also found a way to do streaming decoding and encoding
  between that index and the components it represents, allowing me to stream
  neighbor weights in constant time. This dense index encoding was actually
  really useful in implementing the Javascript demos on this page :)}

Another factor that probably contributed to the overall lull was that Advent of
Code was still running, and we all still enjoyed doing new puzzles every day.
But soon, Christmas passed, the daily rush of doing new puzzles faded, and we
started to return back to tinkering on this hyper-dimensional game of life. It
wouldn't be until January 1st (just over two weeks after the puzzle originally
came out) that a new revelation arose that would pave the way shoot far past
20D.

It was
\href{https://www.reddit.com/r/adventofcode/comments/kfb6zx/day_17_getting_to_t6_at_for_higher_spoilerss/ghre3ce/}{Michal
Marsalek's coset counts post} that set up the stage. From the beginning, Michal
had always tracked the number of cell cosets at the end of the simulation (the
number of active ``normalized'' cells), and had been exploring the relationship
between dimension and coset counts. The discovery was that after a certain
``saturation point'' (6D for Michael's set, 9D for Peter's set, 7D for my set),
all of the coset counts were \emph{perfectly quadratic}! For mine, it followed
the relationship
\includegraphics{https://latex.codecogs.com/png.latex?d\%5E2\%20\%2B\%20109d\%20\%2B\%2070}
exactly for 7D and higher.

My best guess as to why this was happening is that, at 7D and above, we enter a
domain of points where, before t=6, \emph{every} point is at some sort of
reflective boundary. Remember that even for 4D, we had really odd behavior at
the reflective boundaries/edge of the wedge. There wasn't enough room in many
points to ``stretch their wings'' --- every single one is at one reflective
boundary or another. Being a boundary point corresponds to having a ``bins''
encoding with any bin greater than one or anything in the 0 bin (ie,
\texttt{1-0-0-0} and \texttt{0-2-0} are all points on a reflective boundary).

Unfortunately, having a closed-form way to compute coset counts doesn't actually
give us a way to compute the final state itself, since it doesn't tell us
\emph{which} cosets are active, just how many. However, this prompted me to
investigate a little bit more about what was causing this pattern, and how these
cosets were distributed. To do this, tried a new way to visualize things.

In our simulation, x and y components are fundamentally different from the
others; we could actually talk about each point as a tuple
\texttt{(\textless{}x,y\textgreater{},\ \{higher\ dims\})}. Also, things are
usually ``dense'' in \texttt{\textless{}x,y\textgreater{}} (a significant
fraction of the xy space has at least one point), but ``sparse'' in higher
dimensions (a very small fraction of the higher-dimensional space actually has a
point in it). So actually, instead of keeping our active points as a set of
cosets, we can treat it as a map of \texttt{\textless{}x,y\textgreater{}} points
to the higher-dim cosets that live ``under them''. Instead of keeping one giant
set as:

\begin{verbatim}
{<1,2,1,1,3>, <3,1,1,1,4>, <1,2,0,0,5>, <4,2,3,4,4>, <3,1,2,2,2>}
\end{verbatim}

we could instead a map of sets:

\begin{verbatim}
<1,2>: { <1,1,3>, <0,0,5> }
<3,1>: { <1,1,4>, <2,2,2> }
<4,2>: { <3,3,4> }
\end{verbatim}

and propagate \emph{that} in our simulation. I like to call those sets under
each 2d point (ie, the
\texttt{\{\textless{}1,1,3\textgreater{},\ \textless{}0,0,5\textgreater{}\}}) a
``coset stack''.

I did this initially to investigate the nature of the cosets that were showing
up, but once I plotted it and animated things, I realized that in doing this, we
are reducing the entire hyper-dimensional problem \emph{back to a 2D cellular
automata}! This whole thing becomes reframed\ldots instead of a mind-bending
hyper-dimensional deal, it's now simply \emph{normal 2D cellular automata} with
funky rules! It's like a normal 2D game of life, but with funky rules for 2D
points spreading to each other.

\begin{Shaded}
\begin{Highlighting}[]
\KeywordTok{def}\NormalTok{ step\_with\_stacks(stacks):}
\NormalTok{    neighbs }\OperatorTok{=}\NormalTok{ \{\}}
    \ControlFlowTok{for}\NormalTok{ pt\_2d, pt\_stack }\KeywordTok{in}\NormalTok{ stacks.items():}
        \CommentTok{\# higher{-}dimension components}
        \ControlFlowTok{for}\NormalTok{ pt\_nd }\KeywordTok{in}\NormalTok{ pt\_stack:}
\NormalTok{            rev\_neighbs }\OperatorTok{=}\NormalTok{ Counter(reverse\_neighbs(pt\_nd))}
\NormalTok{            rev\_neighbs\_incl\_self }\OperatorTok{=}\NormalTok{ rev\_neighbs }\OperatorTok{+}\NormalTok{ Counter(pt\_nd)}

            \CommentTok{\# the actual propagation}
            \CommentTok{\# 1. add in the same stack; don\textquotesingle{}t include self}
            \ControlFlowTok{if}\NormalTok{ pt\_2d }\KeywordTok{in}\NormalTok{ neighbs:}
\NormalTok{                neighbs[pt\_2d] }\OperatorTok{+=}\NormalTok{ rev\_neighbs}
            \ControlFlowTok{else}\NormalTok{:}
\NormalTok{                neighbs[pt\_2d] }\OperatorTok{=}\NormalTok{ rev\_neighbs}
            \CommentTok{\# 2. add to nieghboring stacks}
            \ControlFlowTok{for}\NormalTok{ ngb\_2 }\KeywordTok{in}\NormalTok{ mk\_neighbs(pt\_2d)[}\DecValTok{1}\NormalTok{:]:}
                \CommentTok{\# add to neighboring stacks; include self}
                \ControlFlowTok{if}\NormalTok{ ngb\_2 }\KeywordTok{in}\NormalTok{ neighbs:}
\NormalTok{                    neighbs[ngb\_2] }\OperatorTok{+=}\NormalTok{ rev\_neighbs\_incl\_self}
                \ControlFlowTok{else}\NormalTok{:}
\NormalTok{                    neighbs[ngb\_2] }\OperatorTok{=}\NormalTok{ rev\_neighbs\_incl\_self}

    \KeywordTok{def}\NormalTok{ validate(pt\_2d, pt\_nd, ncount):}
        \ControlFlowTok{if}\NormalTok{ pt\_nd }\KeywordTok{in}\NormalTok{ stacks[pt\_2d]:}
            \ControlFlowTok{return}\NormalTok{ ncount }\OperatorTok{==} \DecValTok{2} \KeywordTok{or}\NormalTok{ ncount }\OperatorTok{==} \DecValTok{3}
        \ControlFlowTok{else}\NormalTok{:}
            \ControlFlowTok{return}\NormalTok{ ncount }\OperatorTok{==} \DecValTok{3}

    \ControlFlowTok{return}\NormalTok{ \{pt\_2d: }\BuiltInTok{frozenset}\NormalTok{(}
\NormalTok{                       pt\_nd }\ControlFlowTok{for}\NormalTok{ pt\_nd, n }\KeywordTok{in}\NormalTok{ pt\_counts.items()}
                             \ControlFlowTok{if}\NormalTok{ validate(pt\_2d, pt\_nd, n)}
\NormalTok{                   )}
              \ControlFlowTok{for}\NormalTok{ pt\_2d, pt\_counts }\KeywordTok{in}\NormalTok{ neighbs}
\NormalTok{           \}}
\end{Highlighting}
\end{Shaded}

And here is the final animation: we plot a single 2D grid, and each cell is
colored according to the size of the coset stack under that point (how many
points exist with that \texttt{\textless{}x,y\textgreater{}}). You can slide
this one up all the way to 10D to simulate it in your browser!

\leavevmode\hypertarget{golFlat}{}%
\textbf{Element 9:} N-D Game of Life

\leavevmode\hypertarget{golFlatCont}{}%
Please enable Javascript

(A lot of examples with symmetries look nice here, such as
\href{https://blog.jle.im/entry/degenerate-hyper-dimensional-game-of-life.html?points=██▟█.▜███.███▙.█▛██}{spiral
galaxy} and
\href{https://blog.jle.im/entry/degenerate-hyper-dimensional-game-of-life.html?points=▛▀▀▜.▌▛▜▐.▌▙▟▐.▙▄▄▟}{bulls-eye};
the alternating symmetries of
\href{https://blog.jle.im/entry/degenerate-hyper-dimensional-game-of-life.html?points=████.▄▗▖▄.▀▝▘▀.████}{broken
bar code} look nice too! But honestly, at higher-dimensions, most patterns have
appealing gradients, like
\href{https://blog.jle.im/entry/degenerate-hyper-dimensional-game-of-life.html?points=____._▝▖_._▀▘_.____}{glider}
and
\href{https://blog.jle.im/entry/degenerate-hyper-dimensional-game-of-life.html?points=▛▜▙▐.▜▚▗_.█▟▄▘.▟▌▟█}{my
own input})

Play around with it! :D You can move all the way up to 10D; some computers might
struggle, but on my lower-end cell phone it seems to run in less than a second.
If you mouse-over a cell, the text box will show all of the slice cosets where
that xy cell is alive in (the ``coset stack''). If you click on a cell, your
selection will ``lock'' on that \texttt{\textless{}x,y\textgreater{}} coordinate
as you change dimensions and time.

Some interesting things you might notice:

\begin{enumerate}
\def\labelenumi{\arabic{enumi}.}
\item
  At t=6, it looks like 7D, 8D, 9D, 10D all have the \emph{same} exact 2D cells
  ``on''. They're identical except for slightly different stacks above each of
  those cells.

  To see this clearly, set your time to t=6 and drag your dimension slider back
  and forth to see all of the higher-dimensions look identical in shape.
\item
  At t=2, t=4, past 5D or so, the state is exactly the same for all dimensions!
  We could easily find t=4 for 100D or even 200D: they're identical!
\item
  A lot of xy cells share identical coset stacks\ldots more on that later!
\end{enumerate}

Not only is it kinda pretty (in my humble opinion), it also demonstrates that
this whole ordeal is really ``just a normal 2D cellular automata'': it's like a
``multi-valued'' Game of Life, where instead of cells being on and off, they are
one of a few choices of values. Instead of a ``binary'' game of life with a
boolean at each cell, it's an ``integer'' game of life with a finite choice at
each cell.

Because there are
\includegraphics{https://latex.codecogs.com/png.latex?\%7B\%20\%7B\%5Chat\%7Bd\%7D\%7D\%2Bt\%7D\%20\%5Cchoose\%20t}
slice cosets for a given dimension and time, it means that our game is a
\includegraphics{https://latex.codecogs.com/png.latex?2\%5E\%7B\%20\%7B\%20\%5Chat\%7Bd\%7D\%20\%2B\%20t\%7D\%20\%5Cchoose\%20t\%20\%7D}-valued
game of life, where each cell can be one of that many options (each slice coset
and be present or not). That means at 2D
(\includegraphics{https://latex.codecogs.com/png.latex?\%5Chat\%7Bd\%7D\%20\%3D\%200}),
we have a normal 2-valued game of life
(\includegraphics{https://latex.codecogs.com/png.latex?2\%5E1}), at 3D we have
\includegraphics{https://latex.codecogs.com/png.latex?7\%20\%5Cchoose\%206} or 7
possible points at t=6, so that's a
\includegraphics{https://latex.codecogs.com/png.latex?2\%5E7} or 128-valued game
of life, at 4D we have
\includegraphics{https://latex.codecogs.com/png.latex?8\%20\%5Cchoose\%206} or
28 possible points at t=6, and so that's a
\includegraphics{https://latex.codecogs.com/png.latex?2\%5E\%7B28\%7D} or
268435456-valued game of life.

You can see this demonstrated in the simulation above, as well. As you progress,
each 2D cell ``spreads'' to its neighbors according to some complex rule; it's
like watching 2d cells interact with each other in complex ways, without ever
even having to think of higher dimensions.

Implementing things this way (and taking advantage of the fact that coset stacks
are usually very sparse and have few members) gave a big improvement. But
there's one final thing that this view would unlock that would make the biggest
difference.

\hypertarget{repeated-stacks}{%
\subsection{Repeated Stacks}\label{repeated-stacks}}

You might have noticed in the final 10D simulation, if you mouse over an xy
cell, it'll also highlight over all of the other xy cells that share the same
coset stack. For most initial starting positions, you might notice something
maybe even more curious --- a \emph{lot} of those stacks are duplicated over
many xy cells.

In my
\href{https://blog.jle.im/entry/degenerate-hyper-dimensional-game-of-life.html?points=▛▜▙▐.▜▚▗_.█▟▄▘.▟▌▟█}{personal
puzzle input}, \emph{most} of the stacks were duplicated many times across
different xy cells. If you highlight any arbitrary starting condition through
t=6, you'll see too that many (if not most) xy cells have multiple other xy
cells that have identical stacks to them.

This final insight yields the final optimization we have discovered, as of time
of writing. The optimization is that we can treat an \emph{entire stack} as an
``action'' that is spread to the xy neighbors: The stack under
\texttt{\textless{}x,y\textgreater{}=\textless{}3,4\textgreater{}} is spread to
all its eight 2D neighbors identically (and to itself, too, in a way that
excludes the original stack itself). That means if you have a stack, you can
compute the ``single neighbor'' contribution (expensive) it has \emph{one time},
and then \emph{repeat that same contribution} to every occurrence of a stack. So
if a stack is repeated ten times over ten different xy stacks, you only need to
compute it once and propagate it to all
\includegraphics{https://latex.codecogs.com/png.latex?9\%20\%5Ctimes\%2010}
neighbors of those stacks (nine neighbors, including self, times each of the 10
repetitions), for a savings of x90! This can be done by storing map of stacks to
contributions as a cache.

\begin{Shaded}
\begin{Highlighting}[]
\KeywordTok{def}\NormalTok{ step\_with\_stack\_cache(stacks):}
\NormalTok{    neighbs }\OperatorTok{=}\NormalTok{ \{\}}
\NormalTok{    stack\_cache }\OperatorTok{=}\NormalTok{ \{\}}

    \ControlFlowTok{for}\NormalTok{ pt\_2d, pt\_stack }\KeywordTok{in}\NormalTok{ stacks.items():}
        \CommentTok{\# get what to place in the same xy cell, and what to place in neighbor}
        \CommentTok{\# xy cells}
        \ControlFlowTok{if}\NormalTok{ pt\_stack }\KeywordTok{in}\NormalTok{ stack\_cache:}
            \CommentTok{\# get it from the cache if it exists}
\NormalTok{            (rev\_neighbs, rev\_neighbs\_incl\_self) }\OperatorTok{=}\NormalTok{ stack\_cache[pt\_stack]}
        \ControlFlowTok{else}\NormalTok{:}
            \CommentTok{\# otherwise, build it and store it in the cache}
\NormalTok{            rev\_neighbs }\OperatorTok{=}\NormalTok{ Counter()}
            \ControlFlowTok{for}\NormalTok{ pt\_nd }\KeywordTok{in}\NormalTok{ pt\_stack:}
\NormalTok{                rev\_neighbs }\OperatorTok{+=}\NormalTok{ Counter(reverse\_neighbs(pt\_nd))}
\NormalTok{            rev\_neighbs\_incl\_self }\OperatorTok{=}\NormalTok{ rev\_neighbs }\OperatorTok{+}\NormalTok{ Counter(pt\_stack)}
\NormalTok{            stack\_cache[pt\_stack] }\OperatorTok{=}\NormalTok{ (rev\_neighbs, rev\_neighbs\_incl\_self)}

        \CommentTok{\# the actual propagation}
        \CommentTok{\# 1. add in the same stack; don\textquotesingle{}t include self}
        \ControlFlowTok{if}\NormalTok{ pt\_2d }\KeywordTok{in}\NormalTok{ neighbs:}
\NormalTok{            neighbs[pt\_2d] }\OperatorTok{+=}\NormalTok{ rev\_neighbs}
        \ControlFlowTok{else}\NormalTok{:}
\NormalTok{            neighbs[pt\_2d] }\OperatorTok{=}\NormalTok{ rev\_neighbs}
        \CommentTok{\# 2. add to nieghboring stacks}
        \ControlFlowTok{for}\NormalTok{ ngb\_2 }\KeywordTok{in}\NormalTok{ mk\_neighbs(pt\_2d)[}\DecValTok{1}\NormalTok{:]:}
            \CommentTok{\# add to neighboring stacks; include self}
            \ControlFlowTok{if}\NormalTok{ ngb\_2 }\KeywordTok{in}\NormalTok{ neighbs:}
\NormalTok{                neighbs[ngb\_2] }\OperatorTok{+=}\NormalTok{ rev\_neighbs\_incl\_self}
            \ControlFlowTok{else}\NormalTok{:}
\NormalTok{                neighbs[ngb\_2] }\OperatorTok{=}\NormalTok{ rev\_neighbs\_incl\_self}

    \KeywordTok{def}\NormalTok{ validate(pt\_2d, pt\_nd, ncount):}
        \ControlFlowTok{if}\NormalTok{ pt\_2d }\KeywordTok{in}\NormalTok{ stacks }\KeywordTok{and}\NormalTok{ pt\_nd }\KeywordTok{in}\NormalTok{ stacks[pt\_2d]:}
            \ControlFlowTok{return}\NormalTok{ ncount }\OperatorTok{==} \DecValTok{2} \KeywordTok{or}\NormalTok{ ncount }\OperatorTok{==} \DecValTok{3}
        \ControlFlowTok{else}\NormalTok{:}
            \ControlFlowTok{return}\NormalTok{ ncount }\OperatorTok{==} \DecValTok{3}

    \ControlFlowTok{return}\NormalTok{ \{pt\_2d: }\BuiltInTok{frozenset}\NormalTok{(}
\NormalTok{                       pt\_nd }\ControlFlowTok{for}\NormalTok{ pt\_nd, n }\KeywordTok{in}\NormalTok{ pt\_counts.items()}
                             \ControlFlowTok{if}\NormalTok{ validate(pt\_2d, pt\_nd, n)}
\NormalTok{                   )}
              \ControlFlowTok{for}\NormalTok{ pt\_2d, pt\_counts }\KeywordTok{in}\NormalTok{ neighbs}
\NormalTok{           \}}
\end{Highlighting}
\end{Shaded}

With this final piece of the puzzle, I was able to reach 18D \emph{3 seconds} in
my Haskell solution, and 30D in 5 minutes! Michal Marsalek was also able to
build this into their fast Nim solver to
\href{https://www.reddit.com/user/MichalMarsalek/}{reach 40D in 8 minutes, 50D
in 32 minutes, 60D in 120 minutes}.

And as far as I know, this seems to be where things stand today.

\hypertarget{conclusions}{%
\section{Conclusions}\label{conclusions}}

Hope you enjoyed this journey! My hope is that I was able to convey a fraction
of the excitement, wonder, and mystery I felt during the process. At every
point, we had no reason to believe something better would come around the
corner, but we held on to a hope and faith that kept on rewarding us.

Visualization and different perspectives seem to drive almost every revelation
--- from the visually striking symmetries of the 3D and 4D simulations, the
explorations of how neighbor relationships work, the insight that we could treat
the entire problem as a fancy multivalued 2D game of life\ldots all of it came
about from being able to see the problem visually in different ways. At other
times it was a simple change in perspective to find a better way of encoding
variants. I know for myself, the next time I try to explore something like this,
I will try to apply what I learned to always reach for visualization sooner.
Even dead-end visualizations can sometimes provide a new depth to the puzzle
that you might appreciate later on.

Another thing I hope was apparent was the power of community! I know I
definitely would not have had as much fun doing this if it wasn't for the
vibrant Advent of Code ``Ante-Pushing'' community. What I've described is just
\emph{one story} (Day 17, 2020) out of so many that Advent of Code community
members routinely explore together (through 25 puzzles a year over five years).
Most of these discoveries were fun because we always had somebody to share them
with, or a way to encourage each other and strive for a common goal. I'm
definitely lucky to be standing on giants as a part of a talented and
passionately curious community that's excited to explore things like this. Thank
you to so many people --- Michal Marsalek, Peter Tseng, sim64, leftylink ,
Cettbycett, bsterc, flwyd, and so many others that I probably missed. An
especially deep thanks to \href{https://twitter.com/ericwastl}{Eric Wastl} for
hosting a wonderful event like Advent of Code every year. Finally, a profoundly
deep thanks to the late John Conway, who revealed to us how much joy can come
from the exploration of all things mathematical, a genius who was taken away
from this world much too soon.

And of course, in making this post, I'm inviting you, the reader, to join us
along in this journey as well! It's hardly over :) Now that you're up to speed
with all of us, I'd be excited to hear about anything you might discover while
playing around with this too!

Looking forward at least, there are a some open threads still. For example, the
most promising to me: for t=6, the exact same xy cells are always inhabited for
any dimension past 7D or so, and the stacks at each seem to only change slightly
from dimension to dimension. If we can analytically find how the stacks at any
given position change between 9D and 10D, 10D and 11D, etc., then it will be
easy to jump directly to t=6 at 100D without simulating anything. Another way to
say this is --- can we simulate by fixing t and stepping d, instead of fixing d
and stepping t?

Who can tell how far we can go?
\href{https://www.reddit.com/r/adventofcode/comments/kfb6zx/day_17_getting_to_t6_at_for_higher_spoilerss/gia880d/}{Michal}
has as a personal goal something I would also be very happy to reach:

\begin{quote}
I won't be satisfied until I implement a solution that runs in polynomial time
in both t and d.
\end{quote}

Will you, dear reader, be the one to take us there? :)

\hypertarget{signoff}{%
\section{Signoff}\label{signoff}}

Hi, thanks for reading! You can reach me via email at
\href{mailto:justin@jle.im}{\nolinkurl{justin@jle.im}}, or at twitter at
\href{https://twitter.com/mstk}{@mstk}! This post and all others are published
under the \href{https://creativecommons.org/licenses/by-nc-nd/3.0/}{CC-BY-NC-ND
3.0} license. Corrections and edits via pull request are welcome and encouraged
at \href{https://github.com/mstksg/inCode}{the source repository}.

If you feel inclined, or this post was particularly helpful for you, why not
consider \href{https://www.patreon.com/justinle/overview}{supporting me on
Patreon}, or a \href{bitcoin:3D7rmAYgbDnp4gp4rf22THsGt74fNucPDU}{BTC donation}?
:)

\end{document}
