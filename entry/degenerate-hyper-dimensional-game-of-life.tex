\documentclass[]{article}
\usepackage{lmodern}
\usepackage{amssymb,amsmath}
\usepackage{ifxetex,ifluatex}
\usepackage{fixltx2e} % provides \textsubscript
\ifnum 0\ifxetex 1\fi\ifluatex 1\fi=0 % if pdftex
  \usepackage[T1]{fontenc}
  \usepackage[utf8]{inputenc}
\else % if luatex or xelatex
  \ifxetex
    \usepackage{mathspec}
    \usepackage{xltxtra,xunicode}
  \else
    \usepackage{fontspec}
  \fi
  \defaultfontfeatures{Mapping=tex-text,Scale=MatchLowercase}
  \newcommand{\euro}{€}
\fi
% use upquote if available, for straight quotes in verbatim environments
\IfFileExists{upquote.sty}{\usepackage{upquote}}{}
% use microtype if available
\IfFileExists{microtype.sty}{\usepackage{microtype}}{}
\usepackage[margin=1in]{geometry}
\usepackage{color}
\usepackage{fancyvrb}
\newcommand{\VerbBar}{|}
\newcommand{\VERB}{\Verb[commandchars=\\\{\}]}
\DefineVerbatimEnvironment{Highlighting}{Verbatim}{commandchars=\\\{\}}
% Add ',fontsize=\small' for more characters per line
\newenvironment{Shaded}{}{}
\newcommand{\AlertTok}[1]{\textcolor[rgb]{1.00,0.00,0.00}{\textbf{#1}}}
\newcommand{\AnnotationTok}[1]{\textcolor[rgb]{0.38,0.63,0.69}{\textbf{\textit{#1}}}}
\newcommand{\AttributeTok}[1]{\textcolor[rgb]{0.49,0.56,0.16}{#1}}
\newcommand{\BaseNTok}[1]{\textcolor[rgb]{0.25,0.63,0.44}{#1}}
\newcommand{\BuiltInTok}[1]{#1}
\newcommand{\CharTok}[1]{\textcolor[rgb]{0.25,0.44,0.63}{#1}}
\newcommand{\CommentTok}[1]{\textcolor[rgb]{0.38,0.63,0.69}{\textit{#1}}}
\newcommand{\CommentVarTok}[1]{\textcolor[rgb]{0.38,0.63,0.69}{\textbf{\textit{#1}}}}
\newcommand{\ConstantTok}[1]{\textcolor[rgb]{0.53,0.00,0.00}{#1}}
\newcommand{\ControlFlowTok}[1]{\textcolor[rgb]{0.00,0.44,0.13}{\textbf{#1}}}
\newcommand{\DataTypeTok}[1]{\textcolor[rgb]{0.56,0.13,0.00}{#1}}
\newcommand{\DecValTok}[1]{\textcolor[rgb]{0.25,0.63,0.44}{#1}}
\newcommand{\DocumentationTok}[1]{\textcolor[rgb]{0.73,0.13,0.13}{\textit{#1}}}
\newcommand{\ErrorTok}[1]{\textcolor[rgb]{1.00,0.00,0.00}{\textbf{#1}}}
\newcommand{\ExtensionTok}[1]{#1}
\newcommand{\FloatTok}[1]{\textcolor[rgb]{0.25,0.63,0.44}{#1}}
\newcommand{\FunctionTok}[1]{\textcolor[rgb]{0.02,0.16,0.49}{#1}}
\newcommand{\ImportTok}[1]{#1}
\newcommand{\InformationTok}[1]{\textcolor[rgb]{0.38,0.63,0.69}{\textbf{\textit{#1}}}}
\newcommand{\KeywordTok}[1]{\textcolor[rgb]{0.00,0.44,0.13}{\textbf{#1}}}
\newcommand{\NormalTok}[1]{#1}
\newcommand{\OperatorTok}[1]{\textcolor[rgb]{0.40,0.40,0.40}{#1}}
\newcommand{\OtherTok}[1]{\textcolor[rgb]{0.00,0.44,0.13}{#1}}
\newcommand{\PreprocessorTok}[1]{\textcolor[rgb]{0.74,0.48,0.00}{#1}}
\newcommand{\RegionMarkerTok}[1]{#1}
\newcommand{\SpecialCharTok}[1]{\textcolor[rgb]{0.25,0.44,0.63}{#1}}
\newcommand{\SpecialStringTok}[1]{\textcolor[rgb]{0.73,0.40,0.53}{#1}}
\newcommand{\StringTok}[1]{\textcolor[rgb]{0.25,0.44,0.63}{#1}}
\newcommand{\VariableTok}[1]{\textcolor[rgb]{0.10,0.09,0.49}{#1}}
\newcommand{\VerbatimStringTok}[1]{\textcolor[rgb]{0.25,0.44,0.63}{#1}}
\newcommand{\WarningTok}[1]{\textcolor[rgb]{0.38,0.63,0.69}{\textbf{\textit{#1}}}}
\usepackage{graphicx}
\makeatletter
\def\maxwidth{\ifdim\Gin@nat@width>\linewidth\linewidth\else\Gin@nat@width\fi}
\def\maxheight{\ifdim\Gin@nat@height>\textheight\textheight\else\Gin@nat@height\fi}
\makeatother
% Scale images if necessary, so that they will not overflow the page
% margins by default, and it is still possible to overwrite the defaults
% using explicit options in \includegraphics[width, height, ...]{}
\setkeys{Gin}{width=\maxwidth,height=\maxheight,keepaspectratio}
\ifxetex
  \usepackage[setpagesize=false, % page size defined by xetex
              unicode=false, % unicode breaks when used with xetex
              xetex]{hyperref}
\else
  \usepackage[unicode=true]{hyperref}
\fi
\hypersetup{breaklinks=true,
            bookmarks=true,
            pdfauthor={Justin Le},
            pdftitle={Degenerate Hyper-Dimensional Game of Life: Pushing Advent of Code to its Limits},
            colorlinks=true,
            citecolor=blue,
            urlcolor=blue,
            linkcolor=magenta,
            pdfborder={0 0 0}}
\urlstyle{same}  % don't use monospace font for urls
% Make links footnotes instead of hotlinks:
\renewcommand{\href}[2]{#2\footnote{\url{#1}}}
\setlength{\parindent}{0pt}
\setlength{\parskip}{6pt plus 2pt minus 1pt}
\setlength{\emergencystretch}{3em}  % prevent overfull lines
\setcounter{secnumdepth}{0}

\title{Degenerate Hyper-Dimensional Game of Life: Pushing Advent of Code to its Limits}
\author{Justin Le}

\begin{document}
\maketitle

\emph{Originally posted on
\textbf{\href{https://blog.jle.im/entry/degenerate-hyper-dimensional-game-of-life.html}{in
Code}}.}

poopoo

tldr: By exploiting multiple mathematical properties of a ``degenerate''
hyper-dimensional game of life, each discovered one-by-one over the course of a
month, we were able to go from ``10 dimensions may just barely be possible'' to
``10 dimensions in 100ms, 50 dimensions tackled.''

This is a story about breaking the degenerate hyper-dimensional game of life!
Let's travel back in time to the night before December 17, 2020: The release of
\href{https://adventofcode.com/2020/day/17}{``Conway Cubes''}, day 17 of the
``Advent of Code'' of fun little coding puzzles building up to Christmas. One
part I've always found especially fun is that, because the problems are so
self-contained and tidy, they are often \emph{open-ended} in the interesting
ways you can solve them or expand them.

On the surface, it seems to essentially be a straightforward expansion of
\href{https://en.wikipedia.org/wiki/Conway\%27s_Game_of_Life}{Conway's Game Of
Life}. GoL is a simulation played out on an infinite 2d grid, where certain
cells are ``on'' and ``off'', and at each step of the simulation, the on/off
cells spread and propagate in fascinating ways.

The twist of the Advent of Code puzzle is it asks what would happen if we played
out the rules of GoL in 3d, and then 4d! The ``starting conditions'' are a 8x8
2D grid that every puzzle solver receives their own one of, and to solve the
puzzle you must simulate six steps of GoL for your personal input and count the
number of live cells at the end of it. For example, my xy starting conditions
were

\begin{verbatim}
#####..#
#..###.#
###.....
.#.#.#..
##.#..#.
######..
.##..###
###.####
\end{verbatim}

After 6 steps for 4-dimensional game of life, I had 2620 total points. I
submitted my answer with a direct implementation (scoring the 66th spot on the
leader board)\ldots and that was that.

Simple enough, right? Afterwards, when discussing with some friends, we started
talking about the trade-offs of different implementations and realized that
dimensionality was no joke\ldots as you upped the number of dimensions, the
number of points you have to consider grow as
\includegraphics{https://latex.codecogs.com/png.latex?O\%28\%282t\%2B8\%29\%5Ed\%29},
and the number of neighbors of each point to check grows as
\includegraphics{https://latex.codecogs.com/png.latex?O\%283\%5Ed\%29}\ldots{} I
took my solution for d=4 and could \emph{barely} churn out d=6 (in three
minutes), since even at d=6 you have to consider 728 neighbors for (potentially)
each of the 64,000,000 points. I had a dream that t=6, d=10 just \emph{might} be
possible\ldots but clearly not without some major breakthrough. After all, at
d=10, you'd need to check 59,048 neighbors for potentially each of
10,240,000,000,000 points.

And\ldots a breakthrough soon came. Someone brought up that if we look at the 3d
version, we see there's actually a \emph{mirror symmetry}! That is, because
everything starts off on the xy plane, with z=0 and w=0, the resulting
progression must be symmetrical on both sides.

\begin{figure}
\centering
\includegraphics{/img/entries/advent-gol/life3d.gif}
\caption{d=3 animation by
\href{https://www.reddit.com/r/adventofcode/comments/kfa3nr/2020_day_17_godot_cubes_i_think_i_went_a_bit_too/}{u/ZuBsPaCe}}
\end{figure}

This means that we only have to simulate \emph{half} of the points (for each
extra dimension) to get the answer, \emph{halving} the number of points for d=3,
saving a factor of 4 for d=4, saving a factor of 8 for d=5, etc.
(\includegraphics{https://latex.codecogs.com/png.latex?O\%282\%5E\%7Bd-2\%7D\%29}).
After implementing this optimization, I was able to run d=7 in under \emph{six
minutes} (and d=8 in 75 minutes).

What's more is that this made us realize that there were potentially more
breakthroughs we could get by exploiting the fact that we are only given a 2d
slice\ldots we didn't know what those could be yet, but suddenly, d=10 seemed
attainable\ldots maybe?

And such a ``maybe'' (as posed in
\href{https://www.reddit.com/r/adventofcode/comments/kfb6zx/day_17_getting_to_t6_at_for_higher_spoilerss/}{this
reddit thread I started}) turned into a month-long quest of breakthrough after
breakthrough, exploting different aspects of this 2d degeneracy! It was a long,
harrowing journey full of sudden twists and turns and bursts of excitement when
new innovations came. And in the end, the hopeful question ``What if d=10 was
possible?'' turned into ``d=10 in 100ms, d=40 in eight minutes.''

So, let's take a deep dive --- deeper than you probably ever expected to dive
into any particular degenerate starting conditions of a hyper-dimensional game
of life :D

\hypertarget{the-baseline}{%
\section{The Baseline}\label{the-baseline}}

So let's start at the very beginning: how do you solve everything without
anything too fancy?

I discuss my basic Haskell method in
\href{https://github.com/mstksg/advent-of-code-2020/blob/master/reflections-out/day17.md}{this
write-up}

\begin{Shaded}
\begin{Highlighting}[]
\CommentTok{{-}{-} | Given a neighbor set}
\OtherTok{neighbsSet ::} \DataTypeTok{Vector} \DataTypeTok{Int} \OtherTok{{-}>} \DataTypeTok{Set}\NormalTok{ (}\DataTypeTok{Vector} \DataTypeTok{Int}\NormalTok{)}
\NormalTok{neighbsSet }\OtherTok{=}\NormalTok{ S.fromList }\OperatorTok{.} \FunctionTok{tail} \OperatorTok{.} \FunctionTok{traverse}\NormalTok{ (\textbackslash{}x }\OtherTok{{-}>}\NormalTok{ [x,x}\OperatorTok{{-}}\DecValTok{1}\NormalTok{,x}\OperatorTok{+}\DecValTok{1}\NormalTok{])}

\OtherTok{neighborMap ::} \DataTypeTok{Set}\NormalTok{ (}\DataTypeTok{V3} \DataTypeTok{Int}\NormalTok{) }\OtherTok{{-}>} \DataTypeTok{Map}\NormalTok{ (}\DataTypeTok{V3} \DataTypeTok{Int}\NormalTok{) }\DataTypeTok{Int}
\NormalTok{neighborMap ps }\OtherTok{=}\NormalTok{ M.unionsWith (}\OperatorTok{+}\NormalTok{)}
\NormalTok{    [ M.fromSet (}\FunctionTok{const} \DecValTok{1}\NormalTok{) (neighbsSet p)}
    \OperatorTok{|}\NormalTok{ p }\OtherTok{<{-}}\NormalTok{ S.toList ps}
\NormalTok{    ]}

\NormalTok{stepper}
\OtherTok{    ::} \DataTypeTok{Set}\NormalTok{ (}\DataTypeTok{V3} \DataTypeTok{Int}\NormalTok{)}
    \OtherTok{{-}>} \DataTypeTok{Set}\NormalTok{ (}\DataTypeTok{V3} \DataTypeTok{Int}\NormalTok{)}
\NormalTok{stepper ps }\OtherTok{=}\NormalTok{ stayAlive }\OperatorTok{<>}\NormalTok{ comeAlive}
  \KeywordTok{where}
\NormalTok{    neighborCounts }\OtherTok{=}\NormalTok{ neighborMap ps}
\NormalTok{    stayAlive }\OtherTok{=}\NormalTok{ M.keysSet }\OperatorTok{.}\NormalTok{ M.filter (\textbackslash{}n }\OtherTok{{-}>}\NormalTok{ n }\OperatorTok{==} \DecValTok{2} \OperatorTok{||}\NormalTok{ n }\OperatorTok{==} \DecValTok{3}\NormalTok{) }\OperatorTok{$}
\NormalTok{                  neighborCounts }\OtherTok{\textasciigrave{}M.restrictKeys\textasciigrave{}}\NormalTok{ ps}
\NormalTok{    comeAlive }\OtherTok{=}\NormalTok{ M.keysSet }\OperatorTok{.}\NormalTok{ M.filter (}\OperatorTok{==} \DecValTok{3}\NormalTok{) }\OperatorTok{$}
\NormalTok{                  neighborCounts }\OtherTok{\textasciigrave{}M.withoutKeys\textasciigrave{}}\NormalTok{  ps}
\end{Highlighting}
\end{Shaded}

sim642 I wanted to ask this before but forgot: did anyone try to take advantage
of the symmetry, e.g.~in z axis in part 1? sim642 Should halve the amount of
calculations you have to do sim642 Only some extra work at the end to
differentiate z=0 and z\textgreater0 positions to know which to count twice
sim642 And in part 2 I feel like you could also exploit the symmetry in w axis
simultaneously

\hypertarget{signoff}{%
\section{Signoff}\label{signoff}}

Hi, thanks for reading! You can reach me via email at
\href{mailto:justin@jle.im}{\nolinkurl{justin@jle.im}}, or at twitter at
\href{https://twitter.com/mstk}{@mstk}! This post and all others are published
under the \href{https://creativecommons.org/licenses/by-nc-nd/3.0/}{CC-BY-NC-ND
3.0} license. Corrections and edits via pull request are welcome and encouraged
at \href{https://github.com/mstksg/inCode}{the source repository}.

If you feel inclined, or this post was particularly helpful for you, why not
consider \href{https://www.patreon.com/justinle/overview}{supporting me on
Patreon}, or a \href{bitcoin:3D7rmAYgbDnp4gp4rf22THsGt74fNucPDU}{BTC donation}?
:)

\end{document}
