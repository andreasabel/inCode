\documentclass[]{article}
\usepackage{lmodern}
\usepackage{amssymb,amsmath}
\usepackage{ifxetex,ifluatex}
\usepackage{fixltx2e} % provides \textsubscript
\ifnum 0\ifxetex 1\fi\ifluatex 1\fi=0 % if pdftex
  \usepackage[T1]{fontenc}
  \usepackage[utf8]{inputenc}
\else % if luatex or xelatex
  \ifxetex
    \usepackage{mathspec}
    \usepackage{xltxtra,xunicode}
  \else
    \usepackage{fontspec}
  \fi
  \defaultfontfeatures{Mapping=tex-text,Scale=MatchLowercase}
  \newcommand{\euro}{€}
\fi
% use upquote if available, for straight quotes in verbatim environments
\IfFileExists{upquote.sty}{\usepackage{upquote}}{}
% use microtype if available
\IfFileExists{microtype.sty}{\usepackage{microtype}}{}
\usepackage[margin=1in]{geometry}
\usepackage{color}
\usepackage{fancyvrb}
\newcommand{\VerbBar}{|}
\newcommand{\VERB}{\Verb[commandchars=\\\{\}]}
\DefineVerbatimEnvironment{Highlighting}{Verbatim}{commandchars=\\\{\}}
% Add ',fontsize=\small' for more characters per line
\newenvironment{Shaded}{}{}
\newcommand{\KeywordTok}[1]{\textcolor[rgb]{0.00,0.44,0.13}{\textbf{#1}}}
\newcommand{\DataTypeTok}[1]{\textcolor[rgb]{0.56,0.13,0.00}{#1}}
\newcommand{\DecValTok}[1]{\textcolor[rgb]{0.25,0.63,0.44}{#1}}
\newcommand{\BaseNTok}[1]{\textcolor[rgb]{0.25,0.63,0.44}{#1}}
\newcommand{\FloatTok}[1]{\textcolor[rgb]{0.25,0.63,0.44}{#1}}
\newcommand{\ConstantTok}[1]{\textcolor[rgb]{0.53,0.00,0.00}{#1}}
\newcommand{\CharTok}[1]{\textcolor[rgb]{0.25,0.44,0.63}{#1}}
\newcommand{\SpecialCharTok}[1]{\textcolor[rgb]{0.25,0.44,0.63}{#1}}
\newcommand{\StringTok}[1]{\textcolor[rgb]{0.25,0.44,0.63}{#1}}
\newcommand{\VerbatimStringTok}[1]{\textcolor[rgb]{0.25,0.44,0.63}{#1}}
\newcommand{\SpecialStringTok}[1]{\textcolor[rgb]{0.73,0.40,0.53}{#1}}
\newcommand{\ImportTok}[1]{#1}
\newcommand{\CommentTok}[1]{\textcolor[rgb]{0.38,0.63,0.69}{\textit{#1}}}
\newcommand{\DocumentationTok}[1]{\textcolor[rgb]{0.73,0.13,0.13}{\textit{#1}}}
\newcommand{\AnnotationTok}[1]{\textcolor[rgb]{0.38,0.63,0.69}{\textbf{\textit{#1}}}}
\newcommand{\CommentVarTok}[1]{\textcolor[rgb]{0.38,0.63,0.69}{\textbf{\textit{#1}}}}
\newcommand{\OtherTok}[1]{\textcolor[rgb]{0.00,0.44,0.13}{#1}}
\newcommand{\FunctionTok}[1]{\textcolor[rgb]{0.02,0.16,0.49}{#1}}
\newcommand{\VariableTok}[1]{\textcolor[rgb]{0.10,0.09,0.49}{#1}}
\newcommand{\ControlFlowTok}[1]{\textcolor[rgb]{0.00,0.44,0.13}{\textbf{#1}}}
\newcommand{\OperatorTok}[1]{\textcolor[rgb]{0.40,0.40,0.40}{#1}}
\newcommand{\BuiltInTok}[1]{#1}
\newcommand{\ExtensionTok}[1]{#1}
\newcommand{\PreprocessorTok}[1]{\textcolor[rgb]{0.74,0.48,0.00}{#1}}
\newcommand{\AttributeTok}[1]{\textcolor[rgb]{0.49,0.56,0.16}{#1}}
\newcommand{\RegionMarkerTok}[1]{#1}
\newcommand{\InformationTok}[1]{\textcolor[rgb]{0.38,0.63,0.69}{\textbf{\textit{#1}}}}
\newcommand{\WarningTok}[1]{\textcolor[rgb]{0.38,0.63,0.69}{\textbf{\textit{#1}}}}
\newcommand{\AlertTok}[1]{\textcolor[rgb]{1.00,0.00,0.00}{\textbf{#1}}}
\newcommand{\ErrorTok}[1]{\textcolor[rgb]{1.00,0.00,0.00}{\textbf{#1}}}
\newcommand{\NormalTok}[1]{#1}
\ifxetex
  \usepackage[setpagesize=false, % page size defined by xetex
              unicode=false, % unicode breaks when used with xetex
              xetex]{hyperref}
\else
  \usepackage[unicode=true]{hyperref}
\fi
\hypersetup{breaklinks=true,
            bookmarks=true,
            pdfauthor={Justin Le},
            pdftitle={Fixed-Length Vector Types in Haskell (an Update for 2017)},
            colorlinks=true,
            citecolor=blue,
            urlcolor=blue,
            linkcolor=magenta,
            pdfborder={0 0 0}}
\urlstyle{same}  % don't use monospace font for urls
% Make links footnotes instead of hotlinks:
\renewcommand{\href}[2]{#2\footnote{\url{#1}}}
\setlength{\parindent}{0pt}
\setlength{\parskip}{6pt plus 2pt minus 1pt}
\setlength{\emergencystretch}{3em}  % prevent overfull lines
\setcounter{secnumdepth}{0}

\title{Fixed-Length Vector Types in Haskell (an Update for 2017)}
\author{Justin Le}

\begin{document}
\maketitle

\emph{Originally posted on
\textbf{\href{https://blog.jle.im/entry/fixed-length-vector-types-in-haskell.html}{in
Code}}.}

This post is a follow-up to my
\href{https://blog.jle.im/entry/fixed-length-vector-types-in-haskell-2015.html}{fixed-length
vectors in haskell in 2015} post! When I was writing the post originally, I was
new to the whole type-level game in Haskell; I didn't know what I was talking
about, and that post was a way for me to push myself to learn more.

Immediately after it was posted, people taught me where I went wrong in the
idioms I explained, and better and more idiomatic ways to do things.
Unfortunately, I have noticed people referring to the post in a
canonical/authoritative way\ldots{}so the post became an immediate regret to me.
I tried correcting things with my
\href{https://blog.jle.im/entries/series/+practical-dependent-types-in-haskell.html}{practical
dependent types in haskell} series the next year, which incorporated what I had
learned. But I still saw my 2015 post being used as a reference, so I figured
that writing a direct replacement/follow-up as the only way I would ever fix
this!

So here we are in 2017. What's the ``right'' way to do fixed-length vectors in
Haskell?

\section{As a Wrapper with Smart
Constructors}\label{as-a-wrapper-with-smart-constructors}

In most situations, if you use vectors, you want some sort of constant-time
indexed data structure. The best way to do this in Haskell is to wrap the
heavily optimized
\emph{\href{http://hackage.haskell.org/package/vector}{vector}} library with a
newtype wrapper that contains its length as a phantom type parameter.

\begin{Shaded}
\begin{Highlighting}[]
\KeywordTok{import qualified} \DataTypeTok{Data.Vector} \KeywordTok{as} \DataTypeTok{V}

\KeywordTok{data} \DataTypeTok{Vec}\NormalTok{ (}\OtherTok{n ::} \DataTypeTok{Nat}\NormalTok{) a }\FunctionTok{=} \DataTypeTok{UnsafeMkVec}\NormalTok{ \{}\OtherTok{ getVector ::} \DataTypeTok{V.Vector}\NormalTok{ a \}}
    \KeywordTok{deriving} \DataTypeTok{Show}
\end{Highlighting}
\end{Shaded}

A \texttt{Vec\ n\ a} will represent an \texttt{n}-element vector of \texttt{a}s.
So, a \texttt{Vec\ 5\ Int} will be a vector of five \texttt{Int}s, a
\texttt{Vec\ 10\ String} is a vector of 10 \texttt{String}s, etc.

For our numeric types, we're using the fancy ``type literals'' that GHC offers
us with the \texttt{DataKinds} extension. Basically, alongside the normal kinds
\texttt{*}, \texttt{*\ -\textgreater{}\ *}, etc., we also have the \texttt{Nat}
kind; type literals \texttt{1}, \texttt{5}, \texttt{100}, etc. are all
\emph{types} with the \emph{kind} \texttt{Nat}.

\begin{Shaded}
\begin{Highlighting}[]
\NormalTok{ghci}\FunctionTok{>} \FunctionTok{:}\NormalTok{k }\DecValTok{5}
\DataTypeTok{Nat}
\NormalTok{ghci}\FunctionTok{>} \FunctionTok{:}\NormalTok{k }\DataTypeTok{Vec}
\DataTypeTok{Vec}\OtherTok{ ::} \DataTypeTok{Nat} \OtherTok{->} \FunctionTok{*} \OtherTok{->} \FunctionTok{*}
\end{Highlighting}
\end{Shaded}

You can ``reflect'' the type-level numeral as a value using the
\texttt{KnownNat} typeclass, provided by GHC, which lets you gain back the
number as a run-time value using \texttt{natVal}:

\begin{Shaded}
\begin{Highlighting}[]
\OtherTok{natVal ::} \DataTypeTok{KnownNat}\NormalTok{ n }\OtherTok{=>}\NormalTok{ p n }\OtherTok{->} \DataTypeTok{Integer}
\end{Highlighting}
\end{Shaded}

\begin{Shaded}
\begin{Highlighting}[]
\NormalTok{ghci}\FunctionTok{>}\NormalTok{ natVal (}\DataTypeTok{Proxy} \FunctionTok{@}\DecValTok{10}\NormalTok{)   }\CommentTok{-- or, natVal (Proxy :: Proxy 10)}
\DecValTok{10}
\NormalTok{ghci}\FunctionTok{>}\NormalTok{ natVal (}\DataTypeTok{Proxy} \FunctionTok{@}\DecValTok{7}\NormalTok{)}
\DecValTok{7}
\end{Highlighting}
\end{Shaded}

\subsection{The Smart Constructor}\label{the-smart-constructor}

We can use \texttt{natVal} with the \texttt{KnownNat} typeclass to write a
``smart constructor'' for our type -- make a \texttt{Vec} from a
\texttt{Vector}, but only if the length is the correct type:

\begin{Shaded}
\begin{Highlighting}[]
\OtherTok{mkVec ::}\NormalTok{ forall n}\FunctionTok{.} \DataTypeTok{KnownNat}\NormalTok{ n }\OtherTok{=>} \DataTypeTok{V.Vector}\NormalTok{ a }\OtherTok{->} \DataTypeTok{Maybe}\NormalTok{ (}\DataTypeTok{Vec}\NormalTok{ n a)}
\NormalTok{mkVec v }\FunctionTok{|}\NormalTok{ V.length v }\FunctionTok{==}\NormalTok{ l }\FunctionTok{=} \DataTypeTok{Just}\NormalTok{ (}\DataTypeTok{UnsafeMkVec}\NormalTok{ v)}
        \FunctionTok{|}\NormalTok{ otherwise       }\FunctionTok{=} \DataTypeTok{Nothing}
  \KeywordTok{where}
\NormalTok{    l }\FunctionTok{=}\NormalTok{ natVal (}\DataTypeTok{Proxy} \FunctionTok{@}\NormalTok{n)}
\end{Highlighting}
\end{Shaded}

Here, we use \texttt{ScopedTypeVariables} so we can refer to the \texttt{n} in
the type signature in the function body (for \texttt{natVal\ (Proxy\ @n)}). We
need to use an explicit forall, then, to bring the \texttt{n} into scope.

\subsection{Indexing}\label{indexing}

We need an appropriate type for indexing these, but we'd like a type where
indexing is ``safe'' -- that is, that you can't compile a program that will
result in an index error.

For this, we can use the
\emph{\href{http://hackage.haskell.org/package/finite-typelits}{finite-typelits}}
package, which provides the \texttt{Finite\ n} type.

A \texttt{Finite\ n} type is a type with exactly \texttt{n} distinct
inhabitants/values. For example, \texttt{Finite\ 4} contains four ``anonymous''
inhabitants. For convenience, sometimes we like to name them 0, 1, 2, and 3. In
general, we sometimes refer to the values of type \texttt{Finite\ n} as 0
\ldots{} (n - 1).

So, we can imagine that \texttt{Finite\ 6} has inhabitants corresponding to 0,
1, 2, 3, 4, and 5. We can convert back and forth between a \texttt{Finite\ n}
and its \texttt{Integer} representation using \texttt{packFinite} and
\texttt{getFinite}:

\begin{Shaded}
\begin{Highlighting}[]
\OtherTok{packFinite ::} \DataTypeTok{KnownNat}\NormalTok{ n }\OtherTok{=>} \DataTypeTok{Integer}  \OtherTok{->} \DataTypeTok{Maybe}\NormalTok{ (}\DataTypeTok{Finite}\NormalTok{ n)}
\OtherTok{getFinite  ::}               \DataTypeTok{Finite}\NormalTok{ n }\OtherTok{->} \DataTypeTok{Integer}
\end{Highlighting}
\end{Shaded}

\begin{Shaded}
\begin{Highlighting}[]
\NormalTok{ghci}\FunctionTok{>}\NormalTok{ map packFinite [}\DecValTok{0}\FunctionTok{..}\DecValTok{3}\NormalTok{]}\OtherTok{ ::}\NormalTok{ [}\DataTypeTok{Finite} \DecValTok{3}\NormalTok{]}
\NormalTok{[}\DataTypeTok{Just}\NormalTok{ (finite }\DecValTok{0}\NormalTok{), }\DataTypeTok{Just}\NormalTok{ (finite }\DecValTok{1}\NormalTok{), }\DataTypeTok{Just}\NormalTok{ (finite }\DecValTok{2}\NormalTok{), }\DataTypeTok{Nothing}\NormalTok{]}
\NormalTok{ghci}\FunctionTok{>}\NormalTok{ getFinite (finite }\DecValTok{2}\OtherTok{ ::} \DataTypeTok{Finite} \DecValTok{5}\NormalTok{)}
\DecValTok{2}
\end{Highlighting}
\end{Shaded}

We can use a \texttt{Finite\ n} to ``index'' a \texttt{Vector\ n\ a}. A
\texttt{Vector\ n\ a} has exactly \texttt{n} slots, and a \texttt{Finite\ n} has
\texttt{n} possible values. Clearly, \texttt{Finite\ n} only contains valid
indices into our vector!

\begin{Shaded}
\begin{Highlighting}[]
\NormalTok{index}\OtherTok{ ::} \DataTypeTok{Vec}\NormalTok{ n a }\OtherTok{->} \DataTypeTok{Finite}\NormalTok{ n }\OtherTok{->}\NormalTok{ a}
\NormalTok{index v i }\FunctionTok{=}\NormalTok{ getVector v }\FunctionTok{V.!}\NormalTok{ fromIntegral (getFinite i)}
\end{Highlighting}
\end{Shaded}

\texttt{index} will never fail at runtime due to a bad index --- do you see why?
Valid indices of a \texttt{Vector\ 5\ a} are the integers 0 to 4, and that is
precisely the exact things that \texttt{Finite\ 5} can store!

\subsection{Appending and type-level
arithmetic}\label{appending-and-type-level-arithmetic}

Another operation we might want to do with vectors is append them and do things
with them that might change their length. We might want the type of our vectors
to describe the nature of the operations they are undergoing. For example, if
you saw a function:

\begin{Shaded}
\begin{Highlighting}[]
\OtherTok{someFunc ::} \DataTypeTok{Vec}\NormalTok{ n a }\OtherTok{->} \DataTypeTok{Vec}\NormalTok{ m a }\OtherTok{->} \DataTypeTok{Vec}\NormalTok{ (n }\FunctionTok{+}\NormalTok{ m) a}
\end{Highlighting}
\end{Shaded}

You can see that it takes a vector of length \texttt{n} and a vector of length
\texttt{m}, and returns a vector of length \texttt{n\ +\ m}. Clearly, this
function must be appending/concatenating the two input vectors!

In this situation, we can write such an appending function in an ``unsafe'' way,
and then give it our type signature as a form of documentation.

\begin{Shaded}
\begin{Highlighting}[]
\OtherTok{append ::} \DataTypeTok{Vec}\NormalTok{ n a }\OtherTok{->} \DataTypeTok{Vec}\NormalTok{ m a }\OtherTok{->} \DataTypeTok{Vec}\NormalTok{ (n }\FunctionTok{+}\NormalTok{ m) a}
\NormalTok{append v w }\FunctionTok{=} \DataTypeTok{UnsafeMkVec} \FunctionTok{$}\NormalTok{ getVector v }\FunctionTok{V.++}\NormalTok{ getVector w}
\end{Highlighting}
\end{Shaded}

The compiler didn't help us write this function, and we have to be pretty
careful that the guarantees we specify in our types are reflected in the actual
unsafe operations. This is because our types don't \emph{structurally} enforce
their type-level lengths.

So, why bother? For us, here, our fixed-length vector types basically act as
``active documentation'', in a way. Compare:

\begin{Shaded}
\begin{Highlighting}[]
\CommentTok{-- | Appends a vector of length n with a vector of length m to get a vector of}
\NormalTok{length (n }\FunctionTok{+}\NormalTok{ m)}\FunctionTok{.}
\OtherTok{append ::} \DataTypeTok{V.Vector}\NormalTok{ a }\OtherTok{->} \DataTypeTok{V.Vector}\NormalTok{ a }\OtherTok{->} \DataTypeTok{V.Vector}\NormalTok{ a}
\end{Highlighting}
\end{Shaded}

We have to rely on the documentation to \emph{tell} us what the length of the
final resulting vector is, even though it can be known statically if you know
the length of the input vectors. The vectors have a \emph{static relationship}
in their length, but this isn't specified in a way that the compiler can take
advantage of.

By having our
\texttt{append\ ::\ Vec\ n\ a\ -\textgreater{}\ Vec\ m\ a\ -\textgreater{}\ Vec\ (n\ +\ m)\ a},
the relationship between the input lengths and output length is right there in
the types, when you \emph{use} \texttt{append}, GHC is aware of the
relationships and can give you help in the form of typed hole suggestions and
informative type errors. You can even catch errors in logic at compile-time
instead of runtime!

Here, \texttt{(+)} comes from GHC, which provides it as a type family
(type-level function) we can use, with proper meaning and semantics.

Some other examples include:

\begin{Shaded}
\begin{Highlighting}[]
\CommentTok{-- the end vector has the same length as the starting vector}
\NormalTok{map}\OtherTok{ ::}\NormalTok{ (a }\OtherTok{->}\NormalTok{ b) }\OtherTok{->} \DataTypeTok{Vec}\NormalTok{ n a }\OtherTok{->} \DataTypeTok{Vec}\NormalTok{ n b}
\CommentTok{-- you must zip two vectors of the same length}
\NormalTok{zip}\OtherTok{ ::} \DataTypeTok{Vec}\NormalTok{ n a }\OtherTok{->} \DataTypeTok{Vec}\NormalTok{ n b }\OtherTok{->} \DataTypeTok{Vec}\NormalTok{ n (a, b)}
\CommentTok{-- type-level arithmetic to let us 'take'}
\NormalTok{take}\OtherTok{ ::} \DataTypeTok{Vec}\NormalTok{ (n }\FunctionTok{+}\NormalTok{ m) a }\OtherTok{->} \DataTypeTok{Vec}\NormalTok{ n a}
\CommentTok{-- splitAt, as well}
\NormalTok{splitAt}\OtherTok{ ::} \DataTypeTok{Vec}\NormalTok{ (n }\FunctionTok{+}\NormalTok{ m) a }\OtherTok{->}\NormalTok{ (}\DataTypeTok{Vec}\NormalTok{ n a, }\DataTypeTok{Vec}\NormalTok{ m a)}
\end{Highlighting}
\end{Shaded}

\subsection{Generating}\label{generating}

We can directly generate these vectors in interesting ways. Using return-type
polymorphism, we can have the user \emph{directly} request a vector length,
\emph{just} by using type inference or a type annotation. (kind of like
\texttt{read})

For example, we can write a version of \texttt{replicate}:

\begin{Shaded}
\begin{Highlighting}[]
\NormalTok{replicate}\OtherTok{ ::}\NormalTok{ forall n a}\FunctionTok{.} \DataTypeTok{KnownNat}\NormalTok{ n }\OtherTok{=>}\NormalTok{ a }\OtherTok{->} \DataTypeTok{Vec}\NormalTok{ n a}
\NormalTok{replicate x }\FunctionTok{=} \DataTypeTok{UnsafeMkVec} \FunctionTok{$}\NormalTok{ V.replicate l x}
  \KeywordTok{where}
\NormalTok{    l }\FunctionTok{=}\NormalTok{ fromIntegral }\FunctionTok{$}\NormalTok{ natVal (}\DataTypeTok{Proxy} \FunctionTok{@}\NormalTok{n)}
\end{Highlighting}
\end{Shaded}

\begin{Shaded}
\begin{Highlighting}[]
\NormalTok{ghci}\FunctionTok{>}\NormalTok{ replicate }\CharTok{'a'}\OtherTok{ ::} \DataTypeTok{Vec} \DecValTok{5} \DataTypeTok{Char}
\DataTypeTok{UnsafeMkVec}\NormalTok{ (V.fromList [}\CharTok{'a'}\NormalTok{,}\CharTok{'a'}\NormalTok{,}\CharTok{'a'}\NormalTok{,}\CharTok{'a'}\NormalTok{,}\CharTok{'a'}\NormalTok{])}
\end{Highlighting}
\end{Shaded}

Note that normally, \texttt{replicate} takes an \texttt{Int} argument so that
the user can give how long the resulting vector needs to be. However, with our
new \texttt{replicate}, we don't need that \texttt{Int} argument --- the size of
the vector we want can more often than not be inferred automatigically using
type inference!

With this new cleaner type signature, we can actually see that
\texttt{replicate}'s type is something very similar. Look at it carefuly:

\begin{Shaded}
\begin{Highlighting}[]
\NormalTok{replicate}\OtherTok{ ::} \DataTypeTok{KnownNat}\NormalTok{ n }\OtherTok{=>}\NormalTok{ a }\OtherTok{->} \DataTypeTok{Vec}\NormalTok{ n a}
\end{Highlighting}
\end{Shaded}

You might recognize it as the haskellism \texttt{pure}:

\begin{Shaded}
\begin{Highlighting}[]
\NormalTok{pure}\OtherTok{ ::} \DataTypeTok{Applicative}\NormalTok{ f }\OtherTok{=>}\NormalTok{ a }\OtherTok{->}\NormalTok{ f a}
\end{Highlighting}
\end{Shaded}

\texttt{replicate} is actually \texttt{pure} for the Applicative instance of
\texttt{Vec\ n}! As an extra challenge, what would
\texttt{\textless{}*\textgreater{}} be?

\subsubsection{Generating with indices}\label{generating-with-indices}

We can be a little more fancy with \texttt{replicate}, to get what we normally
call \texttt{generate}:

\begin{Shaded}
\begin{Highlighting}[]
\OtherTok{generate ::}\NormalTok{ forall n}\FunctionTok{.} \DataTypeTok{KnownNat}\NormalTok{ n }\OtherTok{=>}\NormalTok{ (}\DataTypeTok{Finite}\NormalTok{ n }\OtherTok{->}\NormalTok{ a) }\OtherTok{->} \DataTypeTok{Vec}\NormalTok{ n a}
\NormalTok{generate f }\FunctionTok{=} \DataTypeTok{UnsafeMkVec} \FunctionTok{$}\NormalTok{ V.generate l (f }\FunctionTok{.}\NormalTok{ finite)}
  \KeywordTok{where}
\NormalTok{    l }\FunctionTok{=}\NormalTok{ fromIntegral }\FunctionTok{$}\NormalTok{ natVal (}\DataTypeTok{Proxy} \FunctionTok{@}\NormalTok{n)}
\end{Highlighting}
\end{Shaded}

\subsubsection{A discussion on the advantages of
type-safety}\label{a-discussion-on-the-advantages-of-type-safety}

I think it's an interesting point that we're using \texttt{Finite\ n} in a
different sense here than in \texttt{index}, for different reasons. In
\texttt{index}, \texttt{Finite} is in the ``negative'' position --- it's
something that the function ``takes''. In \texttt{generate}, \texttt{Finite} is
in the ``positive'' position --- it's something that the function ``gives'' (to
the \texttt{f} in \texttt{generate\ f}).

In the negative position, \texttt{Finite\ n} and type-safety is useful because:

\begin{enumerate}
\def\labelenumi{\arabic{enumi}.}
\tightlist
\item
  It tells the user what sort of values that the function expects. The user
  \emph{knows}, just from the type, that indexing a \texttt{Vec\ 5\ a} requires
  a \texttt{Finite\ 5}, or a number between 0 and 4.
\item
  It guarantees that whatever \texttt{Finite\ n} index you give to
  \texttt{index} is a \emph{valid one}. It's impossible to give \texttt{index}
  an ``invalid index'', so \texttt{index} is allowed to use ``unsafe indexing''
  in its implementation, knowing that nothing bad can be given.
\item
  It lets you develop code in ``typed-hole'' style: if a function requires a
  \texttt{Finite\ 4}, put an underscore there, and GHC will tell you about all
  the \texttt{Finite\ 4}s you have in scope. It can help you write your code for
  you!
\end{enumerate}

In the positive position, \texttt{Finite\ n} and the type-safety have different
uses and advantages: it tells the user what sort of values the function can
return, and also also the type of values that the user has to be expected to
handle. For example, in \texttt{generate}, the fact that the user has to provide
a \texttt{Finite\ n\ -\textgreater{}\ a} tells the user that they have to handle
every number between 0 and n-1, and nothing else.

\subsection{Between Static and Dynamic}\label{between-static-and-dynamic}

So

\end{document}
