\documentclass[]{article}
\usepackage{lmodern}
\usepackage{amssymb,amsmath}
\usepackage{ifxetex,ifluatex}
\usepackage{fixltx2e} % provides \textsubscript
\ifnum 0\ifxetex 1\fi\ifluatex 1\fi=0 % if pdftex
  \usepackage[T1]{fontenc}
  \usepackage[utf8]{inputenc}
\else % if luatex or xelatex
  \ifxetex
    \usepackage{mathspec}
    \usepackage{xltxtra,xunicode}
  \else
    \usepackage{fontspec}
  \fi
  \defaultfontfeatures{Mapping=tex-text,Scale=MatchLowercase}
  \newcommand{\euro}{€}
\fi
% use upquote if available, for straight quotes in verbatim environments
\IfFileExists{upquote.sty}{\usepackage{upquote}}{}
% use microtype if available
\IfFileExists{microtype.sty}{\usepackage{microtype}}{}
\usepackage[margin=1in]{geometry}
\usepackage{color}
\usepackage{fancyvrb}
\newcommand{\VerbBar}{|}
\newcommand{\VERB}{\Verb[commandchars=\\\{\}]}
\DefineVerbatimEnvironment{Highlighting}{Verbatim}{commandchars=\\\{\}}
% Add ',fontsize=\small' for more characters per line
\newenvironment{Shaded}{}{}
\newcommand{\AlertTok}[1]{\textcolor[rgb]{1.00,0.00,0.00}{\textbf{#1}}}
\newcommand{\AnnotationTok}[1]{\textcolor[rgb]{0.38,0.63,0.69}{\textbf{\textit{#1}}}}
\newcommand{\AttributeTok}[1]{\textcolor[rgb]{0.49,0.56,0.16}{#1}}
\newcommand{\BaseNTok}[1]{\textcolor[rgb]{0.25,0.63,0.44}{#1}}
\newcommand{\BuiltInTok}[1]{#1}
\newcommand{\CharTok}[1]{\textcolor[rgb]{0.25,0.44,0.63}{#1}}
\newcommand{\CommentTok}[1]{\textcolor[rgb]{0.38,0.63,0.69}{\textit{#1}}}
\newcommand{\CommentVarTok}[1]{\textcolor[rgb]{0.38,0.63,0.69}{\textbf{\textit{#1}}}}
\newcommand{\ConstantTok}[1]{\textcolor[rgb]{0.53,0.00,0.00}{#1}}
\newcommand{\ControlFlowTok}[1]{\textcolor[rgb]{0.00,0.44,0.13}{\textbf{#1}}}
\newcommand{\DataTypeTok}[1]{\textcolor[rgb]{0.56,0.13,0.00}{#1}}
\newcommand{\DecValTok}[1]{\textcolor[rgb]{0.25,0.63,0.44}{#1}}
\newcommand{\DocumentationTok}[1]{\textcolor[rgb]{0.73,0.13,0.13}{\textit{#1}}}
\newcommand{\ErrorTok}[1]{\textcolor[rgb]{1.00,0.00,0.00}{\textbf{#1}}}
\newcommand{\ExtensionTok}[1]{#1}
\newcommand{\FloatTok}[1]{\textcolor[rgb]{0.25,0.63,0.44}{#1}}
\newcommand{\FunctionTok}[1]{\textcolor[rgb]{0.02,0.16,0.49}{#1}}
\newcommand{\ImportTok}[1]{#1}
\newcommand{\InformationTok}[1]{\textcolor[rgb]{0.38,0.63,0.69}{\textbf{\textit{#1}}}}
\newcommand{\KeywordTok}[1]{\textcolor[rgb]{0.00,0.44,0.13}{\textbf{#1}}}
\newcommand{\NormalTok}[1]{#1}
\newcommand{\OperatorTok}[1]{\textcolor[rgb]{0.40,0.40,0.40}{#1}}
\newcommand{\OtherTok}[1]{\textcolor[rgb]{0.00,0.44,0.13}{#1}}
\newcommand{\PreprocessorTok}[1]{\textcolor[rgb]{0.74,0.48,0.00}{#1}}
\newcommand{\RegionMarkerTok}[1]{#1}
\newcommand{\SpecialCharTok}[1]{\textcolor[rgb]{0.25,0.44,0.63}{#1}}
\newcommand{\SpecialStringTok}[1]{\textcolor[rgb]{0.73,0.40,0.53}{#1}}
\newcommand{\StringTok}[1]{\textcolor[rgb]{0.25,0.44,0.63}{#1}}
\newcommand{\VariableTok}[1]{\textcolor[rgb]{0.10,0.09,0.49}{#1}}
\newcommand{\VerbatimStringTok}[1]{\textcolor[rgb]{0.25,0.44,0.63}{#1}}
\newcommand{\WarningTok}[1]{\textcolor[rgb]{0.38,0.63,0.69}{\textbf{\textit{#1}}}}
\ifxetex
  \usepackage[setpagesize=false, % page size defined by xetex
              unicode=false, % unicode breaks when used with xetex
              xetex]{hyperref}
\else
  \usepackage[unicode=true]{hyperref}
\fi
\hypersetup{breaklinks=true,
            bookmarks=true,
            pdfauthor={Justin Le},
            pdftitle={Introduction to Singletons (Part 4)},
            colorlinks=true,
            citecolor=blue,
            urlcolor=blue,
            linkcolor=magenta,
            pdfborder={0 0 0}}
\urlstyle{same}  % don't use monospace font for urls
% Make links footnotes instead of hotlinks:
\renewcommand{\href}[2]{#2\footnote{\url{#1}}}
\setlength{\parindent}{0pt}
\setlength{\parskip}{6pt plus 2pt minus 1pt}
\setlength{\emergencystretch}{3em}  % prevent overfull lines
\setcounter{secnumdepth}{0}

\title{Introduction to Singletons (Part 4)}
\author{Justin Le}
\date{October 22, 2018}

\begin{document}
\maketitle

\emph{Originally posted on
\textbf{\href{https://blog.jle.im/entry/introduction-to-singletons-4.html}{in
Code}}.}

Hi again! Welcome back; let's jump right into the fourth and final part of our
journey through the \emph{singleton design pattern} and the great
\emph{\href{http://hackage.haskell.org/package/singletons}{singletons}} library.

Please check out
\href{https://blog.jle.im/entries/series/+introduction-to-singletons.html}{the
first three parts of the series} and make sure you are comfortable with them
before reading on. I definitely also recommend trying out some or all of the
exercises, since we are going to be building on the concepts in those posts in a
pretty heavy way.

Today we're going to jump straight into \emph{functional programming} at the
type level. Code in this post is built on \emph{GHC 8.6.1} with the
\emph{\href{https://www.stackage.org/nightly-2018-09-29}{nightly-2018-09-29}}
snapshot (so, \emph{singletons-2.5}). However, unless noted, all of the code
should still work with \emph{GHC 8.4} and \emph{singletons-2.4}.

\hypertarget{review}{%
\section{Review}\label{review}}

Just as a quick review, this entire series we have been working with a
\texttt{Door} type:

\begin{Shaded}
\begin{Highlighting}[]
\CommentTok{{-}{-} source: https://github.com/mstksg/inCode/tree/master/code{-}samples/singletons/Door4.hs\#L24{-}L33}

\OperatorTok{$}\NormalTok{(singletons [d|}
\NormalTok{  data DoorState = Opened | Closed | Locked}
\NormalTok{    deriving (Show, Eq, Ord)}
\NormalTok{  |])}

\KeywordTok{data} \DataTypeTok{Door}\OtherTok{ ::} \DataTypeTok{DoorState} \OtherTok{{-}>} \DataTypeTok{Type} \KeywordTok{where}
    \DataTypeTok{UnsafeMkDoor}\OtherTok{ ::}\NormalTok{ \{}\OtherTok{ doorMaterial ::} \DataTypeTok{String}\NormalTok{ \} }\OtherTok{{-}>} \DataTypeTok{Door}\NormalTok{ s}

\OtherTok{mkDoor ::} \DataTypeTok{Sing}\NormalTok{ s }\OtherTok{{-}>} \DataTypeTok{String} \OtherTok{{-}>} \DataTypeTok{Door}\NormalTok{ s}
\NormalTok{mkDoor \_ }\OtherTok{=} \DataTypeTok{UnsafeMkDoor}
\end{Highlighting}
\end{Shaded}

And we talked about using \texttt{Sing\ s}, or \texttt{SDoorState\ s}, to
represent the state of the door (in its type) as a run-time value. We've been
using a wrapper to existentially hide the door state type, but also stuffing in
a singleton to let us recover the type information once we want it again:

\begin{Shaded}
\begin{Highlighting}[]
\KeywordTok{data} \DataTypeTok{SomeDoor}\OtherTok{ ::} \DataTypeTok{Type} \KeywordTok{where}
    \DataTypeTok{MkSomeDoor}\OtherTok{ ::} \DataTypeTok{Sing}\NormalTok{ s }\OtherTok{{-}>} \DataTypeTok{Door}\NormalTok{ s }\OtherTok{{-}>} \DataTypeTok{SomeDoor}

\OtherTok{mkSomeDoor ::} \DataTypeTok{DoorState} \OtherTok{{-}>} \DataTypeTok{String} \OtherTok{{-}>} \DataTypeTok{SomeDoor}
\NormalTok{mkSomeDoor ds mat }\OtherTok{=}\NormalTok{ withSomeSing ds }\OperatorTok{$}\NormalTok{ \textbackslash{}dsSing }\OtherTok{{-}>}
    \DataTypeTok{MkSomeDoor}\NormalTok{ dsSing (mkDoor dsSing mat)}
\end{Highlighting}
\end{Shaded}

In Part 3 we talked about a \texttt{Pass} data type that we used to talk about
whether or not we can walk through or knock on a door:

\begin{Shaded}
\begin{Highlighting}[]
\OperatorTok{$}\NormalTok{(singletons [d|}
\NormalTok{  data Pass = Obstruct | Allow}
\NormalTok{    deriving (Show, Eq, Ord)}
\NormalTok{  |])}
\end{Highlighting}
\end{Shaded}

And we defined type-level functions on it using \emph{singletons} Template
Haskell:

\begin{Shaded}
\begin{Highlighting}[]
\OperatorTok{$}\NormalTok{(singletons [d|}
\NormalTok{  statePass :: DoorState {-}> Pass}
\NormalTok{  statePass Opened = Allow}
\NormalTok{  statePass Closed = Obstruct}
\NormalTok{  statePass Locked = Obstruct}
\NormalTok{  |])}
\end{Highlighting}
\end{Shaded}

This essentially generates these three things:

\begin{Shaded}
\begin{Highlighting}[]
\OtherTok{statePass ::} \DataTypeTok{DoorState} \OtherTok{{-}>} \DataTypeTok{Pass}
\NormalTok{statePass }\DataTypeTok{Opened} \OtherTok{=} \DataTypeTok{Allow}
\NormalTok{statePass }\DataTypeTok{Closed} \OtherTok{=} \DataTypeTok{Obstruct}
\NormalTok{statePass }\DataTypeTok{Locked} \OtherTok{=} \DataTypeTok{Obstruct}

\KeywordTok{type} \KeywordTok{family} \DataTypeTok{StatePass}\NormalTok{ (}\OtherTok{s ::} \DataTypeTok{DoorState}\NormalTok{)}\OtherTok{ ::} \DataTypeTok{Pass} \KeywordTok{where}
    \DataTypeTok{StatePass} \DataTypeTok{\textquotesingle{}Opened} \OtherTok{=} \DataTypeTok{\textquotesingle{}Allow}
    \DataTypeTok{StatePass} \DataTypeTok{\textquotesingle{}Closed} \OtherTok{=} \DataTypeTok{\textquotesingle{}Obstruct}
    \DataTypeTok{StatePass} \DataTypeTok{\textquotesingle{}Locked} \OtherTok{=} \DataTypeTok{\textquotesingle{}Obstruct}

\OtherTok{sStatePass ::} \DataTypeTok{Sing}\NormalTok{ s }\OtherTok{{-}>} \DataTypeTok{Sing}\NormalTok{ (}\DataTypeTok{StatePass}\NormalTok{ s)}
\NormalTok{sStatePass }\OtherTok{=}\NormalTok{ \textbackslash{}}\KeywordTok{case}
    \DataTypeTok{SOpened} \OtherTok{{-}>} \DataTypeTok{SAllow}
    \DataTypeTok{SClosed} \OtherTok{{-}>} \DataTypeTok{SObstruct}
    \DataTypeTok{SLocked} \OtherTok{{-}>} \DataTypeTok{SObstruct}
\end{Highlighting}
\end{Shaded}

And we can use \texttt{StatePass} as a type-level function while using
\texttt{sStatePass} to manipulate the singletons representing \texttt{s} and
\texttt{StatePass\ s}.

We used this as a constraint to restrict how we can call our functions:

\begin{Shaded}
\begin{Highlighting}[]
\CommentTok{{-}{-} source: https://github.com/mstksg/inCode/tree/master/code{-}samples/singletons/Door3.hs\#L89{-}L90}

\OtherTok{knockP ::}\NormalTok{ (}\DataTypeTok{StatePass}\NormalTok{ s }\OperatorTok{\textasciitilde{}} \DataTypeTok{\textquotesingle{}Obstruct}\NormalTok{) }\OtherTok{=>} \DataTypeTok{Door}\NormalTok{ s }\OtherTok{{-}>} \DataTypeTok{IO}\NormalTok{ ()}
\NormalTok{knockP d }\OtherTok{=} \FunctionTok{putStrLn} \OperatorTok{$} \StringTok{"Knock knock on "} \OperatorTok{++}\NormalTok{ doorMaterial d }\OperatorTok{++} \StringTok{" door!"}
\end{Highlighting}
\end{Shaded}

But then we wondered\ldots is there a way to not only \emph{restrict} our
functions, but to describe how the inputs and outputs are related to each other?

\hypertarget{inputs-and-outputs}{%
\section{Inputs and Outputs}\label{inputs-and-outputs}}

In the past we have settled with very simple relationships, like:

\begin{Shaded}
\begin{Highlighting}[]
\OtherTok{closeDoor ::} \DataTypeTok{Door} \DataTypeTok{\textquotesingle{}Opened} \OtherTok{{-}>} \DataTypeTok{Door} \DataTypeTok{\textquotesingle{}Closed}
\end{Highlighting}
\end{Shaded}

This means that the relationship between the input and output is that the input
is opened\ldots and is then closed.

However, armed with promotion of type-level functions, writing more complex
relationships becomes fairly straightforward!

We can write a function \texttt{mergeDoor} that ``merges'' two doors together,
in sequence:

\begin{Shaded}
\begin{Highlighting}[]
\OtherTok{mergeDoor ::} \DataTypeTok{Door}\NormalTok{ s }\OtherTok{{-}>} \DataTypeTok{Door}\NormalTok{ t }\OtherTok{{-}>} \DataTypeTok{Door} \OperatorTok{????}
\NormalTok{mergeDoor d e }\OtherTok{=} \DataTypeTok{UnsafeMkDoor} \OperatorTok{$}\NormalTok{ doorMaterial d }\OperatorTok{++} \StringTok{" and "} \OperatorTok{++}\NormalTok{ doorMaterial e}
\end{Highlighting}
\end{Shaded}

A merged door will have a material that is composite of the original materials.
But, what will the new \texttt{DoorState} be? What goes in the \texttt{???}
above?

Well, if we can write the function as a normal function in
values\ldots{}\emph{singletons} lets us use it as a function on types. Let's
write that relationship. Let's say merging takes on the higher ``security''
option --- merging opened with locked is locked, merging closed with opened is
closed, merging locked with closed is locked.

\begin{Shaded}
\begin{Highlighting}[]
\OperatorTok{$}\NormalTok{(singletons [d|}
\NormalTok{  mergeState :: DoorState {-}> DoorState {-}> DoorState}
\NormalTok{  mergeState Opened d      = d}
\NormalTok{  mergeState Closed Opened = Closed}
\NormalTok{  mergeState Closed Closed = Closed}
\NormalTok{  mergeState Closed Locked = Locked}
\NormalTok{  mergeState Locked \_      = Locked}
\NormalTok{  |])}

\CommentTok{{-}{-} Alternatively, taking advantage of the derived Ord instance:}
\OperatorTok{$}\NormalTok{(singletons [d|}
\NormalTok{  mergeState :: DoorState {-}> DoorState {-}> DoorState}
\NormalTok{  mergeState = max}
\NormalTok{  |])}
\end{Highlighting}
\end{Shaded}

This makes writing \texttt{mergeDoor}'s type clean to read:

\begin{Shaded}
\begin{Highlighting}[]
\CommentTok{{-}{-} source: https://github.com/mstksg/inCode/tree/master/code{-}samples/singletons/Door4.hs\#L43{-}L47}

\NormalTok{mergeDoor}
\OtherTok{    ::} \DataTypeTok{Door}\NormalTok{ s}
    \OtherTok{{-}>} \DataTypeTok{Door}\NormalTok{ t}
    \OtherTok{{-}>} \DataTypeTok{Door}\NormalTok{ (}\DataTypeTok{MergeState}\NormalTok{ s t)}
\NormalTok{mergeDoor d e }\OtherTok{=} \DataTypeTok{UnsafeMkDoor} \OperatorTok{$}\NormalTok{ doorMaterial d }\OperatorTok{++} \StringTok{" and "} \OperatorTok{++}\NormalTok{ doorMaterial e}
\end{Highlighting}
\end{Shaded}

And, with the help of singletons, we can also write this for our doors where we
don't know the types until runtime:

\begin{Shaded}
\begin{Highlighting}[]
\CommentTok{{-}{-} source: https://github.com/mstksg/inCode/tree/master/code{-}samples/singletons/Door4.hs\#L49{-}L51}

\OtherTok{mergeSomeDoor ::} \DataTypeTok{SomeDoor} \OtherTok{{-}>} \DataTypeTok{SomeDoor} \OtherTok{{-}>} \DataTypeTok{SomeDoor}
\NormalTok{mergeSomeDoor (}\DataTypeTok{MkSomeDoor}\NormalTok{ s d) (}\DataTypeTok{MkSomeDoor}\NormalTok{ t e) }\OtherTok{=}
    \DataTypeTok{MkSomeDoor}\NormalTok{ (sMergeState s t) (mergeDoor d e)}
\end{Highlighting}
\end{Shaded}

To see why this typechecks properly, compare the types of \texttt{sMergeState}
and \texttt{mergeDoor}:

\begin{Shaded}
\begin{Highlighting}[]
\OtherTok{sMergeState ::} \DataTypeTok{Sing}\NormalTok{ s }\OtherTok{{-}>} \DataTypeTok{Sing}\NormalTok{ t }\OtherTok{{-}>} \DataTypeTok{Sing}\NormalTok{ (}\DataTypeTok{MergeState}\NormalTok{ s t)}
\OtherTok{mergeDoor   ::} \DataTypeTok{Door}\NormalTok{ s }\OtherTok{{-}>} \DataTypeTok{Door}\NormalTok{ t }\OtherTok{{-}>} \DataTypeTok{Door}\NormalTok{ (}\DataTypeTok{MergeState}\NormalTok{ s t)}

\DataTypeTok{MkSomeDoor}\OtherTok{  ::} \DataTypeTok{Sing}\NormalTok{ (}\DataTypeTok{MergeState}\NormalTok{ s t) }\OtherTok{{-}>} \DataTypeTok{Door}\NormalTok{ (}\DataTypeTok{MergeState}\NormalTok{ s t) }\OtherTok{{-}>} \DataTypeTok{SomeDoor}
\end{Highlighting}
\end{Shaded}

Because the results both create types \texttt{MergeState\ s\ t},
\texttt{MkSomeDoor} is happy to apply them to each other, and everything
typechecks. However, if, say, we directly stuffed \texttt{s} or \texttt{t} into
\texttt{MkSomeDoor}, things would fall apart and not typecheck.

And so now we have full expressiveness in determining input and output
relationships! Once we unlock the power of type-level functions with
\emph{singletons}, writing type-level relationships become as simple as writing
value-level ones. If you can write a value-level function, you can write a
\emph{type-level} function.

\hypertarget{kicking-it-up-a-notch}{%
\subsection{Kicking it up a notch}\label{kicking-it-up-a-notch}}

How far we can really take this?

Let's make a data type that represents a \emph{series of hallways}, each linked
by a door. A hallway is either an empty stretch with no door, or two hallways
linked by a door. We'll structure it like a linked list, and store the list of
all door states as a type-level list as a type parameter:

\begin{Shaded}
\begin{Highlighting}[]
\CommentTok{{-}{-} source: https://github.com/mstksg/inCode/tree/master/code{-}samples/singletons/Door4.hs\#L53{-}L61}

\KeywordTok{data} \DataTypeTok{Hallway}\OtherTok{ ::}\NormalTok{ [}\DataTypeTok{DoorState}\NormalTok{] }\OtherTok{{-}>} \DataTypeTok{Type} \KeywordTok{where}
    \DataTypeTok{HEnd}\OtherTok{  ::} \DataTypeTok{Hallway}\NormalTok{ \textquotesingle{}[]        }\CommentTok{{-}{-} \^{} end of the hallway, a stretch with no}
                                \CommentTok{{-}{-}   doors}
\OtherTok{    (:<\#) ::} \DataTypeTok{Door}\NormalTok{ s}
          \OtherTok{{-}>} \DataTypeTok{Hallway}\NormalTok{ ss}
          \OtherTok{{-}>} \DataTypeTok{Hallway}\NormalTok{ (s \textquotesingle{}}\OperatorTok{:}\NormalTok{ ss)  }\CommentTok{{-}{-} \^{} A door connected to a hallway is a new}
                                \CommentTok{{-}{-}   hallway, and we track the door\textquotesingle{}s state}
                                \CommentTok{{-}{-}   in the list of hallway door states}
\KeywordTok{infixr} \DecValTok{5} \OperatorTok{:<\#}
\end{Highlighting}
\end{Shaded}

(If you need a refresher on type-level lists, check out
\href{https://blog.jle.im/entry/introduction-to-singletons-1.html\#the-singletons-library}{the
quick introduction in Part 1} and
\href{https://blog.jle.im/entry/introduction-to-singletons-2.html\#exercises}{Exercise
4 in Part 2})

So we might have:

\begin{Shaded}
\begin{Highlighting}[]
\NormalTok{ghci}\OperatorTok{>} \KeywordTok{let}\NormalTok{ door1 }\OtherTok{=}\NormalTok{ mkDoor }\DataTypeTok{SClosed} \StringTok{"Oak"}
\NormalTok{ghci}\OperatorTok{>} \KeywordTok{let}\NormalTok{ door2 }\OtherTok{=}\NormalTok{ mkDoor }\DataTypeTok{SOpened} \StringTok{"Spruce"}
\NormalTok{ghci}\OperatorTok{>} \KeywordTok{let}\NormalTok{ door3 }\OtherTok{=}\NormalTok{ mkDoor }\DataTypeTok{SLocked} \StringTok{"Acacia"}
\NormalTok{ghci}\OperatorTok{>} \OperatorTok{:}\NormalTok{t door1 }\OperatorTok{:<\#}\NormalTok{ door2 }\OperatorTok{:<\#}\NormalTok{ door3 }\OperatorTok{:<\#} \DataTypeTok{HEnd}
\DataTypeTok{Hallway}\NormalTok{ \textquotesingle{}[ }\DataTypeTok{\textquotesingle{}Closed}\NormalTok{, }\DataTypeTok{\textquotesingle{}Opened}\NormalTok{, }\DataTypeTok{\textquotesingle{}Locked}\NormalTok{ ]}
\end{Highlighting}
\end{Shaded}

That is, a \texttt{Hallway\ \textquotesingle{}{[}\ s,\ t,\ u\ {]}} is a hallway
consisting of a \texttt{Door\ s}, a \texttt{Door\ t}, and a \texttt{Door\ u},
constructed like a linked list in Haskell.

Now, let's write a function to \emph{collapse all doors in a hallway down to a
single door}:

\begin{Shaded}
\begin{Highlighting}[]
\OtherTok{collapseHallway ::} \DataTypeTok{Hallway}\NormalTok{ ss }\OtherTok{{-}>} \DataTypeTok{Door} \OperatorTok{?????}
\end{Highlighting}
\end{Shaded}

Basically, we want to merge all of the doors one after the other, collapsing it
until we have a single door state. Luckily, \texttt{MergeState} is both
commutative and associative and has an identity, so this can be defined
sensibly.

First, let's think about the type we want. What will the result of merging
\texttt{ss} be?

We can pattern match and collapse an entire list down item-by-item:

\begin{Shaded}
\begin{Highlighting}[]
\OperatorTok{$}\NormalTok{(singletons [d|}
\NormalTok{  mergeStateList :: [DoorState] {-}> DoorState}
\NormalTok{  mergeStateList []     = Opened               {-}{-} \^{} the identity of mergeState}
\NormalTok{  mergeStateList (s:ss) = s \textasciigrave{}mergeState\textasciigrave{} mergeStateList ss}
\NormalTok{  |])}
\end{Highlighting}
\end{Shaded}

Again, remember that this also defines the type family \texttt{MergeStateList}
and the singleton function
\texttt{sMergeStateList\ ::\ Sing\ ss\ -\textgreater{}\ Sing\ (MergeStateList\ ss)}.

With this, we can write \texttt{collapseHallway}:

\begin{Shaded}
\begin{Highlighting}[]
\CommentTok{{-}{-} source: https://github.com/mstksg/inCode/tree/master/code{-}samples/singletons/Door4.hs\#L69{-}L71}

\OtherTok{collapseHallway ::} \DataTypeTok{Hallway}\NormalTok{ ss }\OtherTok{{-}>} \DataTypeTok{Door}\NormalTok{ (}\DataTypeTok{MergeStateList}\NormalTok{ ss)}
\NormalTok{collapseHallway }\DataTypeTok{HEnd}       \OtherTok{=}\NormalTok{ mkDoor }\DataTypeTok{SOpened} \StringTok{"End of Hallway"}
\NormalTok{collapseHallway (d }\OperatorTok{:<\#}\NormalTok{ ds) }\OtherTok{=}\NormalTok{ d }\OtherTok{\textasciigrave{}mergeDoor\textasciigrave{}}\NormalTok{ collapseHallway ds}
\end{Highlighting}
\end{Shaded}

Now, because the structure of \texttt{collapseHallway} perfectly mirrors the
structure of \texttt{mergeStateList}, this all typechecks, and we're done!

\begin{Shaded}
\begin{Highlighting}[]
\NormalTok{ghci}\OperatorTok{>}\NormalTok{ collapseHallway (door1 }\OperatorTok{:<\#}\NormalTok{ door2 }\OperatorTok{:<\#}\NormalTok{ door3 }\OperatorTok{:<\#} \DataTypeTok{HEnd}\NormalTok{)}
\DataTypeTok{UnsafeMkDoor} \StringTok{"Oak and Spruce and Acacia and End of Hallway"}
\OtherTok{    ::} \DataTypeTok{Door} \DataTypeTok{\textquotesingle{}Locked}
\end{Highlighting}
\end{Shaded}

Note one nice benefit -- the door state of
\texttt{collapseHallway\ (door1\ :\textless{}\#\ door2\ :\textless{}\#\ door3\ :\textless{}\#\ HEnd)}
is known at compile-time to be \texttt{Door\ \textquotesingle{}Locked}, if the
types of all of the component doors are also known!

\hypertarget{functional-programming}{%
\section{Functional Programming}\label{functional-programming}}

We went over that all a bit fast, but some of you might have noticed that the
definition of \texttt{mergeStateList} bears a really strong resemblance to a
very common Haskell list processing pattern:

\begin{Shaded}
\begin{Highlighting}[]
\OtherTok{mergeStateList ::}\NormalTok{ [}\DataTypeTok{DoorState}\NormalTok{] }\OtherTok{{-}>} \DataTypeTok{DoorState}
\NormalTok{mergeStateList []     }\OtherTok{=} \DataTypeTok{Opened}               \CommentTok{{-}{-} \^{} the identity of mergeState}
\NormalTok{mergeStateList (s}\OperatorTok{:}\NormalTok{ss) }\OtherTok{=}\NormalTok{ s }\OtherTok{\textasciigrave{}mergeState\textasciigrave{}}\NormalTok{ mergeStateList ss}
\end{Highlighting}
\end{Shaded}

The algorithm is to basically \texttt{{[}{]}} with \texttt{Opened}, and all
\texttt{(:)} with \texttt{mergeState}. If this sounds familiar, that's because
this is exactly a \emph{right fold}! (In fact,
\href{http://hackage.haskell.org/package/hlint}{hlint} actually made this
suggestion to me while I was writing this)

\begin{Shaded}
\begin{Highlighting}[]
\OtherTok{mergeStateList ::}\NormalTok{ [}\DataTypeTok{DoorState}\NormalTok{] }\OtherTok{{-}>} \DataTypeTok{DoorState}
\NormalTok{mergeStateList }\OtherTok{=} \FunctionTok{foldr}\NormalTok{ mergeState }\DataTypeTok{Opened}
\end{Highlighting}
\end{Shaded}

In Haskell, we are always encouraged to use higher-order functions whenever
possible instead of explicit recursion, both because explicit recursion opens
you up to a lot of potential bugs, and also because using established
higher-order functions make your code more readable.

So, as Haskellers, let us hold ourselves to a higher standard and not be
satisfied with a \texttt{MergeState} written using explicit recursion. Let us
instead go \emph{full fold} --- ONWARD HO!

\hypertarget{the-problem}{%
\subsection{The Problem}\label{the-problem}}

Initial attempts to write a higher-order type-level function as a type family,
however, serve to temper our enthusiasm.

\begin{Shaded}
\begin{Highlighting}[]
\KeywordTok{type} \KeywordTok{family} \DataTypeTok{Foldr}\NormalTok{ (}\OtherTok{f ::}\NormalTok{ j }\OtherTok{{-}>}\NormalTok{ k }\OtherTok{{-}>}\NormalTok{ k) (}\OtherTok{z ::}\NormalTok{ k) (}\OtherTok{xs ::}\NormalTok{ [j])}\OtherTok{ ::}\NormalTok{ k }\KeywordTok{where}
    \DataTypeTok{Foldr}\NormalTok{ f z \textquotesingle{}[]       }\OtherTok{=}\NormalTok{ z}
    \DataTypeTok{Foldr}\NormalTok{ f z (x \textquotesingle{}}\OperatorTok{:}\NormalTok{ xs) }\OtherTok{=}\NormalTok{ f x (}\DataTypeTok{Foldr}\NormalTok{ f z xs)}
\end{Highlighting}
\end{Shaded}

So far so good right? So we should expect to be able to write
\texttt{MergeStateList} using \texttt{Foldr}, \texttt{MergeState}, and
\texttt{\textquotesingle{}Opened}

\begin{Shaded}
\begin{Highlighting}[]
\KeywordTok{type} \DataTypeTok{MergeStateList}\NormalTok{ ss }\OtherTok{=} \DataTypeTok{Foldr} \DataTypeTok{MergeState} \DataTypeTok{\textquotesingle{}Opened}\NormalTok{ ss}
\end{Highlighting}
\end{Shaded}

Ah, but the compiler is here to tell you this isn't allowed in Haskell:

\begin{verbatim}
    • The type family ‘MergeState’ should have 2 arguments, but has been given none
    • In the equations for closed type family ‘MergeStateList’
      In the type family declaration for ‘MergeStateList’
\end{verbatim}

What happened? To figure out, we have to remember that pesky restriction on type
synonyms and type families: they can \emph{not} be used partially applied
(``unsaturated''), and must always be fully applied (``saturated''). For the
most part, only \emph{type constructors} (like \texttt{Maybe}, \texttt{Either},
\texttt{IO}) and lifted DataKinds data constructors (like
\texttt{\textquotesingle{}Just}, \texttt{\textquotesingle{}(:)}) in Haskell can
ever be partially applied at the type level. We therefore can't use
\texttt{MergeState} as an argument to \texttt{Foldr}, because
\texttt{MergeState} must always be fully applied.

Unfortunately for us, this makes our \texttt{Foldr} effectively useless. That's
because we're always going to want to pass in type families (like
\texttt{MergeState}), so there's pretty much literally no way to ever actually
call \texttt{Foldr} except with type constructors or lifted DataKinds data
constructors.

So\ldots back to the drawing board?

\hypertarget{defunctionalization}{%
\section{Defunctionalization}\label{defunctionalization}}

I like to mentally think of the \emph{singletons} library as having two parts:
the first is linking lifted DataKinds types with run-time values to allow us to
manipulate types at runtime as first-class values. The second is a system for
effective \emph{functional programming} at the type level.

To make a working \texttt{Foldr}, we're going to have to jump into that second
half:
\emph{\href{https://en.wikipedia.org/wiki/Defunctionalization}{defunctionalization}}.

Defunctionalization is a technique invented in the early 70's as a way of
compiling higher-order functions into first-order functions in target languages.
The main idea is:

\begin{itemize}
\tightlist
\item
  Instead of working with \emph{functions}, work with \emph{symbols representing
  functions}.
\item
  Build your final functions and values by composing and combining these
  symbols.
\item
  At the end of it all, have a single \texttt{Apply} function interpret all of
  your symbols and produce the value you want.
\end{itemize}

In \emph{singletons} these symbols are implemented as ``dummy'' empty data
constructors, and \texttt{Apply} is a type family.

To help us understand singleton's defunctionalization system better, let's build
our own defunctionalization system from scratch.

First, a little trick to make things easier to read:

\begin{Shaded}
\begin{Highlighting}[]
\CommentTok{{-}{-} source: https://github.com/mstksg/inCode/tree/master/code{-}samples/singletons/Defunctionalization.hs\#L14{-}L17}

\KeywordTok{data} \DataTypeTok{TyFun}\NormalTok{ a b}
\KeywordTok{type}\NormalTok{ a }\OperatorTok{\textasciitilde{}>}\NormalTok{ b }\OtherTok{=} \DataTypeTok{TyFun}\NormalTok{ a b }\OtherTok{{-}>} \DataTypeTok{Type}

\KeywordTok{infixr} \DecValTok{0} \OperatorTok{\textasciitilde{}>}
\end{Highlighting}
\end{Shaded}

\hypertarget{our-first-symbols}{%
\subsection{Our First Symbols}\label{our-first-symbols}}

Now we can define a dummy data type like \texttt{Id}, which represents the
identity function \texttt{id}:

\begin{Shaded}
\begin{Highlighting}[]
\CommentTok{{-}{-} source: https://github.com/mstksg/inCode/tree/master/code{-}samples/singletons/Defunctionalization.hs\#L25{-}L25}

\KeywordTok{data} \DataTypeTok{Id}\OtherTok{ ::}\NormalTok{ a }\OperatorTok{\textasciitilde{}>}\NormalTok{ a}
\end{Highlighting}
\end{Shaded}

The ``actual'' kind of \texttt{Id} is
\texttt{Id\ ::\ TyFun\ a\ a\ -\textgreater{}\ Type}; you can imagine
\texttt{TyFun\ a\ a} as a phantom parameter that signifies that \texttt{Id}
represents a function from \texttt{a} to \texttt{a}. It's essentially a nice
trick to allow you to write
\texttt{Id\ ::\ a\ \textasciitilde{}\textgreater{}\ a} as a kind signature.

Now, \texttt{Id} is \emph{not} a function\ldots it's a \emph{dummy type
constructor} that \emph{represents} a function \texttt{a\ -\textgreater{}\ a}. A
type constructor of kind \texttt{a\ \textasciitilde{}\textgreater{}\ a}
represents a \emph{defunctionalization symbol} -- a type constructor that
represents a function from \texttt{a} to \texttt{a}.

To interpret it, we need to write our global interpreter function:

\begin{Shaded}
\begin{Highlighting}[]
\CommentTok{{-}{-} source: https://github.com/mstksg/inCode/tree/master/code{-}samples/singletons/Defunctionalization.hs\#L19{-}L19}

\KeywordTok{type} \KeywordTok{family} \DataTypeTok{Apply}\NormalTok{ (}\OtherTok{f ::}\NormalTok{ a }\OperatorTok{\textasciitilde{}>}\NormalTok{ b) (}\OtherTok{x ::}\NormalTok{ a)}\OtherTok{ ::}\NormalTok{ b}
\end{Highlighting}
\end{Shaded}

That's the syntax for the definition of an \emph{open} type family in Haskell:
users are free to add their own instances, just like how type classes are
normally open in Haskell.

Let's tell \texttt{Apply} how to interpret \texttt{Id}:

\begin{Shaded}
\begin{Highlighting}[]
\CommentTok{{-}{-} source: https://github.com/mstksg/inCode/tree/master/code{-}samples/singletons/Defunctionalization.hs\#L26{-}L26}

\KeywordTok{type} \KeywordTok{instance} \DataTypeTok{Apply} \DataTypeTok{Id}\NormalTok{ x }\OtherTok{=}\NormalTok{ x}
\end{Highlighting}
\end{Shaded}

The above is the actual function definition, like writing \texttt{id\ x\ =\ x}.
We can now \emph{call} \texttt{Id} to get an actual type in return:

\begin{Shaded}
\begin{Highlighting}[]
\NormalTok{ghci}\OperatorTok{>} \OperatorTok{:}\NormalTok{kind}\OperatorTok{!} \DataTypeTok{Apply} \DataTypeTok{Id} \DataTypeTok{\textquotesingle{}True}
\DataTypeTok{\textquotesingle{}True}
\end{Highlighting}
\end{Shaded}

(Remember, \texttt{:kind!} is the ghci command to evaluate a type family)

Let's define another one! We'll implement \texttt{Not}:

\begin{Shaded}
\begin{Highlighting}[]
\CommentTok{{-}{-} source: https://github.com/mstksg/inCode/tree/master/code{-}samples/singletons/Defunctionalization.hs\#L28{-}L30}

\KeywordTok{data} \DataTypeTok{Not}\OtherTok{ ::} \DataTypeTok{Bool} \OperatorTok{\textasciitilde{}>} \DataTypeTok{Bool}
\KeywordTok{type} \KeywordTok{instance} \DataTypeTok{Apply} \DataTypeTok{Not} \DataTypeTok{\textquotesingle{}False} \OtherTok{=} \DataTypeTok{\textquotesingle{}True}
\KeywordTok{type} \KeywordTok{instance} \DataTypeTok{Apply} \DataTypeTok{Not} \DataTypeTok{\textquotesingle{}True}  \OtherTok{=} \DataTypeTok{\textquotesingle{}False}
\end{Highlighting}
\end{Shaded}

We can try it out:

\begin{Shaded}
\begin{Highlighting}[]
\NormalTok{ghci}\OperatorTok{>} \OperatorTok{:}\NormalTok{kind}\OperatorTok{!} \DataTypeTok{Apply} \DataTypeTok{Not} \DataTypeTok{\textquotesingle{}True}
\DataTypeTok{\textquotesingle{}False}
\NormalTok{ghci}\OperatorTok{>} \OperatorTok{:}\NormalTok{kind}\OperatorTok{!} \DataTypeTok{Apply} \DataTypeTok{Not} \DataTypeTok{\textquotesingle{}False}
\DataTypeTok{\textquotesingle{}True}
\end{Highlighting}
\end{Shaded}

It can be convenient to define an infix synonym for \texttt{Apply}:

\begin{Shaded}
\begin{Highlighting}[]
\CommentTok{{-}{-} source: https://github.com/mstksg/inCode/tree/master/code{-}samples/singletons/Defunctionalization.hs\#L21{-}L23}

\KeywordTok{type}\NormalTok{ f }\OperatorTok{@@}\NormalTok{ a }\OtherTok{=} \DataTypeTok{Apply}\NormalTok{ f a}

\KeywordTok{infixl} \DecValTok{9} \OperatorTok{@@}
\end{Highlighting}
\end{Shaded}

Then we can write:

\begin{Shaded}
\begin{Highlighting}[]
\NormalTok{ghci}\OperatorTok{>} \OperatorTok{:}\NormalTok{kind}\OperatorTok{!} \DataTypeTok{Not} \OperatorTok{@@} \DataTypeTok{\textquotesingle{}False}
\DataTypeTok{\textquotesingle{}True}
\NormalTok{ghci}\OperatorTok{>} \OperatorTok{:}\NormalTok{kind}\OperatorTok{!} \DataTypeTok{Id} \OperatorTok{@@} \DataTypeTok{\textquotesingle{}True}
\DataTypeTok{\textquotesingle{}True}
\end{Highlighting}
\end{Shaded}

Remember, \texttt{Id} and \texttt{Not} are not actual functions --- they're just
dummy data types (``defunctionalization symbols''), and we define the functions
they represent through the global \texttt{Apply} type function.

\hypertarget{a-bit-of-principle}{%
\subsection{A Bit of Principle}\label{a-bit-of-principle}}

So we've got the basics of defunctionalization --- instead of using functions
directly, use dummy symbols that encode your functions that are interpreted
using \texttt{Apply}. Let's add a bit of principle to make this all a bit more
scalable.

The singletons library adopts a few conventions for linking all of these
together. Using the \texttt{Not} function as an example, if we wanted to lift
the function:

\begin{Shaded}
\begin{Highlighting}[]
\FunctionTok{not}\OtherTok{ ::} \DataTypeTok{Bool} \OtherTok{{-}>} \DataTypeTok{Bool}
\FunctionTok{not} \DataTypeTok{False} \OtherTok{=} \DataTypeTok{True}
\FunctionTok{not} \DataTypeTok{True}  \OtherTok{=} \DataTypeTok{Flse}
\end{Highlighting}
\end{Shaded}

We already know about the type family and singleton function this would produce:

\begin{Shaded}
\begin{Highlighting}[]
\KeywordTok{type} \KeywordTok{family} \DataTypeTok{Not}\NormalTok{ (}\OtherTok{x ::} \DataTypeTok{Bool}\NormalTok{)}\OtherTok{ ::} \DataTypeTok{Bool} \KeywordTok{where}
    \DataTypeTok{Not} \DataTypeTok{\textquotesingle{}False} \OtherTok{=} \DataTypeTok{\textquotesingle{}True}
    \DataTypeTok{Not} \DataTypeTok{\textquotesingle{}True}  \OtherTok{=} \DataTypeTok{\textquotesingle{}False}

\OtherTok{sNot ::} \DataTypeTok{Sing}\NormalTok{ x }\OtherTok{{-}>} \DataTypeTok{Sing}\NormalTok{ (}\DataTypeTok{Not}\NormalTok{ x)}
\NormalTok{sNot }\DataTypeTok{SFalse} \OtherTok{=} \DataTypeTok{STrue}
\NormalTok{sNot }\DataTypeTok{STrue}  \OtherTok{=} \DataTypeTok{SFalse}
\end{Highlighting}
\end{Shaded}

But the singletons library also produces the following \emph{defunctionalization
symbols}, according to a naming convention:

\begin{Shaded}
\begin{Highlighting}[]
\KeywordTok{data} \DataTypeTok{NotSym0}\OtherTok{ ::} \DataTypeTok{Bool} \OperatorTok{\textasciitilde{}>} \DataTypeTok{Bool}
\KeywordTok{type} \KeywordTok{instance} \DataTypeTok{Apply} \DataTypeTok{NotSym0}\NormalTok{ x }\OtherTok{=} \DataTypeTok{Not}\NormalTok{ x}

\CommentTok{{-}{-} also generated for consistency}
\KeywordTok{type} \DataTypeTok{NotSym1}\NormalTok{ x }\OtherTok{=} \DataTypeTok{Not}\NormalTok{ x}
\end{Highlighting}
\end{Shaded}

\texttt{NotSym0} is the \emph{defunctionalization symbol} associated with the
\texttt{Not} type family, defined so that \texttt{NotSym0\ @@\ x\ =\ Not\ x}.
Its purpose is to allow us to \emph{pass in} \texttt{Not} as an \emph{un-applied
function}. The \texttt{Sym0} suffix is a naming convention, and the 0 stands for
``expects 0 arguments''. Similarly for \texttt{NotSym1} -- the 1 stands for
``expects 1 argument''.

\hypertarget{two-argument-functions}{%
\subsubsection{Two-Argument Functions}\label{two-argument-functions}}

Let's look at a slightly more complicated example -- a two-argument function.
Let's define the boolean ``and'':

\begin{Shaded}
\begin{Highlighting}[]
\OperatorTok{$}\NormalTok{(singletons [d|}
\NormalTok{  and :: Bool {-}> (Bool {-}> Bool)}
\NormalTok{  and False \_ = False}
\NormalTok{  and True  x = x}
\NormalTok{  ])}
\end{Highlighting}
\end{Shaded}

this will generate:

\begin{Shaded}
\begin{Highlighting}[]
\KeywordTok{type} \KeywordTok{family} \DataTypeTok{And}\NormalTok{ (}\OtherTok{x ::} \DataTypeTok{Bool}\NormalTok{) (}\OtherTok{y ::} \DataTypeTok{Bool}\NormalTok{)}\OtherTok{ ::} \DataTypeTok{Bool} \KeywordTok{where}
    \DataTypeTok{And} \DataTypeTok{\textquotesingle{}False}\NormalTok{ x }\OtherTok{=} \DataTypeTok{\textquotesingle{}False}
    \DataTypeTok{And} \DataTypeTok{\textquotesingle{}True}\NormalTok{  x }\OtherTok{=}\NormalTok{ x}

\OtherTok{sAnd ::} \DataTypeTok{Sing}\NormalTok{ x }\OtherTok{{-}>} \DataTypeTok{Sing}\NormalTok{ y }\OtherTok{{-}>} \DataTypeTok{Sing}\NormalTok{ (}\DataTypeTok{And}\NormalTok{ x y)}
\NormalTok{sAnd }\DataTypeTok{SFalse}\NormalTok{ x }\OtherTok{=} \DataTypeTok{SFalse}
\NormalTok{sAnd }\DataTypeTok{STrue}\NormalTok{  x }\OtherTok{=}\NormalTok{ x}
\end{Highlighting}
\end{Shaded}

And the defunctionalization symbols:

\begin{Shaded}
\begin{Highlighting}[]
\KeywordTok{data} \DataTypeTok{AndSym0}\OtherTok{ ::} \DataTypeTok{Bool} \OperatorTok{\textasciitilde{}>}\NormalTok{ (}\DataTypeTok{Bool} \OperatorTok{\textasciitilde{}>} \DataTypeTok{Bool}\NormalTok{)}
\KeywordTok{type} \KeywordTok{instance} \DataTypeTok{Apply} \DataTypeTok{AndSym0}\NormalTok{ x }\OtherTok{=} \DataTypeTok{AndSym1}\NormalTok{ x}

\KeywordTok{data} \DataTypeTok{AndSym1}\NormalTok{ (}\OtherTok{x ::} \DataTypeTok{Bool}\NormalTok{)}\OtherTok{ ::}\NormalTok{ (}\DataTypeTok{Bool} \OperatorTok{\textasciitilde{}>} \DataTypeTok{Bool}\NormalTok{)}
\CommentTok{{-}{-} or}
\KeywordTok{data} \DataTypeTok{AndSym1}\OtherTok{ ::} \DataTypeTok{Bool} \OtherTok{{-}>}\NormalTok{ (}\DataTypeTok{Bool} \OperatorTok{\textasciitilde{}>} \DataTypeTok{Bool}\NormalTok{)}
\KeywordTok{type} \KeywordTok{instance} \DataTypeTok{Apply}\NormalTok{ (}\DataTypeTok{AndSym1}\NormalTok{ x) y }\OtherTok{=} \DataTypeTok{And}\NormalTok{ x y}

\KeywordTok{type} \DataTypeTok{AndSym2}\NormalTok{ x y }\OtherTok{=} \DataTypeTok{And}\NormalTok{ x y}
\end{Highlighting}
\end{Shaded}

\texttt{AndSym0} is a defunctionalization symbol representing a ``fully
unapplied'' (``completely unsaturated'') version of \texttt{And}.
\texttt{AndSym1\ x} is a defunctionalization symbol representing a ``partially
applied'' version of \texttt{And} --- partially applied to \texttt{x} (its kind
is
\texttt{AndSym1\ ::\ Bool\ -\textgreater{}\ (Bool\ \textasciitilde{}\textgreater{}\ Bool)}).

The application of \texttt{AndSym0} to \texttt{x} gives you \texttt{AndSym1\ x}:

\begin{Shaded}
\begin{Highlighting}[]
\NormalTok{ghci}\OperatorTok{>} \OperatorTok{:}\NormalTok{kind}\OperatorTok{!} \DataTypeTok{AndSym0} \OperatorTok{@@} \DataTypeTok{\textquotesingle{}False}
\DataTypeTok{AndSym1} \DataTypeTok{\textquotesingle{}False}
\end{Highlighting}
\end{Shaded}

Remember its kind
\texttt{AndSym0\ ::\ Bool\ \textasciitilde{}\textgreater{}\ (Bool\ \textasciitilde{}\textgreater{}\ Bool)}
(or just
\texttt{AndSym0\ ::\ Bool\ \textasciitilde{}\textgreater{}\ Bool\ \textasciitilde{}\textgreater{}\ Bool}):
it takes a \texttt{Bool}, and returns a
\texttt{Bool\ \textasciitilde{}\textgreater{}\ Bool} defunctionalization symbol.

The application of \texttt{AndSym1\ x} to \texttt{y} gives you
\texttt{And\ x\ y}:

\begin{Shaded}
\begin{Highlighting}[]
\NormalTok{ghci}\OperatorTok{>} \OperatorTok{:}\NormalTok{kind}\OperatorTok{!} \DataTypeTok{AndSym1} \DataTypeTok{\textquotesingle{}False} \OperatorTok{@@} \DataTypeTok{\textquotesingle{}True}
\DataTypeTok{\textquotesingle{}False}              \CommentTok{{-}{-} or FalseSym0, which is a synonym for \textquotesingle{}False}
\NormalTok{ghci}\OperatorTok{>} \OperatorTok{:}\NormalTok{kind}\OperatorTok{!} \DataTypeTok{AndSym1} \DataTypeTok{\textquotesingle{}True}  \OperatorTok{@@} \DataTypeTok{\textquotesingle{}True}
\DataTypeTok{\textquotesingle{}True}
\end{Highlighting}
\end{Shaded}

A note to remember: \texttt{AndSym1\ \textquotesingle{}True} is the
defunctionalization symbol, and \emph{not} \texttt{AndSym1} itself.
\texttt{AndSym1} has kind
\texttt{Bool\ -\textgreater{}\ (Bool\ \textasciitilde{}\textgreater{}\ Bool)},
but \texttt{AndSym1\ \textquotesingle{}True} has kind
\texttt{Bool\ \textasciitilde{}\textgreater{}\ Bool} --- the kind of a
defunctionalization symbol. \texttt{AndSym1} is a sort of ``defunctionalization
symbol constructor''.

Also note here that we encounter the fact that singletons also provides
``defunctionalization symbols'' for ``nullary'' type functions like
\texttt{False} and \texttt{True}, where:

\begin{Shaded}
\begin{Highlighting}[]
\KeywordTok{type} \DataTypeTok{FalseSym0} \OtherTok{=} \DataTypeTok{\textquotesingle{}False}
\KeywordTok{type} \DataTypeTok{TrueSym0}  \OtherTok{=} \DataTypeTok{\textquotesingle{}True}
\end{Highlighting}
\end{Shaded}

Just like how it defines \texttt{AndSym0} for consistency, as well.

\hypertarget{symbols-for-type-constructors}{%
\subsubsection{Symbols for type
constructors}\label{symbols-for-type-constructors}}

One extra interesting defunctionalization symbol we can write: we turn lift any
type constructor into a ``free'' defunctionalization symbol:

\begin{Shaded}
\begin{Highlighting}[]
\CommentTok{{-}{-} source: https://github.com/mstksg/inCode/tree/master/code{-}samples/singletons/Defunctionalization.hs\#L32{-}L38}

\KeywordTok{data} \DataTypeTok{TyCon1}
\OtherTok{        ::}\NormalTok{ (j }\OtherTok{{-}>}\NormalTok{ k)     }\CommentTok{{-}{-} \^{} take a type constructor}
        \OtherTok{{-}>}\NormalTok{ (j }\OperatorTok{\textasciitilde{}>}\NormalTok{ k)     }\CommentTok{{-}{-} \^{} return a defunctionalization symbol}
\CommentTok{{-}{-} alternatively}
\CommentTok{{-}{-} data TyCon1 (t :: j {-}> k) :: j \textasciitilde{}> k}

\KeywordTok{type} \KeywordTok{instance} \DataTypeTok{Apply}\NormalTok{ (}\DataTypeTok{TyCon1}\NormalTok{ t) a }\OtherTok{=}\NormalTok{ t a}
\end{Highlighting}
\end{Shaded}

Basically the \texttt{Apply} instance just applies the type constructor
\texttt{t} to its input \texttt{a}.

\begin{Shaded}
\begin{Highlighting}[]
\NormalTok{ghci}\OperatorTok{>} \OperatorTok{:}\NormalTok{kind}\OperatorTok{!} \DataTypeTok{TyCon1} \DataTypeTok{Maybe} \OperatorTok{@@} \DataTypeTok{Int}
\DataTypeTok{Maybe} \DataTypeTok{Int}
\NormalTok{ghci}\OperatorTok{>} \OperatorTok{:}\NormalTok{kind}\OperatorTok{!} \DataTypeTok{TyCon1} \DataTypeTok{\textquotesingle{}Right} \OperatorTok{@@} \DataTypeTok{\textquotesingle{}False}
\DataTypeTok{\textquotesingle{}Right} \DataTypeTok{\textquotesingle{}False}
\end{Highlighting}
\end{Shaded}

We can use this to give a normal \texttt{j\ -\textgreater{}\ k} type constructor
to a function that expects a \texttt{j\ \textasciitilde{}\textgreater{}\ k}
defunctionalization symbol.

\hypertarget{bring-me-a-higher-order}{%
\section{Bring Me a Higher Order}\label{bring-me-a-higher-order}}

Okay, so now we have these tokens that represent ``unapplied'' versions of
functions. So what?

Well, remember the problem with our implementation of \texttt{Foldr}? We
couldn't pass in a type family, since type families must be passed fully
applied. So, instead of having \texttt{Foldr} expect a type family\ldots we can
make it expect a \emph{defunctionalization symbol} instead. Remember,
defunctionalization symbols represent the ``unapplied'' versions of type
families, so they are exactly the tools we need!

\begin{Shaded}
\begin{Highlighting}[]
\CommentTok{{-}{-} source: https://github.com/mstksg/inCode/tree/master/code{-}samples/singletons/Defunctionalization.hs\#L40{-}L42}

\KeywordTok{type} \KeywordTok{family} \DataTypeTok{Foldr}\NormalTok{ (}\OtherTok{f ::}\NormalTok{ j }\OperatorTok{\textasciitilde{}>}\NormalTok{ k }\OperatorTok{\textasciitilde{}>}\NormalTok{ k) (}\OtherTok{z ::}\NormalTok{ k) (}\OtherTok{xs ::}\NormalTok{ [j])}\OtherTok{ ::}\NormalTok{ k }\KeywordTok{where}
    \DataTypeTok{Foldr}\NormalTok{ f z \textquotesingle{}[]       }\OtherTok{=}\NormalTok{ z}
    \DataTypeTok{Foldr}\NormalTok{ f z (x \textquotesingle{}}\OperatorTok{:}\NormalTok{ xs) }\OtherTok{=}\NormalTok{ (f }\OperatorTok{@@}\NormalTok{ x) }\OperatorTok{@@} \DataTypeTok{Foldr}\NormalTok{ f z xs}
\end{Highlighting}
\end{Shaded}

The difference is that instead of taking a type family or type constructor
\texttt{f\ ::\ j\ -\textgreater{}\ k\ -\textgreater{}\ k}, we have it take the
\emph{defunctionalization symbol}
\texttt{f\ ::\ j\ \textasciitilde{}\textgreater{}\ (k\ \textasciitilde{}\textgreater{}\ k)}.

Instead of taking a type family or type constructor, we take that dummy type
constructor.

Now we just need to have our defunctionalization symbols for
\texttt{MergeStateList}:

\begin{Shaded}
\begin{Highlighting}[]
\CommentTok{{-}{-} source: https://github.com/mstksg/inCode/tree/master/code{-}samples/singletons/Defunctionalization.hs\#L54{-}L60}

\KeywordTok{data} \DataTypeTok{MergeStateSym0}\OtherTok{ ::} \DataTypeTok{DoorState} \OperatorTok{\textasciitilde{}>} \DataTypeTok{DoorState} \OperatorTok{\textasciitilde{}>} \DataTypeTok{DoorState}
\KeywordTok{type} \KeywordTok{instance} \DataTypeTok{Apply} \DataTypeTok{MergeStateSym0}\NormalTok{ s }\OtherTok{=} \DataTypeTok{MergeStateSym1}\NormalTok{ s}

\KeywordTok{data} \DataTypeTok{MergeStateSym1}\OtherTok{ ::} \DataTypeTok{DoorState} \OtherTok{{-}>} \DataTypeTok{DoorState} \OperatorTok{\textasciitilde{}>} \DataTypeTok{DoorState}
\KeywordTok{type} \KeywordTok{instance} \DataTypeTok{Apply}\NormalTok{ (}\DataTypeTok{MergeStateSym1}\NormalTok{ s) t }\OtherTok{=} \DataTypeTok{MergeState}\NormalTok{ s t}

\KeywordTok{type} \DataTypeTok{MergeStateSym2}\NormalTok{ s t }\OtherTok{=} \DataTypeTok{MergeState}\NormalTok{ s t}
\end{Highlighting}
\end{Shaded}

And now we can write \texttt{MergeStateList}:

\begin{Shaded}
\begin{Highlighting}[]
\CommentTok{{-}{-} source: https://github.com/mstksg/inCode/tree/master/code{-}samples/singletons/Defunctionalization.hs\#L62{-}L62}

\KeywordTok{type} \DataTypeTok{MergeStateList}\NormalTok{ ss }\OtherTok{=} \DataTypeTok{Foldr} \DataTypeTok{MergeStateSym0} \DataTypeTok{\textquotesingle{}Opened}\NormalTok{ ss}
\end{Highlighting}
\end{Shaded}

(If you ``see'' \texttt{MergeStateSym0}, you should \emph{read} it was
\texttt{MergeState}, but partially applied)

This compiles!

\begin{Shaded}
\begin{Highlighting}[]
\NormalTok{ghci}\OperatorTok{>} \OperatorTok{:}\NormalTok{kind}\OperatorTok{!} \DataTypeTok{MergeStateList}\NormalTok{ \textquotesingle{}[ }\DataTypeTok{\textquotesingle{}Closed}\NormalTok{, }\DataTypeTok{\textquotesingle{}Opened}\NormalTok{, }\DataTypeTok{\textquotesingle{}Locked}\NormalTok{ ]}
\DataTypeTok{\textquotesingle{}Locked}
\NormalTok{ghci}\OperatorTok{>} \OperatorTok{:}\NormalTok{kind}\OperatorTok{!} \DataTypeTok{MergeStateList}\NormalTok{ \textquotesingle{}[ }\DataTypeTok{\textquotesingle{}Closed}\NormalTok{, }\DataTypeTok{\textquotesingle{}Opened}\NormalTok{ ]}
\DataTypeTok{\textquotesingle{}Closed}
\end{Highlighting}
\end{Shaded}

\begin{Shaded}
\begin{Highlighting}[]
\CommentTok{{-}{-} source: https://github.com/mstksg/inCode/tree/master/code{-}samples/singletons/Defunctionalization.hs\#L77{-}L79}

\OtherTok{collapseHallway ::} \DataTypeTok{Hallway}\NormalTok{ ss }\OtherTok{{-}>} \DataTypeTok{Door}\NormalTok{ (}\DataTypeTok{MergeStateList}\NormalTok{ ss)}
\NormalTok{collapseHallway }\DataTypeTok{HEnd}       \OtherTok{=} \DataTypeTok{UnsafeMkDoor} \StringTok{"End of Hallway"}
\NormalTok{collapseHallway (d }\OperatorTok{:<\#}\NormalTok{ ds) }\OtherTok{=}\NormalTok{ d }\OtherTok{\textasciigrave{}mergeDoor\textasciigrave{}}\NormalTok{ collapseHallway ds}
\end{Highlighting}
\end{Shaded}

(Note: Unfortunately, we do have to use our our \emph{own} \texttt{Foldr} here,
that we just defined, instead of using the one that comes with
\emph{singletons}, because of some
\href{https://github.com/goldfirere/singletons/issues/339}{outstanding issues}
with how the singletons TH processes alternative implementations of
\texttt{foldr} from Prelude. In general, the issue is that we should only expect
type families to work with singletons if the definition of the type family
perfectly matches the structure of how we implement our value-level functions
like \texttt{collapseHallway})

\hypertarget{singletons-to-make-things-nicer}{%
\subsection{Singletons to make things
nicer}\label{singletons-to-make-things-nicer}}

Admittedly this is all a huge mess of boilerplate. The code we had to write more
than tripled, and we also have an unsightly number of defunctionalization
symbols and \texttt{Apply} instance boilerplate for every function.

Luckily, the \emph{singletons} library is here to help. You can just write:

\begin{Shaded}
\begin{Highlighting}[]
\OperatorTok{$}\NormalTok{(singletons [d|}
\NormalTok{  data DoorState = Opened | Closed | Locked}
\NormalTok{    deriving (Show, Eq, Ord)}

\NormalTok{  mergeState :: DoorState {-}> DoorState {-}> DoorState}
\NormalTok{  mergeState = max}

\NormalTok{  foldr :: (a {-}> b {-}> b) {-}> b {-}> [a] {-}> b}
\NormalTok{  foldr \_ z []     = z}
\NormalTok{  foldr f z (x:xs) = f x (foldr f z xs)}

\NormalTok{  mergeStateList :: [DoorState] {-}> DoorState}
\NormalTok{  mergeStateList = foldr mergeState Opened}
\NormalTok{  |])}
\end{Highlighting}
\end{Shaded}

And all of these defunctionalization symbols are generated for you;
\emph{singletons} is also able to recognize that \texttt{foldr} is a
higher-order function and translate its lifted version to take a
defunctionalization symbol
\texttt{a\ \textasciitilde{}\textgreater{}\ b\ \textasciitilde{}\textgreater{}\ b}.

That the template haskell also generates \texttt{SingI} instances for all of
your defunctionalization symbols, too (more on that in a bit).

It's okay to stay ``in the world of singletons'' for the most part, and let
singletons handle the composition of functions for you. However, it's still
important to know what the \emph{singletons} library generates, because
sometimes it's still useful to manually create defunctionalization symbols and
work with them.

The naming convention for non-symbolic names (non-operators) like
\texttt{myFunction} are just to call them \texttt{MyFunctionSym0} for the
completely unapplied defunctionalization symbol, \texttt{MyFunctionSym1} for the
type constructor that expects one argument before returning a
defunctionalization symbol, \texttt{MyFunctionSym2} for the type constructor
that expects two arguments before returning a defunctionalization symbol, etc.

For operator names like \texttt{++}, the naming convention is to have
\texttt{++@\#@\$} be the completely unapplied defunctionalization symbol,
\texttt{++@\#@\$\$} be the type constructor that expects one argument before
returning a defunctionalization symbol, \texttt{++@\#@\$\$\$} be the type
constructor that takes two arguments before returning a defunctionalization
symbol, etc.

Another helpful thing that \emph{singletons} does is that it also generates
defunctionalization symbols for type families and type synonyms you define in
the Template Haskell, as well --- so if you write

\begin{Shaded}
\begin{Highlighting}[]
\OperatorTok{$}\NormalTok{(singletons [d|}
\NormalTok{  type MyTypeFamily (b :: Bool) :: Type where}
\NormalTok{    MyTypeFamily \textquotesingle{}False = Int}
\NormalTok{    MyTypeFamily \textquotesingle{}True  = String}
\NormalTok{  |])}
\end{Highlighting}
\end{Shaded}

and

\begin{Shaded}
\begin{Highlighting}[]
\OperatorTok{$}\NormalTok{(singletons [d|}
\NormalTok{  type MyTypeSynonym a = (a, [a])}
\NormalTok{  |])}
\end{Highlighting}
\end{Shaded}

\emph{singletons} will generate:

\begin{Shaded}
\begin{Highlighting}[]
\KeywordTok{data} \DataTypeTok{MyTypeFamilySym0}\OtherTok{ ::} \DataTypeTok{Bool} \OperatorTok{\textasciitilde{}>} \DataTypeTok{Type}
\KeywordTok{type} \KeywordTok{instance} \DataTypeTok{Apply} \DataTypeTok{MyTypeFamilySym0}\NormalTok{ b }\OtherTok{=} \DataTypeTok{MyTypeFamily}\NormalTok{ b}

\KeywordTok{type} \DataTypeTok{MyTypeFamilySym1}\NormalTok{ b }\OtherTok{=} \DataTypeTok{MyTypeFamily}\NormalTok{ b}
\end{Highlighting}
\end{Shaded}

and

\begin{Shaded}
\begin{Highlighting}[]
\KeywordTok{data} \DataTypeTok{MyTypeSynonymSym0}\OtherTok{ ::} \DataTypeTok{Type} \OperatorTok{\textasciitilde{}>} \DataTypeTok{Type}
\KeywordTok{type} \KeywordTok{instance} \DataTypeTok{Apply} \DataTypeTok{MyTypeSynonym}\NormalTok{ b }\OtherTok{=} \DataTypeTok{MyTypeSynonym}\NormalTok{ a}

\KeywordTok{type} \DataTypeTok{MyTypeSynonymSym1}\NormalTok{ a }\OtherTok{=} \DataTypeTok{MyTypeSynonym}\NormalTok{ a}
\end{Highlighting}
\end{Shaded}

\hypertarget{bringing-it-all-together}{%
\subsubsection{Bringing it All Together}\label{bringing-it-all-together}}

Just to show off the library, remember that \emph{singletons} also promotes
typeclasses?

Because \texttt{DoorState} is a monoid with respect to merging, we can actually
write and promote a \texttt{Monoid} instance: (requires \emph{singletons-2.5} or
higher)

\begin{Shaded}
\begin{Highlighting}[]
\OperatorTok{$}\NormalTok{(singletons [d|}
\NormalTok{  instance Semigroup DoorState where}
\NormalTok{      (<>) = mergeState}
\NormalTok{  instance Monoid DoorState where}
\NormalTok{      mempty  = Opened}
\NormalTok{      mappend = (<>)}
\NormalTok{  |])}
\end{Highlighting}
\end{Shaded}

We can promote \texttt{fold}:

\begin{Shaded}
\begin{Highlighting}[]
\OperatorTok{$}\NormalTok{(singletons [d|}
\NormalTok{  fold :: Monoid b => [b] {-}> b}
\NormalTok{  fold []     = mempty}
\NormalTok{  fold (x:xs) = x <> fold xs}
\NormalTok{  |])}
\end{Highlighting}
\end{Shaded}

And we can write \texttt{collapseHallway} in terms of those instead :)

\begin{Shaded}
\begin{Highlighting}[]
\CommentTok{{-}{-} source: https://github.com/mstksg/inCode/tree/master/code{-}samples/singletons/Door4Final.hs\#L109{-}L118}

\NormalTok{collapseHallway\textquotesingle{}}
\OtherTok{    ::} \DataTypeTok{Hallway}\NormalTok{ ss}
    \OtherTok{{-}>} \DataTypeTok{Door}\NormalTok{ (}\DataTypeTok{Fold}\NormalTok{ ss)}
\NormalTok{collapseHallway\textquotesingle{} }\DataTypeTok{HEnd}       \OtherTok{=} \DataTypeTok{UnsafeMkDoor} \StringTok{"End of Hallway"}
\NormalTok{collapseHallway\textquotesingle{} (d }\OperatorTok{:<\#}\NormalTok{ ds) }\OtherTok{=}\NormalTok{ d }\OtherTok{\textasciigrave{}mergeDoor\textasciigrave{}}\NormalTok{ collapseHallway\textquotesingle{} ds}

\OtherTok{collapseSomeHallway\textquotesingle{} ::} \DataTypeTok{SomeHallway} \OtherTok{{-}>} \DataTypeTok{SomeDoor}
\NormalTok{collapseSomeHallway\textquotesingle{} (ss }\OperatorTok{:\&:}\NormalTok{ d) }\OtherTok{=}
\NormalTok{        sFold ss}
    \OperatorTok{:\&:}\NormalTok{ collapseHallway\textquotesingle{} d}
\end{Highlighting}
\end{Shaded}

(Note again unfortunately that we have to define our own \texttt{fold} instead
of using the one from \emph{singletons} and the \texttt{SFoldable} typeclass,
because of \href{https://github.com/goldfirere/singletons/issues/339}{issue
\#339})

\hypertarget{thoughts-on-symbols}{%
\section{Thoughts on Symbols}\label{thoughts-on-symbols}}

Defunctionalization symbols may feel like a bit of a mess, and the naming
convention is arguably less than aesthetically satisfying. But, as you work with
them more and more, you start to appreciate them on a deeper level.

At the end of the day, you can compare defunctionalization as turning
``functions'' into just constructors you can \emph{match} on, just like any
other data or type constructor. That's because they \emph{are} just type
constructors!

In a sense, defining defunctionalization symbols is a lot like working with
\emph{pattern synonyms} of your functions, instead of directly passing the
functions themselves. At the type family and type class level, you can ``pattern
match'' on these functions.

For a comparison at the value level -- you can't pattern match on \texttt{(+)},
\texttt{(-)}, \texttt{(*)}, and \texttt{(/)}:

\begin{Shaded}
\begin{Highlighting}[]
\CommentTok{{-}{-} Doesn\textquotesingle{}t work like you think it does}
\OtherTok{invertOperation ::}\NormalTok{ (}\DataTypeTok{Double} \OtherTok{{-}>} \DataTypeTok{Dobule} \OtherTok{{-}>} \DataTypeTok{Double}\NormalTok{) }\OtherTok{{-}>}\NormalTok{ (}\DataTypeTok{Double} \OtherTok{{-}>} \DataTypeTok{Double} \OtherTok{{-}>} \DataTypeTok{Double}\NormalTok{)}
\NormalTok{invertOperation (}\OperatorTok{+}\NormalTok{) }\OtherTok{=}\NormalTok{ (}\OperatorTok{{-}}\NormalTok{)}
\NormalTok{invertOperation (}\OperatorTok{{-}}\NormalTok{) }\OtherTok{=}\NormalTok{ (}\OperatorTok{+}\NormalTok{)}
\NormalTok{invertOperation (}\OperatorTok{*}\NormalTok{) }\OtherTok{=}\NormalTok{ (}\OperatorTok{/}\NormalTok{)}
\NormalTok{invertOperation (}\OperatorTok{/}\NormalTok{) }\OtherTok{=}\NormalTok{ (}\OperatorTok{*}\NormalTok{)}
\end{Highlighting}
\end{Shaded}

You can't quite match on the equality of functions to some list of patterns.
But, what you \emph{can} do is create constructors representing your functions,
and match on those.

This essentially fixes the ``type lambda problem'' of type inference and
typeclass resolution. You can't match on arbitrary lambdas, but you \emph{can}
match on dummy constructors representing type functions.

And a bit of the magic here, also, is the fact that you don't always need to
make our own defunctionalization symbols from scratch --- you can create them
based on other ones in a compositional way. This is the basis of libraries like
\emph{\href{http://hackage.haskell.org/package/decidable}{decidable}}.

For example, suppose we wanted to build defunctionalization symbols for
\texttt{MergeStateList}. We can actually build them directly from
defunctionalization symbols for \texttt{Foldr}.

Check out the defunctionalization symbols for \texttt{Foldr}:

\begin{Shaded}
\begin{Highlighting}[]
\CommentTok{{-}{-} source: https://github.com/mstksg/inCode/tree/master/code{-}samples/singletons/Defunctionalization.hs\#L81{-}L90}

\KeywordTok{data} \DataTypeTok{FoldrSym0}\OtherTok{ ::}\NormalTok{ (j }\OperatorTok{\textasciitilde{}>}\NormalTok{ k }\OperatorTok{\textasciitilde{}>}\NormalTok{ k) }\OperatorTok{\textasciitilde{}>}\NormalTok{ k }\OperatorTok{\textasciitilde{}>}\NormalTok{ [j] }\OperatorTok{\textasciitilde{}>}\NormalTok{ k}
\KeywordTok{type} \KeywordTok{instance} \DataTypeTok{Apply} \DataTypeTok{FoldrSym0}\NormalTok{ f }\OtherTok{=} \DataTypeTok{FoldrSym1}\NormalTok{ f}

\KeywordTok{data} \DataTypeTok{FoldrSym1}\OtherTok{ ::}\NormalTok{ (j }\OperatorTok{\textasciitilde{}>}\NormalTok{ k }\OperatorTok{\textasciitilde{}>}\NormalTok{ k) }\OtherTok{{-}>}\NormalTok{ k }\OperatorTok{\textasciitilde{}>}\NormalTok{ [j] }\OperatorTok{\textasciitilde{}>}\NormalTok{ k}
\KeywordTok{type} \KeywordTok{instance} \DataTypeTok{Apply}\NormalTok{ (}\DataTypeTok{FoldrSym1}\NormalTok{ f) z }\OtherTok{=} \DataTypeTok{FoldrSym2}\NormalTok{ f z}

\KeywordTok{data} \DataTypeTok{FoldrSym2}\OtherTok{ ::}\NormalTok{ (j }\OperatorTok{\textasciitilde{}>}\NormalTok{ k }\OperatorTok{\textasciitilde{}>}\NormalTok{ k) }\OtherTok{{-}>}\NormalTok{ k }\OtherTok{{-}>}\NormalTok{ [j] }\OperatorTok{\textasciitilde{}>}\NormalTok{ k}
\KeywordTok{type} \KeywordTok{instance} \DataTypeTok{Apply}\NormalTok{ (}\DataTypeTok{FoldrSym2}\NormalTok{ f z) xs }\OtherTok{=} \DataTypeTok{Foldr}\NormalTok{ f z xs}

\KeywordTok{type} \DataTypeTok{FoldrSym3}\NormalTok{ f z xs }\OtherTok{=} \DataTypeTok{Foldr}\NormalTok{ f z xs}
\end{Highlighting}
\end{Shaded}

We can actually use these to \emph{define} our \texttt{MergeStateList}
defunctionalization symbols, since \emph{defunctionalization symbols are
first-class}:

\begin{Shaded}
\begin{Highlighting}[]
\CommentTok{{-}{-} source: https://github.com/mstksg/inCode/tree/master/code{-}samples/singletons/Defunctionalization.hs\#L92{-}L92}

\KeywordTok{type} \DataTypeTok{MergeStateListSym0} \OtherTok{=} \DataTypeTok{FoldrSym2} \DataTypeTok{MergeStateSym0} \DataTypeTok{\textquotesingle{}Opened}
\end{Highlighting}
\end{Shaded}

And you can just write \texttt{collapseHallway} as:

\begin{Shaded}
\begin{Highlighting}[]
\OtherTok{collapseHallway ::} \DataTypeTok{Hallway}\NormalTok{ ss }\OtherTok{{-}>} \DataTypeTok{Door}\NormalTok{ (}\DataTypeTok{MergeStateListSym0} \OperatorTok{@@}\NormalTok{ ss)}
\CommentTok{{-}{-} or}
\OtherTok{collapseHallway ::} \DataTypeTok{Hallway}\NormalTok{ ss }\OtherTok{{-}>} \DataTypeTok{Door}\NormalTok{ (}\DataTypeTok{FoldrSym2} \DataTypeTok{MergeStateSym0} \DataTypeTok{\textquotesingle{}Opened} \OperatorTok{@@}\NormalTok{ ss)}
\end{Highlighting}
\end{Shaded}

You never have to actually define \texttt{MergeStateList} as a function or type
family!

The whole time, we're just building defunctionalization symbols in terms of
other defunctionalization symbols. And, at the end, when we finally want to
interpret the complex function we construct, we use \texttt{Apply}, or
\texttt{@@}.

You can think of \texttt{FoldrSym1} and \texttt{FoldrSym2} as
\emph{defunctionalization symbol constructors} -- they're combinators that take
in defunctionalization symbols (like \texttt{MergeStateSym0}) and \emph{return
new ones}.

\hypertarget{sigma}{%
\subsection{Sigma}\label{sigma}}

Let's look at a nice tool that is made possible using defunctionalization
symbols: \emph{dependent pairs}. I talk a bit about dependent pairs (or
dependent sums) in
\href{https://blog.jle.im/entry/introduction-to-singletons-2.html}{part 2} of
this series, and also in my
\href{https://blog.jle.im/entry/practical-dependent-types-in-haskell-2.html}{dependent
types in Haskell} series.

Essentially, a dependent pair is a tuple where the \emph{type} of the second
field depends on the \emph{value} of the first one. This is basically what
\texttt{SomeDoor} was:

\begin{Shaded}
\begin{Highlighting}[]
\KeywordTok{data} \DataTypeTok{SomeDoor}\OtherTok{ ::} \DataTypeTok{Type} \KeywordTok{where}
    \DataTypeTok{MkSomeDoor}\OtherTok{ ::} \DataTypeTok{Sing}\NormalTok{ x }\OtherTok{{-}>} \DataTypeTok{Door}\NormalTok{ x }\OtherTok{{-}>} \DataTypeTok{SomeDoor}
\end{Highlighting}
\end{Shaded}

The \emph{type} of the \texttt{Door\ x} depends on the \emph{value} of the
\texttt{Sing\ x}, which you can read as essentially storing the \texttt{x}.

We made \texttt{SomeDoor} pretty ad-hoc. But what if we wanted to make some
other predicate? Well, we can make a \emph{generic} dependent pair by
\emph{parameterizing it on the dependence} between the first and second field.
Singletons provides the \texttt{Sigma} type, in the
\emph{\href{http://hackage.haskell.org/package/singletons-2.5/docs/Data-Singletons-Sigma.html}{Data.Singletons.Sigma}}
module:

\begin{Shaded}
\begin{Highlighting}[]
\KeywordTok{data} \DataTypeTok{Sigma}\OtherTok{ k ::}\NormalTok{ (k }\OperatorTok{\textasciitilde{}>} \DataTypeTok{Type}\NormalTok{) }\OtherTok{{-}>} \DataTypeTok{Type} \KeywordTok{where}
\OtherTok{    (:\&:) ::} \DataTypeTok{Sing}\NormalTok{ x }\OtherTok{{-}>}\NormalTok{ (f }\OperatorTok{@@}\NormalTok{ x) }\OtherTok{{-}>} \DataTypeTok{Sigma}\NormalTok{ k f}

\CommentTok{{-}{-} also available through fancy type synonym}
\KeywordTok{type}\NormalTok{ Σ k }\OtherTok{=} \DataTypeTok{Sigma}\NormalTok{ k}
\end{Highlighting}
\end{Shaded}

If you squint carefully, you can see that \texttt{Sigma\ k} is just
\texttt{SomeDoor}, but \emph{parameterized over \texttt{Door}}. Instead of
always holding \texttt{Door\ x}, we can have it parameterized on an arbitrary
function \texttt{f} and have it hold an \texttt{f\ @@\ x}.

We can actually define \texttt{SomeDoor} in terms of \texttt{Sigma}:

\begin{Shaded}
\begin{Highlighting}[]
\CommentTok{{-}{-} source: https://github.com/mstksg/inCode/tree/master/code{-}samples/singletons/Door4Final.hs\#L54{-}L58}

\KeywordTok{type} \DataTypeTok{SomeDoor} \OtherTok{=} \DataTypeTok{Sigma} \DataTypeTok{DoorState}\NormalTok{ (}\DataTypeTok{TyCon1} \DataTypeTok{Door}\NormalTok{)}

\OtherTok{mkSomeDoor ::} \DataTypeTok{DoorState} \OtherTok{{-}>} \DataTypeTok{String} \OtherTok{{-}>} \DataTypeTok{SomeDoor}
\NormalTok{mkSomeDoor ds mat }\OtherTok{=}\NormalTok{ withSomeSing ds }\OperatorTok{$}\NormalTok{ \textbackslash{}dsSing }\OtherTok{{-}>}
\NormalTok{    dsSing }\OperatorTok{:\&:}\NormalTok{ mkDoor dsSing mat}
\end{Highlighting}
\end{Shaded}

(Remember \texttt{TyCon1} is the defunctionalization symbol constructor that
turns any normal type constructor \texttt{j\ -\textgreater{}\ k} into a
defunctionalization symbol \texttt{j\ \textasciitilde{}\textgreater{}\ k})

That's because a \texttt{Sigma\ DoorState\ (TyCon1\ Door)} contains a
\texttt{Sing\ (x\ ::\ DoorState)} and a \texttt{TyCon1\ Door\ @@\ x}, or a
\texttt{Door\ x}.

This is a simple relationship, but one can imagine a \texttt{Sigma}
parameterized on an even more complex type-level function. We'll explore more of
these in the exercises.

For some context, \texttt{Sigma} is an interesting data type (the ``dependent
sum'') that is ubiquitous in dependently typed programming.

\hypertarget{singletons-of-defunctionalization-symbols}{%
\subsection{Singletons of Defunctionalization
Symbols}\label{singletons-of-defunctionalization-symbols}}

One last thing to tie it all together -- let's write \texttt{collapseHallway} in
a way that we don't know the types of the doors.

Luckily, we now have a \texttt{SomeHallway} type for free:

\begin{Shaded}
\begin{Highlighting}[]
\CommentTok{{-}{-} source: https://github.com/mstksg/inCode/tree/master/code{-}samples/singletons/Door4Final.hs\#L84{-}L84}

\KeywordTok{type} \DataTypeTok{SomeHallway} \OtherTok{=} \DataTypeTok{Sigma}\NormalTok{ [}\DataTypeTok{DoorState}\NormalTok{] (}\DataTypeTok{TyCon1} \DataTypeTok{Hallway}\NormalTok{)}
\end{Highlighting}
\end{Shaded}

The easy way would be to just use \texttt{sMergeStateList} that we defined:

\begin{Shaded}
\begin{Highlighting}[]
\CommentTok{{-}{-} source: https://github.com/mstksg/inCode/tree/master/code{-}samples/singletons/Door4Final.hs\#L86{-}L88}

\OtherTok{collapseSomeHallway ::} \DataTypeTok{SomeHallway} \OtherTok{{-}>} \DataTypeTok{SomeDoor}
\NormalTok{collapseSomeHallway (ss }\OperatorTok{:\&:}\NormalTok{ d) }\OtherTok{=}\NormalTok{ sMergeStateList ss}
                             \OperatorTok{:\&:}\NormalTok{ collapseHallway d}
\end{Highlighting}
\end{Shaded}

But what if we didn't write \texttt{sMergeStateList}, and we constructed our
defunctionalization symbols from scratch?

\begin{Shaded}
\begin{Highlighting}[]
\CommentTok{{-}{-} source: https://github.com/mstksg/inCode/tree/master/code{-}samples/singletons/Door4Final.hs\#L122{-}L126}

\NormalTok{collapseHallway\textquotesingle{}\textquotesingle{}}
\OtherTok{    ::} \DataTypeTok{Hallway}\NormalTok{ ss}
    \OtherTok{{-}>} \DataTypeTok{Door}\NormalTok{ (}\DataTypeTok{FoldrSym2} \DataTypeTok{MergeStateSym0} \DataTypeTok{\textquotesingle{}Opened} \OperatorTok{@@}\NormalTok{ ss)}
\NormalTok{collapseHallway\textquotesingle{}\textquotesingle{} }\DataTypeTok{HEnd}       \OtherTok{=} \DataTypeTok{UnsafeMkDoor} \StringTok{"End of Hallway"}
\NormalTok{collapseHallway\textquotesingle{}\textquotesingle{} (d }\OperatorTok{:<\#}\NormalTok{ ds) }\OtherTok{=}\NormalTok{ d }\OtherTok{\textasciigrave{}mergeDoor\textasciigrave{}}\NormalTok{ collapseHallway\textquotesingle{}\textquotesingle{} ds}

\OtherTok{collapseSomeHallway\textquotesingle{}\textquotesingle{} ::} \DataTypeTok{SomeHallway} \OtherTok{{-}>} \DataTypeTok{SomeDoor}
\NormalTok{collapseSomeHallway\textquotesingle{}\textquotesingle{} (ss }\OperatorTok{:\&:}\NormalTok{ d) }\OtherTok{=} \OperatorTok{???}    \CommentTok{{-}{-} what goes here?}
                               \OperatorTok{:\&:}\NormalTok{ collapseHallway\textquotesingle{}\textquotesingle{} d}
\end{Highlighting}
\end{Shaded}

This will be our final defunctionalization lesson. How do we turn a singleton of
\texttt{ss} into a singleton of
\texttt{FoldrSym2\ MergeStateSym0\ \textquotesingle{}Opened\ @@\ s} ?

First -- we have \texttt{Foldr} at the value level, as \texttt{sFoldr}. We
glossed over this earlier, but \emph{singletons} generates the following
function for us:

\begin{Shaded}
\begin{Highlighting}[]
\KeywordTok{type} \KeywordTok{family} \DataTypeTok{Foldr}\NormalTok{ (}\OtherTok{f ::}\NormalTok{ j }\OperatorTok{\textasciitilde{}>}\NormalTok{ k }\OperatorTok{\textasciitilde{}>}\NormalTok{ k) (}\OtherTok{z ::}\NormalTok{ k) (}\OtherTok{xs ::}\NormalTok{ [j])}\OtherTok{ ::}\NormalTok{ k }\KeywordTok{where}
    \DataTypeTok{Foldr}\NormalTok{ f z \textquotesingle{}[]       }\OtherTok{=}\NormalTok{ z}
    \DataTypeTok{Foldr}\NormalTok{ f z (x \textquotesingle{}}\OperatorTok{:}\NormalTok{ xs) }\OtherTok{=}\NormalTok{ (f }\OperatorTok{@@}\NormalTok{ x) }\OperatorTok{@@} \DataTypeTok{Foldr}\NormalTok{ f z xs}

\NormalTok{sFoldr}
\OtherTok{    ::} \DataTypeTok{Sing}\NormalTok{ (}\OtherTok{f ::}\NormalTok{ j }\OperatorTok{\textasciitilde{}>}\NormalTok{ k }\OperatorTok{\textasciitilde{}>}\NormalTok{ k)}
    \OtherTok{{-}>} \DataTypeTok{Sing}\NormalTok{ (}\OtherTok{z ::}\NormalTok{ k)}
    \OtherTok{{-}>} \DataTypeTok{Sing}\NormalTok{ (}\OtherTok{xs ::}\NormalTok{ [j])}
    \OtherTok{{-}>} \DataTypeTok{Sing}\NormalTok{ (}\DataTypeTok{Foldr}\NormalTok{ f z}\OtherTok{ xs ::}\NormalTok{ k)}
\NormalTok{sFoldr f z }\DataTypeTok{SNil}           \OtherTok{=}\NormalTok{ z}
\NormalTok{sFoldr f z (x }\OtherTok{\textasciigrave{}SCons\textasciigrave{}}\NormalTok{ xs) }\OtherTok{=}\NormalTok{ (f }\OperatorTok{@@}\NormalTok{ x) }\OperatorTok{@@}\NormalTok{ sFoldr f z xs}
\end{Highlighting}
\end{Shaded}

Where
\texttt{(@@)\ ::\ Sing\ f\ -\textgreater{}\ Sing\ x\ -\textgreater{}\ Sing\ (f\ @@\ x)}
(or \texttt{applySing}) is the singleton/value-level counterpart of
\texttt{Apply} or \texttt{(@@)}.\footnote{\texttt{(@@)} (and as we see shortly,
  the \texttt{singFun} functions) are all implemented in terms of
  \texttt{SLambda}, the ``singleton'' for functions. Understanding the details
  of the implementation of \texttt{SLambda} aren't particularly important for
  the purposes of this introduction.}

So we can write:

\begin{Shaded}
\begin{Highlighting}[]
\OtherTok{collapseSomeHallway\textquotesingle{}\textquotesingle{} ::} \DataTypeTok{SomeHallway} \OtherTok{{-}>} \DataTypeTok{SomeDoor}
\NormalTok{collapseSomeHallway\textquotesingle{}\textquotesingle{} (ss }\OperatorTok{:\&:}\NormalTok{ d) }\OtherTok{=}\NormalTok{ sFoldr }\OperatorTok{????} \DataTypeTok{SOpened}\NormalTok{ ss}
                               \OperatorTok{:\&:}\NormalTok{ collapseHallwa\textquotesingle{}\textquotesingle{}y d}
\end{Highlighting}
\end{Shaded}

But how do we get a \texttt{Sing\ MergeStateSym0}?

We can use the \texttt{singFun} family of functions:

\begin{Shaded}
\begin{Highlighting}[]
\NormalTok{singFun2 }\OperatorTok{@}\DataTypeTok{MergeStateSym0}\NormalTok{ sMergeState}
\OtherTok{    ::} \DataTypeTok{Sing} \DataTypeTok{MergeStateSym0}
\end{Highlighting}
\end{Shaded}

But, also, conveniently, the \emph{singletons} library generates a
\texttt{SingI} instance for \texttt{MergeStateSym0}, if you defined
\texttt{mergeState} using the \emph{singletons} template haskell:

\begin{Shaded}
\begin{Highlighting}[]
\OtherTok{sing ::} \DataTypeTok{Sing} \DataTypeTok{MergeStateSym0}
\CommentTok{{-}{-} or}
\NormalTok{sing }\OperatorTok{@}\NormalTok{\_ }\OperatorTok{@}\DataTypeTok{MergeStateSym0}         \CommentTok{{-}{-} singletons 2.4}
\NormalTok{sing }\OperatorTok{@}\DataTypeTok{MergeStateSym0}            \CommentTok{{-}{-} singletons 2.5}
\end{Highlighting}
\end{Shaded}

And finally, we get our answer:

\begin{Shaded}
\begin{Highlighting}[]
\CommentTok{{-}{-} source: https://github.com/mstksg/inCode/tree/master/code{-}samples/singletons/Door4Final.hs\#L128{-}L133}

\OtherTok{collapseSomeHallway\textquotesingle{}\textquotesingle{} ::} \DataTypeTok{SomeHallway} \OtherTok{{-}>} \DataTypeTok{SomeDoor}
\NormalTok{collapseSomeHallway\textquotesingle{}\textquotesingle{} (ss }\OperatorTok{:\&:}\NormalTok{ d) }\OtherTok{=}
\NormalTok{        sFoldr (singFun2 }\OperatorTok{@}\DataTypeTok{MergeStateSym0}\NormalTok{ sMergeState) }\DataTypeTok{SOpened}\NormalTok{ ss}
     \CommentTok{{-}{-} or}
     \CommentTok{{-}{-} sFoldr (sing @MergeStateSym0) SOpened ss}
    \OperatorTok{:\&:}\NormalTok{ collapseHallway\textquotesingle{}\textquotesingle{} d}
\end{Highlighting}
\end{Shaded}

\hypertarget{closing-up}{%
\section{Closing Up}\label{closing-up}}

Woo! Congratulations, you've made it to the end of the this Introduction to
Singletons tetralogy! This last and final part understandably ramps things up
pretty quickly, so don't be afraid to re-read it a few times until it all sinks
in before jumping into the exercises.

I hope you enjoyed this journey deep into the motivation, philosophy, mechanics,
and usage of this great library. Hopefully these toy examples have been able to
show you a lot of ways that type-level programming can help your programs today,
both in type safety and in writing more expressive programs. And also, I hope
that you can also see now how to leverage the full power of the
\emph{singletons} library to make those gains a reality.

There are a few corners of the library we haven't gone over (like the TypeLits-
and TypeRep-based singletons -- if you're interested, check out
\href{https://blog.jle.im/entry/fixed-length-vector-types-in-haskell.html}{this
post} where I talk a lot about them), but I'd like to hope as well that this
series has equipped you to be able to dive into the library documentation and
decipher what it holds, armed with the knowledge you now have. (We also look at
TypeLits briefly in the exercises)

You can download the source code here ---
\href{https://github.com/mstksg/inCode/tree/master/code-samples/singletons/Door4Final.hs}{Door4Final.hs}
contains the final versions of all our definitions, and
\href{https://github.com/mstksg/inCode/tree/master/code-samples/singletons/Door3.hs}{Defunctionalization.hs}
contains all of our defunctionalization-from-scratch work. These are designed as
stack scripts that you can load into ghci. Just execute the scripts:

\begin{Shaded}
\begin{Highlighting}[]
\NormalTok{$ }\ExtensionTok{./Door4Final.hs}
\ExtensionTok{ghci}\OperatorTok{>}
\end{Highlighting}
\end{Shaded}

And you'll be dropped into a ghci session with all of the definitions in scope.

As always, please try out the exercises, which are designed to help solidify the
concepts we went over here! And if you ever have any future questions, feel free
to leave a comment or find me on \href{https://twitter.com/mstk}{twitter} or in
freenode \texttt{\#haskell}, where I idle as \emph{jle`}.

\hypertarget{looking-forward}{%
\subsection{Looking Forward}\label{looking-forward}}

Some final things to note before truly embracing singletons: remember that, as a
library, \emph{singletons} was always meant to become obsolete. It's a library
that only exists because Haskell doesn't have real dependent types yet.

\href{https://ghc.haskell.org/trac/ghc/wiki/DependentHaskell}{Dependent Haskell}
is coming some day! It's mostly driven by one solo man, Richard Eisenberg, but
every year buzz does get bigger. In a
\href{https://typesandkinds.wordpress.com/2016/07/24/dependent-types-in-haskell-progress-report/\#comment-13327}{recent
progress report}, we do know that we realistically won't have dependent types
before 2020. That means that this tutorial will still remain relevant for at
least another two years :)

How will things be different in a world of Haskell with real dependent types?
Well, for a good guess, take a look at
\href{https://github.com/goldfirere/thesis}{Richard Eisenberg's Dissertation}!

One day, hopefully, we won't need singletons to work with types at the
value-level; we would just be able to directly pattern match and manipulate the
types within the language and use them as first-class values, with a nice story
for dependent sums. And some day, I hope we won't need any more dances with
defunctionalization symbols to write higher-order functions at the type level
--- maybe we'll have a nicer way to work with partially applied type-level
functions (maybe they'll just be normal functions?), and we don't need to think
any different about higher-order or first-order functions.

So, as a final word --- Happy Haskelling, everyone! May you leverage the great
\emph{singletons} library to its full potential, and may we also all dream of a
day where \emph{singletons} becomes obsolete. But may we all enjoy the wonderful
journey along the way.

Until next time!

\hypertarget{exercises}{%
\section{Exercises}\label{exercises}}

Here are your final exercises for this series! Start from
\href{https://github.com/mstksg/inCode/tree/master/code-samples/singletons/Door4Final.hs}{this
sample source code}, which has all of the definitions that the exercises and
their solutions require. Just make sure to delete all of the parts after the
\texttt{-\/-\ Exercises} comment if you don't want to be spoiled. Remember again
to enable \texttt{-Werror=incomplete-patterns} or \texttt{-Wall} to ensure that
all of your functions are total.

\begin{enumerate}
\def\labelenumi{\arabic{enumi}.}
\item
  Let's try combining type families with proofs! In doing so, hopefully we can
  also see the value of using dependent proofs to show how we can manipulate
  proofs as first-class values that the compiler can verify.

  Remember \texttt{Knockable} from Part 3?

\begin{Shaded}
\begin{Highlighting}[]
\CommentTok{{-}{-} source: https://github.com/mstksg/inCode/tree/master/code{-}samples/singletons/Door4Final.hs\#L139{-}L141}

\KeywordTok{data} \DataTypeTok{Knockable}\OtherTok{ ::} \DataTypeTok{DoorState} \OtherTok{{-}>} \DataTypeTok{Type} \KeywordTok{where}
    \DataTypeTok{KnockClosed}\OtherTok{ ::} \DataTypeTok{Knockable} \DataTypeTok{\textquotesingle{}Closed}
    \DataTypeTok{KnockLocked}\OtherTok{ ::} \DataTypeTok{Knockable} \DataTypeTok{\textquotesingle{}Locked}
\end{Highlighting}
\end{Shaded}

  Closed and Locked doors are knockable. But, if you merge two knockable
  doors\ldots is the result \emph{also} always knockable?

  I say yes, but don't take my word for it. Prove it using \texttt{Knockable}!

\begin{Shaded}
\begin{Highlighting}[]
\CommentTok{{-}{-} source: https://github.com/mstksg/inCode/tree/master/code{-}samples/singletons/Door4Final.hs\#L156{-}L159}

\NormalTok{mergedIsKnockable}
\OtherTok{    ::} \DataTypeTok{Knockable}\NormalTok{ s}
    \OtherTok{{-}>} \DataTypeTok{Knockable}\NormalTok{ t}
    \OtherTok{{-}>} \DataTypeTok{Knockable}\NormalTok{ (}\DataTypeTok{MergeState}\NormalTok{ s t)}
\end{Highlighting}
\end{Shaded}

  \texttt{mergedIsKnockable} is only implementable if the merging of two
  DoorStates that are knockable is also knockable. See if you can write the
  implementation!

  \href{https://github.com/mstksg/inCode/tree/master/code-samples/singletons/Door4Final.hs\#L154-L154}{Solution
  here!}
\item
  Write a function to append two hallways together.

\begin{Shaded}
\begin{Highlighting}[]
\NormalTok{appendHallways}
\OtherTok{    ::} \DataTypeTok{Hallway}\NormalTok{ ss}
    \OtherTok{{-}>} \DataTypeTok{Hallway}\NormalTok{ ts}
    \OtherTok{{-}>} \DataTypeTok{Hallway} \OperatorTok{????}
\end{Highlighting}
\end{Shaded}

  from \emph{singletons} --- implement any type families you might need from
  scratch!

  Remember the important principle that your type family must mirror the
  implementation of the functions that use it.

  Next, for fun, use \texttt{appendHallways} to implement
  \texttt{appendSomeHallways}:

\begin{Shaded}
\begin{Highlighting}[]
\CommentTok{{-}{-} source: https://github.com/mstksg/inCode/tree/master/code{-}samples/singletons/Door4Final.hs\#L84{-}L187}

\KeywordTok{type} \DataTypeTok{SomeHallway} \OtherTok{=} \DataTypeTok{Sigma}\NormalTok{ [}\DataTypeTok{DoorState}\NormalTok{] (}\DataTypeTok{TyCon1} \DataTypeTok{Hallway}\NormalTok{)}

\NormalTok{appendSomeHallways}
\OtherTok{    ::} \DataTypeTok{SomeHallway}
    \OtherTok{{-}>} \DataTypeTok{SomeHallway}
    \OtherTok{{-}>} \DataTypeTok{SomeHallway}
\end{Highlighting}
\end{Shaded}

  \href{https://github.com/mstksg/inCode/tree/master/code-samples/singletons/Door4Final.hs\#L169-L169}{Solution
  here!}
\item
  Can you use \texttt{Sigma} to define a door that must be knockable?

  To do this, try directly defining the defunctionalization symbol
  \texttt{KnockableDoor\ ::\ DoorState\ \textasciitilde{}\textgreater{}\ Type}
  (or use singletons to generate it for you --- remember that \emph{singletons}
  can also promote type families) so that:

\begin{Shaded}
\begin{Highlighting}[]
\KeywordTok{type} \DataTypeTok{SomeKnockableDoor} \OtherTok{=} \DataTypeTok{Sigma} \DataTypeTok{DoorState} \DataTypeTok{KnockableDoor}
\end{Highlighting}
\end{Shaded}

  will contain a \texttt{Door} that must be knockable.

  Try doing it for both (a) the ``dependent proof'' version (with the
  \texttt{Knockable} data type) and for (b) the type family version (with the
  \texttt{StatePass} type family).

  \href{https://github.com/mstksg/inCode/tree/master/code-samples/singletons/Door4Final.hs\#L192-L192}{Solutions
  here!} I gave four different ways of doing it, for a full range of manual
  vs.~auto-promoted defunctionalization symbols and \texttt{Knockable}
  vs.~\texttt{Pass}-based methods.

  \emph{Hint:} Look at the definition of \texttt{SomeDoor} in terms of
  \texttt{Sigma}:

\begin{Shaded}
\begin{Highlighting}[]
\KeywordTok{type} \DataTypeTok{SomeDoor} \OtherTok{=} \DataTypeTok{Sigma} \DataTypeTok{DoorState}\NormalTok{ (}\DataTypeTok{TyCon1} \DataTypeTok{Door}\NormalTok{)}
\end{Highlighting}
\end{Shaded}

  \emph{Hint}: Try having \texttt{KnockableDoor} return a tuple.
\item
  Take a look at the API of the
  \emph{\href{http://hackage.haskell.org/package/singletons-2.5/docs/Data-Singletons-TypeLits.html}{Data.Singletons.TypeLits}}
  module, based on the API exposed in
  \emph{\href{http://hackage.haskell.org/package/base-4.12.0.0/docs/GHC-TypeNats.html}{GHC.TypeNats}}
  module from \emph{base}.

  Using this, you can use \texttt{Sigma} to create a predicate that a given
  \texttt{Nat} number is even:

\begin{Shaded}
\begin{Highlighting}[]
\KeywordTok{data} \DataTypeTok{IsHalfOf}\OtherTok{ ::} \DataTypeTok{Nat} \OtherTok{{-}>} \DataTypeTok{Nat} \OperatorTok{\textasciitilde{}>} \DataTypeTok{Type}
\KeywordTok{type} \KeywordTok{instance} \DataTypeTok{Apply}\NormalTok{ (}\DataTypeTok{IsHalfOf}\NormalTok{ n) m }\OtherTok{=}\NormalTok{ n }\OperatorTok{:\textasciitilde{}:}\NormalTok{ (m }\OperatorTok{*} \DecValTok{2}\NormalTok{)}

\KeywordTok{type} \DataTypeTok{IsEven}\NormalTok{ n }\OtherTok{=} \DataTypeTok{Sigma} \DataTypeTok{Nat}\NormalTok{ (}\DataTypeTok{IsHalfOf}\NormalTok{ n)}
\end{Highlighting}
\end{Shaded}

  \texttt{(*)} is multiplication from the
  \emph{\href{http://hackage.haskell.org/package/singletons-2.5/docs/Data-Singletons-Prelude-Num.html}{Data.Singletons.Prelude.Num}}
  module. (\textbf{You must have the \emph{-XNoStarIsType} extension on} for
  this to work in GHC 8.6+), and \texttt{:\textasciitilde{}:} is the predicate
  of equality from Part 3:

\begin{Shaded}
\begin{Highlighting}[]
\KeywordTok{data}\OtherTok{ (:\textasciitilde{}:) ::}\NormalTok{ k }\OtherTok{{-}>}\NormalTok{ k }\OtherTok{{-}>} \DataTypeTok{Type} \KeywordTok{where}
    \DataTypeTok{Refl}\OtherTok{ ::}\NormalTok{ a }\OperatorTok{:\textasciitilde{}:}\NormalTok{ a}
\end{Highlighting}
\end{Shaded}

  (It's only possible to make a value of type \texttt{a\ :\textasciitilde{}:\ b}
  using \texttt{Refl\ ::\ a\ :\textasciitilde{}:\ a}, so it's only possible to
  make a value of that type when \texttt{a} and \texttt{b} are equal. I like to
  use \texttt{Refl} with type application syntax, like \texttt{Refl\ @a}, so
  it's clear what we are saying is the same on both sides;
  \texttt{Refl\ @a\ ::\ a\ :\textasciitilde{}:\ a})

  The only way to construct an \texttt{IsEven\ n} is to provide a number
  \texttt{m} where \texttt{m\ *\ 2} is \texttt{n}. We can do this by using
  \texttt{SNat\ @m}, which is the singleton constructor for the \texttt{Nat}
  kind (just like how \texttt{STrue} and \texttt{SFalse} are the singleton
  constructors for the \texttt{Bool} kind):

\begin{Shaded}
\begin{Highlighting}[]
\OtherTok{tenIsEven ::} \DataTypeTok{IsEven} \DecValTok{10}
\NormalTok{tenIsEven }\OtherTok{=} \DataTypeTok{SNat} \OperatorTok{@}\DecValTok{5} \OperatorTok{:\&:} \DataTypeTok{Refl} \OperatorTok{@}\DecValTok{10}
    \CommentTok{{-}{-} Refl is the constructor of type n :\textasciitilde{}: (m * 2)}
    \CommentTok{{-}{-} here, we use it as Refl @10 :: 10 :\textasciitilde{}: 10}

\CommentTok{{-}{-} won\textquotesingle{}t compile}
\OtherTok{sevenIsEven ::} \DataTypeTok{IsEven} \DecValTok{10}
\NormalTok{sevenIsEven }\OtherTok{=} \DataTypeTok{SNat} \OperatorTok{@}\DecValTok{4} \OperatorTok{:\&:} \DataTypeTok{Refl}
    \CommentTok{{-}{-} won\textquotesingle{}t compile, because we need something of type \textasciigrave{}(4 * 2) :\textasciitilde{}: 7\textasciigrave{},}
    \CommentTok{{-}{-} but Refl must have type \textasciigrave{}a :\textasciitilde{}: a\textasciigrave{}; \textasciigrave{}8 :\textasciitilde{}: 7\textasciigrave{} is not constructable}
    \CommentTok{{-}{-} using \textasciigrave{}Refl\textasciigrave{}.  Neither \textasciigrave{}Refl @8\textasciigrave{} nor \textasciigrave{}Refl @7\textasciigrave{} will work.}
\end{Highlighting}
\end{Shaded}

  Write a similar type \texttt{IsOdd\ n} that can only be constructed if
  \texttt{n} is \emph{odd}.

\begin{Shaded}
\begin{Highlighting}[]
\KeywordTok{type} \DataTypeTok{IsOdd}\NormalTok{ n }\OtherTok{=} \DataTypeTok{Sigma} \DataTypeTok{Nat}\NormalTok{ (}\OperatorTok{????}\NormalTok{ n)}
\end{Highlighting}
\end{Shaded}

  And construct a proof that \texttt{7} is odd:

\begin{Shaded}
\begin{Highlighting}[]
\CommentTok{{-}{-} source: https://github.com/mstksg/inCode/tree/master/code{-}samples/singletons/Door4Final.hs\#L226{-}L226}

\OtherTok{sevenIsOdd ::} \DataTypeTok{IsOdd} \DecValTok{7}
\end{Highlighting}
\end{Shaded}

  \href{https://github.com/mstksg/inCode/tree/master/code-samples/singletons/Door4Final.hs\#L220-L220}{Solution
  here!}

  On a sad note, one exercise I'd like to be able to add is to ask you to write
  decision functions and proofs for \texttt{IsEven} and \texttt{IsOdd}.
  Unfortunately, \texttt{Nat} is not rich enough to support this out of the box
  without a lot of extra tooling!
\item
  A common beginner Haskeller exercise is to implement \texttt{map} in terms of
  \texttt{foldr}:

\begin{Shaded}
\begin{Highlighting}[]
\FunctionTok{map}\OtherTok{ ::}\NormalTok{ (a }\OtherTok{{-}>}\NormalTok{ b) }\OtherTok{{-}>}\NormalTok{ [a] \_}\OperatorTok{>}\NormalTok{ [b]}
\FunctionTok{map}\NormalTok{ f }\OtherTok{=} \FunctionTok{foldr}\NormalTok{ ((}\OperatorTok{:}\NormalTok{) }\OperatorTok{.}\NormalTok{ f) []}
\end{Highlighting}
\end{Shaded}

  Let's do the same thing at the type level, manually.

  Directly implement a type-level \texttt{Map}, with kind
  \texttt{(j\ \textasciitilde{}\textgreater{}\ k)\ -\textgreater{}\ {[}j{]}\ -\textgreater{}\ {[}k{]}},
  in terms of \texttt{Foldr}:

\begin{Shaded}
\begin{Highlighting}[]
\KeywordTok{type} \DataTypeTok{Map}\NormalTok{ f xs }\OtherTok{=} \DataTypeTok{Foldr} \OperatorTok{????} \OperatorTok{????}\NormalTok{ xs}
\end{Highlighting}
\end{Shaded}

  Try to mirror the value-level definition, passing in \texttt{(:)\ .\ f}, and
  use the promoted version of \texttt{(.)} from the \emph{singletons} library,
  in
  \emph{\href{http://hackage.haskell.org/package/singletons-2.5/docs/Data-Singletons-Prelude.html}{Data.Singletons.Prelude}}.
  You might find \texttt{TyCon2} helpful!

  \href{https://github.com/mstksg/inCode/tree/master/code-samples/singletons/Door4Final.hs\#L229-L229}{Solution
  here!}
\item
  Make a \texttt{SomeHallway} from a list of \texttt{SomeDoor}:

\begin{Shaded}
\begin{Highlighting}[]
\CommentTok{{-}{-} source: https://github.com/mstksg/inCode/tree/master/code{-}samples/singletons/Door4Final.hs\#L54{-}L233}

\KeywordTok{type} \DataTypeTok{SomeDoor} \OtherTok{=} \DataTypeTok{Sigma} \DataTypeTok{DoorState}\NormalTok{ (}\DataTypeTok{TyCon1} \DataTypeTok{Door}\NormalTok{)}

\KeywordTok{type} \DataTypeTok{SomeHallway} \OtherTok{=} \DataTypeTok{Sigma}\NormalTok{ [}\DataTypeTok{DoorState}\NormalTok{] (}\DataTypeTok{TyCon1} \DataTypeTok{Hallway}\NormalTok{)}

\OtherTok{mkSomeHallway ::}\NormalTok{ [}\DataTypeTok{SomeDoor}\NormalTok{] }\OtherTok{{-}>} \DataTypeTok{SomeHallway}
\end{Highlighting}
\end{Shaded}

  Remember that the singleton constructors for list are \texttt{SNil} (for
  \texttt{{[}{]}}) and \texttt{SCons} (for \texttt{(:)})!

  \href{https://github.com/mstksg/inCode/tree/master/code-samples/singletons/Door4Final.hs\#L229-L229}{Solution
  here!}
\end{enumerate}

\hypertarget{special-thanks}{%
\section{Special Thanks}\label{special-thanks}}

None of this entire series would be possible without the hard work and effort of
the amazing \emph{singletons} library authors and maintainers --- especially
\href{https://github.com/goldfirere}{Richard Eisenberg} and
\href{https://github.com/RyanGlScott}{Ryan Scott}.

I am very humbled to be supported by an amazing community, who make it possible
for me to devote time to researching and writing these posts. Very special
thanks to my two supporters at the ``Amazing'' level on
\href{https://www.patreon.com/justinle/overview}{patreon}, Sam Stites and Josh
Vera! :)

Thanks also to \href{https://twitter.com/KozRoss}{Koz Ross} for helping
proofread this post!

\hypertarget{signoff}{%
\section{Signoff}\label{signoff}}

Hi, thanks for reading! You can reach me via email at
\href{mailto:justin@jle.im}{\nolinkurl{justin@jle.im}}, or at twitter at
\href{https://twitter.com/mstk}{@mstk}! This post and all others are published
under the \href{https://creativecommons.org/licenses/by-nc-nd/3.0/}{CC-BY-NC-ND
3.0} license. Corrections and edits via pull request are welcome and encouraged
at \href{https://github.com/mstksg/inCode}{the source repository}.

If you feel inclined, or this post was particularly helpful for you, why not
consider \href{https://www.patreon.com/justinle/overview}{supporting me on
Patreon}, or a \href{bitcoin:3D7rmAYgbDnp4gp4rf22THsGt74fNucPDU}{BTC donation}?
:)

\end{document}
