\documentclass[]{article}
\usepackage{lmodern}
\usepackage{amssymb,amsmath}
\usepackage{ifxetex,ifluatex}
\usepackage{fixltx2e} % provides \textsubscript
\ifnum 0\ifxetex 1\fi\ifluatex 1\fi=0 % if pdftex
  \usepackage[T1]{fontenc}
  \usepackage[utf8]{inputenc}
\else % if luatex or xelatex
  \ifxetex
    \usepackage{mathspec}
    \usepackage{xltxtra,xunicode}
  \else
    \usepackage{fontspec}
  \fi
  \defaultfontfeatures{Mapping=tex-text,Scale=MatchLowercase}
  \newcommand{\euro}{€}
\fi
% use upquote if available, for straight quotes in verbatim environments
\IfFileExists{upquote.sty}{\usepackage{upquote}}{}
% use microtype if available
\IfFileExists{microtype.sty}{\usepackage{microtype}}{}
\usepackage[margin=1in]{geometry}
\ifxetex
  \usepackage[setpagesize=false, % page size defined by xetex
              unicode=false, % unicode breaks when used with xetex
              xetex]{hyperref}
\else
  \usepackage[unicode=true]{hyperref}
\fi
\hypersetup{breaklinks=true,
            bookmarks=true,
            pdfauthor={Justin Le},
            pdftitle={Automatic Propagation of Uncertainty with AD},
            colorlinks=true,
            citecolor=blue,
            urlcolor=blue,
            linkcolor=magenta,
            pdfborder={0 0 0}}
\urlstyle{same}  % don't use monospace font for urls
% Make links footnotes instead of hotlinks:
\renewcommand{\href}[2]{#2\footnote{\url{#1}}}
\setlength{\parindent}{0pt}
\setlength{\parskip}{6pt plus 2pt minus 1pt}
\setlength{\emergencystretch}{3em}  % prevent overfull lines
\setcounter{secnumdepth}{0}

\title{Automatic Propagation of Uncertainty with AD}
\author{Justin Le}

\begin{document}
\maketitle

\emph{Originally posted on \textbf{\href{https://blog.jle.im/}{in
Code}}.}

Some of my favorite Haskell ``tricks'' involve working with exotic
numeric types with custom ``overloaded'' numeric functions and literals
that let us work with data in surprisingly elegant and expressive ways.

Here is one example --- from my work in experimental physics and
statistics, we often deal with experimental/sampled values with inherent
uncertainty. If you ever measure something to be \(12.4 \mathrm{cm}\),
that doesn't mean it's \(12.400000 \mathrm{cm}\), it means that it's
somewhere between \(12.3 \mathrm{cm}\) and
\(12.5 \mathrm{cm}\)\ldots{}and we don't know exactly. We can write it
as \(12.4 \pm 0.1 \mathrm{cm}\).

\end{document}
