\documentclass[]{article}
\usepackage{lmodern}
\usepackage{amssymb,amsmath}
\usepackage{ifxetex,ifluatex}
\usepackage{fixltx2e} % provides \textsubscript
\ifnum 0\ifxetex 1\fi\ifluatex 1\fi=0 % if pdftex
  \usepackage[T1]{fontenc}
  \usepackage[utf8]{inputenc}
\else % if luatex or xelatex
  \ifxetex
    \usepackage{mathspec}
    \usepackage{xltxtra,xunicode}
  \else
    \usepackage{fontspec}
  \fi
  \defaultfontfeatures{Mapping=tex-text,Scale=MatchLowercase}
  \newcommand{\euro}{€}
\fi
% use upquote if available, for straight quotes in verbatim environments
\IfFileExists{upquote.sty}{\usepackage{upquote}}{}
% use microtype if available
\IfFileExists{microtype.sty}{\usepackage{microtype}}{}
\usepackage[margin=1in]{geometry}
\usepackage{color}
\usepackage{fancyvrb}
\newcommand{\VerbBar}{|}
\newcommand{\VERB}{\Verb[commandchars=\\\{\}]}
\DefineVerbatimEnvironment{Highlighting}{Verbatim}{commandchars=\\\{\}}
% Add ',fontsize=\small' for more characters per line
\newenvironment{Shaded}{}{}
\newcommand{\KeywordTok}[1]{\textcolor[rgb]{0.00,0.44,0.13}{\textbf{{#1}}}}
\newcommand{\DataTypeTok}[1]{\textcolor[rgb]{0.56,0.13,0.00}{{#1}}}
\newcommand{\DecValTok}[1]{\textcolor[rgb]{0.25,0.63,0.44}{{#1}}}
\newcommand{\BaseNTok}[1]{\textcolor[rgb]{0.25,0.63,0.44}{{#1}}}
\newcommand{\FloatTok}[1]{\textcolor[rgb]{0.25,0.63,0.44}{{#1}}}
\newcommand{\ConstantTok}[1]{\textcolor[rgb]{0.53,0.00,0.00}{{#1}}}
\newcommand{\CharTok}[1]{\textcolor[rgb]{0.25,0.44,0.63}{{#1}}}
\newcommand{\SpecialCharTok}[1]{\textcolor[rgb]{0.25,0.44,0.63}{{#1}}}
\newcommand{\StringTok}[1]{\textcolor[rgb]{0.25,0.44,0.63}{{#1}}}
\newcommand{\VerbatimStringTok}[1]{\textcolor[rgb]{0.25,0.44,0.63}{{#1}}}
\newcommand{\SpecialStringTok}[1]{\textcolor[rgb]{0.73,0.40,0.53}{{#1}}}
\newcommand{\ImportTok}[1]{{#1}}
\newcommand{\CommentTok}[1]{\textcolor[rgb]{0.38,0.63,0.69}{\textit{{#1}}}}
\newcommand{\DocumentationTok}[1]{\textcolor[rgb]{0.73,0.13,0.13}{\textit{{#1}}}}
\newcommand{\AnnotationTok}[1]{\textcolor[rgb]{0.38,0.63,0.69}{\textbf{\textit{{#1}}}}}
\newcommand{\CommentVarTok}[1]{\textcolor[rgb]{0.38,0.63,0.69}{\textbf{\textit{{#1}}}}}
\newcommand{\OtherTok}[1]{\textcolor[rgb]{0.00,0.44,0.13}{{#1}}}
\newcommand{\FunctionTok}[1]{\textcolor[rgb]{0.02,0.16,0.49}{{#1}}}
\newcommand{\VariableTok}[1]{\textcolor[rgb]{0.10,0.09,0.49}{{#1}}}
\newcommand{\ControlFlowTok}[1]{\textcolor[rgb]{0.00,0.44,0.13}{\textbf{{#1}}}}
\newcommand{\OperatorTok}[1]{\textcolor[rgb]{0.40,0.40,0.40}{{#1}}}
\newcommand{\BuiltInTok}[1]{{#1}}
\newcommand{\ExtensionTok}[1]{{#1}}
\newcommand{\PreprocessorTok}[1]{\textcolor[rgb]{0.74,0.48,0.00}{{#1}}}
\newcommand{\AttributeTok}[1]{\textcolor[rgb]{0.49,0.56,0.16}{{#1}}}
\newcommand{\RegionMarkerTok}[1]{{#1}}
\newcommand{\InformationTok}[1]{\textcolor[rgb]{0.38,0.63,0.69}{\textbf{\textit{{#1}}}}}
\newcommand{\WarningTok}[1]{\textcolor[rgb]{0.38,0.63,0.69}{\textbf{\textit{{#1}}}}}
\newcommand{\AlertTok}[1]{\textcolor[rgb]{1.00,0.00,0.00}{\textbf{{#1}}}}
\newcommand{\ErrorTok}[1]{\textcolor[rgb]{1.00,0.00,0.00}{\textbf{{#1}}}}
\newcommand{\NormalTok}[1]{{#1}}
\ifxetex
  \usepackage[setpagesize=false, % page size defined by xetex
              unicode=false, % unicode breaks when used with xetex
              xetex]{hyperref}
\else
  \usepackage[unicode=true]{hyperref}
\fi
\hypersetup{breaklinks=true,
            bookmarks=true,
            pdfauthor={Justin Le},
            pdftitle={Automatic Propagation of Uncertainty with AD},
            colorlinks=true,
            citecolor=blue,
            urlcolor=blue,
            linkcolor=magenta,
            pdfborder={0 0 0}}
\urlstyle{same}  % don't use monospace font for urls
% Make links footnotes instead of hotlinks:
\renewcommand{\href}[2]{#2\footnote{\url{#1}}}
\setlength{\parindent}{0pt}
\setlength{\parskip}{6pt plus 2pt minus 1pt}
\setlength{\emergencystretch}{3em}  % prevent overfull lines
\setcounter{secnumdepth}{0}

\title{Automatic Propagation of Uncertainty with AD}
\author{Justin Le}

\begin{document}
\maketitle

\emph{Originally posted on \textbf{\href{https://blog.jle.im/}{in
Code}}.}

Some of my favorite Haskell ``tricks'' involve working with exotic
numeric types with custom ``overloaded'' numeric functions and literals
that let us work with data in surprisingly elegant and expressive ways.

Here is one example --- from my work in experimental physics and
statistics, we often deal with experimental/sampled values with inherent
uncertainty. If you ever measure something to be \(12.4\,\mathrm{cm}\),
that doesn't mean it's \(12.400000\,\mathrm{cm}\), it means that it's
somewhere between \(12.3\,\mathrm{cm}\) and
\(12.5\,\mathrm{cm}\)\ldots{}and we don't know exactly. We can write it
as \(12.4 \pm 0.1\,\mathrm{cm}\).

The interesting thing happens when we try to add, multiply, divide
numbers with uncertainty. What happens when you ``add'' \(12 \pm 3\) and
\(19 \pm 6\)?

The initial guess might be \(27 \pm 9\), because one is \(\pm 3\) and
the other is \(\pm 6\).

But! If you actually do experiments like this several times, you'll see
that this isn't the case. If you tried this out experimentally and
simulate several hundred trials, you'll see that the answer is actually
something like \(31 \pm 7\).\footnote{If you don't believe me, stop
  reading this article now and try it yourself! You can simulate noisy
  data by using uniform noise distributions, Gaussian distributions, or
  however manner you like. Verify by checking the
  \href{https://en.wikipedia.org/wiki/Variance}{variance} of the sum.}

Let's write ourselves a Haskell data type that lets us work with
``numbers with inherent uncertainty'':

\begin{Shaded}
\begin{Highlighting}[]
\NormalTok{ghci}\FunctionTok{>} \KeywordTok{let} \NormalTok{x }\FunctionTok{=} \FloatTok{14.6} \FunctionTok{+/-} \FloatTok{0.8}
\NormalTok{ghci}\FunctionTok{>} \KeywordTok{let} \NormalTok{y }\FunctionTok{=} \DecValTok{31}   \FunctionTok{+/-} \DecValTok{2}
\NormalTok{ghci}\FunctionTok{>} \NormalTok{x }\FunctionTok{+} \NormalTok{y}
\DecValTok{46} \FunctionTok{+/-} \DecValTok{2}
\NormalTok{ghci}\FunctionTok{>} \NormalTok{x }\FunctionTok{*} \NormalTok{y}
\DecValTok{450} \FunctionTok{+/-} \DecValTok{40}
\NormalTok{ghci}\FunctionTok{>} \NormalTok{sqrt (x }\FunctionTok{+} \NormalTok{y)}
\FloatTok{6.8} \FunctionTok{+/-} \FloatTok{0.2}
\NormalTok{ghci}\FunctionTok{>} \NormalTok{logBase y x}
\FloatTok{0.78} \FunctionTok{+/-} \FloatTok{0.02}
\NormalTok{ghci}\FunctionTok{>} \NormalTok{log (x}\FunctionTok{**}\NormalTok{y)}
\FloatTok{85.9} \FunctionTok{+/-} \FloatTok{0.3}
\end{Highlighting}
\end{Shaded}

Along the way, we'll also learn how to harness the power of awesome
\href{http://hackage.haskell.org/package/ad}{ad} library, a library used
in implementing back-propagation and other optimization algorithms, to
analyze numerical functions in a mathematical way and break down their
derivatives and gradients.

\section{Certain Uncertainty}\label{certain-uncertainty}

First of all, let's think about why adding two ``uncertain'' values
doesn't involve simply adding the uncertainties linearly.

If I have a value \(16 \pm 3\) (maybe I have a ruler whose ticks are 2
units apart, or an instrument that produces measurements with 4 units of
noise), it either means that it's a little below 16 or a little above
16. If I have an independently sampled value \(25 \pm 4\), it means that
it's a little below 25 or a little above 25.

What happens if I want to think about their sum? Well, it's going to be
somewhere around 41. But, the uncertainty won't be \(\pm 7\). In order
for that to be possible, the errors in the two values have to
\emph{always be aligned}. Only when every ``little bit above'' 16 error
lines up perfectly with a ``little bit above'' 25 error, and when every
single ``little bit below'' 16 error lines up perfectly with a ``little
bit above'' 25 error, would you really get something that is \(\pm 7\).

But our two values were sampled independently, and so they are
uncorrelated. You shouldn't expect that if your \(16 \pm 3\) value is
ever ``a little bit above'', then the \(25 \pm 4\) value is also ``a
little bit above'', as well. They're uncorrelated and independent, so
their errors won't actually always align. Instead, you'll get a variance
that's \emph{less than} \(\pm 7\), because a lot of times the deviations
will ``cancel out''. We can mathematically derive that the variance will
be exactly (in a platonic way) \(\pm 5\).

In general, we find that, for \emph{independent} \(X\) and \(Y\):

\[
\operatorname{Var}[aX + bY + c] = a^2 \sigma_X^2 + b^2 \sigma_Y^2
\]

Where \(\sigma_X^2\) is the variance in \(X\). We consider \(\sigma_X\)
to be the standard deviation of \(X\), or the ``plus or minus'' part of
our numbers.

In the simple case of addition, we have
\(\operatorname{Var}[X + Y] = \sigma_X^2 + \sigma_Y^2\), so our new
uncertainty is \(\sqrt{\sigma_X^2 + \sigma_Y^2}\).

However, not all functions that combine \(X\) and \(Y\) can be expressed
as simple linear combinations \(aX + bY + c\). But! If you dig back to
your days of high school calculus, you might remember a method for
expressing any arbitrary function as a linear approximation -- the
\href{https://en.wikipedia.org/wiki/Taylor_series}{Taylor Expansion}!

In general, we can attempt to approximate any well-behaving function as
its tangent hyperplane:

\[
f(x_0 + x, y_0 + y) \approx
\left.\frac{\partial f}{\partial x}\right\vert_{x_0, y_0} x + 
\left.\frac{\partial f}{\partial y}\right\vert_{x_0, y_0} y + 
f(x_0, y_0)
\]

Look familiar? This is exactly the form that we used earlier to
calculate ``combined'' variance!

\[
\operatorname{Var}[f(X,Y)] \approx 
\left.\frac{\partial f}{\partial x}\right\vert_{\mu_X, \mu_Y}^2 \sigma_X^2 +
\left.\frac{\partial f}{\partial x}\right\vert_{\mu_X, \mu_Y}^2 \sigma_Y^2
\]

neat, huh?

\end{document}
