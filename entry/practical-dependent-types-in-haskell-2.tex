\documentclass[]{article}
\usepackage{lmodern}
\usepackage{amssymb,amsmath}
\usepackage{ifxetex,ifluatex}
\usepackage{fixltx2e} % provides \textsubscript
\ifnum 0\ifxetex 1\fi\ifluatex 1\fi=0 % if pdftex
  \usepackage[T1]{fontenc}
  \usepackage[utf8]{inputenc}
\else % if luatex or xelatex
  \ifxetex
    \usepackage{mathspec}
    \usepackage{xltxtra,xunicode}
  \else
    \usepackage{fontspec}
  \fi
  \defaultfontfeatures{Mapping=tex-text,Scale=MatchLowercase}
  \newcommand{\euro}{€}
\fi
% use upquote if available, for straight quotes in verbatim environments
\IfFileExists{upquote.sty}{\usepackage{upquote}}{}
% use microtype if available
\IfFileExists{microtype.sty}{\usepackage{microtype}}{}
\usepackage[margin=1in]{geometry}
\usepackage{color}
\usepackage{fancyvrb}
\newcommand{\VerbBar}{|}
\newcommand{\VERB}{\Verb[commandchars=\\\{\}]}
\DefineVerbatimEnvironment{Highlighting}{Verbatim}{commandchars=\\\{\}}
% Add ',fontsize=\small' for more characters per line
\newenvironment{Shaded}{}{}
\newcommand{\KeywordTok}[1]{\textcolor[rgb]{0.00,0.44,0.13}{\textbf{{#1}}}}
\newcommand{\DataTypeTok}[1]{\textcolor[rgb]{0.56,0.13,0.00}{{#1}}}
\newcommand{\DecValTok}[1]{\textcolor[rgb]{0.25,0.63,0.44}{{#1}}}
\newcommand{\BaseNTok}[1]{\textcolor[rgb]{0.25,0.63,0.44}{{#1}}}
\newcommand{\FloatTok}[1]{\textcolor[rgb]{0.25,0.63,0.44}{{#1}}}
\newcommand{\ConstantTok}[1]{\textcolor[rgb]{0.53,0.00,0.00}{{#1}}}
\newcommand{\CharTok}[1]{\textcolor[rgb]{0.25,0.44,0.63}{{#1}}}
\newcommand{\SpecialCharTok}[1]{\textcolor[rgb]{0.25,0.44,0.63}{{#1}}}
\newcommand{\StringTok}[1]{\textcolor[rgb]{0.25,0.44,0.63}{{#1}}}
\newcommand{\VerbatimStringTok}[1]{\textcolor[rgb]{0.25,0.44,0.63}{{#1}}}
\newcommand{\SpecialStringTok}[1]{\textcolor[rgb]{0.73,0.40,0.53}{{#1}}}
\newcommand{\ImportTok}[1]{{#1}}
\newcommand{\CommentTok}[1]{\textcolor[rgb]{0.38,0.63,0.69}{\textit{{#1}}}}
\newcommand{\DocumentationTok}[1]{\textcolor[rgb]{0.73,0.13,0.13}{\textit{{#1}}}}
\newcommand{\AnnotationTok}[1]{\textcolor[rgb]{0.38,0.63,0.69}{\textbf{\textit{{#1}}}}}
\newcommand{\CommentVarTok}[1]{\textcolor[rgb]{0.38,0.63,0.69}{\textbf{\textit{{#1}}}}}
\newcommand{\OtherTok}[1]{\textcolor[rgb]{0.00,0.44,0.13}{{#1}}}
\newcommand{\FunctionTok}[1]{\textcolor[rgb]{0.02,0.16,0.49}{{#1}}}
\newcommand{\VariableTok}[1]{\textcolor[rgb]{0.10,0.09,0.49}{{#1}}}
\newcommand{\ControlFlowTok}[1]{\textcolor[rgb]{0.00,0.44,0.13}{\textbf{{#1}}}}
\newcommand{\OperatorTok}[1]{\textcolor[rgb]{0.40,0.40,0.40}{{#1}}}
\newcommand{\BuiltInTok}[1]{{#1}}
\newcommand{\ExtensionTok}[1]{{#1}}
\newcommand{\PreprocessorTok}[1]{\textcolor[rgb]{0.74,0.48,0.00}{{#1}}}
\newcommand{\AttributeTok}[1]{\textcolor[rgb]{0.49,0.56,0.16}{{#1}}}
\newcommand{\RegionMarkerTok}[1]{{#1}}
\newcommand{\InformationTok}[1]{\textcolor[rgb]{0.38,0.63,0.69}{\textbf{\textit{{#1}}}}}
\newcommand{\WarningTok}[1]{\textcolor[rgb]{0.38,0.63,0.69}{\textbf{\textit{{#1}}}}}
\newcommand{\AlertTok}[1]{\textcolor[rgb]{1.00,0.00,0.00}{\textbf{{#1}}}}
\newcommand{\ErrorTok}[1]{\textcolor[rgb]{1.00,0.00,0.00}{\textbf{{#1}}}}
\newcommand{\NormalTok}[1]{{#1}}
\ifxetex
  \usepackage[setpagesize=false, % page size defined by xetex
              unicode=false, % unicode breaks when used with xetex
              xetex]{hyperref}
\else
  \usepackage[unicode=true]{hyperref}
\fi
\hypersetup{breaklinks=true,
            bookmarks=true,
            pdfauthor={Justin Le},
            pdftitle={Practical Dependent Types in Haskell: Type-Safe Neural Networks (Part 2)},
            colorlinks=true,
            citecolor=blue,
            urlcolor=blue,
            linkcolor=magenta,
            pdfborder={0 0 0}}
\urlstyle{same}  % don't use monospace font for urls
% Make links footnotes instead of hotlinks:
\renewcommand{\href}[2]{#2\footnote{\url{#1}}}
\setlength{\parindent}{0pt}
\setlength{\parskip}{6pt plus 2pt minus 1pt}
\setlength{\emergencystretch}{3em}  % prevent overfull lines
\setcounter{secnumdepth}{0}

\title{Practical Dependent Types in Haskell: Type-Safe Neural Networks (Part 2)}
\author{Justin Le}

\begin{document}
\maketitle

\emph{Originally posted on
\textbf{\href{https://blog.jle.im/entry/practical-dependent-types-in-haskell-2.html}{in
Code}}.}

We're back to continue on
\href{https://blog.jle.im/entries/series/+practical-dependent-types-in-haskell.html}{our
journey} in using practical dependent types to write type-safe neural networks!
In
\href{https://blog.jle.im/entry/practical-dependent-types-in-haskell-1.html}{Part
1}, we wrote things out in normal, untyped Haskell, and looked at red flags and
general design principles that nudged us in the direction of adding dependent
types to our program. We learned to appreciate what dependent types offered in
terms of guiding us in writing our code, helping the compiler check our
correctness, providing a better interface for users, and more.

We also learned how to use singletons to work around some of Haskell's
fundamental limitations to let us ``pattern match'' on the structure of types,
and how to use typeclasses to generate singletons reflecting the structure of
types we are dealing with.

(If you read
\href{https://blog.jle.im/entry/practical-dependent-types-in-haskell-1.html}{Part
1} \emph{before} the singletons section was re-written to use the
\href{https://hackage.haskell.org/package/singletons}{singletons} library,
\href{https://blog.jle.im/entry/practical-dependent-types-in-haskell-1.html\#singletons-and-induction}{here's
a link to the section} in specific. This tutorial will assume familiarity with
what is discussed there!)

All of what we've dealt with so far has essentially been with types that are
fixed at compile-time. All the networks we've made have had ``static'' types,
with their sizes in their types indicated directly in the source code.

Today, we're going to dive into the world of types that \emph{depend} on factors
unknown until runtime, and see how dependent types in a strongly typed language
like Haskell helps us write safer, more correct, and more maintainable code.
Along the way, we'll encounter and learn first-hand about techniques and guiding
high-level principles that we can apply to our other dependently typed coding
endeavours.

\section{Serializing Networks}\label{serializing-networks}

Just to warm up, let's talk about serializing networks: writing them to binary
and re-reading them.

\subsection{Recap on the Binary Library}\label{recap-on-the-binary-library}

Serializing networks of \emph{known} size --- whose sizes are statically in
their types --- is pretty straightforward. I'm going to be using the
\emph{\href{https://hackage.haskell.org/package/binary}{binary}} library, which
offers a very standard typeclass-based approach for serializing and
deserializing data. There are a lot of tutorials online (and I even
\href{https://blog.jle.im/entry/streaming-huffman-compression-in-haskell-part-2-binary.html}{wrote
a small one} myself a few years ago), but a very high-level view is that the
library offers a typeclass for describing serialization schemes for different
types.

For example, for serializing lists, you have:

\begin{Shaded}
\begin{Highlighting}[]
\OtherTok{putList ::} \DataTypeTok{Binary} \NormalTok{a }\OtherTok{=>} \NormalTok{[a] }\OtherTok{->} \DataTypeTok{Put}
\NormalTok{putList []     }\FunctionTok{=}
    \NormalTok{put }\DataTypeTok{False}           \CommentTok{-- signal the end}
\NormalTok{putList (x}\FunctionTok{:}\NormalTok{xs) }\FunctionTok{=} \KeywordTok{do}
    \NormalTok{put }\DataTypeTok{True}            \CommentTok{-- signal a cons}
    \NormalTok{put x}
    \NormalTok{putList xs}
\end{Highlighting}
\end{Shaded}

Where \texttt{put\ ::\ Binary\ a\ =\textgreater{}\ a\ -\textgreater{}\ Put} is a
polymorphic way of describing serialization of things with a \texttt{Binary}
instance --- every type provides its own \texttt{put}. Sequencing \texttt{Put}s
using do notation is saying ``put this, then that, then that''.

We can deserialize by describing a monadic \texttt{Get} plan:

\begin{Shaded}
\begin{Highlighting}[]
\OtherTok{getList ::} \DataTypeTok{Binary} \NormalTok{a }\OtherTok{=>} \DataTypeTok{Get} \NormalTok{[a]}
\NormalTok{getList }\FunctionTok{=} \KeywordTok{do}
    \NormalTok{isCons }\OtherTok{<-} \NormalTok{get}
    \KeywordTok{if} \NormalTok{isCons}
      \KeywordTok{then} \KeywordTok{do}
        \NormalTok{x  }\OtherTok{<-} \NormalTok{get}
        \NormalTok{xs }\OtherTok{<-} \NormalTok{getList}
        \NormalTok{return (x}\FunctionTok{:}\NormalTok{xs)}
      \KeywordTok{else}
        \NormalTok{return []}
\end{Highlighting}
\end{Shaded}

And \texttt{get\ ::\ Binary\ a\ =\textgreater{}\ Get\ a} is a polymorphic way of
describing a strategy of deserializing things with a \texttt{Binary} instance
--- again, every type provides its own \texttt{get}. Sequencing \texttt{Get}s is
again simply ``do-this-then-that''.

We can write our own instances for our own types by providing \texttt{put} and
\texttt{get} together:

\begin{Shaded}
\begin{Highlighting}[]
\KeywordTok{instance} \DataTypeTok{Binary} \NormalTok{a }\OtherTok{=>} \DataTypeTok{Binary} \NormalTok{[a] }\KeywordTok{where}
    \NormalTok{put }\FunctionTok{=} \NormalTok{putList}
    \NormalTok{get }\FunctionTok{=} \NormalTok{getList}
\end{Highlighting}
\end{Shaded}

In practice, we usually don't write our own instances from scratch, and use
GHC's generics features to give us instances for free:

\begin{Shaded}
\begin{Highlighting}[]
\CommentTok{-- source: https://github.com/mstksg/inCode/tree/master/code-samples/dependent-haskell/NetworkTyped2.hs#L20-36}
\KeywordTok{data} \DataTypeTok{Weights} \NormalTok{i o }\FunctionTok{=} \DataTypeTok{W} \NormalTok{\{}\OtherTok{ wBiases ::} \FunctionTok{!}\NormalTok{(}\DataTypeTok{R} \NormalTok{o)}
                     \NormalTok{,}\OtherTok{ wNodes  ::} \FunctionTok{!}\NormalTok{(}\DataTypeTok{L} \NormalTok{o i)}
                     \NormalTok{\}}
  \KeywordTok{deriving} \NormalTok{(}\DataTypeTok{Show}\NormalTok{, }\DataTypeTok{Generic}\NormalTok{)}

\KeywordTok{instance} \NormalTok{(}\DataTypeTok{KnownNat} \NormalTok{i, }\DataTypeTok{KnownNat} \NormalTok{o) }\OtherTok{=>} \DataTypeTok{Binary} \NormalTok{(}\DataTypeTok{Weights} \NormalTok{i o)}
\end{Highlighting}
\end{Shaded}

For simple types like \texttt{Weights}, which simply ``contain'' serializable
things, the \emph{binary} is smart enough to write your instances automatically!

\subsection{\texorpdfstring{\texttt{Binary} for
\texttt{Network}}{Binary for Network}}\label{binary-for-network}

Writing \texttt{putNet} and \texttt{getNet} is pretty nice because the entire
structure is already known ahead of time, and we don't need to do any tricks
with flags like for lists.

\begin{Shaded}
\begin{Highlighting}[]
\CommentTok{-- source: https://github.com/mstksg/inCode/tree/master/code-samples/dependent-haskell/NetworkTyped2.hs#L57-61}
\OtherTok{putNet ::} \NormalTok{(}\DataTypeTok{KnownNat} \NormalTok{i, }\DataTypeTok{KnownNat} \NormalTok{o)}
       \OtherTok{=>} \DataTypeTok{Network} \NormalTok{i hs o}
       \OtherTok{->} \DataTypeTok{Put}
\NormalTok{putNet }\FunctionTok{=} \NormalTok{\textbackslash{}}\KeywordTok{case} \DataTypeTok{O} \NormalTok{w     }\OtherTok{->} \NormalTok{put w}
               \NormalTok{w }\FunctionTok{:&~} \NormalTok{n }\OtherTok{->} \NormalTok{put w }\FunctionTok{*>} \NormalTok{putNet n}
\end{Highlighting}
\end{Shaded}

Even simpler than for lists! If it's an \texttt{O\ w}, just serialize the
\texttt{w}. If it's a \texttt{w\ :\&\textasciitilde{}\ net}, serialize the
\texttt{w} then the rest of the \texttt{net}. The reason we can get away without
any flags is because we already \emph{know} how many
\texttt{:\&\textasciitilde{}} layers to expect \emph{just from the type}. If we
want to deserialize/load a
\texttt{Network\ 5\ \textquotesingle{}{[}10,6,3{]}\ 2}, we \emph{know} we want
three \texttt{(:\&\textasciitilde{})}'s and one \texttt{O} --- no need for
dynamically sized networks like we had to handle for lists.

We'll write \texttt{getNet} the similarly to how wrote
\href{https://github.com/mstksg/inCode/tree/master/code-samples/dependent-haskell/NetworkTyped.hs\#L67-75}{\texttt{randomNet}}
from the last post:

\begin{Shaded}
\begin{Highlighting}[]
\CommentTok{-- source: https://github.com/mstksg/inCode/tree/master/code-samples/dependent-haskell/NetworkTyped2.hs#L63-67}
\OtherTok{getNet ::} \NormalTok{forall i hs o}\FunctionTok{.} \NormalTok{(}\DataTypeTok{KnownNat} \NormalTok{i, }\DataTypeTok{KnownNat} \NormalTok{o)}
       \OtherTok{=>} \DataTypeTok{Sing} \NormalTok{hs}
       \OtherTok{->} \DataTypeTok{Get} \NormalTok{(}\DataTypeTok{Network} \NormalTok{i hs o)}
\NormalTok{getNet }\FunctionTok{=} \NormalTok{\textbackslash{}}\KeywordTok{case} \DataTypeTok{SNil}            \OtherTok{->}     \DataTypeTok{O} \FunctionTok{<$>} \NormalTok{get}
               \DataTypeTok{SNat} \OtherTok{`SCons`} \NormalTok{ss }\OtherTok{->} \NormalTok{(}\FunctionTok{:&~}\NormalTok{) }\FunctionTok{<$>} \NormalTok{get }\FunctionTok{<*>} \NormalTok{getNet ss}
\end{Highlighting}
\end{Shaded}

Now would be a good time to refresh on the
\href{https://blog.jle.im/entry/practical-dependent-types-in-haskell-1.html\#singletons-and-induction}{singletons
section} of my last post again. To deserialize a \texttt{Network\ i\ hs\ o}, we
have to ``pattern match'' on \texttt{hs} to see what constructor we are
expecting to deserialize, and we do that by using singletons with constructors
we \emph{can} literally pattern match on (\texttt{SNil} and \texttt{SCons}),
which tell GHC what type to expect through ``dependent pattern matching''.

If you see a \texttt{SNil\ ::\ Sing\ \textquotesingle{}{[}{]}}, that means that
\texttt{hs} is \texttt{\textquotesingle{}{[}{]}}, so expect a \texttt{O}
constructor. If you see
\texttt{SCons\ s\ ss\ ::\ Sing\ (h\ \textquotesingle{}:\ hs)}, that means that
\texttt{hs} is \texttt{h\ \textquotesingle{}:\ h\textquotesingle{}}, so expect a
\texttt{(:\&\textasciitilde{})} constructor. (If this is a little confusing
still, try re-reading the
\href{https://blog.jle.im/entry/practical-dependent-types-in-haskell-1.html\#singletons-and-induction}{singleton
section} from the last post again for a more thorough description!)

Note that here we decide to implement \texttt{getNet} by asking for an explicit
singleton input (\texttt{Sing\ hs\ -\textgreater{}}) instead of an implicit one
(\texttt{SingI\ hs\ =\textgreater{}}) like we did for \texttt{randomNet}.
Remember that the two methods are technically equivalent, really, and compile to
the same thing at runtime. We need one or the other because of type erasure ---
so either we pass in \texttt{Sing\ hs}, or provide a \texttt{SingI\ hs}
constraint so that we can use \texttt{sing\ ::\ Sing\ hs} to construct the
\texttt{Sing\ hs}.

There's a trade-off either way, and it can be a bit annoying because switching
between different modes can potentially be diverse. For the most part, always
try to \emph{take explicit singleton arguments} where you can, especially for
internal functions. The simple reason is because in Haskell, we like to really
only do typeclass-level programming as a last, last resort. Typeclasses in
Haskell are very magical and not really first-class in a satisfying way. Normal
values (like singletons) \emph{are} first-class and easily passed.

Explicit singleton arguments can sometimes pose a burden for the caller, so my
personal approach is to \emph{always} use explicit \texttt{Sing\ a} whenever
possible for \emph{internal functions}, and to expose a
\texttt{SingI\ a\ =\textgreater{}} interface for \emph{user-facing functions}
(including for typeclass instances)

So, we're going to write our \texttt{Binary} instance for \texttt{Network}. Of
course, we can't have \texttt{put} or \texttt{get} take a \texttt{Sing\ hs}
(that'd change the arity/type of the function), so what we can do is have their
\texttt{Binary} instances require a \texttt{SingI\ hs} constraint, essentially
doing the same thing:

\begin{Shaded}
\begin{Highlighting}[]
\CommentTok{-- source: https://github.com/mstksg/inCode/tree/master/code-samples/dependent-haskell/NetworkTyped2.hs#L69-71}
\KeywordTok{instance} \NormalTok{(}\DataTypeTok{KnownNat} \NormalTok{i, }\DataTypeTok{SingI} \NormalTok{hs, }\DataTypeTok{KnownNat} \NormalTok{o) }\OtherTok{=>} \DataTypeTok{Binary} \NormalTok{(}\DataTypeTok{Network} \NormalTok{i hs o) }\KeywordTok{where}
    \NormalTok{put }\FunctionTok{=} \NormalTok{putNet}
    \NormalTok{get }\FunctionTok{=} \NormalTok{getNet sing}
\end{Highlighting}
\end{Shaded}

To go from ``\texttt{SingI} world'' to ``\texttt{Sing} world'', we use
\texttt{sing} to generate the explicit \texttt{Sing\ hs} from
\texttt{SingI\ hs\ =\textgreater{}}.

\section{Existential Crisis}\label{existential-crisis}

Now, having the entire structure of your neural network in the type is nice and
all for cool tricks like \texttt{randomNet}\ldots{}but do you \emph{really} want
to work with this directly? After all, from the user's perspective, the user
really only ever needs to know \texttt{i} and \texttt{o}: What vectors the
network \emph{expects}, and what vectors the network \emph{outputs}. In the end,
all a (feed-forward) Neural Network really is is an abstraction over a function
\texttt{R\ i\ -\textgreater{}\ R\ o}.

Remember, the main benefits of having the entire structure in the type was to
help us \emph{implement} our functions more safely, with the compiler's help,
and also for cute return type polymorphism tricks like \texttt{randomNet} and
\texttt{getNet}. The \emph{first} type of benefit really doesn't benefit the
\emph{user} of the network. So let's talk about a way to ``abstract'' away the
internal structure of hidden nodes from the type, and maybe even have it depend
on runtime values!

The big key to ``hiding'' parts of types and letting them depend on runtime
values is called the \emph{existential type}. Existential types are sort of the
``opposite'' of the normal polymorphic (universally quantified) types you
normally see in Haskell.

For a function like

\begin{Shaded}
\begin{Highlighting}[]
\NormalTok{map}\OtherTok{ ::} \NormalTok{(a }\OtherTok{->} \NormalTok{b) }\OtherTok{->} \NormalTok{[a] }\OtherTok{->} \NormalTok{[b]}
\end{Highlighting}
\end{Shaded}

\texttt{a} and \texttt{b} are universally quantified, which means that the
person who \emph{uses} \texttt{map} gets to \emph{decide} what \texttt{a} and
\texttt{b} are. To be more explicit, that type signature can be written as:

\begin{Shaded}
\begin{Highlighting}[]
\NormalTok{map}\OtherTok{ ::} \NormalTok{forall a b}\FunctionTok{.} \NormalTok{(a }\OtherTok{->} \NormalTok{b) }\OtherTok{->} \NormalTok{[a] }\OtherTok{->} \NormalTok{[b]}
\end{Highlighting}
\end{Shaded}

For a universally quantified type, the \emph{caller} gets to decide what is
what\ldots{}and the function has to \emph{adapt} to handle it.

For the \emph{existentially} quantified type, the \emph{function} gets to decide
what is what, and the \emph{caller} has to adapt to handle it.

There are two main ways to work with existential types in Haskell, and we'll go
over both of them now and talk about their relative strengths and weaknesses and
what situations to use either of them in. It pays to be aware of both!

\subsection{Existential Data Type}\label{existential-data-type}

Arguably the more natural way in Haskell to work with existential types is to
wrap them in a data type:

\begin{Shaded}
\begin{Highlighting}[]
\CommentTok{-- source: https://github.com/mstksg/inCode/tree/master/code-samples/dependent-haskell/NetworkTyped2.hs#L73-74}
\KeywordTok{data} \DataTypeTok{OpaqueNet}\OtherTok{ ::} \DataTypeTok{Nat} \OtherTok{->} \DataTypeTok{Nat} \OtherTok{->} \FunctionTok{*} \KeywordTok{where}
    \DataTypeTok{ONet}\OtherTok{ ::} \DataTypeTok{Sing} \NormalTok{hs }\OtherTok{->} \DataTypeTok{Network} \NormalTok{i hs o }\OtherTok{->} \DataTypeTok{OpaqueNet} \NormalTok{i o}
\end{Highlighting}
\end{Shaded}

So, if you have \texttt{net\ ::\ Network\ 6\ \textquotesingle{}{[}10,6,3{]}\ 2},
you can create \texttt{ONet\ sing\ net\ ::\ OpaqueNet\ 6\ 2}. When you use the
\texttt{ONet} constructor, the structure of the hidden layers disappears from
the type!

How do we use this type? When we \emph{pattern match} on \texttt{ONet}, we get
the singleton and the net back!

\begin{Shaded}
\begin{Highlighting}[]
\CommentTok{-- source: https://github.com/mstksg/inCode/tree/master/code-samples/dependent-haskell/NetworkTyped2.hs#L76-81}
\OtherTok{numHiddens ::} \DataTypeTok{OpaqueNet} \NormalTok{i o }\OtherTok{->} \DataTypeTok{Int}
\NormalTok{numHiddens }\FunctionTok{=} \NormalTok{\textbackslash{}}\KeywordTok{case} \DataTypeTok{ONet} \NormalTok{ss _ }\OtherTok{->} \NormalTok{lengthSing ss}
  \KeywordTok{where}
\OtherTok{    lengthSing ::} \DataTypeTok{Sing} \NormalTok{(}\OtherTok{hs ::} \NormalTok{[}\DataTypeTok{Nat}\NormalTok{]) }\OtherTok{->} \DataTypeTok{Int}
    \NormalTok{lengthSing }\FunctionTok{=} \NormalTok{\textbackslash{}}\KeywordTok{case} \DataTypeTok{SNil}         \OtherTok{->} \DecValTok{0}
                       \NormalTok{_ }\OtherTok{`SCons`} \NormalTok{ss }\OtherTok{->} \DecValTok{1} \FunctionTok{+} \NormalTok{lengthSing ss}
\end{Highlighting}
\end{Shaded}

Note that it's important for us to stuff in the singleton in addition to the
network itself, because of type erasure. If we didn't pop the singleton in,
there'd be no way for us to recover the original \texttt{hs}! (Note that we
could have had
\texttt{ONet\ ::\ SingI\ hs\ =\textgreater{}\ Network\ i\ hs\ o\ -\textgreater{}\ OpaqueNet\ i\ o},
which is essentially the same thing)

Once you \emph{do} pattern match on \texttt{ONet}, you have to handle the
\texttt{hs} in a \emph{completely polymorphic way}. You're not allowed to assume
anything about \texttt{hs}\ldots{}you have to provide a completely
parametrically polymorphic way of dealing with it!

Note that this function is completely not ok:

\begin{Shaded}
\begin{Highlighting}[]
\OtherTok{bad ::} \DataTypeTok{OpaqueNet} \NormalTok{i o }\OtherTok{->} \DataTypeTok{Network} \NormalTok{i hs o}
\NormalTok{bad }\FunctionTok{=} \NormalTok{\textbackslash{}}\KeywordTok{case} \DataTypeTok{ONet} \NormalTok{_ n }\OtherTok{->} \NormalTok{n}
\end{Highlighting}
\end{Shaded}

Why not? Because a type signature like
\texttt{OpaqueNet\ i\ o\ -\textgreater{}\ Network\ i\ hs\ o} means that the
\emph{caller} can decide what \texttt{hs} can be --- just like
\texttt{read\ ::\ Read\ a\ =\textgreater{}\ String\ -\textgreater{}\ a}, where
the caller decides what \texttt{a} is.

Of course, this \emph{isn't} the case with the way we've written the
function\ldots{}the function only returns a \emph{specific} \texttt{hs} that the
\emph{function} decides. The \emph{caller} has to accommodate whatever is inside
\texttt{ONet}.

\subsubsection{Binary}\label{binary}

Now, let's find out a way to serialize this type! Now, the structure of our
network is \emph{not} known in the type, so we do have to plant flags in our
data somehow. We need to store a witness to the structure of the network, as
well.

To do this, we can move \texttt{hs} from the type level to the value level. In
Haskell, this is called \textbf{reflection}. The \emph{singletons} library
provides the \texttt{fromSing} function for this purpose:

\begin{Shaded}
\begin{Highlighting}[]
\NormalTok{ghci}\FunctionTok{>} \NormalTok{fromSing (}\OtherTok{sing ::} \DataTypeTok{Sing} \CharTok{'[1,2,3])}
\NormalTok{[}\DecValTok{1}\NormalTok{,}\DecValTok{2}\NormalTok{,}\DecValTok{3}\NormalTok{]}
\NormalTok{ghci}\FunctionTok{>} \NormalTok{fromSing (}\OtherTok{sing ::} \DataTypeTok{Sing} \CharTok{'[True, False])}
\NormalTok{[}\DataTypeTok{True}\NormalTok{, }\DataTypeTok{False}\NormalTok{]}
\end{Highlighting}
\end{Shaded}

And with that, we can write a serializer for \texttt{OpaqueNet}:

\begin{Shaded}
\begin{Highlighting}[]
\CommentTok{-- source: https://github.com/mstksg/inCode/tree/master/code-samples/dependent-haskell/NetworkTyped2.hs#L83-88}
\OtherTok{putONet ::} \NormalTok{(}\DataTypeTok{KnownNat} \NormalTok{i, }\DataTypeTok{KnownNat} \NormalTok{o)}
        \OtherTok{=>} \DataTypeTok{OpaqueNet} \NormalTok{i o}
        \OtherTok{->} \DataTypeTok{Put}
\NormalTok{putONet }\FunctionTok{=} \NormalTok{\textbackslash{}}\KeywordTok{case} \DataTypeTok{ONet} \NormalTok{ss net }\OtherTok{->} \KeywordTok{do}
                  \NormalTok{put (fromSing ss)}
                  \NormalTok{putNet net}
\end{Highlighting}
\end{Shaded}

Put the structure (as a flag), and then put the network itself.

Now, to deserialize, we want to \emph{load} the list of \texttt{Integer}s and
move that \emph{back} into the type level. In Haskell, this is called
\textbf{reification}, the dual of reflection.

The \emph{singletons} library provides the \texttt{toSing} function, which
returns a \texttt{SomeSing} (an existentially quantified \texttt{Sing} wrapped
in a constructor that we can pattern match on):

\begin{Shaded}
\begin{Highlighting}[]
\CommentTok{-- source: https://github.com/mstksg/inCode/tree/master/code-samples/dependent-haskell/NetworkTyped2.hs#L90-97}
\OtherTok{getONet ::} \NormalTok{(}\DataTypeTok{KnownNat} \NormalTok{i, }\DataTypeTok{KnownNat} \NormalTok{o)}
        \OtherTok{=>} \DataTypeTok{Get} \NormalTok{(}\DataTypeTok{OpaqueNet} \NormalTok{i o)}
\NormalTok{getONet }\FunctionTok{=} \KeywordTok{do}
    \NormalTok{hs }\OtherTok{<-} \NormalTok{get}
    \KeywordTok{case} \NormalTok{toSing hs }\KeywordTok{of}
      \DataTypeTok{SomeSing} \NormalTok{ss }\OtherTok{->} \KeywordTok{do}
        \NormalTok{n }\OtherTok{<-} \NormalTok{getNet ss}
        \NormalTok{return (}\DataTypeTok{ONet} \NormalTok{ss n)}
\end{Highlighting}
\end{Shaded}

We first \texttt{get} the \texttt{{[}Integer{]}}, then \emph{reify} the list of
integers into the type level by getting our \texttt{ss\ ::\ Sing\ hs}. Then we
\texttt{getNet\ ss}, remembering that \texttt{getNet} takes a singleton to
figure out what structure to get. Then we wrap it all up in the \texttt{ONet}
constructor.

Phew! We load our flag, reify it, and once we're back in the typed land again,
we can do our normal business!

\begin{Shaded}
\begin{Highlighting}[]
\CommentTok{-- source: https://github.com/mstksg/inCode/tree/master/code-samples/dependent-haskell/NetworkTyped2.hs#L107-109}
\KeywordTok{instance} \NormalTok{(}\DataTypeTok{KnownNat} \NormalTok{i, }\DataTypeTok{KnownNat} \NormalTok{o) }\OtherTok{=>} \DataTypeTok{Binary} \NormalTok{(}\DataTypeTok{OpaqueNet} \NormalTok{i o) }\KeywordTok{where}
    \NormalTok{put }\FunctionTok{=} \NormalTok{putONet}
    \NormalTok{get }\FunctionTok{=} \NormalTok{getONet}
\end{Highlighting}
\end{Shaded}

\subsubsection{The Boundary}\label{the-boundary}

Did you notice what we just did there? The \emph{type} of \texttt{Sing\ hs} was
\emph{dynamically generated} based on the \emph{value} of the
\texttt{{[}Integer{]}} we load at runtime. We just worked with types that
\emph{depended} on runtime values.

With the power of existentially quantified types (like in \texttt{SomeSing}), we
essentially gained the ability to work with types that depend on runtime
results.

In a way, you can consider the \texttt{toSing} and the \texttt{SomeSing} as our
``boundary'' between the ``untyped world'' and the ``typed world''. This layer
(and the process of reification) cleanly separates the two.

This ``boundary'' can be thought of as a lot like the boundary we talk about
between ``pure'' functions and values and ``impure'' (IO, etc.) ones. We say to
always write as much of your program as possible in the ``pure'' world, and to
separate and pull out as much logic as you can to be pure logic. That's sort of
one of the first things you learn about as a Haskell programmer: how to separate
logic that \emph{can} be pure from logic that is ``impure'' (IO, etc.), and then
``combine them'' at the very end, as late as possible.

Well, if the final program is going to be IO in the end anyway, why bother
separating out pure and impure parts of your logic? Separation of concerns, the
increased ability to reason with your code and analyze what it does, the
compiler's ability to check what you write, the limitation of implementations,
and etc. are all reasons any Haskeller should be familiar with reciting.

You can think of the general philosophy of working with typed/untyped worlds as
being the same thing. You can write as much of your program as possible in the
``typed'' world, like we did in Part 1. Take advantage of the increased ability
to reason with your code, parametric polymorphism helping you \emph{write} your
code, limit your implementations, nab you compiler help, etc. All of those are
benefits of working in the typed world.

Then, write what you must in your ``untyped'' world, such as dealing with values
that pop up at runtime like the \texttt{{[}Integer{]}} above.

Finally, at the end, \emph{unite} them at the boundary. Pass the control
football from the untyped world to the typed world!

\subsubsection{Reifying for Fun and Profit}\label{reifying-for-fun-and-profit}

Before we move on, let's see a more direct example of reification: creating
networks with variable internal structure based on runtime input, like
configuration files.

Maybe you're reading the internal size from a configuration file, or maybe you
want it to be determined by user input. Our old \texttt{randomNet} won't work:

\begin{Shaded}
\begin{Highlighting}[]
\OtherTok{randomNet ::} \NormalTok{(}\DataTypeTok{MonadRandom} \NormalTok{m, }\DataTypeTok{SingI} \NormalTok{hs)}
          \OtherTok{=>} \NormalTok{m (}\DataTypeTok{Network} \NormalTok{i hs o)}
\end{Highlighting}
\end{Shaded}

Because we need a static type signature to use it directly. But we can return an
\texttt{OpaqueNet}!

\begin{Shaded}
\begin{Highlighting}[]
\CommentTok{-- source: https://github.com/mstksg/inCode/tree/master/code-samples/dependent-haskell/NetworkTyped2.hs#L99-105}
\OtherTok{randomONet ::} \NormalTok{(}\DataTypeTok{MonadRandom} \NormalTok{m, }\DataTypeTok{KnownNat} \NormalTok{i, }\DataTypeTok{KnownNat} \NormalTok{o)}
           \OtherTok{=>} \NormalTok{[}\DataTypeTok{Integer}\NormalTok{]}
           \OtherTok{->} \NormalTok{m (}\DataTypeTok{OpaqueNet} \NormalTok{i o)}
\NormalTok{randomONet hs }\FunctionTok{=} \KeywordTok{case} \NormalTok{toSing hs }\KeywordTok{of}
                  \DataTypeTok{SomeSing} \NormalTok{ss }\OtherTok{->} \KeywordTok{case} \NormalTok{singInstance ss }\KeywordTok{of}
                    \DataTypeTok{SingInstance} \OtherTok{->}
                      \DataTypeTok{ONet} \NormalTok{ss }\FunctionTok{<$>} \NormalTok{randomNet}
\end{Highlighting}
\end{Shaded}

Note that the implementation is slightly awkward because we had the lack of
foresight to implement \texttt{randomNet} using
\texttt{SingI\ hs\ =\textgreater{}} instead of
\texttt{Sing\ hs\ -\textgreater{}}, so we have to ``go from \texttt{Sing} to
\texttt{SingI}''.

You go from the \texttt{SingI} style to the \texttt{Sing} style with
\texttt{sing}, like we saw earlier, and you can go backwards with
\texttt{singInstance}:

\begin{Shaded}
\begin{Highlighting}[]
\OtherTok{singInstance ::} \DataTypeTok{Sing} \NormalTok{a }\OtherTok{->} \DataTypeTok{SingInstance} \NormalTok{a}
\end{Highlighting}
\end{Shaded}

Where the data type \texttt{SingInstance} has a single constructor that's a lot
like \texttt{SNat} from the last part:

\begin{Shaded}
\begin{Highlighting}[]
\KeywordTok{data} \DataTypeTok{SingInstance} \NormalTok{a }\KeywordTok{where}
    \DataTypeTok{SingInstance}\OtherTok{ ::} \DataTypeTok{SingI} \NormalTok{a }\OtherTok{=>} \DataTypeTok{SingInstance} \NormalTok{a}
\end{Highlighting}
\end{Shaded}

Basically, it's impossible to \emph{use} the \texttt{SingInstance} constructor
unless you have a \texttt{SingI\ a} instance in scope, so if you ever
\emph{pattern match} on it, it's a ``proof'' that \texttt{SingI\ a} exists. It's
the same as how, for the constructor
\texttt{SNat\ ::\ KnownNat\ n\ =\textgreater{}\ Sing\ n}, if you pattern match
on \texttt{SNat} and see that it's not something silly like
\texttt{undefined}/\texttt{error}/bottom, GHC knows that there is a
\texttt{KnownNat\ n} instance. It's sort of like pattern matching out the
instance itself.

Now you can get from your untyped world into the world of dependent types ---

\begin{Shaded}
\begin{Highlighting}[]
\CommentTok{-- source: https://github.com/mstksg/inCode/tree/master/code-samples/dependent-haskell/NetworkTyped2.hs#L150-156}
\OtherTok{main ::} \DataTypeTok{IO} \NormalTok{()}
\NormalTok{main }\FunctionTok{=} \KeywordTok{do}
    \NormalTok{putStrLn }\StringTok{"What size random net?"}
    \NormalTok{hs }\OtherTok{<-} \NormalTok{readLn}
    \DataTypeTok{ONet} \NormalTok{ss (}\OtherTok{net ::} \DataTypeTok{Network} \DecValTok{10} \NormalTok{hs }\DecValTok{3}\NormalTok{) }\OtherTok{<-} \NormalTok{randomONet hs}
    \NormalTok{print net}
    \CommentTok{-- blah blah stuff with our dynamically generated net}
\end{Highlighting}
\end{Shaded}

\subsection{Continuation-Based
Existentials}\label{continuation-based-existentials}

There's another way in Haskell that we work with existential types that can be
more natural and easy to work with in a lot of cases.

Note that when we pattern match on an existential data type, you have to work
with the values in the constructor in a parametrically polymorphic way. For
example, if we had:

\begin{Shaded}
\begin{Highlighting}[]
\OtherTok{oNetToFoo ::} \DataTypeTok{OpaqueNet} \NormalTok{i o }\OtherTok{->} \DataTypeTok{Foo}
\NormalTok{oNetToFoo }\FunctionTok{=} \NormalTok{\textbackslash{}}\KeywordTok{case} \DataTypeTok{ONet} \NormalTok{s n }\OtherTok{->} \NormalTok{f s n}
\end{Highlighting}
\end{Shaded}

What does the type of \texttt{f} have to be? It has to take a \texttt{Sing\ hs}
and a \texttt{Network\ i\ hs\ o}, but deal with it in a way that works \emph{for
all} \texttt{hs}. It has to be:

\begin{Shaded}
\begin{Highlighting}[]
\OtherTok{f ::} \NormalTok{forall (}\OtherTok{hs ::} \NormalTok{[}\DataTypeTok{Nat}\NormalTok{])}\FunctionTok{.} \DataTypeTok{Sing} \NormalTok{hs }\OtherTok{->} \DataTypeTok{Network} \NormalTok{i hs o }\OtherTok{->} \DataTypeTok{Foo}
\end{Highlighting}
\end{Shaded}

That is, it can't be written for \emph{only}
\texttt{Sing\ \textquotesingle{}{[}5{]}} or \emph{only}
\texttt{Sing\ \textquotesingle{}{[}6,3{]}}\ldots{}it has to work for \emph{any}
\texttt{hs}.

Well, we could really also just skip the data type together and represent an
existential type as something \emph{taking} the continuation \texttt{f} and
giving it what it needs.

\begin{Shaded}
\begin{Highlighting}[]
\CommentTok{-- source: https://github.com/mstksg/inCode/tree/master/code-samples/dependent-haskell/NetworkTyped2.hs#L111-111}
\KeywordTok{type} \DataTypeTok{OpaqueNet'} \NormalTok{i o r }\FunctionTok{=} \NormalTok{(forall hs}\FunctionTok{.} \DataTypeTok{Sing} \NormalTok{hs }\OtherTok{->} \DataTypeTok{Network} \NormalTok{i hs o }\OtherTok{->} \NormalTok{r) }\OtherTok{->} \NormalTok{r}
\end{Highlighting}
\end{Shaded}

``Tell me how you would make an \texttt{r} if you had a \texttt{Sing\ hs} and a
\texttt{Network\ i\ hs\ o}, and I'll make it for you!''

This ``continuation transformation'' is known as formally
\textbf{skolemization}.\footnote{Skolemization is probably one of the coolest
  words you'll encounter working with dependent types in Haskell, and sometimes
  just knowing that you're ``skolemizing'' something makes you feel cooler.
  Thank you \href{https://en.wikipedia.org/wiki/Thoralf_Skolem}{Thoralf Skolem}.
  If you ever see a ``rigid, skolem'' error in GHC, you can thank him for that
  too!} We can ``wrap'' a \texttt{Network\ i\ hs\ o} into an
\texttt{OpaqueNet\textquotesingle{}\ i\ o\ r} pretty straightforwardly:

\begin{Shaded}
\begin{Highlighting}[]
\CommentTok{-- source: https://github.com/mstksg/inCode/tree/master/code-samples/dependent-haskell/NetworkTyped2.hs#L113-114}
\OtherTok{oNet' ::} \DataTypeTok{Sing} \NormalTok{hs }\OtherTok{->} \DataTypeTok{Network} \NormalTok{i hs o }\OtherTok{->} \DataTypeTok{OpaqueNet'} \NormalTok{i o r}
\NormalTok{oNet' s n }\FunctionTok{=} \NormalTok{\textbackslash{}f }\OtherTok{->} \NormalTok{f s n}
\end{Highlighting}
\end{Shaded}

To prove that the two \texttt{OpaqueNet}s are the same (and to help us see more
about how they relate), we can write functions that convert back and forth from
them:

\begin{Shaded}
\begin{Highlighting}[]
\CommentTok{-- source: https://github.com/mstksg/inCode/tree/master/code-samples/dependent-haskell/NetworkTyped2.hs#L116-121}
\CommentTok{-- withONet :: OpaqueNet i o -> (forall hs. Sing hs -> Network i hs o -> r) -> r}
\OtherTok{withONet ::} \DataTypeTok{OpaqueNet} \NormalTok{i o }\OtherTok{->} \DataTypeTok{OpaqueNet'} \NormalTok{i o r}
\NormalTok{withONet }\FunctionTok{=} \NormalTok{\textbackslash{}}\KeywordTok{case} \DataTypeTok{ONet} \NormalTok{s n }\OtherTok{->} \NormalTok{(\textbackslash{}f }\OtherTok{->} \NormalTok{f s n)}

\OtherTok{toONet ::} \DataTypeTok{OpaqueNet'} \NormalTok{i o (}\DataTypeTok{OpaqueNet} \NormalTok{i o) }\OtherTok{->} \DataTypeTok{OpaqueNet} \NormalTok{i o}
\NormalTok{toONet oN' }\FunctionTok{=} \NormalTok{oN' (\textbackslash{}s n }\OtherTok{->} \DataTypeTok{ONet} \NormalTok{s n)}
\end{Highlighting}
\end{Shaded}

Note the expanded type signature of \texttt{withONet}, which you can sort of
interpret as, ``do \emph{this function} on the existentially quantified contents
of an \texttt{OpaqueNet}.''

\subsubsection{Trying it out}\label{trying-it-out}

To sort of compare how the two methods look like in practice, we're going to
Rosetta stone it up and re-implement serialization with the continuation-based
existentials:

\begin{Shaded}
\begin{Highlighting}[]
\CommentTok{-- source: https://github.com/mstksg/inCode/tree/master/code-samples/dependent-haskell/NetworkTyped2.hs#L123-138}
\OtherTok{putONet' ::} \NormalTok{(}\DataTypeTok{KnownNat} \NormalTok{i, }\DataTypeTok{KnownNat} \NormalTok{o)}
         \OtherTok{=>} \DataTypeTok{OpaqueNet'} \NormalTok{i o }\DataTypeTok{Put}
         \OtherTok{->} \DataTypeTok{Put}
\NormalTok{putONet' oN }\FunctionTok{=} \NormalTok{oN }\FunctionTok{$} \NormalTok{\textbackslash{}ss net }\OtherTok{->} \KeywordTok{do}
                      \NormalTok{put (fromSing ss)}
                      \NormalTok{putNet net}

\OtherTok{getONet' ::} \NormalTok{(}\DataTypeTok{KnownNat} \NormalTok{i, }\DataTypeTok{KnownNat} \NormalTok{o)}
         \OtherTok{=>} \NormalTok{(forall hs}\FunctionTok{.} \DataTypeTok{Sing} \NormalTok{hs }\OtherTok{->} \DataTypeTok{Network} \NormalTok{i hs o }\OtherTok{->} \DataTypeTok{Get} \NormalTok{r)}
         \OtherTok{->} \DataTypeTok{Get} \NormalTok{r}
\CommentTok{--  aka, => OpaqueNet' i o (Get r)}
\NormalTok{getONet' f }\FunctionTok{=} \KeywordTok{do}
    \NormalTok{hs }\OtherTok{<-} \NormalTok{get}
    \NormalTok{withSomeSing (}\OtherTok{hs ::} \NormalTok{[}\DataTypeTok{Integer}\NormalTok{]) }\FunctionTok{$} \NormalTok{\textbackslash{}ss }\OtherTok{->} \KeywordTok{do}
      \NormalTok{n }\OtherTok{<-} \NormalTok{getNet ss}
      \NormalTok{f ss n}
\end{Highlighting}
\end{Shaded}

To be cute, I used the skolemized partners of \texttt{toSing} and
\texttt{SomeSing}:

\begin{Shaded}
\begin{Highlighting}[]
\OtherTok{withSomeSing ::} \NormalTok{[}\DataTypeTok{Integer}\NormalTok{]}
             \OtherTok{->} \NormalTok{(forall (}\OtherTok{hs ::} \NormalTok{[}\DataTypeTok{Nat}\NormalTok{])}\FunctionTok{.} \DataTypeTok{Sing} \NormalTok{hs }\OtherTok{->} \NormalTok{r)}
             \OtherTok{->} \NormalTok{r}
\end{Highlighting}
\end{Shaded}

Instead of returning a \texttt{SomeSing} like \texttt{toSing} does,
\texttt{withSomeSing} returns the continuation-based existential.

The expanded type signature of \texttt{getONet\textquotesingle{}} can be read:
``Give what you would do if you \emph{had} a \texttt{Sing\ hs} and a
\texttt{Network\ i\ hs\ o}'', and I'll get them for you and give you the
result."

Let's also see how we'd return a random network with a continuation:

\begin{Shaded}
\begin{Highlighting}[]
\CommentTok{-- source: https://github.com/mstksg/inCode/tree/master/code-samples/dependent-haskell/NetworkTyped2.hs#L140-148}
\OtherTok{withRandomONet' ::} \NormalTok{(}\DataTypeTok{MonadRandom} \NormalTok{m, }\DataTypeTok{KnownNat} \NormalTok{i, }\DataTypeTok{KnownNat} \NormalTok{o)}
                \OtherTok{=>} \NormalTok{[}\DataTypeTok{Integer}\NormalTok{]}
                \OtherTok{->} \NormalTok{(forall hs}\FunctionTok{.} \DataTypeTok{Sing} \NormalTok{hs }\OtherTok{->} \DataTypeTok{Network} \NormalTok{i hs o }\OtherTok{->} \NormalTok{m r)}
                \OtherTok{->} \NormalTok{m r}
\CommentTok{--         aka, => OpaqueNet' i o (m r)}
\NormalTok{withRandomONet' hs f }\FunctionTok{=} \NormalTok{withSomeSing hs }\FunctionTok{$} \NormalTok{\textbackslash{}ss }\OtherTok{->}
                       \NormalTok{withSingI ss    }\FunctionTok{$} \KeywordTok{do}
                         \NormalTok{net }\OtherTok{<-} \NormalTok{randomNet}
                         \NormalTok{f ss net}
\end{Highlighting}
\end{Shaded}

Again, to be cute, I used the continuation-based version of
\texttt{singInstance}, \texttt{withSingI}:

\begin{Shaded}
\begin{Highlighting}[]
\OtherTok{withSingI ::} \DataTypeTok{Sing} \NormalTok{a }\OtherTok{->} \NormalTok{(}\DataTypeTok{SingI} \NormalTok{a }\OtherTok{=>} \NormalTok{r) }\OtherTok{->} \NormalTok{r}
\end{Highlighting}
\end{Shaded}

The signature, in English, is ``give me a \texttt{Sing\ a} and a value that you
could make \emph{if only you had} a \texttt{SingI} instance, and I'll give you
that value as if you had the instance, magically!''

Of course, we know it's not magic:\footnote{Actually, it kind of \emph{is}
  magic, because \texttt{singInstance} is implemented with
  \texttt{unsafeCoerce}. But don't tell anyone ;)}

\begin{Shaded}
\begin{Highlighting}[]
\OtherTok{withSingI ::} \DataTypeTok{Sing} \NormalTok{a }\OtherTok{->} \NormalTok{(}\DataTypeTok{SingI} \NormalTok{a }\OtherTok{=>} \NormalTok{r) }\OtherTok{->} \NormalTok{r}
\NormalTok{withSingI s x }\FunctionTok{=} \KeywordTok{case} \NormalTok{singInstance s }\KeywordTok{of}
                  \DataTypeTok{SingInstance} \OtherTok{->} \NormalTok{x}
\end{Highlighting}
\end{Shaded}

And we see another way we can ``move past the untyped/typed boundary'':

\begin{Shaded}
\begin{Highlighting}[]
\CommentTok{-- source: https://github.com/mstksg/inCode/tree/master/code-samples/dependent-haskell/NetworkTyped2.hs#L158-164}
\OtherTok{main' ::} \DataTypeTok{IO} \NormalTok{()}
\NormalTok{main' }\FunctionTok{=} \KeywordTok{do}
    \NormalTok{putStrLn }\StringTok{"What size random net?"}
    \NormalTok{hs }\OtherTok{<-} \NormalTok{readLn}
    \NormalTok{withRandomONet' hs }\FunctionTok{$} \NormalTok{\textbackslash{}ss (}\OtherTok{net ::} \DataTypeTok{Network} \DecValTok{10} \NormalTok{hs }\DecValTok{3}\NormalTok{) }\OtherTok{->} \KeywordTok{do}
      \NormalTok{print net}
      \CommentTok{-- blah blah stuff with our dynamically generated net}
\end{Highlighting}
\end{Shaded}

\section{A Tale of Two Styles}\label{a-tale-of-two-styles}

\end{document}
