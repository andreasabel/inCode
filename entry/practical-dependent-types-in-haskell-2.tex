\documentclass[]{article}
\usepackage{lmodern}
\usepackage{amssymb,amsmath}
\usepackage{ifxetex,ifluatex}
\usepackage{fixltx2e} % provides \textsubscript
\ifnum 0\ifxetex 1\fi\ifluatex 1\fi=0 % if pdftex
  \usepackage[T1]{fontenc}
  \usepackage[utf8]{inputenc}
\else % if luatex or xelatex
  \ifxetex
    \usepackage{mathspec}
    \usepackage{xltxtra,xunicode}
  \else
    \usepackage{fontspec}
  \fi
  \defaultfontfeatures{Mapping=tex-text,Scale=MatchLowercase}
  \newcommand{\euro}{€}
\fi
% use upquote if available, for straight quotes in verbatim environments
\IfFileExists{upquote.sty}{\usepackage{upquote}}{}
% use microtype if available
\IfFileExists{microtype.sty}{\usepackage{microtype}}{}
\usepackage[margin=1in]{geometry}
\ifxetex
  \usepackage[setpagesize=false, % page size defined by xetex
              unicode=false, % unicode breaks when used with xetex
              xetex]{hyperref}
\else
  \usepackage[unicode=true]{hyperref}
\fi
\hypersetup{breaklinks=true,
            bookmarks=true,
            pdfauthor={},
            pdftitle={},
            colorlinks=true,
            citecolor=blue,
            urlcolor=blue,
            linkcolor=magenta,
            pdfborder={0 0 0}}
\urlstyle{same}  % don't use monospace font for urls
% Make links footnotes instead of hotlinks:
\renewcommand{\href}[2]{#2\footnote{\url{#1}}}
\setlength{\parindent}{0pt}
\setlength{\parskip}{6pt plus 2pt minus 1pt}
\setlength{\emergencystretch}{3em}  % prevent overfull lines
\setcounter{secnumdepth}{0}


\begin{document}

\% Practical Dependent Types in Haskell 2: Existential Neural Networks and Types
at Runtime \% Justin Le \% June 30, 2016

\emph{Originally posted on
\textbf{\href{https://blog.jle.im/entry/practical-dependent-types-in-haskell-2.html}{in
Code}}.}

We're back to continue on
\href{https://blog.jle.im/entries/series/+practical-dependent-types-in-haskell.html}{our
journey} in using dependent types to write type-safe neural networks! In
\href{https://blog.jle.im/entry/practical-dependent-types-in-haskell-1.html}{Part
1}, we wrote things out in normal, untyped Haskell, and looked at red flags and
general design principles that nudged us in the direction of adding dependent
types to our program. We learned to appreciate what dependent types offered in
terms of guiding us in writing our code, helping the compiler check our
correctness, providing a better interface for users, and more.

We also learned how to use singletons to work around some of Haskell's
fundamental limitations to let us "pattern match" on the structure of types, and
how to use typeclasses to generate singletons reflecting the structure of types
we are dealing with.

(If you read
\href{https://blog.jle.im/entry/practical-dependent-types-in-haskell-1.html}{Part
1} \emph{before} the singletons section was re-written to use the
\href{https://hackage.haskell.org/package/singletons}{singletons} library,
\href{https://blog.jle.im/entry/practical-dependent-types-in-haskell-1.html\#singletons-and-induction}{here's
a link to the section} in specific. This tutorial will assume familiarity with
what is discussed there!)

All of what we've dealt with so far has essentially been with types that are
fixed at compile-time. All the networks we've made have had "static" types, with
their sizes in their types indicated directly in the source code. In this post,
we're going to dive into the world of types that \emph{depend} on factors
unknown until runtime, and see how dependent types in a strongly typed language
like Haskell helps us write safer, more correct, and more maintainable code.

This post was written for GHC 8 on stackage snapshot
\href{https://www.stackage.org/nightly-2016-06-28}{nightly-2016-06-28}, but
should work with GHC 7.10 for the most part. All of the set-up instructions and
caveats (like the \emph{singletons-2.0.1} bug affecting GHC 7.10 users and the
unreleased \emph{hmatrix} version) are the same as for
\href{https://blog.jle.im/entry/practical-dependent-types-in-haskell-1.html\#setup}{part
1's setup}.

All of the code in this post is
\href{https://github.com/mstksg/inCode/tree/master/code-samples/dependent-haskell/NetworkTyped2.hs}{downloadable
as a standalone source file} so you can follow along!

A fair disclosure: a lot of this post doesn't actually directly deal with
machine learning or neural networks. Most of it will be learning general
principles for working with dependent types through implementing things you'd
want to do with neural networks. More stuff with ML will come in the next posts!

\section{Types at Runtime}

Recall the type we had for our neural networks:

\textasciitilde{}\textasciitilde{}\textasciitilde{}haskell ghci\textgreater{} :k
Network Network :: Nat -\textgreater{} {[}Nat{]} -\textgreater{} Nat
-\textgreater{} * \textasciitilde{}\textasciitilde{}\textasciitilde{}

They're of the form \texttt{Network\ i\ hs\ o}, where \texttt{i} is the size of
the input vector it expects, \texttt{hs} is the list of hidden layer sizes, and
\texttt{o} is the size of the output vector it produces. Something of type
\texttt{Network\ 10\ \textquotesingle{}{[}6,\ 4{]}\ 3} is a network with 10
input nodes, two input layers of size 6 and 4, and 3 output nodes.

This is great and all, but there's an severe limitation to this: Haskell is a
statically typed language, right? So doesn't this mean that using a network
requires that you know the entire structure of the network at compile-time?

It's conceivable that you might be able to have the input and output sizes known
at compile-time, but it's possible that you don't care or know your hidden layer
structure. You might load it from a configuration file, or have it depend on
user input. You might even want to receive one over a network channel without
knowing what the internal structure is. But can a type really depend on things
that you can't know until runtime?

To illustrate more clearly:

\textasciitilde{}\textasciitilde{}\textasciitilde{}haskell main :: IO () main =
do putStrLn "What hidden layer structure do you want?" hs \textless{}- readLn ::
IO {[}Integer{]} net \textless{}- randomNet :: IO (Network 10 ??? 3) -\/- what
is ??? -\/- ...? \textasciitilde{}\textasciitilde{}\textasciitilde{}

We \emph{want} to put \texttt{hs} there where \texttt{???} is,
but...\texttt{???} has to be a type (of kind \texttt{{[}Nat{]}}). \texttt{hs} is
a value (of type \texttt{{[}Integer{]}}). It's clear here that the \emph{type}
of our network depends on something we can't write down or decide until runtime.

\subsection{An Existential Crisis}

There are two main ways to go about solving this issue in Haskell. We'll look at
both, and then see that they are really actually just two styles of doing the
same thing.

\subsubsection{Types hiding behind constructors}

Now, having the entire structure of your neural network in the type is nice and
all for cool tricks like \texttt{randomNet}...but do you \emph{really} want to
work with this directly? After all, from the user's perspective, the user really
only ever needs to know \texttt{i} and \texttt{o}: What vectors the network
\emph{expects} and what vectors the network \emph{outputs}. In the end, a
(feed-forward) Neural Network is really just a fancy
\texttt{R\ i\ -\textgreater{}\ R\ o}.

Remember, the main benefits of having the entire structure in the type was to
help us \emph{implement} our functions more safely, with the compiler's help,
and also for cute return type polymorphism tricks like \texttt{randomNet} and
\texttt{getNet} and some stronger documentation. The \emph{user} of the network
really only benefits from the second two types of benefits.

One practical downside of having the structure in the type is that you can't
store them in the same list or data structure. A
\texttt{Network\ 10\ \textquotesingle{}{[}5,3{]}\ 1} won't share a list with a
\texttt{Network\ 10\ \textquotesingle{}{[}5,2{]}\ 1}, despite having the same
inputs/outputs (and API).

Imagine that we had written a \texttt{Network} type that \emph{didn't} have the
internal structure in the type -\/-\/-

\textasciitilde{}\textasciitilde{}\textasciitilde{}haskell data OpaqueNet i o
\textasciitilde{}\textasciitilde{}\textasciitilde{}

Recall that our issue earlier was that we had to write
\texttt{Network\ i\ ???\ o}, but we had no idea what to put in for \texttt{???}.
But, if we worked with an \texttt{OpaqueNet\ i\ o}, we wouldn't even care! We
wouldn't have to tell GHC what the internal structure is.

I'd actually argue that \texttt{OpaqueNet} might often be the more useful type
to offer to your users (or to use yourself), because it only exposes the types
that are \emph{relevant} to its usage/API. You can store them in a list or MVar
-\/-\/- \texttt{{[}OpaqueNet\ 10\ 3{]}} and \texttt{MVar\ (OpaqueNet\ 10\ 3)},
serialize/deserialize them without knowing their internal structure in advance
(\texttt{loadNet\ ::\ FilePath\ -\textgreater{}\ IO\ (OpaqueNet\ 10\ 3)}), etc.
(if you wanted to load a \texttt{Network}, you would need to know exactly what
internal structure was stored, in advance). Though \texttt{Network} is a much
easier type to \emph{implement}, \texttt{OpaqueNet} is a often a more ideal type
to \emph{use}.

We can implement our vision for \texttt{OpaqueNet} as an "existential" wrapper
over \texttt{Network}, actually:

\textasciitilde{}\textasciitilde{}\textasciitilde{}haskell -\/- source:
https://github.com/mstksg/inCode/tree/master/code-samples/dependent-haskell/NetworkTyped2.hs\#L110-L111
data OpaqueNet :: Nat -\textgreater{} Nat -\textgreater{} * where ONet ::
Network i hs o -\textgreater{} OpaqueNet i o
\textasciitilde{}\textasciitilde{}\textasciitilde{}

So, if you have \texttt{net\ ::\ Network\ 6\ \textquotesingle{}{[}10,6,3{]}\ 2},
you can create \texttt{ONet\ net\ ::\ OpaqueNet\ 6\ 2}. When you use the
\texttt{ONet} constructor, the structure of the hidden layers disappears from
the type!

We can use the network inside by \emph{pattern matching} on \texttt{ONet}:

\textasciitilde{}\textasciitilde{}\textasciitilde{}haskell -\/- source:
https://github.com/mstksg/inCode/tree/master/code-samples/dependent-haskell/NetworkTyped2.hs\#L113-L125
runOpaqueNet :: (KnownNat i, KnownNat o) =\textgreater{} OpaqueNet i o
-\textgreater{} R i -\textgreater{} R o runOpaqueNet (ONet n) x = runNet n x

numHiddens :: OpaqueNet i o -\textgreater{} Int numHiddens (ONet n) = go n where
go :: Network i hs o -\textgreater{} Int go = \textbackslash{}case O \_
-\textgreater{} 0 \_ :\&\textasciitilde{} n' -\textgreater{} 1 + go n'
\textasciitilde{}\textasciitilde{}\textasciitilde{}

With the \emph{ScopedTypeVariables} extension, we can even bring \texttt{hs}
back into scope, as in:

\textasciitilde{}\textasciitilde{}\textasciitilde{}haskell case oN of ONet (n ::
Network i hs o) -\textgreater{} ...
\textasciitilde{}\textasciitilde{}\textasciitilde{}

This pattern is sometimes called the \textbf{dependent pair}, because pattern
matching on \texttt{ONet} yields the hidden \textbf{existentially quantified}
type (\texttt{hs}) and also a value whose type is based on it
(\texttt{Network\ i\ hs\ o}). It's like \texttt{hs} "paired" with
\texttt{Network\ i\ hs\ o}. Pattern match on the results to give both the type
(\texttt{hs}) \emph{and} the data structure. (To make this more explicit, we
could have implemented it as
\texttt{ONet\ ::\ Sing\ hs\ -\textgreater{}\ Network\ i\ hs\ o\ -\textgreater{}\ OpaqueNet\ i\ o})

And here's the key to making this all work: once you pattern match on
\texttt{ONet}, you have to handle the \texttt{hs} in a \emph{completely
polymorphic way}. You're not allowed to assume anything about \texttt{hs}...you
have to provide a completely parametrically polymorphic way of dealing with it!

For example, this function is completely \emph{not} ok:

\textasciitilde{}\textasciitilde{}\textasciitilde{}haskell bad :: OpaqueNet i o
-\textgreater{} Network i hs o bad (ONet n) = n -\/- nope, not ok at all.
\textasciitilde{}\textasciitilde{}\textasciitilde{}

Why not? Well, a type signature like
\texttt{OpaqueNet\ i\ o\ -\textgreater{}\ Network\ i\ hs\ o} means that the
\emph{caller} can decide what \texttt{hs} can be -\/-\/- just like
\texttt{read\ ::\ Read\ a\ =\textgreater{}\ String\ -\textgreater{}\ a}, where
the caller decides what \texttt{a} is.

Of course, this isn't the case in the way we've written the function...the
function can only return a \emph{specific} \texttt{hs} (namely, the \texttt{hs}
of the network that \texttt{ONet} hides). The \emph{caller} has to accommodate
whatever is inside \texttt{ONet}.

\subsubsection{The Universal and the Existential}

We just brushed here on something at the heart of using existential types in
Haskell: the issue of who has the power to decide what the types will be
instantiated as. Most polymorphic functions you work with in Haskell are
"universally qualified". For example, for a function like

\textasciitilde{}\textasciitilde{}\textasciitilde{}haskell map :: (a
-\textgreater{} b) -\textgreater{} {[}a{]} -\textgreater{} {[}b{]}
\textasciitilde{}\textasciitilde{}\textasciitilde{}

\texttt{a} and \texttt{b} are universally quantified, which means that the
person who \emph{uses} \texttt{map} gets to decide what \texttt{a} and
\texttt{b} are. To be more explicit, that type signature can be written as:

\textasciitilde{}\textasciitilde{}\textasciitilde{}haskell map :: forall a b. (a
-\textgreater{} b) -\textgreater{} {[}a{]} -\textgreater{} {[}b{]}
\textasciitilde{}\textasciitilde{}\textasciitilde{}

This means that \texttt{map} is defined in a way that will work for \emph{any}
\texttt{a} and \texttt{b} that the \emph{caller} wants. As a caller, you can
request:

\textasciitilde{}\textasciitilde{}\textasciitilde{}haskell map :: (Int
-\textgreater{} Bool) -\textgreater{} {[}Int{]} -\textgreater{} {[}Bool{]} map
:: (Double -\textgreater{} Void) -\textgreater{} {[}Double{]} -\textgreater{}
{[}Void{]} map :: (String -\textgreater{} (Bool -\textgreater{} Char))
-\textgreater{} {[}String{]} -\textgreater{} {[}Bool -\textgreater{} Char{]}
\textasciitilde{}\textasciitilde{}\textasciitilde{}

Or anything else!

Consequentially, the function has to be implemented in a way that will work for
\emph{any} \texttt{a} and \texttt{b}. The function's implementation has the
burden of being flexible enough to handle whatever the caller asks for.

But, for a function like:

\textasciitilde{}\textasciitilde{}\textasciitilde{}haskell foo :: {[}Int{]}
-\textgreater{} OpaqueNet i o
\textasciitilde{}\textasciitilde{}\textasciitilde{}

While the caller can choose what \texttt{i} and \texttt{o} are, the
\emph{function} gets to choose what \texttt{hs} (in the hidden
\texttt{Network\ i\ hs\ o}) is. If I want to \emph{use} the thing that
\texttt{foo} returns...then \emph{I} have to be flexible. \emph{I} have the
burden of being flexible enough to handle whatever \texttt{hs} the
\emph{function} returns.

In summary:

\begin{itemize}
\item
  For universally quantified types, the \emph{caller} chooses the type being
  instanced, and the \emph{function's implementation} has to accommodate any
  choice.
\item
  For existentially quantified types, the \emph{function's implementation}
  chooses the type being instanced, and the \emph{caller} has to accommodate any
  choice.
\end{itemize}

Indeed, we saw earlier that if we ever wanted to \emph{use} the
\texttt{Network\ i\ hs\ o} inside the \texttt{OpaqueNet\ i\ o}, we were forced
to deal with it in a parametrically polymorphic way. We had to be able to handle
\emph{any} \texttt{hs} that the \texttt{ONet} could throw at us!

\subsubsection{A familiar friend}

I called \texttt{OpaqueNet\ i\ o} a "dependent pair" earlier, pairing
\texttt{hs} with \texttt{Network\ i\ hs\ o}. But there's another common term for
it: a \textbf{dependent sum}.

People familiar with Haskell might recognize that "sum types" are
\texttt{Either}-like types that can be one thing or another. Sum types are one
of the first things you learn about in Haskell -\/-\/- heck, even
\texttt{Maybe\ a} is the sum of \texttt{a} and \texttt{()}. Dependent
pairs/existential types actually are very similar to \texttt{Either}/sum types,
in spirit, and it might help to see the parallel so that you can see that
they're nothing scary, and that the fundamentals/intuition of working with
existential types in Haskell is no different than working with \texttt{Either}!

If I had:

\textasciitilde{}\textasciitilde{}\textasciitilde{}haskell foo :: String
-\textgreater{} Either Int Bool
\textasciitilde{}\textasciitilde{}\textasciitilde{}

I have to handle the result for both the case where I get an \texttt{Int} and
the case where I get a \texttt{Bool}. The \emph{function} gets to pick what type
I have to handle (\texttt{Int} or \texttt{Bool}), and \emph{I} have to adapt to
whatever it returns. Sound familiar? In fact, you can even imagine that
\texttt{OpaqueNet\ i\ o} as being just a infinite \emph{Either} over
\texttt{\textquotesingle{}{[}{]}}, \texttt{\textquotesingle{}{[}1{]}},
\texttt{\textquotesingle{}{[}1,2{]}}, etc.{[}\^{}eithers{]}

Remember that the basic way of handling an \texttt{Either} and figuring out what
the type of the value is inside is through \emph{pattern matching} on it. You
can't know if an \texttt{Either\ Int\ Bool} contains an \texttt{Int} or
\texttt{Bool} until you pattern match. But, once you do, all is revealed, and
GHC lets you take advantage of knowing the type.

For \texttt{OpaqueNet\ i\ o}, it's the same! You don't know the actual type of
the \texttt{Network\ i\ hs\ o} it contains until you \emph{pattern match} on the
network (This time, it's a "dependent pattern match"). Once you pattern match on
it, all is revealed...and GHC lets you take advantage of knowing the type!

\subsection{Reification}

For simplicity, let's re-write \texttt{randomNet} the more sensible way -\/-\/-
with the explicit singleton input style:

\textasciitilde{}\textasciitilde{}\textasciitilde{}haskell -\/- source:
https://github.com/mstksg/inCode/tree/master/code-samples/dependent-haskell/NetworkTyped2.hs\#L79-L87
randomNet' :: forall m i hs o. (MonadRandom m, KnownNat i, KnownNat o)
=\textgreater{} Sing hs -\textgreater{} m (Network i hs o) randomNet' =
\textbackslash{}case SNil -\textgreater{} O \textless{}\$\textgreater{}
randomWeights SNat \texttt{SCons} ss -\textgreater{} (:\&\textasciitilde{})
\textless{}\$\textgreater{} randomWeights \textless{}*\textgreater{} randomNet'
ss

randomNet :: forall m i hs o. (MonadRandom m, KnownNat i, SingI hs, KnownNat o)
=\textgreater{} m (Network i hs o) randomNet = randomNet' sing
\textasciitilde{}\textasciitilde{}\textasciitilde{}

We use \texttt{sing\ ::\ SingI\ hs\ =\textgreater{}\ Sing\ hs} to go call the
\texttt{Sing\ hs\ -\textgreater{}}-style function from the
\texttt{SingI\ hs\ =\textgreater{}} one.{[}\^{}style{]}

Now, we still need to somehow get our list of integers to the type level so that
we can create a \texttt{Network\ i\ hs\ o} to stuff into our \texttt{ONet}. For
that, the \emph{singletons} library offers the necessary tooling. It gives us
\texttt{SomeSing}, which is a lot like our \texttt{OpaqueNet} above, wrapping
the \texttt{Sing\ a} inside an existential data constructor. \texttt{toSing}
takes the term-level value (for us, an \texttt{{[}Integer{]}}) and returns a
\texttt{SomeSing} wrapping the type-level value (for us, a \texttt{{[}Nat{]}}).
When we pattern match on the \texttt{SomeSing} constructor, we get \texttt{a} in
scope!

As of \emph{singletons-2.2} and GHC 8{[}\^{}old-singletons{]}, \texttt{SomeSing}
is implemented as:

\textasciitilde{}\textasciitilde{}\textasciitilde{}haskell data SomeSing :: *
-\textgreater{} * where SomeSing :: Sing (a :: k) -\textgreater{} SomeSing k
\textasciitilde{}\textasciitilde{}\textasciitilde{}

And you have:

\textasciitilde{}\textasciitilde{}\textasciitilde{}haskell foo :: SomeSing Bool
foo = SomeSing STrue

bar :: SomeSing Nat bar = SomeSing (SNat :: Sing 10)
\textasciitilde{}\textasciitilde{}\textasciitilde{}

Pattern matching looks like:

\textasciitilde{}\textasciitilde{}\textasciitilde{}haskell main :: IO () main =
do putStrLn "How many cats do you own?" c \textless{}- readLn :: IO Integer case
toSing c of SomeSing (SNat :: Sing n) -\textgreater{} -\/- ...
\textasciitilde{}\textasciitilde{}\textasciitilde{}

Now, inside the case statement branch (the \texttt{...}), we have \emph{type}
\texttt{n\ ::\ Nat} in scope! And by pattern matching on the \texttt{SNat}
constructor, we also have a \texttt{KnownNat\ n} instance (As discussed in
\href{https://blog.jle.im/entry/practical-dependent-types-in-haskell-1.html\#on-typeclasses-and-dictionaries}{previous
part}).

\texttt{toSing} works using a simple typeclass mechanism with an associated type
whose job is to connect the types of values with the kinds of their singletons.
It associates \texttt{Bool} (the type) with \texttt{Bool} (the kind),
\texttt{Integer} (the type) with \texttt{Nat} (the kind), \texttt{{[}Integer{]}}
(the type) with \texttt{{[}Nat{]}} (the kind), etc., and it does it with simple
applications of type families (here's a
\href{https://ocharles.org.uk/blog/posts/2014-12-12-type-families.html}{nice
tutorial on type families} courtesy of Oliver Charles, as a refresher). With it,
we can convert any normal value \texttt{x} of type \texttt{a} to a singleton
representing type \texttt{x} with kind \texttt{a}.

We now have enough to write our \texttt{randomONet}:

\textasciitilde{}\textasciitilde{}\textasciitilde{}haskell -\/- source:
https://github.com/mstksg/inCode/tree/master/code-samples/dependent-haskell/NetworkTyped2.hs\#L127-L131
randomONet :: (MonadRandom m, KnownNat i, KnownNat o) =\textgreater{}
{[}Integer{]} -\textgreater{} m (OpaqueNet i o) randomONet hs = case toSing hs
of SomeSing ss -\textgreater{} ONet \textless{}\$\textgreater{} randomNet' ss
\textasciitilde{}\textasciitilde{}\textasciitilde{}

This process of bringing a term-level value into the type level is known in
Haskell as \textbf{reification}. With this, our original goal is (finally)
within reach:

\textasciitilde{}\textasciitilde{}\textasciitilde{}haskell -\/- source:
https://github.com/mstksg/inCode/tree/master/code-samples/dependent-haskell/NetworkTyped2.hs\#L205-L213
main :: IO () main = do putStrLn "What hidden layer structure do you want?" hs
\textless{}- readLn n \textless{}- randomONet hs case n of ONet (net :: Network
10 hs 3) -\textgreater{} do print net -\/- blah blah stuff with our dynamically
generated net \textasciitilde{}\textasciitilde{}\textasciitilde{}

\subsubsection{The Boundary}

With the power of existentially quantified types (like in \texttt{SomeSing}), we
essentially gained the ability to work with types that depend on runtime
results.

In a way, you can consider the \texttt{toSing} and the \texttt{SomeSing} as our
"boundary" between the "untyped world" and the "typed world". This layer (and
the process of reification) cleanly separates the two.

This boundary can be thought of as a lot like the boundary we talk about between
"pure" functions and values and "impure" (IO, etc.) ones. People say to always
write as much of your program as possible in the "pure" world -\/-\/- to
separate and pull out as much logic as you can to be pure logic. That's one of
the first things you learn about as a Haskell programmer: how to separate logic
that \emph{can} be pure from logic that is "impure" (IO, etc.), and then finally
combine them at the very end, as late as possible.

It's easy to think that just because the final program is going to "be in IO in
the end anyway", there isn't any point in separating out pure and impure parts
of your program logic. But we know that we gain separation of concerns, the
increased ability to reason with your code and analyze what it does, the
compiler's ability to check what you write, the limitation of implementations,
etc.

You can think of the general philosophy of working with typed/untyped worlds as
being the same thing. You try to write as much of your program as possible in
the "typed" world, like we did in Part 1. Take advantage of the increased
ability to reason with your code, parametric polymorphism helping you
\emph{write} your code, limit your implementations, nab you compiler help, etc.
All of those are benefits of working in the typed world.

Then, write what you absolutely must in your "untyped" world, such as dealing
with values that pop up at runtime like the \texttt{{[}Integer{]}} above.

Finally, at the very end, \emph{unite} them at the boundary. Pass the control
football from the untyped world to the typed world!

The great part about this all is that GHC and the type system is there at every
step holding your hand, guiding you as you implement your programs and making
sure everything is type-safe and fits together! (This, after all, is why
dependently typed programming with dynamically generated types is \emph{not} the
same thing as "\emph{dynamically} typed programming"!)

\subsection{Continuation-Based Existentials}

There's another way in Haskell that we work with existential types that can be
more natural to use in a lot of cases. Remember that when we pattern match on an
existential data type, we have to work with the values in the constructor in a
parametrically polymorphic way. For example, if we had:

\textasciitilde{}\textasciitilde{}\textasciitilde{}haskell oNetToFoo ::
OpaqueNet i o -\textgreater{} Foo oNetToFoo (ONet n) = f n
\textasciitilde{}\textasciitilde{}\textasciitilde{}

\texttt{f} has to take a \texttt{Network\ i\ hs\ o} but deal with it in a way
that works \emph{for all} \texttt{hs}. It can't be written for \emph{only}
\texttt{\textquotesingle{}{[}5{]}} or \emph{only}
\texttt{\textquotesingle{}{[}6,3{]}}...it has to work for \emph{any}
\texttt{hs}. That's the whole "existential vs. universal quantification" thing
we just talked about.

Well, we could really also just skip the constructor altogether and represent an
existential type as something \emph{taking} the continuation \texttt{f} and
giving it what it needs.

\textasciitilde{}\textasciitilde{}\textasciitilde{}haskell -\/- source:
https://github.com/mstksg/inCode/tree/master/code-samples/dependent-haskell/NetworkTyped2.hs\#L154-L154
type OpaqueNet' i o r = (forall hs. Network i hs o -\textgreater{} r)
-\textgreater{} r \textasciitilde{}\textasciitilde{}\textasciitilde{}

"Tell me how you would make an \texttt{r} if you had a
\texttt{Network\ i\ hs\ o} (that works for any \texttt{hs}) and I'll make it for
you!"

(This takes advantage of Rank-N types. If you're unfamiliar with them, Gregor
Riegler has a
\href{http://sleepomeno.github.io/blog/2014/02/12/Explaining-Haskell-RankNTypes-for-all/}{nice
tutorial} on the subject.)

\emph{Using} these types is very similar to using the constructor-style ones:

\textasciitilde{}\textasciitilde{}\textasciitilde{}haskell -\/- source:
https://github.com/mstksg/inCode/tree/master/code-samples/dependent-haskell/NetworkTyped2.hs\#L159-L176
runOpaqueNet' :: (KnownNat i, KnownNat o) =\textgreater{} OpaqueNet' i o (R o)
-\textgreater{} R i -\textgreater{} R o runOpaqueNet' oN x = oN
(\textbackslash{}n -\textgreater{} runNet n x) -\/- :: ((forall hs. Network i hs
o -\textgreater{} R o) -\textgreater{} R o) -\/- -\textgreater{} R i -\/-
-\textgreater{} R o

numHiddens' :: OpaqueNet' i o Int -\textgreater{} Int numHiddens' oN = oN go
where go :: Network i hs o -\textgreater{} Int go = \textbackslash{}case O \_
-\textgreater{} 0 \_ :\&\textasciitilde{} n' -\textgreater{} 1 + go n' -\/- ::
((forall hs. Network i hs o -\textgreater{} Int) -\textgreater{} Int) -\/-
-\textgreater{} Int \textasciitilde{}\textasciitilde{}\textasciitilde{}

This "continuation transformation" is formally known as
\textbf{skolemization}.{[}\^{}skolemization{]}

We can "wrap" a \texttt{Network\ i\ hs\ o} into an
\texttt{OpaqueNet\textquotesingle{}\ i\ o\ r}:

\textasciitilde{}\textasciitilde{}\textasciitilde{}haskell -\/- source:
https://github.com/mstksg/inCode/tree/master/code-samples/dependent-haskell/NetworkTyped2.hs\#L156-L157
oNet' :: Network i hs o -\textgreater{} OpaqueNet' i o r oNet' n =
\textbackslash{}f -\textgreater{} f n
\textasciitilde{}\textasciitilde{}\textasciitilde{}

Let's write a version of \texttt{randomONet} that returns a continuation-style
existential:

\textasciitilde{}\textasciitilde{}\textasciitilde{}haskell withRandomONet' ::
(MonadRandom m, KnownNat i, KnownNat o) =\textgreater{} {[}Integer{]}
-\textgreater{} (forall hs. Network i hs o -\textgreater{} m r) -\textgreater{}
m r -\/- aka, =\textgreater{} {[}Integer{]} -\/- -\textgreater{} OpaqueNet' i o
(m r) withRandomONet' hs f = case toSing hs of SomeSing ss -\textgreater{} do
net \textless{}- randomNet' ss f net

\textasciitilde{}\textasciitilde{}\textasciitilde{}

But, hey, because we're skolemizing everything, let's do it with the skolemized
version of \texttt{toSing}/\texttt{SomeSing}, \texttt{withSomeSing}:

\textasciitilde{}\textasciitilde{}\textasciitilde{}haskell -\/- a version of
\texttt{toSing} that returns a skolemized \texttt{SomeSing} withSomeSing ::
{[}Integer{]} -\textgreater{} (forall (hs :: {[}Nat{]}). Sing hs -\textgreater{}
r) -\textgreater{} r \textasciitilde{}\textasciitilde{}\textasciitilde{}

Because why not? Skolemize all the things!

\textasciitilde{}\textasciitilde{}\textasciitilde{}haskell -\/- source:
https://github.com/mstksg/inCode/tree/master/code-samples/dependent-haskell/NetworkTyped2.hs\#L178-L186
withRandomONet' :: (MonadRandom m, KnownNat i, KnownNat o) =\textgreater{}
{[}Integer{]} -\textgreater{} (forall hs. Network i hs o -\textgreater{} m r)
-\textgreater{} m r -\/- aka, =\textgreater{} {[}Integer{]} -\/- -\textgreater{}
OpaqueNet' i o (m r) withRandomONet' hs f = withSomeSing hs \$
\textbackslash{}ss -\textgreater{} do net \textless{}- randomNet' ss f net
\textasciitilde{}\textasciitilde{}\textasciitilde{}

We can use it to do the same things we used the constructor-based existential
for, as well...and, in a way, it actually seems (oddly) more natural.

\textasciitilde{}\textasciitilde{}\textasciitilde{}haskell -\/- source:
https://github.com/mstksg/inCode/tree/master/code-samples/dependent-haskell/NetworkTyped2.hs\#L215-L221
main' :: IO () main' = do putStrLn "What hidden layer structure do you want?" hs
\textless{}- readLn withRandomONet' hs \$ (net :: Network 10 hs 3)
-\textgreater{} do print net -\/- blah blah stuff with our dynamically generated
net \textasciitilde{}\textasciitilde{}\textasciitilde{}

Like the case statement pattern match represented the lexical "wall"/"boundary"
between the untyped and typed world when using constructor-style existentials,
the \texttt{...\ \$\ \textbackslash{}net\ -\textgreater{}\ ...} can be thought
of the "wall" for the continuation-style existentials.

\section{A Tale of Two Styles}

We've just discussed two ways of doing the same thing. Two styles of
representing/working with existential types. The two are equivalent, in that you
can always "convert" between one or the other, but the choice of which one you
use/reach for/offer can make a difference in code clarity. After working with
both styles a lot (sometimes, libraries only offer one style), you start to get
a feel for which one you like more in which situations. In the end, I don't
think there are any hard or fast rules. Just use whichever one you feel is more
readable!

That being said, here are some general Pros and Cons that I've noticed over the
years:

\begin{itemize}
\item
  Most obviously, continuation-style doesn't require you to define a throwaway
  data type/constructor. While new types are cheap in Haskell, they force your
  users to learn a new set of types and constructors for every single
  existential type you return. If you or the library you're writing uses/returns
  a \emph{lot} of different existentially qualified types, all those extra dumb
  wrappers are a huge hassle.
\item
  Continuation-style existentials are in general smoother to use than
  constructor-style ones when functions \emph{return} existentials. Especially
  if you intend to immediately use them, continuation-style basically saves you
  an extraneous pattern match.
\item
  When you have to use several existentials at once, continuation-style is much
  better because each nested existential doesn't force another level of
  indentation:

  \textasciitilde{}\textasciitilde{}\textasciitilde{}haskell foo = withSomeSing
  x \$ \textbackslash{}sx -\textgreater{} withSomeSing y \$ \textbackslash{}sy
  -\textgreater{} withSomeSing z \$ \textbackslash{}sz -\textgreater{} -\/- ...
  \textasciitilde{}\textasciitilde{}\textasciitilde{}

  vs.

  \textasciitilde{}\textasciitilde{}\textasciitilde{}haskell foo = case toSing x
  of SomeSing sx -\textgreater{} case toSing y of SomeSing sy -\textgreater{}
  case toSing z of SomeSing sz -\textgreater{} -\/- ...
  \textasciitilde{}\textasciitilde{}\textasciitilde{}

  Every time you nest a case statement, you actually waste \emph{two} levels of
  indentation, which can be annoying even at 2-space indentation. But you don't
  need \emph{any} to nest in the continuation style!
\item
  If you're working monadically, though, you can take advantage of do notation
  and \emph{ScopedTypeVariables} for a nicer style that doesn't require any
  nesting at all:

  \textasciitilde{}\textasciitilde{}\textasciitilde{}haskell main = do ONet n1
  \textless{}- randomONet {[}7,5,3{]} :: IO (OpaqueNet 10 1) ONet n2
  \textless{}- randomONet {[}5,5,5{]} :: IO (OpaqueNet 10 1) ONet n3
  \textless{}- randomONet {[}5,4,3{]} :: IO (OpaqueNet 10 1) hs \textless{}-
  readLn ONet (n4 :: Network 10 hs 1) \textless{}- randomONet hs -\/- ...
  \textasciitilde{}\textasciitilde{}\textasciitilde{}

  Which is arguably nicer than

  \textasciitilde{}\textasciitilde{}\textasciitilde{}haskell main =
  withRandomONet' {[}7,5,3{]} \$ \textbackslash{}n1 -\textgreater{}
  withRandomONet' {[}5,5,5{]} \$ \textbackslash{}n2 -\textgreater{}
  withRandomONet' {[}5,4,3{]} \$ \textbackslash{}n3 -\textgreater{} do hs
  \textless{}- readLn withRandomONet' hs \$ (n4 :: Network 10 hs 1)
  -\textgreater{} do -\/- ...
  \textasciitilde{}\textasciitilde{}\textasciitilde{}

  A lot of libraries return existentials in \texttt{Maybe}'s
  (\href{http://hackage.haskell.org/package/base-4.9.0.0/docs/GHC-TypeLits.html\#v:someNatVal}{base
  is guilty}), so this trick can be useful for those, too!

  This trick is less useful for functions like \texttt{toSing} where things are
  \emph{not} returned in a monad. You could wrap it in Identity, but that's kind
  of silly:

  \textasciitilde{}\textasciitilde{}\textasciitilde{}haskell foo = runIdentity
  \$ do SomeSing sx \textless{}- Identity \$ toSing x SomeSing sy \textless{}-
  Identity \$ toSing y SomeSing sz \textless{}- Identity \$ toSing z return \$
  -\/- ... \textasciitilde{}\textasciitilde{}\textasciitilde{}
\item
  Constructor-style is necessary for writing typeclass instances. You can't
  write a \texttt{Show} instance for
  \texttt{(forall\ hs.\ Network\ i\ hs\ o\ -\textgreater{}\ r)\ -\textgreater{}\ r},
  but you can write one for \texttt{OpaqueNet\ i\ o}. We'll also be writing
  \texttt{Binary} instances later for serialization/deserialization, and it all
  only works in constructor-style.
\item
  Haskell doesn't allow you to use Rank-N types as arguments to type
  constructors, so you can have \texttt{{[}OpaqueNet\ i\ o{]}}, but \emph{not}
  \texttt{{[}OpaqueNet\textquotesingle{}\ i\ o\ r{]}} or
  \texttt{{[}(forall\ hs.\ Network\ i\ hs\ o\ -\textgreater{}\ r)\ -\textgreater{}\ r{]}}.
  You can have \texttt{MVar\ (OpaqueNet\ i\ o)}, but not
  \texttt{MVar\ ((forall\ hs.\ Network\ i\ hs\ o\ -\textgreater{}\ r)\ -\textgreater{}\ r)}.
  The latter are known as \emph{impredicative} types, which are a big no-no in
  GHC Haskell. Don't even go there! The constructor style is necessary in these
  situations.

  If the type constructor is a Monad, you can get away with a ContT-style
  skolemization, like
  \texttt{(forall\ hs.\ Network\ i\ hs\ o\ -\textgreater{}\ {[}r{]})\ -\textgreater{}\ {[}r{]}}
  and
  \texttt{(forall\ hs.\ Network\ i\ hs\ o\ -\textgreater{}\ IO\ r)\ -\textgreater{}\ IO\ r}.
  But this doesn't work for \texttt{MVar} and other useful type constructors you
  might want to put \texttt{OpaqueNet} in.
\item
  When writing functions that \emph{take} existentials as inputs, the
  constructor-style is arguably more natural. But barely.

  For example, we wrote a function to find the number of hidden layers in a
  network earlier:

  \textasciitilde{}\textasciitilde{}\textasciitilde{}haskell numHiddens ::
  OpaqueNet i o -\textgreater{} Int
  \textasciitilde{}\textasciitilde{}\textasciitilde{}

  But the continuation-style version has a slightly messier type:

  \textasciitilde{}\textasciitilde{}\textasciitilde{}haskell numHiddens' ::
  ((forall hs. Network i hs o -\textgreater{} Int) -\textgreater{} Int)
  -\textgreater{} Int \textasciitilde{}\textasciitilde{}\textasciitilde{}

  Even with with the type synonym, it's still a little awkward:

  \textasciitilde{}\textasciitilde{}\textasciitilde{}haskell numHiddens' ::
  OpaqueNet' i o Int -\textgreater{} Int
  \textasciitilde{}\textasciitilde{}\textasciitilde{}

  This is why you'll encounter many more functions \emph{returning}
  continuation-style existentials in libraries than \emph{taking} them, for the
  most part.
\end{itemize}

These are just general principles, not hard fast rules. This list is nowhere
near exhaustive and reflects my current progress in my journey towards a
dependently typed lifestyle. If you come back in a month, you might see more
things listed here!

All said, I do find myself very happy when I see that a library I'm using offers
\emph{both} styles for me to use. And I've been known to submit PR's to a
library to have it offer one style or another, if it's lacking.

Be judicious. If you're writing a library, don't spam it with too many throwaway
constructors. After a while, you'll begin to intuitively see which style shines
in which situations! (And, in some case, there might not even be a definitive
"better" style to use.)

\section{Serializing Networks}

To drive things home, let's apply what we learned about existential types and
reification to another simple application: serialization.

\subsection{Recap on the Binary Library}

Serializing networks of \emph{known} size -\/-\/- whose sizes are statically in
their types -\/-\/- is pretty straightforward, and its ease is one of the
often-quoted advantages of having sizes in your types.{[}\^{}storable{]} I'm
going to be using the
\emph{\href{https://hackage.haskell.org/package/binary}{binary}} library, which
offers a very standard typeclass-based approach for serializing and
deserializing data. There are a lot of tutorials online (and I even
\href{https://blog.jle.im/entry/streaming-huffman-compression-in-haskell-part-2-binary.html}{wrote
a small one} myself a few years ago), but a very high-level view is that the
library offers monads (\texttt{Get}, \texttt{Put}) for describing serialization
schemes and also a typeclass used to provide serialization instructions for
different types.

In practice, we usually don't write our own instances from scratch. Instead, we
use GHC's generics features to give us instances for free:

\textasciitilde{}\textasciitilde{}\textasciitilde{}haskell -\/- source:
https://github.com/mstksg/inCode/tree/master/code-samples/dependent-haskell/NetworkTyped2.hs\#L25-L30
data Weights i o = W \{ wBiases :: !(R o) , wNodes :: !(L o i) \} deriving
(Show, Generic)

instance (KnownNat i, KnownNat o) =\textgreater{} Binary (Weights i o)
\textasciitilde{}\textasciitilde{}\textasciitilde{}

For simple types like \texttt{Weights}, which simply contain serializable
things, the \emph{binary} library is smart enough to write your instances
automatically for you! This gives us \texttt{get} and \texttt{put}:

\textasciitilde{}\textasciitilde{}\textasciitilde{}haskell get :: (KnownNat i,
KnownNat o) =\textgreater{} Get (Weights i o)

put :: (KnownNat i, KnownNat o) =\textgreater{} Weights i o -\textgreater{} Put
\textasciitilde{}\textasciitilde{}\textasciitilde{}

However, for GADTs like \texttt{Network}, we have to things manually.

\subsubsection{\texorpdfstring{Serializing
\texttt{Network}}{Serializing Network}}

Taking advantage of having the entire structure in the type, \texttt{put} is
simple:

\textasciitilde{}\textasciitilde{}\textasciitilde{}haskell -\/- source:
https://github.com/mstksg/inCode/tree/master/code-samples/dependent-haskell/NetworkTyped2.hs\#L89-L94
putNet :: (KnownNat i, KnownNat o) =\textgreater{} Network i hs o
-\textgreater{} Put putNet = \textbackslash{}case O w -\textgreater{} put w w
:\&\textasciitilde{} n -\textgreater{} put w *\textgreater{} putNet n
\textasciitilde{}\textasciitilde{}\textasciitilde{}

If it's an \texttt{O\ w}, just serialize the \texttt{w}. If it's a
\texttt{w\ :\&\textasciitilde{}\ net}, serialize the \texttt{w} then the rest of
the \texttt{net}. Normally, we might have to put a "flag" to tell what
constructor we serializing, so that the deserializer can know what constructor
to deserialize at every step. But for \texttt{Network}, we don't have to:

\textasciitilde{}\textasciitilde{}\textasciitilde{}haskell -\/- source:
https://github.com/mstksg/inCode/tree/master/code-samples/dependent-haskell/NetworkTyped2.hs\#L96-L101
getNet :: forall i hs o. (KnownNat i, KnownNat o) =\textgreater{} Sing hs
-\textgreater{} Get (Network i hs o) getNet = \textbackslash{}case SNil
-\textgreater{} O \textless{}\$\textgreater{} get SNat \texttt{SCons} ss
-\textgreater{} (:\&\textasciitilde{}) \textless{}\$\textgreater{} get
\textless{}*\textgreater{} getNet ss
\textasciitilde{}\textasciitilde{}\textasciitilde{}

\texttt{getNet} doesn't need flags because we already \emph{know} how many
\texttt{:\&\textasciitilde{}} layers to expect \emph{just from the type}. If we
want to deserialize a \texttt{Network\ 5\ \textquotesingle{}{[}10,6,3{]}\ 2}, we
\emph{know} we want three \texttt{(:\&\textasciitilde{})}'s and one \texttt{O}.

\texttt{getNet} is written similarly to how we wrote
\href{https://github.com/mstksg/inCode/tree/master/code-samples/dependent-haskell/NetworkTyped2.hs\#L79-L83}{\texttt{randomNet\textquotesingle{}}}.
We "pattern match" on \texttt{hs} (using singletons) to get the constructors we
are expecting to deserialize and just follow what the singleton's structure
tells us.

To write a \texttt{Binary} instance for \texttt{Network}, we can't have
\texttt{get} take a \texttt{Sing\ hs} input -\/-\/- that'd change the arity/type
of the function. We have to switch to \texttt{SingI}-style had have their
\texttt{Binary} instances require a \texttt{SingI\ hs} constraint.

\textasciitilde{}\textasciitilde{}\textasciitilde{}haskell -\/- source:
https://github.com/mstksg/inCode/tree/master/code-samples/dependent-haskell/NetworkTyped2.hs\#L103-L105
instance (KnownNat i, SingI hs, KnownNat o) =\textgreater{} Binary (Network i hs
o) where put = putNet get = getNet sing
\textasciitilde{}\textasciitilde{}\textasciitilde{}

\subsection{\texorpdfstring{Serializating
\texttt{OpaqueNet}}{Serializating OpaqueNet}}

It's arguably much more useful to serialize/deserialize \texttt{OpaqueNet}s.
Between different iterations of your program, you might have the same
inputs/outputs, but want to try out different internal structures. You'd want to
store them and access them uniformly, or send them over a network without
requiring the receiver to know the internal structure beforehand. Remember, you
can't even \emph{load} a \texttt{Network\ i\ hs\ o} without knowing its complete
structure!

Because the complete structure of the network is not in the type, we need to
encode it as a flag in the binary serialization so that the deserializer will
know what constructors to expect and deserialize. We can write a simple function
to get the \texttt{{[}Integer{]}} of a network's structure:

\textasciitilde{}\textasciitilde{}\textasciitilde{}haskell -\/- source:
https://github.com/mstksg/inCode/tree/master/code-samples/dependent-haskell/NetworkTyped2.hs\#L72-L77
hiddenStruct :: Network i hs o -\textgreater{} {[}Integer{]} hiddenStruct =
\textbackslash{}case O \_ -\textgreater{} {[}{]} \_ :\&\textasciitilde{} (n' ::
Network h hs' o) -\textgreater{} natVal (Proxy @h) : hiddenStruct n'
\textasciitilde{}\textasciitilde{}\textasciitilde{}

Recall that
\texttt{natVal\ ::\ KnownNat\ n\ =\textgreater{}\ Proxy\ n\ -\textgreater{}\ Integer}
returns the value-level \texttt{Integer} corresponding to the type-level
\texttt{n\ ::\ Nat}. (I'm also using GHC 8's fancy \emph{TypeApplications}
syntax, and \texttt{Proxy\ @h} is the same as \texttt{Proxy\ ::\ Proxy\ h}).

\texttt{natVal} and \texttt{hiddenStruct} are kind of interesting -\/-\/- they
take type-level information (\texttt{n}, \texttt{hs}) and turns them into
term-level values (\texttt{Integer}s, \texttt{{[}Integer{]}}s). They are the
opposites of our reification functions (like \texttt{toSing}). Going from the
"type level" to the "value level" is known in Haskell as \textbf{reflection},
and is the dual concept of reification. (The \emph{singletons} library offers
reflectors for all of its singletons, as \texttt{fromSing}.)

That's all we need!

\textasciitilde{}\textasciitilde{}\textasciitilde{}haskell -\/- source:
https://github.com/mstksg/inCode/tree/master/code-samples/dependent-haskell/NetworkTyped2.hs\#L133-L138
putONet :: (KnownNat i, KnownNat o) =\textgreater{} OpaqueNet i o
-\textgreater{} Put putONet (ONet net) = do put (hiddenStruct net) putNet net
\textasciitilde{}\textasciitilde{}\textasciitilde{}

"Put the structure (as a binary flag), and then put the network itself."

Now, to deserialize, we want to \emph{load} the list of \texttt{Integer}s and
reify it back to the type level to know what type of network we're expecting to
load:

\textasciitilde{}\textasciitilde{}\textasciitilde{}haskell -\/- source:
https://github.com/mstksg/inCode/tree/master/code-samples/dependent-haskell/NetworkTyped2.hs\#L140-L145
getONet :: (KnownNat i, KnownNat o) =\textgreater{} Get (OpaqueNet i o) getONet
= do hs \textless{}- get withSomeSing hs \$ \textbackslash{}ss -\textgreater{}
ONet \textless{}\$\textgreater{} getNet ss
\textasciitilde{}\textasciitilde{}\textasciitilde{}

We load our flag, reify it, and once we're back in the typed land again, we can
do our normal business. Isn't it nice that GHC is also there at every step to
make sure we make the transition safely?

Our final instance:

\textasciitilde{}\textasciitilde{}\textasciitilde{}haskell -\/- source:
https://github.com/mstksg/inCode/tree/master/code-samples/dependent-haskell/NetworkTyped2.hs\#L147-L149
instance (KnownNat i, KnownNat o) =\textgreater{} Binary (OpaqueNet i o) where
put = putONet get = getONet \textasciitilde{}\textasciitilde{}\textasciitilde{}

And, of course, we used the constructor-style existential this whole time
instead of the continuation-style one because we can't directly write typeclass
instances for the latter.

\section{An Existence For All}

We've learned about how to "cross" from the untyped world to the typed world and
bring about contexts involving types that can depend on runtime factors. To me,
this is really the point where dependently typed programming starts -\/-\/- when
you start having to work with types that depend on run-time factors.

We've already been able to reap a lot of benefits. All of the type safety we
discovered in the last part is now available to us in a fully dynamic world, as
well. We also learned the advantages of \emph{separating} the typed world from
the untyped world and how the compiler helps us make the transition safely.

But really, this is all just the \emph{start} of dependently typed programming.
This is where things \emph{really} start to get fun.

Stepping into this new world can be disorienting at first. There's a lot of
unexpected things that come up when we start working more with these fancy new
types. We have to deal with types coming from different sources, convince the
type system about their properties and relationships between them, and deal with
a whole bunch of other concerns that just don't happen when you program only at
the value level. But don't worry! Like all things, it will more naturally with
practice.

Now that we have existential types and run-time types out of the way, come back
for the next post in the series, where we start to have the \emph{real} fun! :D

\subsubsection{Exercises}

Here are some fun exercises you can try, if you want to test your understanding!
Links are to the solutions.

\begin{enumerate}
\item
  Implement
  \href{https://github.com/mstksg/inCode/tree/master/code-samples/dependent-haskell/NetworkTyped2.hs\#L188-L193}{\texttt{putONet\textquotesingle{}}}
  and
  \href{https://github.com/mstksg/inCode/tree/master/code-samples/dependent-haskell/NetworkTyped2.hs\#L195-L203}{\texttt{getONet\textquotesingle{}}}
  using the continuation-style existentials, instead.
\item
  Work with an existential wrapper over the \emph{entire} network structure
  (inputs and outputs, too):

  \textasciitilde{}\textasciitilde{}\textasciitilde{}haskell data SomeNet where
  SNet :: (KnownNat i, KnownNat o) =\textgreater{} Network i hs o
  -\textgreater{} SomeNet \textasciitilde{}\textasciitilde{}\textasciitilde{}

  (We need the \texttt{KnownNat} constraints because of type erasure, to recover
  the original input/output dimensions back once we pattern match)

  And write:

  \begin{itemize}
  \item
    A function to
    \href{https://github.com/mstksg/inCode/tree/master/code-samples/dependent-haskell/NetworkTyped2.hs\#L231-L234}{convert
    \texttt{SomeNet}s to \texttt{OpaqueNet}s}. Return the \texttt{OpaqueNet}
    with existentially quantified \texttt{i} and \texttt{o} in
    continuation-style. (You can write a data type to return it in
    constructor-style, too, for funsies.)
  \item
    \href{https://github.com/mstksg/inCode/tree/master/code-samples/dependent-haskell/NetworkTyped2.hs\#L236-L245}{\texttt{randomSNet}},
    returning \texttt{m\ SomeNet}.
  \item
    While you're at it, write it to return
    \href{https://github.com/mstksg/inCode/tree/master/code-samples/dependent-haskell/NetworkTyped2.hs\#L247-L258}{a
    random continuation-style \texttt{SomeNet}, too}! (See the type of
    \href{https://github.com/mstksg/inCode/tree/master/code-samples/dependent-haskell/NetworkTyped2.hs\#L178-L186}{\texttt{withRandomONet\textquotesingle{}}}
    for reference on how to write the type)
  \item
    The
    \href{https://github.com/mstksg/inCode/tree/master/code-samples/dependent-haskell/NetworkTyped2.hs\#L260-L274}{binary
    instance} for \texttt{SomeNet}.

    Hint: Remember
    \texttt{natVal\ ::\ KnownNat\ n\ =\textgreater{}\ Proxy\ n\ -\textgreater{}\ Integer}!
  \end{itemize}

  Hint: Remember that \texttt{toSomeSing} also works for \texttt{Integer}s, to
  get \texttt{Sing}s for \texttt{Nat}s, too!
\end{enumerate}

\end{document}
