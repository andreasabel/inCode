\documentclass[]{article}
\usepackage{lmodern}
\usepackage{amssymb,amsmath}
\usepackage{ifxetex,ifluatex}
\usepackage{fixltx2e} % provides \textsubscript
\ifnum 0\ifxetex 1\fi\ifluatex 1\fi=0 % if pdftex
  \usepackage[T1]{fontenc}
  \usepackage[utf8]{inputenc}
\else % if luatex or xelatex
  \ifxetex
    \usepackage{mathspec}
    \usepackage{xltxtra,xunicode}
  \else
    \usepackage{fontspec}
  \fi
  \defaultfontfeatures{Mapping=tex-text,Scale=MatchLowercase}
  \newcommand{\euro}{€}
\fi
% use upquote if available, for straight quotes in verbatim environments
\IfFileExists{upquote.sty}{\usepackage{upquote}}{}
% use microtype if available
\IfFileExists{microtype.sty}{\usepackage{microtype}}{}
\usepackage[margin=1in]{geometry}
\usepackage{color}
\usepackage{fancyvrb}
\newcommand{\VerbBar}{|}
\newcommand{\VERB}{\Verb[commandchars=\\\{\}]}
\DefineVerbatimEnvironment{Highlighting}{Verbatim}{commandchars=\\\{\}}
% Add ',fontsize=\small' for more characters per line
\newenvironment{Shaded}{}{}
\newcommand{\KeywordTok}[1]{\textcolor[rgb]{0.00,0.44,0.13}{\textbf{{#1}}}}
\newcommand{\DataTypeTok}[1]{\textcolor[rgb]{0.56,0.13,0.00}{{#1}}}
\newcommand{\DecValTok}[1]{\textcolor[rgb]{0.25,0.63,0.44}{{#1}}}
\newcommand{\BaseNTok}[1]{\textcolor[rgb]{0.25,0.63,0.44}{{#1}}}
\newcommand{\FloatTok}[1]{\textcolor[rgb]{0.25,0.63,0.44}{{#1}}}
\newcommand{\ConstantTok}[1]{\textcolor[rgb]{0.53,0.00,0.00}{{#1}}}
\newcommand{\CharTok}[1]{\textcolor[rgb]{0.25,0.44,0.63}{{#1}}}
\newcommand{\SpecialCharTok}[1]{\textcolor[rgb]{0.25,0.44,0.63}{{#1}}}
\newcommand{\StringTok}[1]{\textcolor[rgb]{0.25,0.44,0.63}{{#1}}}
\newcommand{\VerbatimStringTok}[1]{\textcolor[rgb]{0.25,0.44,0.63}{{#1}}}
\newcommand{\SpecialStringTok}[1]{\textcolor[rgb]{0.73,0.40,0.53}{{#1}}}
\newcommand{\ImportTok}[1]{{#1}}
\newcommand{\CommentTok}[1]{\textcolor[rgb]{0.38,0.63,0.69}{\textit{{#1}}}}
\newcommand{\DocumentationTok}[1]{\textcolor[rgb]{0.73,0.13,0.13}{\textit{{#1}}}}
\newcommand{\AnnotationTok}[1]{\textcolor[rgb]{0.38,0.63,0.69}{\textbf{\textit{{#1}}}}}
\newcommand{\CommentVarTok}[1]{\textcolor[rgb]{0.38,0.63,0.69}{\textbf{\textit{{#1}}}}}
\newcommand{\OtherTok}[1]{\textcolor[rgb]{0.00,0.44,0.13}{{#1}}}
\newcommand{\FunctionTok}[1]{\textcolor[rgb]{0.02,0.16,0.49}{{#1}}}
\newcommand{\VariableTok}[1]{\textcolor[rgb]{0.10,0.09,0.49}{{#1}}}
\newcommand{\ControlFlowTok}[1]{\textcolor[rgb]{0.00,0.44,0.13}{\textbf{{#1}}}}
\newcommand{\OperatorTok}[1]{\textcolor[rgb]{0.40,0.40,0.40}{{#1}}}
\newcommand{\BuiltInTok}[1]{{#1}}
\newcommand{\ExtensionTok}[1]{{#1}}
\newcommand{\PreprocessorTok}[1]{\textcolor[rgb]{0.74,0.48,0.00}{{#1}}}
\newcommand{\AttributeTok}[1]{\textcolor[rgb]{0.49,0.56,0.16}{{#1}}}
\newcommand{\RegionMarkerTok}[1]{{#1}}
\newcommand{\InformationTok}[1]{\textcolor[rgb]{0.38,0.63,0.69}{\textbf{\textit{{#1}}}}}
\newcommand{\WarningTok}[1]{\textcolor[rgb]{0.38,0.63,0.69}{\textbf{\textit{{#1}}}}}
\newcommand{\AlertTok}[1]{\textcolor[rgb]{1.00,0.00,0.00}{\textbf{{#1}}}}
\newcommand{\ErrorTok}[1]{\textcolor[rgb]{1.00,0.00,0.00}{\textbf{{#1}}}}
\newcommand{\NormalTok}[1]{{#1}}
\ifxetex
  \usepackage[setpagesize=false, % page size defined by xetex
              unicode=false, % unicode breaks when used with xetex
              xetex]{hyperref}
\else
  \usepackage[unicode=true]{hyperref}
\fi
\hypersetup{breaklinks=true,
            bookmarks=true,
            pdfauthor={Justin Le},
            pdftitle={Practical Dependent Types in Haskell: Type-Safe Neural Networks (Part 2)},
            colorlinks=true,
            citecolor=blue,
            urlcolor=blue,
            linkcolor=magenta,
            pdfborder={0 0 0}}
\urlstyle{same}  % don't use monospace font for urls
% Make links footnotes instead of hotlinks:
\renewcommand{\href}[2]{#2\footnote{\url{#1}}}
\setlength{\parindent}{0pt}
\setlength{\parskip}{6pt plus 2pt minus 1pt}
\setlength{\emergencystretch}{3em}  % prevent overfull lines
\setcounter{secnumdepth}{0}

\title{Practical Dependent Types in Haskell: Type-Safe Neural Networks (Part 2)}
\author{Justin Le}

\begin{document}
\maketitle

\emph{Originally posted on
\textbf{\href{https://blog.jle.im/entry/practical-dependent-types-in-haskell-2.html}{in
Code}}.}

We're back to continue on
\href{https://blog.jle.im/entries/series/+practical-dependent-types-in-haskell.html}{our
journey} in using practical dependent types to write type-safe neural networks!
In
\href{https://blog.jle.im/entry/practical-dependent-types-in-haskell-1.html}{Part
1}, we wrote things out in normal, untyped Haskell, and looked at red flags and
general design principles that nudged us in the direction of adding dependent
types to our program. We learned to appreciate what dependent types offered in
terms of guiding us in writing our code, helping the compiler check our
correctness, providing a better interface for users, and more.

We also learned how to use singletons to work around some of Haskell's
fundamental limitations to let us ``pattern match'' on the structure of types,
and how to use typeclasses to generate singletons reflecting the structure of
types we are dealing with.

(If you read
\href{https://blog.jle.im/entry/practical-dependent-types-in-haskell-1.html}{Part
1} \emph{before} the singletons section was re-written to use the
\href{https://hackage.haskell.org/package/singletons}{singletons} library,
\href{https://blog.jle.im/entry/practical-dependent-types-in-haskell-1.html\#singletons-and-induction}{here's
a link to the section} in specific. This tutorial will assume familiarity with
what is discussed there!)

All of what we've dealt with so far has essentially been with types that are
fixed at compile-time. All the networks we've made have had ``static'' types,
with their sizes in their types indicated directly in the source code.

Today, we're going to dive into the world of types that \emph{depend} on factors
unknown until runtime, and see how dependent types in a strongly typed language
like Haskell helps us write safer, more correct, and more maintainable code.
Along the way, we'll encounter and learn first-hand about techniques and guiding
high-level principles that we can apply to our other dependently typed coding
endeavours.

\section{Serializing Networks}\label{serializing-networks}

To warm up, let's talk about serializing networks: writing them to binary and
re-reading them.

\subsection{Binary}\label{binary}

Serializing networks of \emph{known} size --- whose sizes are statically in
their types --- is pretty straightforward. I'm going to be using the
\emph{\href{https://hackage.haskell.org/package/binary}{binary}} library, which
offers a very standard typeclass-based approach for serializing and
deserializing data. There are a lot of tutorials online (and I even
\href{https://blog.jle.im/entry/streaming-huffman-compression-in-haskell-part-2-binary.html}{wrote
a small one} myself a few years ago), but a very high-level view is that the
library offers a typeclass for describing serialization schemes for different
types.

For example, for serializing lists, you have:

\begin{Shaded}
\begin{Highlighting}[]
\OtherTok{putList ::} \DataTypeTok{Binary} \NormalTok{a }\OtherTok{=>} \NormalTok{[a] }\OtherTok{->} \DataTypeTok{Put}
\NormalTok{putList []     }\FunctionTok{=}
    \NormalTok{put }\DataTypeTok{False}           \CommentTok{-- signal the end}
\NormalTok{putList (x}\FunctionTok{:}\NormalTok{xs) }\FunctionTok{=} \KeywordTok{do}
    \NormalTok{put }\DataTypeTok{True}            \CommentTok{-- signal a cons}
    \NormalTok{put x}
    \NormalTok{putList xs}
\end{Highlighting}
\end{Shaded}

Where \texttt{put\ ::\ Binary\ a\ =\textgreater{}\ a\ -\textgreater{}\ Put} is a
polymorphic way of describing serialization of things with a \texttt{Binary}
instance --- every type provides its own \texttt{put}. Sequencing \texttt{Put}s
using do notation is saying ``put this, then that, then that''.

We can deserialize by describing a monadic \texttt{Get} plan:

\begin{Shaded}
\begin{Highlighting}[]
\OtherTok{getList ::} \DataTypeTok{Binary} \NormalTok{a }\OtherTok{=>} \DataTypeTok{Get} \NormalTok{[a]}
\NormalTok{getList }\FunctionTok{=} \KeywordTok{do}
    \NormalTok{isCons }\OtherTok{<-} \NormalTok{get}
    \KeywordTok{if} \NormalTok{isCons}
      \KeywordTok{then} \KeywordTok{do}
        \NormalTok{x  }\OtherTok{<-} \NormalTok{get}
        \NormalTok{xs }\OtherTok{<-} \NormalTok{getList}
        \NormalTok{return (x}\FunctionTok{:}\NormalTok{xs)}
      \KeywordTok{else}
        \NormalTok{return []}
\end{Highlighting}
\end{Shaded}

And \texttt{get\ ::\ Binary\ a\ =\textgreater{}\ Get\ a} is a polymorphic way of
describing a strategy of deserializing things with a \texttt{Binary} instance
--- again, every type provides its own \texttt{get}. Sequencing \texttt{Get}s is
again simply ``do-this-then-that''.

We can write our own instances for our own types by providing \texttt{put} and
\texttt{get} together:

\begin{Shaded}
\begin{Highlighting}[]
\KeywordTok{instance} \DataTypeTok{Binary} \NormalTok{a }\OtherTok{=>} \DataTypeTok{Binary} \NormalTok{[a] }\KeywordTok{where}
    \NormalTok{put }\FunctionTok{=} \NormalTok{putList}
    \NormalTok{get }\FunctionTok{=} \NormalTok{getList}
\end{Highlighting}
\end{Shaded}

In practice, we usually don't write our own instances from scratch, and use
GHC's generics features to give us instances for free:

\begin{Shaded}
\begin{Highlighting}[]
\CommentTok{-- source: https://github.com/mstksg/inCode/tree/master/code-samples/dependent-haskell/NetworkTyped2.hs#L16-30}
\KeywordTok{data} \DataTypeTok{Weights} \NormalTok{i o }\FunctionTok{=} \DataTypeTok{W} \NormalTok{\{}\OtherTok{ wBiases ::} \FunctionTok{!}\NormalTok{(}\DataTypeTok{R} \NormalTok{o)}
                     \NormalTok{,}\OtherTok{ wNodes  ::} \FunctionTok{!}\NormalTok{(}\DataTypeTok{L} \NormalTok{o i)}
                     \NormalTok{\}}
  \KeywordTok{deriving} \NormalTok{(}\DataTypeTok{Generic}\NormalTok{)}

\KeywordTok{instance} \NormalTok{(}\DataTypeTok{KnownNat} \NormalTok{i, }\DataTypeTok{KnownNat} \NormalTok{o) }\OtherTok{=>} \DataTypeTok{Binary} \NormalTok{(}\DataTypeTok{Weights} \NormalTok{i o)}
\end{Highlighting}
\end{Shaded}

For simple types like \texttt{Weights}, which simply ``contain'' serializable
things, the \emph{binary} is smart enough to write your instances automatically!

\subsection{\texorpdfstring{\texttt{Binary} for
\texttt{Network}}{Binary for Network}}\label{binary-for-network}

Writing \texttt{putNet} and \texttt{getNet} is pretty nice because the entire
structure is already known ahead of time, and we don't need to do any tricks
with flags like for lists.

\begin{Shaded}
\begin{Highlighting}[]
\CommentTok{-- source: https://github.com/mstksg/inCode/tree/master/code-samples/dependent-haskell/NetworkTyped2.hs#L32-36}
\OtherTok{putNet ::} \NormalTok{(}\DataTypeTok{KnownNat} \NormalTok{i, }\DataTypeTok{KnownNat} \NormalTok{o)}
       \OtherTok{=>} \DataTypeTok{Network} \NormalTok{i hs o}
       \OtherTok{->} \DataTypeTok{Put}
\NormalTok{putNet }\FunctionTok{=} \NormalTok{\textbackslash{}}\KeywordTok{case} \DataTypeTok{O} \NormalTok{w     }\OtherTok{->} \NormalTok{put w}
               \NormalTok{w }\FunctionTok{:&~} \NormalTok{n }\OtherTok{->} \NormalTok{put w }\FunctionTok{*>} \NormalTok{putNet n}
\end{Highlighting}
\end{Shaded}

Even simpler than for lists! If it's an \texttt{O\ w}, just serialize the
\texttt{w}. If it's a \texttt{w\ :\&\textasciitilde{}\ net}, serialize the
\texttt{w} then the rest of the \texttt{net}. The reason we can get away without
any flags is because we already \emph{know} how many
\texttt{:\&\textasciitilde{}} layers to expect \emph{just from the type}. If we
want to deserialize/load a
\texttt{Network\ 5\ \textquotesingle{}{[}10,6,3{]}\ 2}, we \emph{know} we want
three \texttt{(:\&\textasciitilde{})}'s and one \texttt{O} --- no need for
dynamically sized networks like we had to handle for lists.

We'll write \texttt{getNet} the similarly to how wrote
\href{https://github.com/mstksg/inCode/tree/master/code-samples/dependent-haskell/NetworkTyped.hs\#L67-75}{\texttt{randomNet}}
from the last post:

\begin{Shaded}
\begin{Highlighting}[]
\CommentTok{-- source: https://github.com/mstksg/inCode/tree/master/code-samples/dependent-haskell/NetworkTyped2.hs#L38-42}
\OtherTok{getNet ::} \NormalTok{forall i hs o}\FunctionTok{.} \NormalTok{(}\DataTypeTok{KnownNat} \NormalTok{i, }\DataTypeTok{KnownNat} \NormalTok{o)}
       \OtherTok{=>} \DataTypeTok{Sing} \NormalTok{hs}
       \OtherTok{->} \DataTypeTok{Get} \NormalTok{(}\DataTypeTok{Network} \NormalTok{i hs o)}
\NormalTok{getNet }\FunctionTok{=} \NormalTok{\textbackslash{}}\KeywordTok{case} \DataTypeTok{SNil}            \OtherTok{->}     \DataTypeTok{O} \FunctionTok{<$>} \NormalTok{get}
               \DataTypeTok{SNat} \OtherTok{`SCons`} \NormalTok{ss }\OtherTok{->} \NormalTok{(}\FunctionTok{:&~}\NormalTok{) }\FunctionTok{<$>} \NormalTok{get }\FunctionTok{<*>} \NormalTok{getNet ss}
\end{Highlighting}
\end{Shaded}

Now would be a good time to refresh on the
\href{https://blog.jle.im/entry/practical-dependent-types-in-haskell-1.html\#singletons-and-induction}{singletons
section} of my last post again. To deserialize a \texttt{Network\ i\ hs\ o}, we
have to ``pattern match'' on \texttt{hs} to see what constructor we are
expecting to deserialize, and we do that by using singletons with constructors
we \emph{can} literally pattern match on (\texttt{SNil} and \texttt{SCons}),
which tell GHC what type to expect through ``dependent pattern matching''.

If you see a \texttt{SNil\ ::\ Sing\ \textquotesingle{}{[}{]}}, that means that
\texttt{hs} is \texttt{\textquotesingle{}{[}{]}}, so expect a \texttt{O}
constructor. If you see
\texttt{SCons\ s\ ss\ ::\ Sing\ (h\ \textquotesingle{}:\ hs)}, that means that
\texttt{hs} is \texttt{h\ \textquotesingle{}:\ h\textquotesingle{}}, so expect a
\texttt{(:\&\textasciitilde{})} constructor. (If this is a little confusing
still, try re-reading the
\href{https://blog.jle.im/entry/practical-dependent-types-in-haskell-1.html\#singletons-and-induction}{singleton
section} from the last post again for a more thorough description!)

Note that here we decide to implement \texttt{getNet} by asking for an explicit
singleton input (\texttt{Sing\ hs\ -\textgreater{}}) instead of an implicit one
(\texttt{SingI\ hs\ =\textgreater{}}) like we did for \texttt{randomNet}.
Remember that the two methods are technically equivalent, really, and compile to
the same thing at runtime. We need one or the other because of type erasure ---
so either we pass in \texttt{Sing\ hs}, or provide a \texttt{SingI\ hs}
constraint so that we can use \texttt{sing\ ::\ Sing\ hs} to construct the
\texttt{Sing\ hs}.

There's a trade-off either way, and it can be a bit annoying because switching
between different modes can potentially be diverse. For the most part, always
try to \emph{take explicit singleton arguments} where you can, especially for
internal functions. The simple reason is because in Haskell, we like to really
only do typeclass-level programming as a last, last resort. Typeclasses in
Haskell are very magical and not really first-class in a satisfying way. Normal
values (like singletons) \emph{are} first-class and easily passed.

Explicit singleton arguments can sometimes pose a burden for the caller, so my
personal approach is to \emph{always} use explicit \texttt{Sing\ a} whenever
possible for \emph{internal functions}, and to expose a
\texttt{SingI\ a\ =\textgreater{}} interface for \emph{user-facing functions}
(including for typeclass instances)

So, we're going to write our \texttt{Binary} instance for \texttt{Network}. Of
course, we can't have \texttt{put} or \texttt{get} take a \texttt{Sing\ hs}
(that'd change the arity/type of the function), so what we can do is have their
\texttt{Binary} instances require a \texttt{SingI\ hs} constraint, essentially
doing the same thing:

\begin{Shaded}
\begin{Highlighting}[]
\CommentTok{-- source: https://github.com/mstksg/inCode/tree/master/code-samples/dependent-haskell/NetworkTyped2.hs#L44-46}
\KeywordTok{instance} \NormalTok{(}\DataTypeTok{KnownNat} \NormalTok{i, }\DataTypeTok{SingI} \NormalTok{hs, }\DataTypeTok{KnownNat} \NormalTok{o) }\OtherTok{=>} \DataTypeTok{Binary} \NormalTok{(}\DataTypeTok{Network} \NormalTok{i hs o) }\KeywordTok{where}
    \NormalTok{put }\FunctionTok{=} \NormalTok{putNet}
    \NormalTok{get }\FunctionTok{=} \NormalTok{getNet sing}
\end{Highlighting}
\end{Shaded}

To go from ``\texttt{SingI} world'' to ``\texttt{Sing} world'', we use
\texttt{sing} to generate the explicit \texttt{Sing\ hs} from
\texttt{SingI\ hs\ =\textgreater{}}.

We can go ``backwards'' too by using
\texttt{withSingI\ ::\ Sing\ a\ -\textgreater{}\ (SingI\ a\ =\textgreater{}\ r)\ -\textgreater{}\ r}:

\begin{Shaded}
\begin{Highlighting}[]
\CommentTok{-- source: https://github.com/mstksg/inCode/tree/master/code-samples/dependent-haskell/NetworkTyped2.hs#L48-51}
\OtherTok{sillyGetNet ::} \NormalTok{(}\DataTypeTok{KnownNat} \NormalTok{i, }\DataTypeTok{KnownNat} \NormalTok{o)}
            \OtherTok{=>} \DataTypeTok{Sing} \NormalTok{hs}
            \OtherTok{->} \DataTypeTok{Get} \NormalTok{(}\DataTypeTok{Network} \NormalTok{i hs o)}
\NormalTok{sillyGetNet ss }\FunctionTok{=} \NormalTok{withSingI ss get}
\end{Highlighting}
\end{Shaded}

\texttt{withSingI} takes a \texttt{Sing\ a} and a ``thing you can evaluate if
only you had a \texttt{SingI\ a} instance available'', and evaluates it. We can
only evaluate
\texttt{get\ ::\ (KnownNat\ i,\ SingI\ hs,\ KnownNat\ o)\ =\textgreater{}\ Get\ (Network\ i\ hs\ o)}
if we have that \texttt{SingI} instance, so we can pass it into
\texttt{withSingI} as the second argument, and, voilà --- we get that
\texttt{Get\ (Network\ i\ hs\ o)} right out.

\end{document}
