\documentclass[]{article}
\usepackage{lmodern}
\usepackage{amssymb,amsmath}
\usepackage{ifxetex,ifluatex}
\usepackage{fixltx2e} % provides \textsubscript
\ifnum 0\ifxetex 1\fi\ifluatex 1\fi=0 % if pdftex
  \usepackage[T1]{fontenc}
  \usepackage[utf8]{inputenc}
\else % if luatex or xelatex
  \ifxetex
    \usepackage{mathspec}
    \usepackage{xltxtra,xunicode}
  \else
    \usepackage{fontspec}
  \fi
  \defaultfontfeatures{Mapping=tex-text,Scale=MatchLowercase}
  \newcommand{\euro}{€}
\fi
% use upquote if available, for straight quotes in verbatim environments
\IfFileExists{upquote.sty}{\usepackage{upquote}}{}
% use microtype if available
\IfFileExists{microtype.sty}{\usepackage{microtype}}{}
\usepackage[margin=1in]{geometry}
\usepackage{graphicx}
\makeatletter
\def\maxwidth{\ifdim\Gin@nat@width>\linewidth\linewidth\else\Gin@nat@width\fi}
\def\maxheight{\ifdim\Gin@nat@height>\textheight\textheight\else\Gin@nat@height\fi}
\makeatother
% Scale images if necessary, so that they will not overflow the page
% margins by default, and it is still possible to overwrite the defaults
% using explicit options in \includegraphics[width, height, ...]{}
\setkeys{Gin}{width=\maxwidth,height=\maxheight,keepaspectratio}
\ifxetex
  \usepackage[setpagesize=false, % page size defined by xetex
              unicode=false, % unicode breaks when used with xetex
              xetex]{hyperref}
\else
  \usepackage[unicode=true]{hyperref}
\fi
\hypersetup{breaklinks=true,
            bookmarks=true,
            pdfauthor={Justin Le},
            pdftitle={Hamiltonian Dynamics in Haskell},
            colorlinks=true,
            citecolor=blue,
            urlcolor=blue,
            linkcolor=magenta,
            pdfborder={0 0 0}}
\urlstyle{same}  % don't use monospace font for urls
% Make links footnotes instead of hotlinks:
\renewcommand{\href}[2]{#2\footnote{\url{#1}}}
\setlength{\parindent}{0pt}
\setlength{\parskip}{6pt plus 2pt minus 1pt}
\setlength{\emergencystretch}{3em}  % prevent overfull lines
\setcounter{secnumdepth}{0}

\title{Hamiltonian Dynamics in Haskell}
\author{Justin Le}

\begin{document}
\maketitle

\emph{Originally posted on
\textbf{\href{https://blog.jle.im/entry/hamiltonian-dynamics-in-haskell.html}{in
Code}}.}

As promised in my
\href{https://blog.jle.im/entry/introducing-the-hamilton-library.html}{\emph{hamilton}
introduction post}, I'm going to go over implementing of the
\emph{\href{http://hackage.haskell.org/package/hamilton}{hamilton}} library
using \emph{\href{http://hackage.haskell.org/package/ad}{ad}} and dependent
types.

This post will be a bit heavy in some mathematics and Haskell concepts. The
expected audience is intermediate Haskell programmers, and no previous knowledge
of dependent types is expected.

The mathematics and physics are ``extra'' flavor text and could potentially be
skipped, but you'll get the most out of this article if you have basic
familiarity with:

\begin{enumerate}
\def\labelenumi{\arabic{enumi}.}
\tightlist
\item
  Basic concepts of multivariable calculus (like partial and total derivatives).
\item
  Concepts of linear algebra (like dot products, matrix multiplication, and
  matrix inverses)
\end{enumerate}

No physics knowledge is assumed, but knowing a little bit of first semester
physics would help you gain a bit more of an appreciation for the end result!

\section{Hamiltonian Mechanics}\label{hamiltonian-mechanics}

As mentioned in the previous post, Hamiltonian mechanics is a re-imagining of
dynamics and mechanics (think ``the world post-\(F = m a\)'') that not only
opened up new doors to solving problems in classical, but also ended up being
the right angle of viewing the world to unlock statistical mechanics and
thermodynamics, and later even quantum mechanics.

Hamiltonian mechanics lets you parameterize your system's ``position'' in
arbitrary ways (like the angle of rotation, for pendulum problems) and then
posits that the full state of the system exists in something called \emph{phase
space}, and that the system's dynamics is its motion through phase space that is
dictated by the geometry of the \emph{Hamiltonian} of that phase space.

The system's \emph{Hamiltonian} is a \(\mathbb{R}^{2n} \rightarrow \mathbb{R}\)
function on the phase space (\(\mathbb{R}^{2n}\), where \(n\) is the number of
coordinates parameterizing your system) to \(\mathbb{R}\). And, for a
time-independent system, the picture is quite simple: the system moves along the
\emph{contour lines} of the \emph{Hamiltonian} -- the lines of equal ``height''.

\begin{figure}[htbp]
\centering
\includegraphics{/img/entries/hamilton/contour-lines.jpg}
\caption{Example of contour lines of a \(\mathbb{R}^2 \rightarrow \mathbb{R}\)
function -- the elevation of land. From the
\href{https://www.ordnancesurvey.co.uk/blog/2015/11/map-reading-skills-making-sense-of-contour-lines/}{Ordinace
Survey} website.}
\end{figure}

In the example above, if we imagine that phase space is the 2D location, then
the \emph{Hamiltonian} is the mountain. And for a system dropped anywhere on the
mountain, its motion would be along the contour lines. For example, if a system
started somewhere along the 10 contour line, it would begin to oscillate the
entire phase space along the 10 contour line.\footnote{The picture with a
  time-dependent Hamiltonian is different, but only slightly. In the
  time-dependent case, the system still \emph{tries} to move along contour lines
  at every point in time, but the mountain is constantly changing underneath it
  and the contour lines keep on shifting underneath it. Sounds like life!}

\emph{Every} \href{https://en.wikipedia.org/wiki/Smooth_jazz}{smooth}
\(\mathbb{R}^{2n} \rightarrow \mathbb{R}\) function on phase space can be used
as a Hamiltonian to describe the physics of some system. So, given any
``mountain range'' on phase space, any ``elevation map'' or real-valued function
on phase space, you can treat it as a description of the dynamics of some
physical system.

The \emph{trick}, then, to using Hamiltonian dynamics to model your system, is:

\begin{enumerate}
\def\labelenumi{\arabic{enumi}.}
\item
  Finding the phase space to describe your system. This can be done based on any
  continuous parameterization of your system (``generalized coordinates''), like
  angles of pendulums and so on.
\item
  Finding the Hamiltonian on that phase space to describe your system.
\end{enumerate}

And then Hamilton's dynamics will give you the rest! All you do is ``follow the
contour lines'' on that Hamiltonian!

\subsection{Phase Space}\label{phase-space}

I've sort of hand-waved away describing what phase space is. The only thing I've
really said in detail is that if your system's state has \(n\) parameters, then
the corresponding phase space is \(2n\)-dimensional (and that Hamiltonian
mechanics is somehow about systems moving around in phase space).

\emph{Phase space} is a \(2n\)-dimensional space consisting of:

\begin{enumerate}
\def\labelenumi{\arabic{enumi}.}
\tightlist
\item
  All of the current values of the \(n\) parameters (``generalized
  coordinates'')
\item
  All of the current ``generalized momenta'' of those \(n\) parameters
\end{enumerate}

So if you were parameterizing your pendulum system by, say, the angle of the
pendulum, the phase space would be the current angle of the pendulum along with
the current ``generalized momentum'' associated with the angle of the pendulum.

What exactly \emph{is} generalized momentum? Well, we'll go over calculating it
eventually. But what does it represent\ldots{}\emph{physically}?

I could give you some spiel here about the underlying Lie algebra of the Lie
group associated with the generalized coordinates, but I don't think that it
would be very intuitively appealing in a physical sense. It'd also be out of the
scope of the math prerequisites that I promised I'd stick to going into this
post!

But, what I can say is that the generalized momenta associated with (``conjugate
to'') certain sets of familiar coordinates yield things that we typically call
``momenta'':

\begin{enumerate}
\def\labelenumi{\arabic{enumi}.}
\item
  The momentum conjugate to normal Cartesian coordinates is just our normal
  run-of-the-mill \emph{linear momentum} (in the \(\mathbf{p} = m \mathbf{v}\))
  from first semester physics.
\item
  The momentum conjugate to the angle \(\theta\) in polar coordinates is
  \emph{angular momentum} (\(\mathbf{l} = m \mathbf{r} \times \mathbf{v}\), or
  \(\mathbf{l} = m r \mathbf{\theta}\)) from first semester physics.
\end{enumerate}

So maybe this can help you feel comfortable with calling it ``generalized
momenta'', in the sense that it's our normal momentum (for linear and polar
coordinates) generalized to arbitrary coordinates.

\subsection{Hamiltonian Dynamics}\label{hamiltonian-dynamics}

I've explained Hamiltonian dynamics for time-independent Hamiltonians as
``follow the contour lines'', but I didn't really say how quickly to move along
the contour lines and in what direction (clockwise, or counter-clockwise? left,
or right?). The actual equations of motion are:

\[
\dot{q} = \frac{\partial}{\partial p_q} \mathcal{H}(\mathbf{q},\mathbf{p})
\]

\[
\dot{p_q} = - \frac{\partial}{\partial q} \mathcal{H}(\mathbf{q},\mathbf{p})
\]

Which holds for every generalized coordinate \(q\), where \(p_q\) is the
momentum conjugate to that coordinate. \(\mathcal{H}\) is the Hamiltonian
function, \(\dot{q}\) is the rate of change of \(q\), and \(\dot{p_q}\) is the
rate of change of \(p_q\).

Essentially, these give you ``updating functions'' for \(q\) and \(p_q\) --
given \(\mathcal{H}(\mathbf{q},\mathbf{p})\), you have a way to ``update'' the
particle's position in phase space. Just take the partial derivatives of
\(\mathcal{H}\) at every step in time!

\end{document}
