\documentclass[]{article}
\usepackage{lmodern}
\usepackage{amssymb,amsmath}
\usepackage{ifxetex,ifluatex}
\usepackage{fixltx2e} % provides \textsubscript
\ifnum 0\ifxetex 1\fi\ifluatex 1\fi=0 % if pdftex
  \usepackage[T1]{fontenc}
  \usepackage[utf8]{inputenc}
\else % if luatex or xelatex
  \ifxetex
    \usepackage{mathspec}
    \usepackage{xltxtra,xunicode}
  \else
    \usepackage{fontspec}
  \fi
  \defaultfontfeatures{Mapping=tex-text,Scale=MatchLowercase}
  \newcommand{\euro}{€}
\fi
% use upquote if available, for straight quotes in verbatim environments
\IfFileExists{upquote.sty}{\usepackage{upquote}}{}
% use microtype if available
\IfFileExists{microtype.sty}{\usepackage{microtype}}{}
\usepackage[margin=1in]{geometry}
\usepackage{graphicx}
\makeatletter
\def\maxwidth{\ifdim\Gin@nat@width>\linewidth\linewidth\else\Gin@nat@width\fi}
\def\maxheight{\ifdim\Gin@nat@height>\textheight\textheight\else\Gin@nat@height\fi}
\makeatother
% Scale images if necessary, so that they will not overflow the page
% margins by default, and it is still possible to overwrite the defaults
% using explicit options in \includegraphics[width, height, ...]{}
\setkeys{Gin}{width=\maxwidth,height=\maxheight,keepaspectratio}
\ifxetex
  \usepackage[setpagesize=false, % page size defined by xetex
              unicode=false, % unicode breaks when used with xetex
              xetex]{hyperref}
\else
  \usepackage[unicode=true]{hyperref}
\fi
\hypersetup{breaklinks=true,
            bookmarks=true,
            pdfauthor={Justin Le},
            pdftitle={Hamiltonian Dynamics in Haskell},
            colorlinks=true,
            citecolor=blue,
            urlcolor=blue,
            linkcolor=magenta,
            pdfborder={0 0 0}}
\urlstyle{same}  % don't use monospace font for urls
% Make links footnotes instead of hotlinks:
\renewcommand{\href}[2]{#2\footnote{\url{#1}}}
\setlength{\parindent}{0pt}
\setlength{\parskip}{6pt plus 2pt minus 1pt}
\setlength{\emergencystretch}{3em}  % prevent overfull lines
\setcounter{secnumdepth}{0}

\title{Hamiltonian Dynamics in Haskell}
\author{Justin Le}

\begin{document}
\maketitle

\emph{Originally posted on
\textbf{\href{https://blog.jle.im/entry/hamiltonian-dynamics-in-haskell.html}{in
Code}}.}

As promised in my
\href{https://blog.jle.im/entry/introducing-the-hamilton-library.html}{\emph{hamilton}
introduction post}, I'm going to go over implementing of the
\emph{\href{http://hackage.haskell.org/package/hamilton}{hamilton}} library
using \emph{\href{http://hackage.haskell.org/package/ad}{ad}} and dependent
types.

This post will be a bit heavy in some mathematics and Haskell concepts. The
expected audience is intermediate Haskell programmers, and no previous knowledge
of dependent types is expected.

The mathematics and physics are ``extra'' flavor text and could potentially be
skipped, but you'll get the most out of this article if you have basic
familiarity with:

\begin{enumerate}
\def\labelenumi{\arabic{enumi}.}
\tightlist
\item
  Basic concepts of multivariable calculus (like partial and total derivatives).
\item
  Concepts of linear algebra (like dot products, matrix multiplication, and
  matrix inverses)
\end{enumerate}

No physics knowledge is assumed, but knowing a little bit of first semester
physics would help you gain a bit more of an appreciation for the end result!

\section{Hamiltonian Mechanics}\label{hamiltonian-mechanics}

As mentioned in the previous post, Hamiltonian mechanics is a re-imagining of
dynamics and mechanics (think ``the world post-\(F = m a\)'') that not only
opened up new doors to solving problems in classical, but also ended up being
the right angle of viewing the world to unlock statistical mechanics and
thermodynamics, and later even quantum mechanics.

Hamiltonian mechanics lets you parameterize your system's ``position'' in
arbitrary ways (like the angle of rotation, for pendulum problems) and then
posits that the full state of the system exists in something called \emph{phase
space}, and that the system's dynamics is its motion through phase space that is
dictated by the geometry of the \emph{Hamiltonian} of that phase space.

The system's \emph{Hamiltonian} is a \(\mathbb{R}^{2n} \rightarrow \mathbb{R}\)
function on phase space (where \(n\) is the number of coordinates parameterizing
your system) to \(\mathbb{R}\). For a time-independent system, the picture of
the dynamics is pretty simple: the system moves along the \emph{contour lines}
of the \emph{Hamiltonian} -- the lines of equal ``height''.

\begin{figure}[htbp]
\centering
\includegraphics{/img/entries/hamilton/contour-lines.jpg}
\caption{Example of contour lines of a \(\mathbb{R}^2 \rightarrow \mathbb{R}\)
function -- the elevation of land. From the
\href{https://www.ordnancesurvey.co.uk/blog/2015/11/map-reading-skills-making-sense-of-contour-lines/}{Ordinace
Survey} website.}
\end{figure}

In the example above, if we imagine that phase space is the 2D location, then
the \emph{Hamiltonian} is the mountain. And for a system dropped anywhere on the
mountain, its motion would be along the contour lines. For example, if a system
started somewhere along the 10 contour line, it would begin to oscillate the
entire phase space along the 10 contour line.\footnote{The picture with a
  time-dependent Hamiltonian is different, but only slightly. In the
  time-dependent case, the system still \emph{tries} to move along contour lines
  at every point in time, but the mountain is constantly changing underneath it
  and the contour lines keep on shifting underneath it. Sounds like life!}

\emph{Every} \href{https://en.wikipedia.org/wiki/Smooth_jazz}{smooth}
\(\mathbb{R}^{2n} \rightarrow \mathbb{R}\) function on phase space can be used
as a Hamiltonian to describe the physics of some system. So, given any
``mountain range'' on phase space, any ``elevation map'' or real-valued function
on phase space, you can treat it as a description of the dynamics of some
physical system.

The \emph{trick}, then, to using Hamiltonian dynamics to model your system, is:

\begin{enumerate}
\def\labelenumi{\arabic{enumi}.}
\item
  Finding the phase space to describe your system. This can be done based on any
  continuous parameterization of your system (``generalized coordinates''), like
  angles of pendulums and so on.
\item
  Finding the Hamiltonian on that phase space to describe your system.
\end{enumerate}

And then Hamilton's dynamics will give you the rest! All you do is ``follow the
contour lines'' on that Hamiltonian!

\subsection{Phase Space}\label{phase-space}

So far the only thing I've really said in detail is that if your system's state
has \(n\) parameters, then the corresponding phase space is \(2n\)-dimensional
(and that Hamiltonian mechanics is somehow about systems moving around in phase
space). \emph{Phase space} is a \(2n\)-dimensional space parameterized by:

\begin{enumerate}
\def\labelenumi{\arabic{enumi}.}
\tightlist
\item
  All of the current values of the \(n\) parameters (``generalized
  coordinates'')
\item
  All of the current ``generalized momenta'' of those \(n\) parameters
\end{enumerate}

So if you were parameterizing your pendulum system by, say, the angle of the
pendulum, the phase space would be the current angle of the pendulum along with
the current ``generalized momentum'' associated with the angle of the pendulum.
What exactly \emph{is} generalized momentum? We'll go over calculating it
eventually, but what does it represent\ldots{}\emph{physically}?

I could give you some spiel here about the underlying Lie algebra of the Lie
group associated with the generalized coordinates, but I don't think that it
would be very intuitively appealing in a physical sense. But what I \emph{can}
say is that the generalized momenta associated with (``conjugate to'') certain
sets of familiar coordinates yield things that we typically call ``momenta'':

\begin{enumerate}
\def\labelenumi{\arabic{enumi}.}
\item
  The momentum conjugate to normal Cartesian coordinates is just our normal
  run-of-the-mill \emph{linear momentum} (in the \(\mathbf{p} = m \mathbf{v}\))
  from first semester physics.
\item
  The momentum conjugate to the angle \(\theta\) in polar coordinates is
  \emph{angular momentum} (\(\mathbf{L} = m \mathbf{r} \times \mathbf{v}\), or
  \(L = m r^2 \dot{\theta}\)) from first semester physics.

  The momentum conjugate to the radial coordinate \(r\) in polar coordinates is
  also just boring old linear momentum \(p_r = m \dot{r}\).
\end{enumerate}

So, it's our normal momentum (for linear and polar coordinates)
\emph{generalized} to arbitrary coordinates.

\subsection{Hamiltonian Dynamics}\label{hamiltonian-dynamics}

I've explained Hamiltonian dynamics for time-independent Hamiltonians as
``follow the contour lines''. If you remember your basic multi-variable calculus
course, you'll know that the line of ``steepest ascent'' is the gradient. If we
call the Hamiltonian \(\mathcal{H}(\mathbf{q},\mathbf{p})\) (where
\(\mathbf{q}\) is the vector of positions and \(\mathbf{p}\) is the vector of
momenta), then the direction of steepest ascent is
\(\left \langle \frac{\partial}{\partial \mathbf{q}} \mathcal{H}(\mathbf{q},\mathbf{p}), \frac{\partial}{\partial \mathbf{p}} \mathcal{H}(\mathbf{q},\mathbf{p}) \right \rangle\).

But we want to move along the \emph{contour lines}\ldots{}and these are the
lines \emph{perpendicular} to the direction of steepest descent. The vector
perpendicular to \(\langle x, y \rangle\) is \(\langle y, -x \rangle\), so we
just derived the actual Hamiltonian equations of motion: just move in the
direction perpendicular to the steepest descent!

\[
\dot{q} = \frac{\partial}{\partial p_q} \mathcal{H}(\mathbf{q},\mathbf{p})
\]

\[
\dot{p_q} = - \frac{\partial}{\partial q} \mathcal{H}(\mathbf{q},\mathbf{p})
\]

Which holds for every generalized coordinate \(q\), where \(p_q\) is the
momentum conjugate to that coordinate.

Essentially, these give you ``updating functions'' for \(q\) and \(p_q\) --
given \(\mathcal{H}(\mathbf{q},\mathbf{p})\), you have a way to ``update'' the
particle's position in phase space. Just take the partial derivatives of
\(\mathcal{H}\) at every step in time! To update \(q\), nudge it by
\(\frac{\partial}{\partial p_q} \mathcal{H}(\mathbf{q},\mathbf{p})\). To update
\(p_q\), nudge it by
\(\frac{\partial}{\partial q} \mathcal{H}(\mathbf{q},\mathbf{p})\)!

This picture is appealing to me in a visceral way because it sort of seems like
the system is ``surfing'' along the Hamiltonian's contour lines. It's being
``pushed'' \emph{faster} when the Hamiltonian is steeper, and slower when it's
more shallow. If you have super steep Hamiltonians, you'll just zip right
through phase space, but if you have shallow Hamiltonians, you'll take a comfy
glide along that contour line you're on. I can apply all my intuition as a
surfer\footnote{Disclaimer: I am not a surfer} to Hamiltonian mechanics!

\section{Hamiltonian Dynamics and Physical
Systems}\label{hamiltonian-dynamics-and-physical-systems}

Earlier I mentioned that the two steps for applying Hamiltonian mechanics to
your system was figuring out your system's conjugate momenta and the appropriate
Hamiltonian. Let's get onto that. I'm going to make a couple of simplifying
assumptions that make the job easier for the purposes of this article:

\begin{enumerate}
\def\labelenumi{\arabic{enumi}.}
\tightlist
\item
  Your coordinates and potential energy are time-independent.
\item
  Your potential energy function only depends on \emph{positions}, and not
  \emph{velocities}. (So nothing like friction or wind resistance or magnetic
  field vector potentials)
\end{enumerate}

With these assumptions, I'm going to skip over discussing the
\href{https://en.wikipedia.org/wiki/Lagrangian_mechanics}{Lagrangian} of the
system, which is the traditional way to do this. You can think of this section
as me presenting derived conclusions and skipping the derivations.

\subsection{Conjugate Momenta}\label{conjugate-momenta}

We'll define the momentum conjugate to coordinate \(q\) as

\[
\frac{\partial}{\partial \dot{q}} KE(\dot{\mathbf{q}})
\]

Where \(KE(\dot{\mathbf{q}))\) is the kinetic energy of the system, which is an
explicit function on the rate of changes of the coordinates. For example, for
normal Cartesian coordinates in one dimension,
\(KE(\dot{x}) = \frac{1}{2} m \dot{x}^2\). So the momentum conjugate to \(x\)
is:

\[
p_x = \frac{\partial}{\partial \dot{x}} \frac{1}{2} m \dot{x}^2 = m \dot{x}
\]

Just linear momentum, like I claimed before!

Alright, now let's generalize this to arbitrary coordinates. In general, for
\emph{Cartesian} coordinates, the kinetic energy will always be

\[
KE(\dot{\mathbf{x}}) = \frac{1}{2} \left[ m_1 x_1^2 + m_2 x_2^2 + m_3 x_3^2 \dots \right]
\]

Where \(m\) is the inertia associated with each coordinate\ldots{}for example,
if there's an object of mass \(m\) at \(\langle x_1, x_2 \rangle\), then
\(m_1 = m_2 = m\).

To make things more convenient, we'll treat this as a quadratic form over an
inertia matrix:

\[
KE(\dot{\mathbf{x}}) = \frac{1}{2} \mathbf{x}^T \hat{M} \mathbf{x}
\]

\end{document}
