\documentclass[]{article}
\usepackage{lmodern}
\usepackage{amssymb,amsmath}
\usepackage{ifxetex,ifluatex}
\usepackage{fixltx2e} % provides \textsubscript
\ifnum 0\ifxetex 1\fi\ifluatex 1\fi=0 % if pdftex
  \usepackage[T1]{fontenc}
  \usepackage[utf8]{inputenc}
\else % if luatex or xelatex
  \ifxetex
    \usepackage{mathspec}
    \usepackage{xltxtra,xunicode}
  \else
    \usepackage{fontspec}
  \fi
  \defaultfontfeatures{Mapping=tex-text,Scale=MatchLowercase}
  \newcommand{\euro}{€}
\fi
% use upquote if available, for straight quotes in verbatim environments
\IfFileExists{upquote.sty}{\usepackage{upquote}}{}
% use microtype if available
\IfFileExists{microtype.sty}{\usepackage{microtype}}{}
\usepackage[margin=1in]{geometry}
\usepackage{graphicx}
\makeatletter
\def\maxwidth{\ifdim\Gin@nat@width>\linewidth\linewidth\else\Gin@nat@width\fi}
\def\maxheight{\ifdim\Gin@nat@height>\textheight\textheight\else\Gin@nat@height\fi}
\makeatother
% Scale images if necessary, so that they will not overflow the page
% margins by default, and it is still possible to overwrite the defaults
% using explicit options in \includegraphics[width, height, ...]{}
\setkeys{Gin}{width=\maxwidth,height=\maxheight,keepaspectratio}
\ifxetex
  \usepackage[setpagesize=false, % page size defined by xetex
              unicode=false, % unicode breaks when used with xetex
              xetex]{hyperref}
\else
  \usepackage[unicode=true]{hyperref}
\fi
\hypersetup{breaklinks=true,
            bookmarks=true,
            pdfauthor={Justin Le},
            pdftitle={Lenses embody Products, Prisms embody Sums},
            colorlinks=true,
            citecolor=blue,
            urlcolor=blue,
            linkcolor=magenta,
            pdfborder={0 0 0}}
\urlstyle{same}  % don't use monospace font for urls
% Make links footnotes instead of hotlinks:
\renewcommand{\href}[2]{#2\footnote{\url{#1}}}
\setlength{\parindent}{0pt}
\setlength{\parskip}{6pt plus 2pt minus 1pt}
\setlength{\emergencystretch}{3em}  % prevent overfull lines
\setcounter{secnumdepth}{0}

\title{Lenses embody Products, Prisms embody Sums}
\author{Justin Le}

\begin{document}
\maketitle

\emph{Originally posted on
\textbf{\href{https://blog.jle.im/entry/lenses-products-prisms-sums.html}{in
Code}}.}

I've written about a variety of topics on this blog, but one thing I haven't
touched in too much detail is the topic of lenses and optics. A big part of this
is because there are already so many great resources on lenses, like the famous
(and my favorite) \href{https://artyom.me/lens-over-tea-1}{lenses over tea}
series.

This post won't be a ``lens tutorial'', but rather a dive into a (what I believe
is an) insightful perspective on lenses and prisms that I've heard repeated many
times, but not yet all gathered together into a single place. In particular, I'm
going to talk about the perspective of lenses and prisms as embodying the
essences of products and sums (respectively), and how that observation can help
you with a more ``practical'' understanding of lenses and prisms.

\hypertarget{products-and-sums}{%
\section{Products and Sums}\label{products-and-sums}}

In Haskell, ``products and sums'' can roughly be said to correspond to ``tuples
and \texttt{Either}''. If I have two types \texttt{A} and \texttt{B},
\texttt{(A,\ B)} is their ``product'' type. It's often called an ``anonymous
product'', because we can make one without having to give it a fancy name. It's
called a product type because \texttt{A} has
\includegraphics{https://latex.codecogs.com/png.latex?n} possible values and
\texttt{B} has \includegraphics{https://latex.codecogs.com/png.latex?m} possible
values, then \texttt{(A,\ B)} has
\includegraphics{https://latex.codecogs.com/png.latex?n\%20\%5Ctimes\%20m}
possible values\footnote{All of this is disregarding the notorious ``bottom''
  value that inhabits every type.}. And, \texttt{Either\ A\ B} is their
(anonymous) ``sum'' type. It's called a sum type because \texttt{Either\ A\ B}
has \includegraphics{https://latex.codecogs.com/png.latex?n\%20\%2B\%20m}
possible values. I won't go much deeper into this, but there are
\href{https://codewords.recurse.com/issues/three/algebra-and-calculus-of-algebraic-data-types}{many
useful tutorials already online} on this topic!

It's easy to recognize \texttt{(Int,\ Double)} as a product between \texttt{Int}
and \texttt{Bool}. However, did you know that some types are secretly product
types in disguise?

Here's an easy one!

\texttt{haskell\ top\ foo\ =\ 10}

\end{document}
