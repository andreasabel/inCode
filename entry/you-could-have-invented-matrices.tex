\documentclass[]{article}
\usepackage{lmodern}
\usepackage{amssymb,amsmath}
\usepackage{ifxetex,ifluatex}
\usepackage{fixltx2e} % provides \textsubscript
\ifnum 0\ifxetex 1\fi\ifluatex 1\fi=0 % if pdftex
  \usepackage[T1]{fontenc}
  \usepackage[utf8]{inputenc}
\else % if luatex or xelatex
  \ifxetex
    \usepackage{mathspec}
    \usepackage{xltxtra,xunicode}
  \else
    \usepackage{fontspec}
  \fi
  \defaultfontfeatures{Mapping=tex-text,Scale=MatchLowercase}
  \newcommand{\euro}{€}
\fi
% use upquote if available, for straight quotes in verbatim environments
\IfFileExists{upquote.sty}{\usepackage{upquote}}{}
% use microtype if available
\IfFileExists{microtype.sty}{\usepackage{microtype}}{}
\usepackage[margin=1in]{geometry}
\usepackage{graphicx}
\makeatletter
\def\maxwidth{\ifdim\Gin@nat@width>\linewidth\linewidth\else\Gin@nat@width\fi}
\def\maxheight{\ifdim\Gin@nat@height>\textheight\textheight\else\Gin@nat@height\fi}
\makeatother
% Scale images if necessary, so that they will not overflow the page
% margins by default, and it is still possible to overwrite the defaults
% using explicit options in \includegraphics[width, height, ...]{}
\setkeys{Gin}{width=\maxwidth,height=\maxheight,keepaspectratio}
\ifxetex
  \usepackage[setpagesize=false, % page size defined by xetex
              unicode=false, % unicode breaks when used with xetex
              xetex]{hyperref}
\else
  \usepackage[unicode=true]{hyperref}
\fi
\hypersetup{breaklinks=true,
            bookmarks=true,
            pdfauthor={Justin Le},
            pdftitle={You Could Have Invented Matrices!},
            colorlinks=true,
            citecolor=blue,
            urlcolor=blue,
            linkcolor=magenta,
            pdfborder={0 0 0}}
\urlstyle{same}  % don't use monospace font for urls
% Make links footnotes instead of hotlinks:
\renewcommand{\href}[2]{#2\footnote{\url{#1}}}
\setlength{\parindent}{0pt}
\setlength{\parskip}{6pt plus 2pt minus 1pt}
\setlength{\emergencystretch}{3em}  % prevent overfull lines
\setcounter{secnumdepth}{0}

\title{You Could Have Invented Matrices!}
\author{Justin Le}

\begin{document}
\maketitle

\emph{Originally posted on
\textbf{\href{https://blog.jle.im/entry/you-could-have-invented-matrices.html}{in
Code}}.}

You could have invented matrices!

Let's talk about vectors. A \textbf{vector} (denoted as
\includegraphics{https://latex.codecogs.com/png.latex?\%5Cmathbf\%7Bv\%7D}, a
lower-case bold italicized letter) is an element in a \textbf{vector space},
which means that it can be ``scaled'', like
\includegraphics{https://latex.codecogs.com/png.latex?c\%20\%5Cmathbf\%7Bv\%7D}
(the \includegraphics{https://latex.codecogs.com/png.latex?c} is called a
``scalar'' --- creative name, right?) and added, like
\includegraphics{https://latex.codecogs.com/png.latex?\%5Cmathbf\%7Bv\%7D\%20\%2B\%20\%5Cmathbf\%7Bu\%7D}.

In order for vector spaces and their operations to be valid, they just have to
obey some common-sense rules (like associativity, commutativity, distributivity,
etc.) that allow us to make meaningful conclusions.\footnote{In short, vector
  spaces form an Abelian group (which is another way of just saying that
  addition is commutative, associative, has an identity, and an inverse), and
  scalars have to play nice with addition
  (\includegraphics{https://latex.codecogs.com/png.latex?c\%28\%5Cmathbf\%7Bv\%7D\%20\%2B\%20\%5Cmathbf\%7Bu\%7D\%29\%20\%3D\%20c\%20\%5Cmathbf\%7Bv\%7D\%20\%2B\%20c\%20\%5Cmathbf\%7Bu\%7D},
  and
  \includegraphics{https://latex.codecogs.com/png.latex?\%28c\%20\%2B\%20d\%29\%5Cmathbf\%7Bv\%7D\%20\%3D\%20c\%20\%5Cmathbf\%7Bv\%7D\%20\%2B\%20d\%20\%5Cmathbf\%7Bv\%7D}).
  Also, scalars themselves form a field.}

\hypertarget{dimensionality}{%
\section{Dimensionality}\label{dimensionality}}

One neat thing about vector spaces is that \emph{some} of them (if you're lucky)
have a notion of \textbf{dimensionality}. We say that a vector space is
N-dimensional if there exists N ``basis'' vectors
\includegraphics{https://latex.codecogs.com/png.latex?\%5Cmathbf\%7Be\%7D_1\%2C\%20\%5Cmathbf\%7Be\%7D_2\%20\%5Cldots\%20\%5Cmathbf\%7Be\%7D_N}
where \emph{any} vector can be described as scaled sums of all of them, and that
N is the lowest number of basis vectors needed. For example, if a vector space
is 3-dimensional, then it means that \emph{any} vector
\includegraphics{https://latex.codecogs.com/png.latex?\%5Cmathbf\%7Bv\%7D} in
that space can be broken down as:

{[} \textbackslash{}mathbf\{v\} = a \textbackslash{}mathbf\{e\}\_1 + b
\textbackslash{}mathbf\{e\}\_2 + c
\textbackslash{}mathbf\{e\}\_3{]}(https://latex.codecogs.com/png.latex?\%0A\%5Cmathbf\%7Bv\%7D\%20\%3D\%20a\%20\%5Cmathbf\%7Be\%7D\_1\%20\%2B\%20b\%20\%5Cmathbf\%7Be\%7D\_2\%20\%2B\%20c\%20\%5Cmathbf\%7Be\%7D\_3\%0A
" \mathbf{v} = a \mathbf{e}\_1 + b \mathbf{e}\_2 + c \mathbf{e}\_3 ``)

Where \includegraphics{https://latex.codecogs.com/png.latex?a},
\includegraphics{https://latex.codecogs.com/png.latex?b}, and
\includegraphics{https://latex.codecogs.com/png.latex?c} are scalars.

Dimensionality is really a statement about being able to decompose any vector in
that vector space into a useful set of bases. For a 3-dimensional vector space,
you need at least 3 vectors to make a bases that can reproduce \emph{any} vector
in your space.

In physics, we often treat reality as taking place in a three-dimensional vector
space. The basis vectors are often called
\includegraphics{https://latex.codecogs.com/png.latex?\%5Chat\%7B\%5Cmathbf\%7Bi\%7D\%7D},
\includegraphics{https://latex.codecogs.com/png.latex?\%5Chat\%7B\%5Cmathbf\%7Bj\%7D\%7D},
and
\includegraphics{https://latex.codecogs.com/png.latex?\%5Chat\%7B\%5Cmathbf\%7Bk\%7D\%7D},
and so we say that we can describe our 3D physics vectors as
\includegraphics{https://latex.codecogs.com/png.latex?\%5Cmathbf\%7Bv\%7D\%20\%3D\%20v_x\%20\%5Chat\%7B\%5Cmathbf\%7Bi\%7D\%7D\%20\%2B\%20v_y\%20\%5Chat\%7B\%5Cmathbf\%7Bj\%7D\%7D\%20\%2B\%20v_x\%20\%5Chat\%7B\%5Cmathbf\%7Bk\%7D\%7D}

\hypertarget{encoding}{%
\subsection{Encoding}\label{encoding}}

One neat thing that physicists take advantage of all the time is that if we
\emph{agree} on a set of basis vectors and a specific ordering, we can actually
\emph{encode} any vector
\includegraphics{https://latex.codecogs.com/png.latex?\%5Cmathbf\%7Bv\%7D} in
terms of those basis vectors.

So in physics, we can say ``Let's encode vectors in terms of
\includegraphics{https://latex.codecogs.com/png.latex?\%5Chat\%7B\%5Cmathbf\%7Bi\%7D\%7D},
\includegraphics{https://latex.codecogs.com/png.latex?\%5Chat\%7B\%5Cmathbf\%7Bj\%7D\%7D},
and
\includegraphics{https://latex.codecogs.com/png.latex?\%5Chat\%7B\%5Cmathbf\%7Bk\%7D\%7D},
in that order.'' Then, we can \emph{write}
\includegraphics{https://latex.codecogs.com/png.latex?\%5Cmathbf\%7Bv\%7D} as
\includegraphics{https://latex.codecogs.com/png.latex?\%5Clangle\%20v_x\%2C\%20v_y\%2C\%20v_z\%20\%5Crangle},
such that
\includegraphics{https://latex.codecogs.com/png.latex?\%5Cmathbf\%7Bv\%7D\%20\%3D\%20v_x\%20\%5Chat\%7B\%5Cmathbf\%7Bi\%7D\%7D\%20\%2B\%20v_y\%20\%5Chat\%7B\%5Cmathbf\%7Bj\%7D\%7D\%20\%2B\%20v_x\%20\%5Chat\%7B\%5Cmathbf\%7Bk\%7D\%7D}.

Note that
\includegraphics{https://latex.codecogs.com/png.latex?\%5Clangle\%20v_x\%2C\%20v_y\%2C\%20v_z\%20\%5Crangle}
is \textbf{not} the same thing, conceptually, as the \textbf{vector}
\includegraphics{https://latex.codecogs.com/png.latex?\%5Cmathbf\%7Bv\%7D}. It
is \emph{an encoding} of that vector, that only makes sense once we choose to
\emph{agree} on a specific set of basis.

For an N-dimensional vector space, it means that, with N items, we can represent
any vector in that space. And, if we agree on those N items, we can devise an
encoding, such that:

{[} \textbackslash{}langle v\_1, v\_2 \textbackslash{}dots v\_N
\textbackslash{}rangle{]}(https://latex.codecogs.com/png.latex?\%0A\%5Clangle\%20v\_1\%2C\%20v\_2\%20\%5Cdots\%20v\_N\%20\%5Crangle\%0A
" \langle v\_1, v\_2 \dots v\_N \rangle ``)

will \emph{represent} the vector:

{[} v\_1 \textbackslash{}mathbf\{e\}\_1 + v\_2 \textbackslash{}mathbf\{e\}\_2 +
\textbackslash{}ldots + v\_N
\textbackslash{}mathbf\{e\}\_N{]}(https://latex.codecogs.com/png.latex?\%0Av\_1\%20\%5Cmathbf\%7Be\%7D\_1\%20\%2B\%20v\_2\%20\%5Cmathbf\%7Be\%7D\_2\%20\%2B\%20\%5Cldots\%20\%2B\%20v\_N\%20\%5Cmathbf\%7Be\%7D\_N\%0A
" v\_1 \mathbf{e}\_1 + v\_2 \mathbf{e}\_2 + \ldots + v\_N \mathbf{e}\_N ``)

Note that what this encoding represents is \emph{completely dependent} on what
\includegraphics{https://latex.codecogs.com/png.latex?\%5Cmathbf\%7Be\%7D_1\%2C\%20\%5Cmathbf\%7Be\%7D_2\%20\%5Cldots\%20\%5Cmathbf\%7Be\%7D_N}
we pick, and in what order. The basis vectors we pick are arbitrary, and
determine what our encoding looks like.

\end{document}
