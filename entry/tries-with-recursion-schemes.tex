\documentclass[]{article}
\usepackage{lmodern}
\usepackage{amssymb,amsmath}
\usepackage{ifxetex,ifluatex}
\usepackage{fixltx2e} % provides \textsubscript
\ifnum 0\ifxetex 1\fi\ifluatex 1\fi=0 % if pdftex
  \usepackage[T1]{fontenc}
  \usepackage[utf8]{inputenc}
\else % if luatex or xelatex
  \ifxetex
    \usepackage{mathspec}
    \usepackage{xltxtra,xunicode}
  \else
    \usepackage{fontspec}
  \fi
  \defaultfontfeatures{Mapping=tex-text,Scale=MatchLowercase}
  \newcommand{\euro}{€}
\fi
% use upquote if available, for straight quotes in verbatim environments
\IfFileExists{upquote.sty}{\usepackage{upquote}}{}
% use microtype if available
\IfFileExists{microtype.sty}{\usepackage{microtype}}{}
\usepackage[margin=1in]{geometry}
\usepackage{color}
\usepackage{fancyvrb}
\newcommand{\VerbBar}{|}
\newcommand{\VERB}{\Verb[commandchars=\\\{\}]}
\DefineVerbatimEnvironment{Highlighting}{Verbatim}{commandchars=\\\{\}}
% Add ',fontsize=\small' for more characters per line
\newenvironment{Shaded}{}{}
\newcommand{\AlertTok}[1]{\textcolor[rgb]{1.00,0.00,0.00}{\textbf{#1}}}
\newcommand{\AnnotationTok}[1]{\textcolor[rgb]{0.38,0.63,0.69}{\textbf{\textit{#1}}}}
\newcommand{\AttributeTok}[1]{\textcolor[rgb]{0.49,0.56,0.16}{#1}}
\newcommand{\BaseNTok}[1]{\textcolor[rgb]{0.25,0.63,0.44}{#1}}
\newcommand{\BuiltInTok}[1]{#1}
\newcommand{\CharTok}[1]{\textcolor[rgb]{0.25,0.44,0.63}{#1}}
\newcommand{\CommentTok}[1]{\textcolor[rgb]{0.38,0.63,0.69}{\textit{#1}}}
\newcommand{\CommentVarTok}[1]{\textcolor[rgb]{0.38,0.63,0.69}{\textbf{\textit{#1}}}}
\newcommand{\ConstantTok}[1]{\textcolor[rgb]{0.53,0.00,0.00}{#1}}
\newcommand{\ControlFlowTok}[1]{\textcolor[rgb]{0.00,0.44,0.13}{\textbf{#1}}}
\newcommand{\DataTypeTok}[1]{\textcolor[rgb]{0.56,0.13,0.00}{#1}}
\newcommand{\DecValTok}[1]{\textcolor[rgb]{0.25,0.63,0.44}{#1}}
\newcommand{\DocumentationTok}[1]{\textcolor[rgb]{0.73,0.13,0.13}{\textit{#1}}}
\newcommand{\ErrorTok}[1]{\textcolor[rgb]{1.00,0.00,0.00}{\textbf{#1}}}
\newcommand{\ExtensionTok}[1]{#1}
\newcommand{\FloatTok}[1]{\textcolor[rgb]{0.25,0.63,0.44}{#1}}
\newcommand{\FunctionTok}[1]{\textcolor[rgb]{0.02,0.16,0.49}{#1}}
\newcommand{\ImportTok}[1]{#1}
\newcommand{\InformationTok}[1]{\textcolor[rgb]{0.38,0.63,0.69}{\textbf{\textit{#1}}}}
\newcommand{\KeywordTok}[1]{\textcolor[rgb]{0.00,0.44,0.13}{\textbf{#1}}}
\newcommand{\NormalTok}[1]{#1}
\newcommand{\OperatorTok}[1]{\textcolor[rgb]{0.40,0.40,0.40}{#1}}
\newcommand{\OtherTok}[1]{\textcolor[rgb]{0.00,0.44,0.13}{#1}}
\newcommand{\PreprocessorTok}[1]{\textcolor[rgb]{0.74,0.48,0.00}{#1}}
\newcommand{\RegionMarkerTok}[1]{#1}
\newcommand{\SpecialCharTok}[1]{\textcolor[rgb]{0.25,0.44,0.63}{#1}}
\newcommand{\SpecialStringTok}[1]{\textcolor[rgb]{0.73,0.40,0.53}{#1}}
\newcommand{\StringTok}[1]{\textcolor[rgb]{0.25,0.44,0.63}{#1}}
\newcommand{\VariableTok}[1]{\textcolor[rgb]{0.10,0.09,0.49}{#1}}
\newcommand{\VerbatimStringTok}[1]{\textcolor[rgb]{0.25,0.44,0.63}{#1}}
\newcommand{\WarningTok}[1]{\textcolor[rgb]{0.38,0.63,0.69}{\textbf{\textit{#1}}}}
\usepackage{graphicx}
\makeatletter
\def\maxwidth{\ifdim\Gin@nat@width>\linewidth\linewidth\else\Gin@nat@width\fi}
\def\maxheight{\ifdim\Gin@nat@height>\textheight\textheight\else\Gin@nat@height\fi}
\makeatother
% Scale images if necessary, so that they will not overflow the page
% margins by default, and it is still possible to overwrite the defaults
% using explicit options in \includegraphics[width, height, ...]{}
\setkeys{Gin}{width=\maxwidth,height=\maxheight,keepaspectratio}
\ifxetex
  \usepackage[setpagesize=false, % page size defined by xetex
              unicode=false, % unicode breaks when used with xetex
              xetex]{hyperref}
\else
  \usepackage[unicode=true]{hyperref}
\fi
\hypersetup{breaklinks=true,
            bookmarks=true,
            pdfauthor={Justin Le},
            pdftitle={Tries with Recursion Schemes},
            colorlinks=true,
            citecolor=blue,
            urlcolor=blue,
            linkcolor=magenta,
            pdfborder={0 0 0}}
\urlstyle{same}  % don't use monospace font for urls
% Make links footnotes instead of hotlinks:
\renewcommand{\href}[2]{#2\footnote{\url{#1}}}
\setlength{\parindent}{0pt}
\setlength{\parskip}{6pt plus 2pt minus 1pt}
\setlength{\emergencystretch}{3em}  % prevent overfull lines
\setcounter{secnumdepth}{0}

\title{Tries with Recursion Schemes}
\author{Justin Le}

\begin{document}
\maketitle

\emph{Originally posted on
\textbf{\href{https://blog.jle.im/entry/tries-with-recursion-schemes.html}{in
Code}}.}

Not too long ago, I was browsing the
\href{https://www.reddit.com/r/PrequelMemes}{prequel memes subreddit} --- a
community built around creative ways of remixing and re-contextualizing quotes
from the cinematic corpus of the three Star Wars prequel movies --- when I
noticed that a fad was in progress
\href{https://www.reddit.com/r/PrequelMemes/comments/9w59t4/i_expanded_it/}{constructing
tries based on quotes as keys} indexing stills from the movie corresponding to
those quotes.

This inspired me to try playing around with some tries myself, and it gave me an
excuse to play around with
\emph{\href{https://hackage.haskell.org/package/recursion-schemes}{recursion-schemes}}
(one of my favorite Haskell libraries). If you haven't heard about it yet,
\emph{recursion-schemes} (and the similar library
\emph{\href{https://hackage.haskell.org/package/data-fix}{data-fix}}) abstracts
over common recursive functions written on recursive data types. It exploits the
fact that a lot of recursive functions for different recursive data types all
really follow the same pattern and gives us powerful tools for writing cleaner
and safer code.

Recursion schemes is a perfect example of those amazing accidents that happen
throughout the Haskell ecosystem: an extremely ``theoretically beautiful''
abstraction that also happens to be extremely useful for writing industrially
rigorous code.

Tries are a common intermediate-level recursive data type, and recursion-schemes
is a common intermediate-level library. So, as a fun intermediate-level Haskell
project, let's build a trie data type in Haskell based on recursion-schemes, to
see what it has to offer! The resulting data type will definitely not be a
``toy'' --- it'll be something you can actually use to build meme diagrams of
your own!

\hypertarget{trie}{%
\section{Trie}\label{trie}}

A \href{https://en.wikipedia.org/wiki/Trie}{trie} (prefix tree) is a classic
example of a simple yet powerful data type most people encounter in school (I
remember being introduced to it through a project implementing a boggle solver).

Wikipedia has a nice picture:

\begin{figure}
\centering
\includegraphics{/img/entries/trie/wiki-trie.png}
\caption{Sample Trie from Wikipedia, indexing lists of Char to Ints}
\end{figure}

API-wise, it is very similar to an \emph{associative map}, like the \texttt{Map}
type from
\emph{\href{https://hackage.haskell.org/package/containers/docs/Data-Map-Lazy.html}{containers}}.
It stores ``keys'' to ``values'', and you can insert a value at a given key,
lookup the value stored at a given key, or delete the value at a given key.

The main difference is in implementation: the keys are \emph{strings of tokens},
and it is internally represented as a tree: if your keys are words, then the
first level is the first letter, the second level is the letter, etc. In the
example above, the trie stores the keys \texttt{to}, \texttt{tea}, \texttt{ted},
\texttt{ten}, \texttt{A}, \texttt{i}, \texttt{in}, and \texttt{inn}, to the
values 7, 3, 4, 12, 15, 11, 5, and 9, respectively. Note that it is possible for
one key to completely overlap another (like \texttt{in} storing 5, and
\texttt{inn} storing 9). In the usual case, however, we have partial overlaps
(like \texttt{tea}, storing 3, and \texttt{ted} storing 4), whose common prefix
(\texttt{te}) has no value stored under it.

\hypertarget{haskell-tries}{%
\section{Haskell Tries}\label{haskell-tries}}

We can represent this in Haskell by representing each layer as a \texttt{Map} of
a token to the next ``level'' of the trie:

\begin{Shaded}
\begin{Highlighting}[]
\CommentTok{-- source: https://github.com/mstksg/inCode/tree/master/code-samples/trie/trie.hs#L30-L31}

\KeywordTok{data} \DataTypeTok{Trie}\NormalTok{ k v }\FunctionTok{=} \DataTypeTok{MkT}\NormalTok{ (}\DataTypeTok{Maybe}\NormalTok{ v) (}\DataTypeTok{Map}\NormalTok{ k (}\DataTypeTok{Trie}\NormalTok{ k v))}
  \KeywordTok{deriving} \DataTypeTok{Show}
\end{Highlighting}
\end{Shaded}

A \texttt{Trie\ k\ v} will have keys of type \texttt{{[}k{]}}, where \texttt{k}
is the key token type, and values of type \texttt{v}. Each layer might have a
value (\texttt{Maybe\ v}), and branches out to each new layer.

We could write the trie storing \texttt{(to,\ 9)}, \texttt{(ton,\ 3)}, and
\texttt{(tax,\ 2)} as:

\begin{Shaded}
\begin{Highlighting}[]
\CommentTok{-- source: https://github.com/mstksg/inCode/tree/master/code-samples/trie/trie.hs#L46-L59}

\OtherTok{testTrie ::} \DataTypeTok{Trie} \DataTypeTok{Char} \DataTypeTok{Int}
\NormalTok{testTrie }\FunctionTok{=} \DataTypeTok{MkT} \DataTypeTok{Nothing} \FunctionTok{$}\NormalTok{ M.fromList [}
\NormalTok{      (}\CharTok{'t'}\NormalTok{, }\DataTypeTok{MkT} \DataTypeTok{Nothing} \FunctionTok{$}\NormalTok{ M.fromList [}
\NormalTok{          (}\CharTok{'o'}\NormalTok{, }\DataTypeTok{MkT}\NormalTok{ (}\DataTypeTok{Just} \DecValTok{9}\NormalTok{) }\FunctionTok{$}\NormalTok{ M.fromList [}
\NormalTok{              ( }\CharTok{'n'}\NormalTok{, }\DataTypeTok{MkT}\NormalTok{ (}\DataTypeTok{Just} \DecValTok{3}\NormalTok{) M.empty )}
\NormalTok{            ]}
\NormalTok{          )}
\NormalTok{        , (}\CharTok{'a'}\NormalTok{, }\DataTypeTok{MkT} \DataTypeTok{Nothing} \FunctionTok{$}\NormalTok{ M.fromList [}
\NormalTok{              ( }\CharTok{'x'}\NormalTok{, }\DataTypeTok{MkT}\NormalTok{ (}\DataTypeTok{Just} \DecValTok{2}\NormalTok{) M.empty )}
\NormalTok{            ]}
\NormalTok{          )}
\NormalTok{        ]}
\NormalTok{      )}
\NormalTok{    ]}
\end{Highlighting}
\end{Shaded}

Note that this implementation isn't particularly structurally sound, since it's
possible to represent invalid keys that have branches that lead to nothing. This
mostly becomes troublesome when we implement \texttt{delete}, but we won't be
worrying about that for now. The nice thing about Haskell is that we can be as
safe as we want or need, as a judgement call on a case-by-case basis. However, a
``correct-by-construction'' trie is in the next part of this series :)

\hypertarget{recursion-schemes-an-elegant-weapon}{%
\subsection{Recursion Schemes: An Elegant
Weapon}\label{recursion-schemes-an-elegant-weapon}}

Now, \texttt{Trie} as written up there is an explicitly recursive data type.
While this is common practice, it's not a particularly ideal situation. The
problem with explicitly recursive data types is that to work with them, you
often rely on explicitly recursive functions.

Explicitly recursive functions are notoriously difficult to write, understand,
and maintain. It's extremely easy to accidentally write an infinite loop, and is
often called ``the GOTO of functional programming''.

So, there's a trick we can use to ``factor out'' the recursion in our data type.
The trick is to replace the recursive occurrence of \texttt{Trie\ a} (in the
\texttt{Cons} constructor) with a ``placeholder'' variable:

\begin{Shaded}
\begin{Highlighting}[]
\CommentTok{-- source: https://github.com/mstksg/inCode/tree/master/code-samples/trie/trie.hs#L33-L34}

\KeywordTok{data} \DataTypeTok{TrieF}\NormalTok{ k v x }\FunctionTok{=} \DataTypeTok{MkTF}\NormalTok{ (}\DataTypeTok{Maybe}\NormalTok{ v) (}\DataTypeTok{Map}\NormalTok{ k x)}
  \KeywordTok{deriving}\NormalTok{ (}\DataTypeTok{Functor}\NormalTok{, }\DataTypeTok{Show}\NormalTok{)}
\end{Highlighting}
\end{Shaded}

\texttt{TrieF} now represents, essentially, ``one layer'' of a \texttt{Trie}.

There are now two paths we can go down: we can re-implement \texttt{Trie} in
terms of \texttt{TrieF} (something that most tutorials and introductions do,
using something like \texttt{Trie\ k\ v\ =\ Fix\ (TrieF\ k\ v)}), or we can
think of \texttt{TrieF} as a ``non-recursive view'' into \texttt{Trie}. It's a
way of \emph{working} with \texttt{Trie\ a} \emph{as if} it were a non-recursive
data type. Specifically, in our case, it's a non-recursive view of a ``single
layer'' of a \texttt{Trie}.

We can do this because \emph{recursion-schemes} gives combinators (called
``recursion schemes'') to abstract over common explicit recursion patterns. The
key to using \emph{recursion-schemes} is to recognize which combinators
abstracts over the type of recursion you're using. You then give that combinator
an algebra or a coalgebra (more on this later), and you're done!

Learning how to use \emph{recursion-schemes} effectively is basically picking
the right recursion scheme that abstracts over the type of function you want to
write for your data type. It's all about becoming familiar with the ``zoo'' of
(colorfully named) recursion schemes you can pick from, and identifying which
one does the job in your situation.

That's the high-level view --- let's dive into writing out the API of our
\texttt{Trie}!

\hypertarget{linking-the-base}{%
\subsection{Linking the base}\label{linking-the-base}}

One thing we need to do before we can start: we need to tell
\emph{recursion-schemes} to link \texttt{TrieF} with \texttt{Trie}. In the
nomenclature of \emph{recursion-schemes}, \texttt{TrieF} is known as the ``base
type'', and \texttt{Trie} is called ``the fixed-point''.

Linking them requires some boilerplate, which is basically converting back and
forth from \texttt{Trie} to \texttt{TrieF}.

\begin{Shaded}
\begin{Highlighting}[]
\CommentTok{-- source: https://github.com/mstksg/inCode/tree/master/code-samples/trie/trie.hs#L36-L44}

\KeywordTok{type} \KeywordTok{instance} \DataTypeTok{Base}\NormalTok{ (}\DataTypeTok{Trie}\NormalTok{ k v) }\FunctionTok{=} \DataTypeTok{TrieF}\NormalTok{ k v}

\KeywordTok{instance} \DataTypeTok{Recursive}\NormalTok{ (}\DataTypeTok{Trie}\NormalTok{ k v) }\KeywordTok{where}
\OtherTok{    project ::} \DataTypeTok{Trie}\NormalTok{ k v }\OtherTok{->} \DataTypeTok{TrieF}\NormalTok{ k v (}\DataTypeTok{Trie}\NormalTok{ k v)}
\NormalTok{    project (}\DataTypeTok{MkT}\NormalTok{ v xs) }\FunctionTok{=} \DataTypeTok{MkTF}\NormalTok{ v xs}

\KeywordTok{instance} \DataTypeTok{Corecursive}\NormalTok{ (}\DataTypeTok{Trie}\NormalTok{ k v) }\KeywordTok{where}
\OtherTok{    embed ::} \DataTypeTok{TrieF}\NormalTok{ k v (}\DataTypeTok{Trie}\NormalTok{ k v) }\OtherTok{->} \DataTypeTok{Trie}\NormalTok{ k v}
\NormalTok{    embed (}\DataTypeTok{MkTF}\NormalTok{ v xs) }\FunctionTok{=} \DataTypeTok{MkT}\NormalTok{ v xs}
\end{Highlighting}
\end{Shaded}

Basically we just link the constructors and fields of \texttt{MkT} and
\texttt{MkTF} together.

As with all boilerplate, it is sometimes useful to clean it up a bit using
Template Haskell. The \emph{recursion-schemes} library offers such splice:

\begin{Shaded}
\begin{Highlighting}[]
\KeywordTok{data} \DataTypeTok{Trie}\NormalTok{ k v }\FunctionTok{=} \DataTypeTok{MkT}\NormalTok{ (}\DataTypeTok{Maybe}\NormalTok{ v) (}\DataTypeTok{Map}\NormalTok{ k (}\DataTypeTok{Trie}\NormalTok{ k v))}
  \KeywordTok{deriving} \DataTypeTok{Show}

\NormalTok{makeBaseFunctor ''}\DataTypeTok{Trie}
\end{Highlighting}
\end{Shaded}

This will define \texttt{TrieF} with the \texttt{MkTF} constructor, the
\texttt{Base} type family instance, and the \texttt{Recursive} and
\texttt{Corecursive} instances (in possibly a more efficient way than the way we
wrote by hand, too).

\hypertarget{exploring-the-zoo}{%
\section{Exploring the Zoo}\label{exploring-the-zoo}}

Time to explore the zoo a bit! This is where the fun begins.

Whenever you get a new recursive type and base functor, a good ``first thing''
to try out is testing out \texttt{cata} and \texttt{ana} (catamorphisms and
anamorphisms), the basic ``folder'' and ``unfolder''.

\hypertarget{hakuna-my-cata}{%
\subsection{Hakuna My Cata}\label{hakuna-my-cata}}

Catamorphisms are functions that ``combine'' or ``fold'' every layer of our
recursive type into a single value. If we want to write a function of type
\texttt{Trie\ k\ v\ -\textgreater{}\ A}, we can reach first for a catamorphism.

Catamorphisms work by folding layer-by-layer, from the bottom up. We can write
one by defining ``what to do with each layer''. This description comes in the
form of an ``algebra'' in terms of the base functor:

\begin{Shaded}
\begin{Highlighting}[]
\OtherTok{myAlg ::} \DataTypeTok{TrieF}\NormalTok{ k v }\DataTypeTok{A} \OtherTok{->} \DataTypeTok{A}
\end{Highlighting}
\end{Shaded}

If we think of \texttt{TrieF\ k\ v\ a} as ``one layer'' of a
\texttt{Trie\ k\ v}, then \texttt{TrieF\ k\ v\ A\ -\textgreater{}\ A} describes
how to fold up one layer of our \texttt{Trie\ k\ v} into our final result value
(here, of type \texttt{A}). Remember that a \texttt{TrieF\ k\ v\ A} contains a
\texttt{Maybe\ v} and a \texttt{Map\ k\ A}. The \texttt{A} (the values of the
map) contains the result of folding up all of the original subtries along each
key; it's the ``results so far''.

And then we can use \texttt{cata} to ``fold'' our value along the algebra:

\begin{Shaded}
\begin{Highlighting}[]
\NormalTok{cata}\OtherTok{ myAlg ::} \DataTypeTok{Trie}\NormalTok{ k v }\OtherTok{->} \DataTypeTok{A}
\end{Highlighting}
\end{Shaded}

\texttt{cata} starts from the bottom-most layer, runs \texttt{myAlg} on that,
then goes up a layer, running \texttt{myAlg} on the results, then goes up
another layer, running \texttt{myAlg} on those results, etc., until it reaches
the top layer and runs \texttt{myAlg} again to produce the final result.

For example, we'll write a catamorphism that counts how many values/leaves we
have in our Trie into an \texttt{Int}.

\begin{Shaded}
\begin{Highlighting}[]
\OtherTok{countAlg ::} \DataTypeTok{TrieF}\NormalTok{ k v }\DataTypeTok{Int} \OtherTok{->} \DataTypeTok{Int}
\end{Highlighting}
\end{Shaded}

This is the basic structure of an algebra: our final result becomes the
parameter of \texttt{TrieF\ k\ v}, and also the result of our algebra.

Remember that a \texttt{Trie\ k\ v} contains a \texttt{Maybe\ v} and a
\texttt{Map\ k\ (Trie\ k\ v)}, and a \texttt{TrieF\ k\ v\ Int} contains a
\texttt{Maybe\ v} and a \texttt{Map\ k\ Int}. The \texttt{Map}, here in
\texttt{countAlg}, represents the count of the original subtries along each key.

Basically, our task is ``How to find a count, given a map of sub-counts''.

With this in mind, we can write \texttt{countAlg}:

\begin{Shaded}
\begin{Highlighting}[]
\CommentTok{-- source: https://github.com/mstksg/inCode/tree/master/code-samples/trie/trie.hs#L64-L69}

\OtherTok{countAlg ::} \DataTypeTok{TrieF}\NormalTok{ k v }\DataTypeTok{Int} \OtherTok{->} \DataTypeTok{Int}
\NormalTok{countAlg (}\DataTypeTok{MkTF}\NormalTok{ v subtrieCounts)}
    \FunctionTok{|}\NormalTok{ isJust v  }\FunctionTok{=} \DecValTok{1} \FunctionTok{+}\NormalTok{ subtrieTotal}
    \FunctionTok{|}\NormalTok{ otherwise }\FunctionTok{=}\NormalTok{ subtrieTotal}
  \KeywordTok{where}
\NormalTok{    subtrieTotal }\FunctionTok{=}\NormalTok{ sum subtrieCounts}
\end{Highlighting}
\end{Shaded}

If \texttt{v} is indeed a leaf (it's \texttt{Just}), then it's one plus the
total counts of all of the subtees (remember, the \texttt{Map\ k\ Int} contains
the counts of all of the original subtries, under each key). Otherwise, it's
just the total counts of all of the original subtries.

Our final \texttt{count} is, then:

\begin{Shaded}
\begin{Highlighting}[]
\CommentTok{-- source: https://github.com/mstksg/inCode/tree/master/code-samples/trie/trie.hs#L61-L62}

\OtherTok{count ::} \DataTypeTok{Trie}\NormalTok{ k v }\OtherTok{->} \DataTypeTok{Int}
\NormalTok{count }\FunctionTok{=}\NormalTok{ cata countAlg}
\end{Highlighting}
\end{Shaded}

\begin{Shaded}
\begin{Highlighting}[]
\NormalTok{ghci}\FunctionTok{>}\NormalTok{ count testTrie}
\DecValTok{3}
\end{Highlighting}
\end{Shaded}

We can do something similar by writing a summer, as well, to sum up all values
inside a trie:

\begin{Shaded}
\begin{Highlighting}[]
\CommentTok{-- source: https://github.com/mstksg/inCode/tree/master/code-samples/trie/trie.hs#L71-L75}

\OtherTok{trieSumAlg ::} \DataTypeTok{Num}\NormalTok{ a }\OtherTok{=>} \DataTypeTok{TrieF}\NormalTok{ k a a }\OtherTok{->}\NormalTok{ a}
\NormalTok{trieSumAlg (}\DataTypeTok{MkTF}\NormalTok{ v subtrieSums) }\FunctionTok{=}\NormalTok{ fromMaybe }\DecValTok{0}\NormalTok{ v }\FunctionTok{+}\NormalTok{ sum subtrieSums}

\OtherTok{trieSum ::} \DataTypeTok{Num}\NormalTok{ a }\OtherTok{=>} \DataTypeTok{Trie}\NormalTok{ k a }\OtherTok{->}\NormalTok{ a}
\NormalTok{trieSum }\FunctionTok{=}\NormalTok{ cata trieSumAlg}
\end{Highlighting}
\end{Shaded}

\begin{Shaded}
\begin{Highlighting}[]
\NormalTok{ghci}\FunctionTok{>}\NormalTok{ trieSum testTrie}
\DecValTok{14}
\end{Highlighting}
\end{Shaded}

In the algebra, the \texttt{subtrieSums\ ::\ Map\ k\ a} contains the sum of all
of the subtries. The algebra therefore just adds up all of the subtrie sums with
the value at that layer. ``How to find a sum, given a map of sub-sums''.

\hypertarget{outside-in}{%
\subsubsection{Outside-In}\label{outside-in}}

Catamorphisms are naturally ``inside-out'', or ``bottom-up''. However, some
operations are more naturally ``outside-in'', or ``top-down''. One immediate
example is
\texttt{lookup\ ::\ {[}k{]}\ -\textgreater{}\ Trie\ k\ v\ -\textgreater{}\ Maybe\ v},
which is clearly ``top-down'': it first descends down the first item in the
\texttt{{[}k{]}}, then the second, then the third, etc. until you reach the end,
and return the \texttt{Maybe\ v} at that layer.

In this case, it helps to invert control: instead of folding into a
\texttt{Maybe\ v} directly, fold into a ``looker upper'', a
\texttt{{[}k{]}\ -\textgreater{}\ Maybe\ v}. We generate a ``lookup function''
from the bottom-up, and then run that all in the end on the key we want to look
up.

Our algebra will therefore have type:

\begin{Shaded}
\begin{Highlighting}[]
\NormalTok{lookupperAlg}
\OtherTok{    ::} \DataTypeTok{Ord}\NormalTok{ k}
    \OtherTok{=>} \DataTypeTok{TrieF}\NormalTok{ k v ([k] }\OtherTok{->} \DataTypeTok{Maybe}\NormalTok{ v)}
    \OtherTok{->}\NormalTok{ ([k] }\OtherTok{->} \DataTypeTok{Maybe}\NormalTok{ v)}
\end{Highlighting}
\end{Shaded}

A \texttt{TrieF\ k\ v\ ({[}k{]}\ -\textgreater{}\ Maybe\ v)} contains a
\texttt{Maybe\ v} and a \texttt{Map\ k\ ({[}k{]}\ -\textgreater{}\ Maybe\ v)},
or a map of ``lookuppers''. Indexed at each key is function of how to look up a
given key in the original subtrie.

So, we are tasked with ``how to implement a lookupper, given a map of
sub-lookuppers''.

To do this, we can pattern match on the key we are looking up. If it's
\texttt{{[}{]}}, then we just return the current leaf (if it exists). Otherwise,
if it's \texttt{k:ks}, we can \emph{run the lookupper of the subtrie at key
\texttt{k}}.

\begin{Shaded}
\begin{Highlighting}[]
\CommentTok{-- source: https://github.com/mstksg/inCode/tree/master/code-samples/trie/trie.hs#L85-L100}

\NormalTok{lookupperAlg}
\OtherTok{    ::} \DataTypeTok{Ord}\NormalTok{ k}
    \OtherTok{=>} \DataTypeTok{TrieF}\NormalTok{ k v ([k] }\OtherTok{->} \DataTypeTok{Maybe}\NormalTok{ v)}
    \OtherTok{->}\NormalTok{ ([k] }\OtherTok{->} \DataTypeTok{Maybe}\NormalTok{ v)}
\NormalTok{lookupperAlg (}\DataTypeTok{MkTF}\NormalTok{ v lookuppers) }\FunctionTok{=}\NormalTok{ \textbackslash{}}\KeywordTok{case}
\NormalTok{    []   }\OtherTok{->}\NormalTok{ v}
\NormalTok{    k}\FunctionTok{:}\NormalTok{ks }\OtherTok{->} \KeywordTok{case}\NormalTok{ M.lookup k lookuppers }\KeywordTok{of}
      \DataTypeTok{Nothing}        \OtherTok{->} \DataTypeTok{Nothing}
      \DataTypeTok{Just}\NormalTok{ lookupper }\OtherTok{->}\NormalTok{ lookupper ks}

\NormalTok{lookup}
\OtherTok{    ::} \DataTypeTok{Ord}\NormalTok{ k}
    \OtherTok{=>}\NormalTok{ [k]}
    \OtherTok{->} \DataTypeTok{Trie}\NormalTok{ k v}
    \OtherTok{->} \DataTypeTok{Maybe}\NormalTok{ v}
\NormalTok{lookup ks t }\FunctionTok{=}\NormalTok{ cata lookupperAlg t ks}
\end{Highlighting}
\end{Shaded}

(written using the -XLambdaCase extension, allowing for
\texttt{\textbackslash{}case} syntax)

\begin{Shaded}
\begin{Highlighting}[]
\NormalTok{ghci}\FunctionTok{>}\NormalTok{ lookup }\StringTok{"to"}\NormalTok{ testTrie}
\DataTypeTok{Just} \DecValTok{9}
\NormalTok{ghci}\FunctionTok{>}\NormalTok{ lookup }\StringTok{"ton"}\NormalTok{ testTrie}
\DataTypeTok{Just} \DecValTok{3}
\NormalTok{ghci}\FunctionTok{>}\NormalTok{ lookup }\StringTok{"tone"}\NormalTok{ testTrie}
\DataTypeTok{Nothing}
\end{Highlighting}
\end{Shaded}

Note that because \texttt{Map}s have lazy values by default, we only ever
actually generate ``lookuppers'' for subtries under keys that we eventually
descend on; any other subtries will be ignored (and no lookuppers are ever
generated for them).

In the end, this version has all of the same performance characteristics as the
explicitly recursive one; we're assembling a ``lookupper'' that stops as soon as
it sees either a failed lookup (so it doesn't cause any more evaluation later
on), or stops when it reaches the end of its list.

\hypertarget{i-think-the-system-works}{%
\subsubsection{I Think the System Works}\label{i-think-the-system-works}}

We've now written a couple of non-recursive functions to ``query''
\texttt{Trie}. But what was the point, again? What do we gain over writing
explicit versions to query Trie? Why couldn't we just write:

\begin{Shaded}
\begin{Highlighting}[]
\CommentTok{-- source: https://github.com/mstksg/inCode/tree/master/code-samples/trie/trie.hs#L77-L79}

\OtherTok{trieSumExplicit ::} \DataTypeTok{Num}\NormalTok{ a }\OtherTok{=>} \DataTypeTok{Trie}\NormalTok{ k a }\OtherTok{->}\NormalTok{ a}
\NormalTok{trieSumExplicit (}\DataTypeTok{MkT}\NormalTok{ v subtries) }\FunctionTok{=}
\NormalTok{    fromMaybe }\DecValTok{0}\NormalTok{ v }\FunctionTok{+}\NormalTok{ sum (fmap trieSumExplicit subtries)}
\end{Highlighting}
\end{Shaded}

instead of

\begin{Shaded}
\begin{Highlighting}[]
\CommentTok{-- source: https://github.com/mstksg/inCode/tree/master/code-samples/trie/trie.hs#L81-L83}

\OtherTok{trieSumCata ::} \DataTypeTok{Num}\NormalTok{ a }\OtherTok{=>} \DataTypeTok{Trie}\NormalTok{ k a }\OtherTok{->}\NormalTok{ a}
\NormalTok{trieSumCata }\FunctionTok{=}\NormalTok{ cata }\FunctionTok{$}\NormalTok{ \textbackslash{}(}\DataTypeTok{MkTF}\NormalTok{ v subtrieSums) }\OtherTok{->}
\NormalTok{    fromMaybe }\DecValTok{0}\NormalTok{ v }\FunctionTok{+}\NormalTok{ sum subtrieSums}
\end{Highlighting}
\end{Shaded}

One major reason, like I mentioned before, is to avoid using \emph{explicit
recursion}. It's extremely easy when using explicit recursion to accidentally
write an infinite loop, or to mess up your control flow somehow. It's basically
like using \texttt{GOTO} instead of \texttt{while} or \texttt{for} loops in
imperative languages. \texttt{while} and \texttt{for} loops are meant to
abstract over a common type of looping control flow, and provide a disciplined
structure for them. They also are often much easier to read, because as soon as
you see ``while'' or ``for'', it gives you a hint at programmer intent in ways
that an explicit GOTO might not.

Another major reason is to allow you to separate concerns. Writing
\texttt{trieSumExplicit} forces you to think ``how to fold this entire trie''.
Writing \texttt{trieSumAlg} allows us to just focus on ``how to fold \emph{this
immediate} layer''. You only need to ever focus on the immediate layer you are
trying to sum --- and never the entire trie. \texttt{cata} takes your ``how to
fold this immediate layer'' function and turns it into a function that folds an
entire trie, taking care of the recursive descent for you.

\begin{center}\rule{0.5\linewidth}{\linethickness}\end{center}

\textbf{Aside}

Before we move on, I just wanted to mention that \texttt{cata} is not a magic
function. From the clues above, you might actually be able to implement it
yourself. For our \texttt{Trie}, it's:

\begin{Shaded}
\begin{Highlighting}[]
\CommentTok{-- source: https://github.com/mstksg/inCode/tree/master/code-samples/trie/trie.hs#L102-L103}

\OtherTok{cata' ::}\NormalTok{ (}\DataTypeTok{TrieF}\NormalTok{ k v a }\OtherTok{->}\NormalTok{ a) }\OtherTok{->} \DataTypeTok{Trie}\NormalTok{ k v }\OtherTok{->}\NormalTok{ a}
\NormalTok{cata' alg }\FunctionTok{=}\NormalTok{ alg }\FunctionTok{.}\NormalTok{ fmap (cata' alg) }\FunctionTok{.}\NormalTok{ project}
\end{Highlighting}
\end{Shaded}

First we
\texttt{project\ ::\ Trie\ k\ v\ -\textgreater{}\ TrieF\ k\ v\ (Trie\ k\ v)}, to
``de-recursive'' our type. Then we fmap our entire
\texttt{cata\ alg\ ::\ Trie\ k\ v\ -\textgreater{}\ a}. Then we run the
\texttt{alg\ ::\ TrieF\ k\ v\ a\ -\textgreater{}\ a} on the result. Basically,
it's fmap-collapse-then-collapse.

\begin{center}\rule{0.5\linewidth}{\linethickness}\end{center}

\hypertarget{ana-montana}{%
\subsection{Ana Montana}\label{ana-montana}}

\emph{Anamorphisms}, the dual of catamorphisms, are functions that ``generate''
or ``unfold'' a value of a recursive type, layer-by-layer. If we want to write a
function of type \texttt{A\ -\textgreater{}\ Trie\ k\ v}, we can reach first for
an anamorphism.

Anamorphisms work by unfolding ``layer-by-layer'', from the outside-in (or
top-down). We write one by defining ``how to generate the next layer''. This
description comes in the form of a ``coalgebra'' (pronounced like
``co-algebra'', and not like coal energy ``coal-gebra''), in terms of the base
functor:

\begin{Shaded}
\begin{Highlighting}[]
\OtherTok{myCoalg ::} \DataTypeTok{A} \OtherTok{->} \DataTypeTok{TrieF}\NormalTok{ k v }\DataTypeTok{A}
\end{Highlighting}
\end{Shaded}

If we think of \texttt{TrieF\ k\ v\ a} as ``one layer'' of a
\texttt{Trie\ k\ v}, then \texttt{A\ -\textgreater{}\ TrieF\ k\ v\ A} describes
how to generate a new nested layer of our \texttt{Trie\ k\ v} from our initial
``seed'' (here, of type \texttt{A}). It tells us how to generate the next
immediate layer. Remember that a \texttt{TrieF\ k\ v\ A} contains a
\texttt{Maybe\ v} and a \texttt{Map\ k\ A}. The \texttt{A} (the values of the
map) are then used to seed the \emph{new} subtries. The \texttt{A} is the
``continue expanding with\ldots{}'' value.

And then we can use \texttt{ana} to ``unfold'' our value along the coalgebra:

\begin{Shaded}
\begin{Highlighting}[]
\NormalTok{ana}\OtherTok{ myCoalg ::} \DataTypeTok{A} \OtherTok{->} \DataTypeTok{Trie}\NormalTok{ k v}
\end{Highlighting}
\end{Shaded}

\texttt{ana} starts from the an initial seed \texttt{A}, runs \texttt{myCoalg}
on that, and then goes down a layer, running \texttt{myCoalg} on each value in
the map to create new layers, etc., forever and ever. In practice, it usually
stops when we return a \texttt{TrieF} with an empty map, since there are no more
seeds to expand down. However, it's nice to remember we don't have to
special-case this behavior: it arises naturally from the structure of maps.

While I don't have a concrete ``universal'' example (like how we had
\texttt{count} and \texttt{sum} for `cata1), the general idea is that if you
want to create a value by repeatedly ``expanding leaves'', an anamorphism is a
perfect fit.

An example here that fits will with the nature of a trie is to produce a
``singleton trie'': a trie that has only a single value at a single trie.

\begin{Shaded}
\begin{Highlighting}[]
\CommentTok{-- source: https://github.com/mstksg/inCode/tree/master/code-samples/trie/trie.hs#L108-L112}

\OtherTok{mkSingletonCoalg ::}\NormalTok{ v }\OtherTok{->}\NormalTok{ ([k] }\OtherTok{->} \DataTypeTok{TrieF}\NormalTok{ k v [k])}
\NormalTok{mkSingletonCoalg v }\FunctionTok{=}\NormalTok{ singletonCoalg}
  \KeywordTok{where}
\NormalTok{    singletonCoalg []     }\FunctionTok{=} \DataTypeTok{MkTF}\NormalTok{ (}\DataTypeTok{Just}\NormalTok{ v) M.empty}
\NormalTok{    singletonCoalg (k}\FunctionTok{:}\NormalTok{ks) }\FunctionTok{=} \DataTypeTok{MkTF} \DataTypeTok{Nothing}\NormalTok{  (M.singleton k ks)}
\end{Highlighting}
\end{Shaded}

Given a \texttt{v} value, we'll make a coalgebra
\texttt{{[}k{]}\ -\textgreater{}\ TrieF\ k\ v\ {[}k{]}}. Our ``seed'' will be
the \texttt{{[}k{]}} key we want to insert, and we'll generate our singleton key
by making sub-maps with sub-keys.

Our coalgebra (``layer generating function'') goes like this:

\begin{enumerate}
\def\labelenumi{\arabic{enumi}.}
\item
  If our key-to-insert is empty \texttt{{[}{]}}, then we're here! We're at
  \emph{the layer} where we want to insert the value at, so
  \texttt{MkTF\ (Just\ v)\ M.empty}. Returning \texttt{M.empty} means that we
  don't want to expand anymore, since there are no new seeds to expand into
  subtries.
\item
  If our key-to-insert is \emph{not} empty, then we're \emph{not} here! We
  return \texttt{MkTF\ Nothing}\ldots{}but we know we leave a singleton map
  \texttt{M.singleton\ k\ ks\ ::\ Map\ k\ {[}k{]}} leaving a single seed. When
  we run our coalgebra with \texttt{ana}, \texttt{ana} will go down and expand
  out that single seed (with our coalgebra) into an entire new sub-trie, with
  \texttt{ks} as its seed.
\end{enumerate}

So, we have \texttt{singleton}:

\begin{Shaded}
\begin{Highlighting}[]
\CommentTok{-- source: https://github.com/mstksg/inCode/tree/master/code-samples/trie/trie.hs#L105-L106}

\OtherTok{singleton ::}\NormalTok{ [k] }\OtherTok{->}\NormalTok{ v }\OtherTok{->} \DataTypeTok{Trie}\NormalTok{ k v}
\NormalTok{singleton k v }\FunctionTok{=}\NormalTok{ ana (mkSingletonCoalg v) k}
\end{Highlighting}
\end{Shaded}

We run the coalgebra on our initial seed (the key), and ana will run
\texttt{singletonCoalg} repeatedly, expanding out all of the seeds we deposit,
forever and ever (or at least until there are no more values of the seed type
left, which happens if we return an empty map).

\begin{Shaded}
\begin{Highlighting}[]
\NormalTok{ghci}\FunctionTok{>}\NormalTok{ singleton }\StringTok{"hi"} \DecValTok{7}
\DataTypeTok{MkT} \DataTypeTok{Nothing} \FunctionTok{$}\NormalTok{ M.fromList [}
\NormalTok{    (}\CharTok{'h'}\NormalTok{, }\DataTypeTok{MkT} \DataTypeTok{Nothing} \FunctionTok{$}\NormalTok{ M.fromList [}
\NormalTok{        (}\CharTok{'i'}\NormalTok{, }\DataTypeTok{MkT}\NormalTok{ (}\DataTypeTok{Just} \DecValTok{7}\NormalTok{) M.empty )}
\NormalTok{      ]}
\NormalTok{    )}
\NormalTok{  ]}
\end{Highlighting}
\end{Shaded}

\hypertarget{trie-from-map}{%
\subsection{Trie from Map}\label{trie-from-map}}

Now that we've got the basics, let's look at a more interesting anamorphism,
where we leave multiple ``seeds'' along many different keys in the map, to
generate many different subtries from our root.

Let's write a function to generate a \texttt{Trie\ k\ v} from a
\texttt{Map\ {[}k{]}\ v}: Given a map of keys (as token strings), generate a
prefix trie containing every key-value pair in the map.

This might sound complicated, but let's remember the philosophy and approach of
writing an anamorphism: ``How do I generate \emph{one layer}''?

Our \texttt{fromMapCoalg} will take a \texttt{Map\ {[}k{]}\ v} and generate
\texttt{TrieF\ k\ v\ (Map\ {[}k{]}\ v)}: \emph{one single layer} of our new Trie
(in particular, the \emph{top layer}). And the values in each of the resultant
maps will be later then watered and expanded into their own fully mature
subtries.

So, how do we generate the \emph{top layer} of a prefix trie from a map? Well,
remember, to make a \texttt{TrieF\ k\ v\ (Map\ {[}k{]}\ v)}, we need a
\texttt{Maybe\ v} (the value at this layer) and a
\texttt{Map\ k\ (Map\ {[}k{]}\ v)} (the map of seeds that will each expand into
full subtries).

\begin{itemize}
\tightlist
\item
  If the map contains \texttt{{[}{]}} (the empty string), then there \emph{is a
  value} at this layer. We will return \texttt{Just}.
\item
  In the \texttt{Map\ k\ (Map\ {[}k{]}\ v)}, the value at key \texttt{k} will
  contain all of the key-value pairs in the original map that \emph{start with
  k}, not including the \texttt{k}.
\end{itemize}

For a concrete example, if we start with
\texttt{M.fromList\ {[}("to",\ 9),\ ("ton",\ 3),\ ("tax",\ 2){]}}, then we want
\texttt{fromMapCoalg} to produce:

\begin{Shaded}
\begin{Highlighting}[]
\NormalTok{fromMap (M.fromList [(}\StringTok{"to"}\NormalTok{, }\DecValTok{9}\NormalTok{), (}\StringTok{"ton"}\NormalTok{, }\DecValTok{3}\NormalTok{), (}\StringTok{"tax"}\NormalTok{, }\DecValTok{2}\NormalTok{)])}
    \FunctionTok{=} \DataTypeTok{MkTF} \DataTypeTok{Nothing}\NormalTok{ (}
\NormalTok{          M.fromList [}
\NormalTok{            (}\CharTok{'t'}\NormalTok{, M.fromList [}
\NormalTok{                (}\StringTok{"o"}\NormalTok{ , }\DecValTok{9}\NormalTok{)}
\NormalTok{              , (}\StringTok{"on"}\NormalTok{, }\DecValTok{3}\NormalTok{)}
\NormalTok{              , (}\StringTok{"ax"}\NormalTok{, }\DecValTok{2}\NormalTok{)}
\NormalTok{              ]}
\NormalTok{            )}
\NormalTok{          ]}
\NormalTok{        )}
\end{Highlighting}
\end{Shaded}

The value is \texttt{Nothing} because we don't have the empty string, and the
map at \texttt{t} contains all of the original key-value pairs that began with
\texttt{t}.

Now that we have the concept, we can implement it using \texttt{Data.Map}
combinators like \texttt{M.lookup}, \texttt{M.mapMaybeWithKey}, and
\texttt{M.unionsWith\ M.union}:

\begin{Shaded}
\begin{Highlighting}[]
\CommentTok{-- source: https://github.com/mstksg/inCode/tree/master/code-samples/trie/trie.hs#L114-L128}

\NormalTok{fromMapCoalg}
\OtherTok{    ::} \DataTypeTok{Ord}\NormalTok{ k}
    \OtherTok{=>} \DataTypeTok{Map}\NormalTok{ [k] v}
    \OtherTok{->} \DataTypeTok{TrieF}\NormalTok{ k v (}\DataTypeTok{Map}\NormalTok{ [k] v)}
\NormalTok{fromMapCoalg mp }\FunctionTok{=} \DataTypeTok{MkTF}\NormalTok{ (M.lookup [] mp)}
\NormalTok{                       (M.fromListWith M.union (M.foldMapWithKey descend mp))}
  \KeywordTok{where}
\NormalTok{    descend []     _ }\FunctionTok{=}\NormalTok{ []}
\NormalTok{    descend (k}\FunctionTok{:}\NormalTok{ks) v }\FunctionTok{=}\NormalTok{ [(k, M.singleton ks v)]}

\NormalTok{fromMap}
\OtherTok{    ::} \DataTypeTok{Ord}\NormalTok{ k}
    \OtherTok{=>} \DataTypeTok{Map}\NormalTok{ [k] v}
    \OtherTok{->} \DataTypeTok{Trie}\NormalTok{ k v}
\NormalTok{fromMap }\FunctionTok{=}\NormalTok{ ana fromMapCoalg}
\end{Highlighting}
\end{Shaded}

And just like that, we have a way to turn a \texttt{Map\ {[}k{]}} into a
\texttt{Trie\ k}\ldots{}all just from describing how to make \emph{the top-most
layer}. \texttt{ana} extrapolates the rest!

Again, we can ask what the point of this is: why couldn't we just write it
directly recursively?

The answers again are the same: first, to avoid potential bugs from explicit
recursion. Second, to separate concerns: instead of thinking about how to
generate an entire trie, we only need to be think about how to generate a single
layer. \texttt{ana} reads our mind here, and extrapolates out the entire trie.

\begin{center}\rule{0.5\linewidth}{\linethickness}\end{center}

\textbf{Aside}

Again, let's take some time to reassure ourselves that \texttt{ana} is not a
magic function. You might have been able to guess how it's implemented: it runs
the coalgebra, and then fmaps re-expansion recursively.

\begin{Shaded}
\begin{Highlighting}[]
\CommentTok{-- source: https://github.com/mstksg/inCode/tree/master/code-samples/trie/trie.hs#L130-L131}

\OtherTok{ana' ::}\NormalTok{ (a }\OtherTok{->} \DataTypeTok{TrieF}\NormalTok{ k v a) }\OtherTok{->}\NormalTok{ a }\OtherTok{->} \DataTypeTok{Trie}\NormalTok{ k v}
\NormalTok{ana' coalg }\FunctionTok{=}\NormalTok{ embed }\FunctionTok{.}\NormalTok{ fmap (ana' coalg) }\FunctionTok{.}\NormalTok{ coalg}
\end{Highlighting}
\end{Shaded}

First, we run the \texttt{coalg\ ::\ a\ -\textgreater{}\ TrieF\ k\ v\ a}, then
we fmap our entire \texttt{ana\ coalg\ ::\ a\ -\textgreater{}\ Trie\ k\ v}, then
we \texttt{embed\ ::\ TrieF\ k\ v\ (Trie\ k\ v)\ -\textgreater{}\ Trie\ k\ v}
back into our recursive type.

\begin{center}\rule{0.5\linewidth}{\linethickness}\end{center}

\hypertarget{down-to-business}{%
\section{Down to Business}\label{down-to-business}}

So those are some examples to get our feet wet; now it's time to build our
prequel meme trie!

To render our tree, we're going to be using the
\emph{\href{https://hackage.haskell.org/package/graphviz}{graphviz}} library,
which generates a
\emph{\href{https://en.wikipedia.org/wiki/DOT_(graph_description_language)}{DOT
file}} which the \href{https://www.graphviz.org/}{graphviz application} can
render. The \emph{graphviz} library directly renders a value of the graph data
type from \emph{\href{https://hackage.haskell.org/package/fgl}{fgl}}, the
functional graph library that is the de-facto fleshed-out graph manipulation
library of the Haskell ecosystem.

So, the roadmap seems straightforward:

\begin{enumerate}
\def\labelenumi{\arabic{enumi}.}
\tightlist
\item
  Load our prequel memes into a \texttt{Map\ String\ PrequelMeme}, a map of
  quotes to their associated macro images
\item
  Use \texttt{ana} to turn a \texttt{Map\ String\ PrequelMeme} into a
  \texttt{Trie\ Char\ PrequelMeme}
\item
  Use \texttt{cata} to turn a \texttt{Trie\ Char\ PrequelMeme} into a graph of
  nodes linked by letters, with prequel meme leaves
\item
  Use the \emph{graphviz} library to turn that graph into a DOT file, to be
  rendered by the external graphviz application.
\end{enumerate}

1 and 4 are mainly fumbling around with IO and interfacing with libraries, so 2
and 3 are the interesting steps in our case. We actually already wrote 2 (in the
previous section --- surprise!), so that just leaves 3 to investigate.

\hypertarget{generating-the-graph}{%
\subsection{Generating the Graph}\label{generating-the-graph}}

\emph{fgl} provides a two (interchangeable) graph types; for the sake of this
article, we're going to be using \texttt{Gr} from the
\emph{Data.Graph.Inductive.PatriciaTree} module.

The type \texttt{Gr\ a\ b} represents a graph of vertices with labels of type
\texttt{a}, and edges with labels of type \texttt{b}. In our case, for a
\texttt{Trie\ k\ v}, we'll have a graph with nodes of type \texttt{Maybe\ v}
(the leaves, if they exist) and edges of type \texttt{k} (the token linking one
node to the next).

Our end goal, then, is to write a function
\texttt{Trie\ k\ v\ -\textgreater{}\ Gr\ (Maybe\ v)\ k}. Knowing this, we can
jump directly into writing an algebra:

\begin{Shaded}
\begin{Highlighting}[]
\NormalTok{trieGraphAlg}
\OtherTok{    ::} \DataTypeTok{TrieF}\NormalTok{ k v (}\DataTypeTok{Gr}\NormalTok{ (}\DataTypeTok{Maybe}\NormalTok{ v) k)}
    \OtherTok{->} \DataTypeTok{Gr}\NormalTok{ (}\DataTypeTok{Maybe}\NormalTok{ v) k}
\end{Highlighting}
\end{Shaded}

and then using
\texttt{cata\ trieGraphAlg\ ::\ Trie\ k\ v\ -\textgreater{}\ Gr\ (Maybe\ v)\ k}.

This isn't a bad way to go about it, and you won't have \emph{too} many
problems. However, this might be a good learning opportunity to try writing
``monadic'' catamorphisms.

That's because to create a graph using \emph{fgl}, you need to manage Node ID's,
which are represented as \texttt{Int}s. To add a node, you need to generate a
fresh Node ID. \emph{fgl} has some nice tools for managing this, but we can have
some fun by taking care of it ourselves using the so-called ``state monad'',
\texttt{State\ Int}.

Hylomorphisms We can use \texttt{State\ Int} as a way to generate ``fresh'' node
ID's on-demand, with the action \texttt{fresh}:

\begin{Shaded}
\begin{Highlighting}[]
\CommentTok{-- source: https://github.com/mstksg/inCode/tree/master/code-samples/trie/trie.hs#L149-L150}

\OtherTok{fresh ::} \DataTypeTok{State} \DataTypeTok{Int} \DataTypeTok{Int}
\NormalTok{fresh }\FunctionTok{=}\NormalTok{ state }\FunctionTok{$}\NormalTok{ \textbackslash{}i }\OtherTok{->}\NormalTok{ (i, i}\FunctionTok{+}\DecValTok{1}\NormalTok{)}
\end{Highlighting}
\end{Shaded}

\texttt{fresh} will return the current counter state to produce a new node ID,
and then increment the counter so that the next invocation will return a new
node ID.

In this light, we can frame our big picture as writing a
\texttt{Trie\ k\ v\ -\textgreater{}\ State\ Int\ (Gr\ (Maybe\ v)\ k)}: turn a
\texttt{Trie\ k\ v} into a state action to generate a graph.

To write this, we lay out our algebra:

\begin{Shaded}
\begin{Highlighting}[]
\NormalTok{trieGraphAlg}
\OtherTok{    ::} \DataTypeTok{TrieF}\NormalTok{ k v (}\DataTypeTok{State} \DataTypeTok{Int}\NormalTok{ (}\DataTypeTok{Gr}\NormalTok{ (}\DataTypeTok{Maybe}\NormalTok{ v) k))}
    \OtherTok{->} \DataTypeTok{State} \DataTypeTok{Int}\NormalTok{ (}\DataTypeTok{Gr}\NormalTok{ (}\DataTypeTok{Maybe}\NormalTok{ v) k)}
\end{Highlighting}
\end{Shaded}

We have to write a function ``how to make a state action creating a graph, given
a map of state actions creating sub-graphs''.

One interesting thing to note is that we have a lot to gain from using
``first-class effects'': \texttt{State\ Int\ (Gr\ (Maybe\ v)\ k)} is just a
normal, inert Haskell value that we can manipulate and sequence however we want.
State is not only explicit, but the sequencing of actions (as first-class
values) is also explicit.

We can write this using \emph{fgl} combinators:

\begin{Shaded}
\begin{Highlighting}[]
\CommentTok{-- source: https://github.com/mstksg/inCode/tree/master/code-samples/trie/trie.hs#L157-L169}

\NormalTok{trieGraphAlg}
\OtherTok{    ::}\NormalTok{ forall k v}\FunctionTok{.}\NormalTok{ ()}
    \OtherTok{=>} \DataTypeTok{TrieF}\NormalTok{ k v (}\DataTypeTok{State} \DataTypeTok{Int}\NormalTok{ (}\DataTypeTok{Gr}\NormalTok{ (}\DataTypeTok{Maybe}\NormalTok{ v) k))}
    \OtherTok{->} \DataTypeTok{State} \DataTypeTok{Int}\NormalTok{ (}\DataTypeTok{Gr}\NormalTok{ (}\DataTypeTok{Maybe}\NormalTok{ v) k)}
\NormalTok{trieGraphAlg (}\DataTypeTok{MkTF}\NormalTok{ v xs) }\FunctionTok{=} \KeywordTok{do}
\NormalTok{    n         }\OtherTok{<-}\NormalTok{ fresh}
\NormalTok{    subgraphs }\OtherTok{<-}\NormalTok{ sequence xs}
    \KeywordTok{let}\OtherTok{ subroots ::}\NormalTok{ [(k, }\DataTypeTok{Int}\NormalTok{)]}
\NormalTok{        subroots }\FunctionTok{=}\NormalTok{ M.toList }\FunctionTok{.}\NormalTok{ fmap (fst }\FunctionTok{.}\NormalTok{ G.nodeRange) }\FunctionTok{$}\NormalTok{ subgraphs}
\NormalTok{    pure }\FunctionTok{$}\NormalTok{ G.insEdges ((\textbackslash{}(k,i) }\OtherTok{->}\NormalTok{ (n,i,k)) }\FunctionTok{<$>}\NormalTok{ subroots)   }\CommentTok{-- insert root-to-subroots}
         \FunctionTok{.}\NormalTok{ G.insNode (n, v)                     }\CommentTok{-- insert new root}
         \FunctionTok{.}\NormalTok{ M.foldr (G.ufold (}\FunctionTok{G.&}\NormalTok{)) G.empty      }\CommentTok{-- merge all subgraphs}
         \FunctionTok{$}\NormalTok{ subgraphs}
\end{Highlighting}
\end{Shaded}

\begin{enumerate}
\def\labelenumi{\arabic{enumi}.}
\item
  First, generate a fresh node label
\item
  Then, sequence all of the state actions inside the map of sub-graph
  generators. Remember, a
  \texttt{TrieF\ k\ v\ (State\ Int\ (Gr\ (Maybe\ v)\ k))} contains a
  \texttt{Maybe\ v} and a \texttt{Map\ k\ (State\ Int\ (Gr\ (Maybe\ v)\ k))}.
  The map contains State actions to create the sub-graphs, and we use:

\begin{Shaded}
\begin{Highlighting}[]
\NormalTok{sequence}
\OtherTok{    ::} \DataTypeTok{Map}\NormalTok{ k (}\DataTypeTok{State} \DataTypeTok{Int}\NormalTok{ (}\DataTypeTok{Gr}\NormalTok{ (}\DataTypeTok{Maybe}\NormalTok{ v) k))}
    \OtherTok{->} \DataTypeTok{State} \DataTypeTok{Int}\NormalTok{ (}\DataTypeTok{Map}\NormalTok{ k (}\DataTypeTok{Gr}\NormalTok{ (}\DataTypeTok{Maybe}\NormalTok{ v) k))}
\end{Highlighting}
\end{Shaded}

  To turn a map of subgraph-producing actions into an action producing a map of
  subgraphs.
\item
  Next, it's useful to collect all of the subroots,
  \texttt{subroots\ ::\ {[}(k,\ Int){]}}. These are all of the node id's of the
  roots of each of the subtrees, paired with the token leading to that subtree.
\item
  Now to generate our result:

  \begin{enumerate}
  \def\labelenumii{\alph{enumii}.}
  \tightlist
  \item
    First we merge all subgraphs (using \texttt{G.ufold\ (G.\&)} to merge
    together two graphs)
  \item
    Then, we insert the new root, with our fresh node ID and the new
    \texttt{Maybe\ v} label.
  \item
    Then, we insert all of the edges connecting our new root to the root of all
    our subgraphs (in \texttt{subroots}).
  \end{enumerate}
\end{enumerate}

We can then write our graph generating function using this algebra, and then
running the resulting \texttt{State\ Int\ (Gr\ (Maybe\ v)\ k)} action:

\begin{Shaded}
\begin{Highlighting}[]
\CommentTok{-- source: https://github.com/mstksg/inCode/tree/master/code-samples/trie/trie.hs#L152-L155}

\NormalTok{trieGraph}
\OtherTok{    ::} \DataTypeTok{Trie}\NormalTok{ k v}
    \OtherTok{->} \DataTypeTok{Gr}\NormalTok{ (}\DataTypeTok{Maybe}\NormalTok{ v) k}
\NormalTok{trieGraph }\FunctionTok{=}\NormalTok{ flip evalState }\DecValTok{0} \FunctionTok{.}\NormalTok{ cata trieGraphAlg}
\end{Highlighting}
\end{Shaded}

Finally, we can write our \texttt{mapToGraph}:

\begin{Shaded}
\begin{Highlighting}[]
\NormalTok{mapToGraph}
\OtherTok{    ::} \DataTypeTok{Ord}\NormalTok{ k}
    \OtherTok{=>} \DataTypeTok{Map}\NormalTok{ [k] v}
    \OtherTok{->} \DataTypeTok{Gr}\NormalTok{ (}\DataTypeTok{Maybe}\NormalTok{ v) k}
\NormalTok{mapToGraph }\FunctionTok{=}\NormalTok{ flip evalState }\DecValTok{0}
           \FunctionTok{.}\NormalTok{ cata trieGraphAlg}
           \FunctionTok{.}\NormalTok{ ana fromGraphCoalg}
\end{Highlighting}
\end{Shaded}

\hypertarget{hylomorphisms}{%
\subsection{Hylomorphisms}\label{hylomorphisms}}

Actually, writing things out as \texttt{mapToGraph} gives us some interesting
insight: our function takes a \texttt{Map\ {[}k{]}\ v}, and returns a
\texttt{Gr\ (Maybe\ v)\ k}. Notice that \texttt{Trie\ k\ v} isn't anywhere in
the type signature. This means that, to, the external user, \texttt{Trie}'s role
is completely ``internal''.

In other words, \texttt{Trie} ``doesn't matter'' at all. We really want a
\texttt{Map\ {[}k{]}\ v\ -\textgreater{}\ Graph\ (Maybe\ v)\ k}. We're using
\texttt{Trie} as an \emph{intermediate data structure}. We are exploiting its
structure to do write our full function, and we don't care about it outside of
that. We build it up with \texttt{ana} and then immediately tear it down with
\texttt{cata}, and it is completely invisible to the outside world.

One neat thing about \emph{recursion-schemes} is that it lets us capture this
``the actual fixed-point is only intermediate and is not directly consequential
to the outside world'' pattern. The logic goes like this:

\begin{itemize}
\tightlist
\item
  We don't care about \texttt{Trie} itself as a result our input. We only care
  about it because we exploit its internal structure.
\item
  \texttt{TrieF} already expresses the internal structure of \texttt{Trie}
\item
  Therefore, if we only want to take advantage of the structure (and not use
  \texttt{Trie} as a direct input or output), we can use \texttt{TrieF}
  \emph{only}, completely bypassing \texttt{Trie}.
\end{itemize}

This \emph{should} make sense, because the only reason we use \texttt{Trie} is
for its internal structure. But \texttt{TrieF} already captures the internal
structure so we really only need to ever worry about \texttt{TrieF}. We don't
actually care about the recursive data type --- we never did!

So, \emph{recursion-schemes} offers the \emph{hylomorphism}:

\begin{Shaded}
\begin{Highlighting}[]
\NormalTok{hylo}
\OtherTok{    ::}\NormalTok{ (}\DataTypeTok{TrieF}\NormalTok{ k v b }\OtherTok{->}\NormalTok{ b)   }\CommentTok{-- ^ an algebra}
    \OtherTok{->}\NormalTok{ (a }\OtherTok{->} \DataTypeTok{TrieF}\NormalTok{ k v a)   }\CommentTok{-- ^ a coalgebra}
    \OtherTok{->}\NormalTok{ a}
    \OtherTok{->}\NormalTok{ b}
\end{Highlighting}
\end{Shaded}

If we see the coalgebra \texttt{a\ -\textgreater{}\ TrieF\ k\ v\ a} as a
``building'' function, and the algebra
\texttt{TrieF\ k\ v\ b\ -\textgreater{}\ b} as a ``consuming'' function, then
\texttt{hylo} will \emph{build, then immediately consume}. It'll build with the
coalgebra on \texttt{TrieF}, then immediately consume with the algebra on
\texttt{TrieF}. No \texttt{Trie} is ever generated, because it's never
necessary: we're literally just building and immediately consuming
\texttt{TrieF} values.

We could even implement \texttt{hylo} ourselves, to illustrate the ``build and
immediately consume'' property:

\begin{Shaded}
\begin{Highlighting}[]
\CommentTok{-- source: https://github.com/mstksg/inCode/tree/master/code-samples/trie/trie.hs#L177-L184}

\NormalTok{hylo'}
\OtherTok{    ::}\NormalTok{ (}\DataTypeTok{TrieF}\NormalTok{ k v b }\OtherTok{->}\NormalTok{ b)   }\CommentTok{-- ^ an algebra}
    \OtherTok{->}\NormalTok{ (a }\OtherTok{->} \DataTypeTok{TrieF}\NormalTok{ k v a)   }\CommentTok{-- ^ a coalgebra}
    \OtherTok{->}\NormalTok{ a}
    \OtherTok{->}\NormalTok{ b}
\NormalTok{hylo' consume build }\FunctionTok{=}\NormalTok{ consume}
                    \FunctionTok{.}\NormalTok{ fmap (hylo' consume build)}
                    \FunctionTok{.}\NormalTok{ build}
\end{Highlighting}
\end{Shaded}

Note that the implementation of \texttt{hylo} works for any \texttt{Functor}
instance: we build and consume along any \texttt{Functor}, taking advantage of
the specific functor's structure.

To me, implementing a function in terms of \texttt{hylo} (or its cousin
\texttt{chrono}, the chronomorphism) represents the ultimate ``victory'' in
using \emph{recursion-schemes} to refactor out your recursive functions. That's
because it helps us realize that we never really \emph{cared} about having a
recursive data type. \texttt{Trie} was never the actual thing we wanted: we just
wanted the layer-by-layer structure. This whole time, we just cared about the
structure of \texttt{TrieF} (and its structure), \emph{not} \texttt{Trie}. Being
able to use \texttt{hylo} lets us see that the original recursive data type was
nothing more than a distraction. Through it, we see the light.

\hypertarget{the-full-package}{%
\subsection{The Full Package}\label{the-full-package}}

Now time to wrap things up. I made a text file storing all of the prequel quotes
in the original reference trie, along with images stored on my drive:

\begin{verbatim}
-- source: https://github.com/mstksg/inCode/tree/master/code-samples/trie/quotes.txt

I DON'T THINK SO,img/idts.jpg
I DON'T THINK THE SYSTEM WORKS,img/idttsw.jpg
I HAVE BEEN LOOKING FORWARD TO THIS,img/iblftt.jpg
I HAVE A BAD FEELING ABOUT THIS,img/ihabfat.jpg
IT'S TREASON THEN,img/itt.jpg
IT'S OUTRAGEOUS IT'S UNFAIR,img/tioiu.jpg
\end{verbatim}

We can write a quick parser and aggregator into a
\texttt{Map\ {[}Char{]}\ Label}, where \texttt{Label} is from the
\emph{graphviz} library, a renderable object to display on the final image.

\begin{Shaded}
\begin{Highlighting}[]
\CommentTok{-- source: https://github.com/mstksg/inCode/tree/master/code-samples/trie/trie.hs#L186-L192}

\OtherTok{memeMap ::} \DataTypeTok{String} \OtherTok{->} \DataTypeTok{Map} \DataTypeTok{String} \DataTypeTok{HTML.Label}
\NormalTok{memeMap }\FunctionTok{=}\NormalTok{ M.fromList }\FunctionTok{.}\NormalTok{ map (uncurry processLine }\FunctionTok{.}\NormalTok{ span (}\FunctionTok{/=} \CharTok{','}\NormalTok{)) }\FunctionTok{.}\NormalTok{ lines}
  \KeywordTok{where}
\NormalTok{    processLine qt (drop }\DecValTok{1}\OtherTok{->}\NormalTok{img) }\FunctionTok{=}\NormalTok{ (filter (not }\FunctionTok{.}\NormalTok{ isSpace) qt, }\DataTypeTok{HTML.Table}\NormalTok{ (}\DataTypeTok{HTML.HTable} \DataTypeTok{Nothing}\NormalTok{ [] [r1,r2]))}
      \KeywordTok{where}
\NormalTok{        r1 }\FunctionTok{=} \DataTypeTok{HTML.Cells}\NormalTok{ [}\DataTypeTok{HTML.LabelCell}\NormalTok{ [] (}\DataTypeTok{HTML.Text}\NormalTok{ [}\DataTypeTok{HTML.Str}\NormalTok{ (T.pack qt)])]}
\NormalTok{        r2 }\FunctionTok{=} \DataTypeTok{HTML.Cells}\NormalTok{ [}\DataTypeTok{HTML.ImgCell}\NormalTok{   [] (}\DataTypeTok{HTML.Img}\NormalTok{ [}\DataTypeTok{HTML.Src}\NormalTok{ img])]}
\end{Highlighting}
\end{Shaded}

A small utility function to clean up our final graph; it deletes nodes that only
have one child and compacts them into the node above. It's just to ``compress''
together strings of nodes that don't have any forks.

\begin{Shaded}
\begin{Highlighting}[]
\CommentTok{-- source: https://github.com/mstksg/inCode/tree/master/code-samples/trie/trie.hs#L212-L223}

\NormalTok{compactify}
\OtherTok{    ::} \DataTypeTok{Gr}\NormalTok{ (}\DataTypeTok{Maybe}\NormalTok{ v) k}
    \OtherTok{->} \DataTypeTok{Gr}\NormalTok{ (}\DataTypeTok{Maybe}\NormalTok{ v) [k]}
\NormalTok{compactify g0 }\FunctionTok{=}\NormalTok{ foldl' go (G.emap (}\FunctionTok{:}\NormalTok{[]) g0) (G.labNodes g0)}
  \KeywordTok{where}
\NormalTok{    go g (i, v) }\FunctionTok{=} \KeywordTok{case}\NormalTok{ (G.inn g i, G.out g i) }\KeywordTok{of}
\NormalTok{      ([(j, _, lj)], [(_, k, lk)])}
        \FunctionTok{|}\NormalTok{ isNothing v }\OtherTok{->}\NormalTok{ G.insEdge (j, k, lj }\FunctionTok{++}\NormalTok{ lk)}
                       \FunctionTok{.}\NormalTok{ G.delNode i}
                       \FunctionTok{.}\NormalTok{ G.delEdges [(j,i),(i,k)]}
                       \FunctionTok{$}\NormalTok{ g}
\NormalTok{      _               }\OtherTok{->}\NormalTok{ g}
\end{Highlighting}
\end{Shaded}

We can directly output a compacted graph from \texttt{graphAlg}, but for the
sake of this post it's a bit cleaner to separate out these concerns.

We'll write a function to turn a \texttt{Gr\ (Maybe\ v)\ {[}Char{]}} into a dot
file, using \emph{graphviz} to do most of the work:

\begin{Shaded}
\begin{Highlighting}[]
\CommentTok{-- source: https://github.com/mstksg/inCode/tree/master/code-samples/trie/trie.hs#L194-L205}

\NormalTok{graphDot}
\OtherTok{    ::} \DataTypeTok{GV.Labellable}\NormalTok{ v}
    \OtherTok{=>} \DataTypeTok{Gr}\NormalTok{ (}\DataTypeTok{Maybe}\NormalTok{ v) }\DataTypeTok{String}
    \OtherTok{->} \DataTypeTok{T.Text}
\NormalTok{graphDot }\FunctionTok{=}\NormalTok{ GV.printIt }\FunctionTok{.}\NormalTok{ GV.graphToDot params}
  \KeywordTok{where}
\NormalTok{    params }\FunctionTok{=}\NormalTok{ GV.nonClusteredParams}
\NormalTok{      \{ fmtNode }\FunctionTok{=}\NormalTok{ \textbackslash{}(_,  l) }\OtherTok{->} \KeywordTok{case}\NormalTok{ l }\KeywordTok{of}
          \DataTypeTok{Nothing} \OtherTok{->}\NormalTok{ [GV.shape }\DataTypeTok{GV.PointShape}\NormalTok{]}
          \DataTypeTok{Just}\NormalTok{ l' }\OtherTok{->}\NormalTok{ [GV.toLabel l', GV.shape }\DataTypeTok{GV.PlainText}\NormalTok{]}
\NormalTok{      , fmtEdge }\FunctionTok{=}\NormalTok{ \textbackslash{}(_,_,l) }\OtherTok{->}\NormalTok{ [GV.toLabel (concat [}\StringTok{"["}\NormalTok{, l, }\StringTok{"]"}\NormalTok{])]}
\NormalTok{      \}}
\end{Highlighting}
\end{Shaded}

And finally, to wrap it all together, the entire pipeline:

\begin{Shaded}
\begin{Highlighting}[]
\CommentTok{-- source: https://github.com/mstksg/inCode/tree/master/code-samples/trie/trie.hs#L207-L210}

\NormalTok{memeDot}
\OtherTok{    ::} \DataTypeTok{String}
    \OtherTok{->} \DataTypeTok{T.Text}
\NormalTok{memeDot }\FunctionTok{=}\NormalTok{ graphDot }\FunctionTok{.}\NormalTok{ compactify }\FunctionTok{.}\NormalTok{ mapToGraph }\FunctionTok{.}\NormalTok{ memeMap}
\end{Highlighting}
\end{Shaded}

Giving us our final result:

\begin{figure}
\centering
\includegraphics{/img/entries/trie/meme-trie.png}
\caption{Our rendered dotfile, using graphviz}
\end{figure}

\hypertarget{signoff}{%
\section{Signoff}\label{signoff}}

Hi, thanks for reading! You can reach me via email at
\href{mailto:justin@jle.im}{\nolinkurl{justin@jle.im}}, or at twitter at
\href{https://twitter.com/mstk}{@mstk}! This post and all others are published
under the \href{https://creativecommons.org/licenses/by-nc-nd/3.0/}{CC-BY-NC-ND
3.0} license. Corrections and edits via pull request are welcome and encouraged
at \href{https://github.com/mstksg/inCode}{the source repository}.

If you feel inclined, or this post was particularly helpful for you, why not
consider \href{https://www.patreon.com/justinle/overview}{supporting me on
Patreon}, or a \href{bitcoin:3D7rmAYgbDnp4gp4rf22THsGt74fNucPDU}{BTC donation}?
:)

\end{document}
