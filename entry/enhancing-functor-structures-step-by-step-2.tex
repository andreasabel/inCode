\documentclass[]{article}
\usepackage{lmodern}
\usepackage{amssymb,amsmath}
\usepackage{ifxetex,ifluatex}
\usepackage{fixltx2e} % provides \textsubscript
\ifnum 0\ifxetex 1\fi\ifluatex 1\fi=0 % if pdftex
  \usepackage[T1]{fontenc}
  \usepackage[utf8]{inputenc}
\else % if luatex or xelatex
  \ifxetex
    \usepackage{mathspec}
    \usepackage{xltxtra,xunicode}
  \else
    \usepackage{fontspec}
  \fi
  \defaultfontfeatures{Mapping=tex-text,Scale=MatchLowercase}
  \newcommand{\euro}{€}
\fi
% use upquote if available, for straight quotes in verbatim environments
\IfFileExists{upquote.sty}{\usepackage{upquote}}{}
% use microtype if available
\IfFileExists{microtype.sty}{\usepackage{microtype}}{}
\usepackage[margin=1in]{geometry}
\usepackage{color}
\usepackage{fancyvrb}
\newcommand{\VerbBar}{|}
\newcommand{\VERB}{\Verb[commandchars=\\\{\}]}
\DefineVerbatimEnvironment{Highlighting}{Verbatim}{commandchars=\\\{\}}
% Add ',fontsize=\small' for more characters per line
\newenvironment{Shaded}{}{}
\newcommand{\AlertTok}[1]{\textcolor[rgb]{1.00,0.00,0.00}{\textbf{#1}}}
\newcommand{\AnnotationTok}[1]{\textcolor[rgb]{0.38,0.63,0.69}{\textbf{\textit{#1}}}}
\newcommand{\AttributeTok}[1]{\textcolor[rgb]{0.49,0.56,0.16}{#1}}
\newcommand{\BaseNTok}[1]{\textcolor[rgb]{0.25,0.63,0.44}{#1}}
\newcommand{\BuiltInTok}[1]{#1}
\newcommand{\CharTok}[1]{\textcolor[rgb]{0.25,0.44,0.63}{#1}}
\newcommand{\CommentTok}[1]{\textcolor[rgb]{0.38,0.63,0.69}{\textit{#1}}}
\newcommand{\CommentVarTok}[1]{\textcolor[rgb]{0.38,0.63,0.69}{\textbf{\textit{#1}}}}
\newcommand{\ConstantTok}[1]{\textcolor[rgb]{0.53,0.00,0.00}{#1}}
\newcommand{\ControlFlowTok}[1]{\textcolor[rgb]{0.00,0.44,0.13}{\textbf{#1}}}
\newcommand{\DataTypeTok}[1]{\textcolor[rgb]{0.56,0.13,0.00}{#1}}
\newcommand{\DecValTok}[1]{\textcolor[rgb]{0.25,0.63,0.44}{#1}}
\newcommand{\DocumentationTok}[1]{\textcolor[rgb]{0.73,0.13,0.13}{\textit{#1}}}
\newcommand{\ErrorTok}[1]{\textcolor[rgb]{1.00,0.00,0.00}{\textbf{#1}}}
\newcommand{\ExtensionTok}[1]{#1}
\newcommand{\FloatTok}[1]{\textcolor[rgb]{0.25,0.63,0.44}{#1}}
\newcommand{\FunctionTok}[1]{\textcolor[rgb]{0.02,0.16,0.49}{#1}}
\newcommand{\ImportTok}[1]{#1}
\newcommand{\InformationTok}[1]{\textcolor[rgb]{0.38,0.63,0.69}{\textbf{\textit{#1}}}}
\newcommand{\KeywordTok}[1]{\textcolor[rgb]{0.00,0.44,0.13}{\textbf{#1}}}
\newcommand{\NormalTok}[1]{#1}
\newcommand{\OperatorTok}[1]{\textcolor[rgb]{0.40,0.40,0.40}{#1}}
\newcommand{\OtherTok}[1]{\textcolor[rgb]{0.00,0.44,0.13}{#1}}
\newcommand{\PreprocessorTok}[1]{\textcolor[rgb]{0.74,0.48,0.00}{#1}}
\newcommand{\RegionMarkerTok}[1]{#1}
\newcommand{\SpecialCharTok}[1]{\textcolor[rgb]{0.25,0.44,0.63}{#1}}
\newcommand{\SpecialStringTok}[1]{\textcolor[rgb]{0.73,0.40,0.53}{#1}}
\newcommand{\StringTok}[1]{\textcolor[rgb]{0.25,0.44,0.63}{#1}}
\newcommand{\VariableTok}[1]{\textcolor[rgb]{0.10,0.09,0.49}{#1}}
\newcommand{\VerbatimStringTok}[1]{\textcolor[rgb]{0.25,0.44,0.63}{#1}}
\newcommand{\WarningTok}[1]{\textcolor[rgb]{0.38,0.63,0.69}{\textbf{\textit{#1}}}}
\usepackage{fancyvrb}
\ifxetex
  \usepackage[setpagesize=false, % page size defined by xetex
              unicode=false, % unicode breaks when used with xetex
              xetex]{hyperref}
\else
  \usepackage[unicode=true]{hyperref}
\fi
\hypersetup{breaklinks=true,
            bookmarks=true,
            pdfauthor={Justin Le},
            pdftitle={Enhancing Functor Structures Step-By-Step (Part 2)},
            colorlinks=true,
            citecolor=blue,
            urlcolor=blue,
            linkcolor=magenta,
            pdfborder={0 0 0}}
\urlstyle{same}  % don't use monospace font for urls
% Make links footnotes instead of hotlinks:
\renewcommand{\href}[2]{#2\footnote{\url{#1}}}
\setlength{\parindent}{0pt}
\setlength{\parskip}{6pt plus 2pt minus 1pt}
\setlength{\emergencystretch}{3em}  % prevent overfull lines
\setcounter{secnumdepth}{0}
\VerbatimFootnotes % allows verbatim text in footnotes

\title{Enhancing Functor Structures Step-By-Step (Part 2)}
\author{Justin Le}

\begin{document}
\maketitle

\emph{Originally posted on
\textbf{\href{https://blog.jle.im/entry/enhancing-functor-structures-step-by-step-2.html}{in
Code}}.}

\hypertarget{parsing-and-serializing-invariantly}{%
\section{Parsing and Serializing
Invariantly}\label{parsing-and-serializing-invariantly}}

At this point, we have:

\begin{enumerate}
\def\labelenumi{\arabic{enumi}.}
\tightlist
\item
  Started with a simple ADT representing the structure we want to be able to
  express
\item
  Enhanced that simple ADT with Covariant Functor capabilities, in order to
  interpret it as a parser
\item
  Enhanced that original simple ADT with Contravariant Functor, in order to
  interpret it as a serializer.
\end{enumerate}

From this, it seems the next logical step would be to add \emph{both}
enhancements to the same structure!

There are some clear benefits to this --- for example, we can now ensure that
our ``serialization'' and ``parsing'' functions are always ``in sync''. If we
defined a separate process/type for serializing and a separate process/type for
parsing, then it's possible we might accidentally make errors in keeping them in
sync\ldots one might use a different tag, or we might make changes to one but
not the other during refactoring.

Like before, the main thing we need to change at the fundamental level is
\texttt{Primitive}:

\begin{Shaded}
\begin{Highlighting}[]
\CommentTok{{-}{-} source: https://github.com/mstksg/inCode/tree/master/code{-}samples/functor{-}structures/invariant.hs\#L45{-}L48}

\KeywordTok{data} \DataTypeTok{Primitive}\NormalTok{ a }\OtherTok{=}
      \DataTypeTok{PString}\NormalTok{ (a }\OtherTok{{-}>} \DataTypeTok{String}\NormalTok{)     (}\DataTypeTok{String}     \OtherTok{{-}>} \DataTypeTok{Maybe}\NormalTok{ a)}
    \OperatorTok{|} \DataTypeTok{PNumber}\NormalTok{ (a }\OtherTok{{-}>} \DataTypeTok{Scientific}\NormalTok{) (}\DataTypeTok{Scientific} \OtherTok{{-}>} \DataTypeTok{Maybe}\NormalTok{ a)}
    \OperatorTok{|} \DataTypeTok{PBool}\NormalTok{   (a }\OtherTok{{-}>} \DataTypeTok{Bool}\NormalTok{)       (}\DataTypeTok{Bool}       \OtherTok{{-}>} \DataTypeTok{Maybe}\NormalTok{ a)}
\end{Highlighting}
\end{Shaded}

We're just basically combining the additions we made to enable parsing with the
additions we made to enable serialization. Our new \texttt{Primitive} type gives
us the capability to do both!

We call this new \texttt{Primitive} an
\href{https://hackage.haskell.org/package/invariant/docs/Data-Functor-Invariant.html}{``Invariant''
Functor}: these are functors that give you ``both'' capabilities: interpreting
covariantly \emph{and} contravariantly.

\hypertarget{divap-and-decalt}{%
\subsection{DivAp and DecAlt}\label{divap-and-decalt}}

By now, we know the drill. We also need to change our \texttt{RecordType} and
\texttt{SumType} constructors to get the right type of container.

\begin{Shaded}
\begin{Highlighting}[]
\CommentTok{{-}{-} Covariant Schema}
\CommentTok{{-}{-} source: https://github.com/mstksg/inCode/tree/master/code{-}samples/functor{-}structures/parse.hs\#L27{-}L31}

\KeywordTok{data} \DataTypeTok{Schema}\NormalTok{ a }\OtherTok{=}
      \DataTypeTok{RecordType}\NormalTok{  (}\DataTypeTok{Ap} \DataTypeTok{Field}\NormalTok{ a)}
    \OperatorTok{|} \DataTypeTok{SumType}\NormalTok{     (}\DataTypeTok{ListF} \DataTypeTok{Choice}\NormalTok{ a)}
    \OperatorTok{|} \DataTypeTok{SchemaLeaf}\NormalTok{  (}\DataTypeTok{Primitive}\NormalTok{ a)}
  \KeywordTok{deriving} \DataTypeTok{Functor}

\CommentTok{{-}{-} Contravariant Schema}
\CommentTok{{-}{-} source: https://github.com/mstksg/inCode/tree/master/code{-}samples/functor{-}structures/serialize.hs\#L27{-}L30}

\KeywordTok{data} \DataTypeTok{Schema}\NormalTok{ a }\OtherTok{=}
      \DataTypeTok{RecordType}\NormalTok{  (}\DataTypeTok{Div} \DataTypeTok{Field}\NormalTok{ a)}
    \OperatorTok{|} \DataTypeTok{SumType}\NormalTok{     (}\DataTypeTok{Dec} \DataTypeTok{Choice}\NormalTok{ a)}
    \OperatorTok{|} \DataTypeTok{SchemaLeaf}\NormalTok{  (}\DataTypeTok{Primitive}\NormalTok{ a)}
\end{Highlighting}
\end{Shaded}

For the covariant \texttt{RecordType}, we used \texttt{Ap\ Field\ a}. For the
contravariant \texttt{RecordType}, we used \texttt{Div\ Field\ a}. Is there a
type that combines \emph{both} \texttt{Ap} and \texttt{Div}?

Ah, we're in luck! We have
\emph{\href{https://hackage.haskell.org/package/functor-combinators/docs/Data-Functor-Invariant-DivAp.html}{DivAp}}
from the \emph{functor-combinatotrs} library\ldots which is named to invoke the
idea of having both \texttt{Ap} and \texttt{Div} capabilities, combined
together.

For the covariant \texttt{SumType}, we used \texttt{ListF\ Choice\ a}. For the
contravariant \texttt{SumType}, we used \texttt{Dec\ Choice\ a}. Is there a type
that combines \emph{both} \texttt{ListF} and \texttt{Dec}?

Ah hah, if we look nearby \texttt{DivAp}, we see the answer:
\emph{\href{https://hackage.haskell.org/package/functor-combinators/docs/Data-Functor-Invariant-DecAlt.html}{DecAlt}}!
It combines both \texttt{ListF} and \texttt{Dec}.

\hypertarget{building-an-invariant-schema}{%
\subsection{Building an Invariant Schema}\label{building-an-invariant-schema}}

Now let's wire it up:

\begin{Shaded}
\begin{Highlighting}[]
\CommentTok{{-}{-} source: https://github.com/mstksg/inCode/tree/master/code{-}samples/functor{-}structures/invariant.hs\#L30{-}L48}

\KeywordTok{data} \DataTypeTok{Schema}\NormalTok{ a }\OtherTok{=}
      \DataTypeTok{RecordType}\NormalTok{  (}\DataTypeTok{DivAp}   \DataTypeTok{Field}\NormalTok{  a)}
    \OperatorTok{|} \DataTypeTok{SumType}\NormalTok{     (}\DataTypeTok{DecAlt} \DataTypeTok{Choice}\NormalTok{ a)}
    \OperatorTok{|} \DataTypeTok{SchemaLeaf}\NormalTok{  (}\DataTypeTok{Primitive}\NormalTok{ a)}

\KeywordTok{data} \DataTypeTok{Field}\NormalTok{ a }\OtherTok{=} \DataTypeTok{Field}
\NormalTok{    \{}\OtherTok{ fieldName  ::} \DataTypeTok{String}
\NormalTok{    ,}\OtherTok{ fieldValue ::} \DataTypeTok{Schema}\NormalTok{ a}
\NormalTok{    \}}

\KeywordTok{data} \DataTypeTok{Choice}\NormalTok{ a }\OtherTok{=} \DataTypeTok{Choice}
\NormalTok{    \{}\OtherTok{ choiceName  ::} \DataTypeTok{String}
\NormalTok{    ,}\OtherTok{ choiceValue ::} \DataTypeTok{Schema}\NormalTok{ a}
\NormalTok{    \}}

\KeywordTok{data} \DataTypeTok{Primitive}\NormalTok{ a }\OtherTok{=}
      \DataTypeTok{PString}\NormalTok{ (a }\OtherTok{{-}>} \DataTypeTok{String}\NormalTok{)     (}\DataTypeTok{String}     \OtherTok{{-}>} \DataTypeTok{Maybe}\NormalTok{ a)}
    \OperatorTok{|} \DataTypeTok{PNumber}\NormalTok{ (a }\OtherTok{{-}>} \DataTypeTok{Scientific}\NormalTok{) (}\DataTypeTok{Scientific} \OtherTok{{-}>} \DataTypeTok{Maybe}\NormalTok{ a)}
    \OperatorTok{|} \DataTypeTok{PBool}\NormalTok{   (a }\OtherTok{{-}>} \DataTypeTok{Bool}\NormalTok{)       (}\DataTypeTok{Bool}       \OtherTok{{-}>} \DataTypeTok{Maybe}\NormalTok{ a)}
\end{Highlighting}
\end{Shaded}

Writing a schema using this type is going to be very similar to writing the
contravariant version.

\begin{Shaded}
\begin{Highlighting}[]
\CommentTok{{-}{-} source: https://github.com/mstksg/inCode/tree/master/code{-}samples/functor{-}structures/invariant.hs\#L61{-}L76}

\OtherTok{customerSchema ::} \DataTypeTok{Schema} \DataTypeTok{Customer}
\NormalTok{customerSchema }\OtherTok{=} \DataTypeTok{SumType} \OperatorTok{$}
\NormalTok{    swerve (\textbackslash{}}\KeywordTok{case} \DataTypeTok{CPerson}\NormalTok{ x y }\OtherTok{{-}>} \DataTypeTok{Left}\NormalTok{ (x,y); }\DataTypeTok{CBusiness}\NormalTok{ x }\OtherTok{{-}>} \DataTypeTok{Right}\NormalTok{ x) (}\FunctionTok{uncurry} \DataTypeTok{CPerson}\NormalTok{) }\DataTypeTok{CBusiness}
\NormalTok{        (inject }\DataTypeTok{Choice}
\NormalTok{          \{ choiceName  }\OtherTok{=} \StringTok{"Person"}
\NormalTok{          , choiceValue }\OtherTok{=} \DataTypeTok{RecordType} \OperatorTok{$}\NormalTok{ gathered}
\NormalTok{              (inject }\DataTypeTok{Field}\NormalTok{ \{ fieldName }\OtherTok{=} \StringTok{"Name"}\NormalTok{, fieldValue }\OtherTok{=} \DataTypeTok{SchemaLeaf}\NormalTok{ pString \})}
\NormalTok{              (inject }\DataTypeTok{Field}\NormalTok{ \{ fieldName }\OtherTok{=} \StringTok{"Age"}\NormalTok{ , fieldValue }\OtherTok{=} \DataTypeTok{SchemaLeaf}\NormalTok{ pInt    \})}
\NormalTok{          \}}
\NormalTok{        )}
\NormalTok{        (inject }\DataTypeTok{Choice}
\NormalTok{          \{ choiceName  }\OtherTok{=} \StringTok{"Business"}
\NormalTok{          , choiceValue }\OtherTok{=} \DataTypeTok{RecordType} \OperatorTok{$}
\NormalTok{              inject }\DataTypeTok{Field}\NormalTok{ \{ fieldName }\OtherTok{=} \StringTok{"Age"}\NormalTok{ , fieldValue }\OtherTok{=} \DataTypeTok{SchemaLeaf}\NormalTok{ pInt \}}
\NormalTok{          \}}
\NormalTok{        )}
\end{Highlighting}
\end{Shaded}

The main difference is, while \texttt{decide} expects the
\texttt{a\ -\textgreater{}\ Either\ b\ c} splitting function, \texttt{swerve}
(the invariant \texttt{DecAlt} equivalent) expects also the functions to
``recombine'' the \texttt{b} and \texttt{c} back to \texttt{a}. We also note
that the invariant version of \texttt{divided} is \texttt{gathered}.

\begin{Shaded}
\begin{Highlighting}[]
\NormalTok{swerve}
\OtherTok{    ::}\NormalTok{ (a }\OtherTok{{-}>} \DataTypeTok{Either}\NormalTok{ b c)    }\CommentTok{{-}{-} \^{} break into branches}
    \OtherTok{{-}>}\NormalTok{ (b }\OtherTok{{-}>}\NormalTok{ a)             }\CommentTok{{-}{-} \^{} put the branch back into the original input}
    \OtherTok{{-}>}\NormalTok{ (c }\OtherTok{{-}>}\NormalTok{ a)             }\CommentTok{{-}{-} \^{} put the branch back into the original input}
    \OtherTok{{-}>} \DataTypeTok{DecAlt}\NormalTok{ f b           }\CommentTok{{-}{-} \^{} handle first branch}
    \OtherTok{{-}>} \DataTypeTok{DecAlt}\NormalTok{ f c           }\CommentTok{{-}{-} \^{} handle second branch}
    \OtherTok{{-}>} \DataTypeTok{DecAlt}\NormalTok{ f a           }\CommentTok{{-}{-} \^{} overall handler}

\NormalTok{swerve}
\OtherTok{    ::}\NormalTok{ (}\DataTypeTok{Customer} \OtherTok{{-}>} \DataTypeTok{Either}\NormalTok{ (}\DataTypeTok{String}\NormalTok{, }\DataTypeTok{Int}\NormalTok{) }\DataTypeTok{Int}\NormalTok{)   }\CommentTok{{-}{-} \^{} break into branches}
    \OtherTok{{-}>}\NormalTok{ ((}\DataTypeTok{String}\NormalTok{, }\DataTypeTok{Int}\NormalTok{) }\OtherTok{{-}>} \DataTypeTok{Customer}\NormalTok{)              }\CommentTok{{-}{-} \^{} put the CPerson branch back into a Customer}
    \OtherTok{{-}>}\NormalTok{ (}\DataTypeTok{Int} \OtherTok{{-}>} \DataTypeTok{Customer}\NormalTok{)                        }\CommentTok{{-}{-} \^{} put the CBusiness branch back into a Customer}
    \OtherTok{{-}>} \DataTypeTok{DecAlt} \DataTypeTok{Choice}\NormalTok{ (}\DataTypeTok{String}\NormalTok{, }\DataTypeTok{Int}\NormalTok{)              }\CommentTok{{-}{-} \^{} handle CPerson branch}
    \OtherTok{{-}>} \DataTypeTok{DecAlt} \DataTypeTok{Choice} \DataTypeTok{Int}                        \CommentTok{{-}{-} \^{} handle CBusiness branch}
    \OtherTok{{-}>} \DataTypeTok{DecAlt} \DataTypeTok{Choice} \DataTypeTok{Customer}
\end{Highlighting}
\end{Shaded}

And \texttt{gathered} works like:

\begin{Shaded}
\begin{Highlighting}[]
\NormalTok{gathered}
\OtherTok{    ::} \DataTypeTok{DivAp}\NormalTok{ f a          }\CommentTok{{-}{-} \^{} first handler}
    \OtherTok{{-}>} \DataTypeTok{DivAp}\NormalTok{ f b          }\CommentTok{{-}{-} \^{} second handler}
    \OtherTok{{-}>} \DataTypeTok{DivAp}\NormalTok{ f (a, b)     }\CommentTok{{-}{-} \^{} merged handler}

\NormalTok{gathered}
\OtherTok{    ::} \DataTypeTok{DivAp} \DataTypeTok{Field} \DataTypeTok{String}          \CommentTok{{-}{-} \^{} handle the cpName field}
    \OtherTok{{-}>} \DataTypeTok{DivAp} \DataTypeTok{Field} \DataTypeTok{Int}             \CommentTok{{-}{-} \^{} handle the cpAge field}
    \OtherTok{{-}>} \DataTypeTok{DivAp} \DataTypeTok{Field}\NormalTok{ (}\DataTypeTok{String}\NormalTok{, }\DataTypeTok{Int}\NormalTok{)   }\CommentTok{{-}{-} \^{} handle both together}
\end{Highlighting}
\end{Shaded}

\hypertarget{using-invariant-schema}{%
\subsection{Using Invariant Schema}\label{using-invariant-schema}}

It looks like we mostly did all the work already. Writing \texttt{schemaDoc},
\texttt{schemaParser}, and \texttt{schemaToValue}, we can re-use pretty much all
of our code! The main (unfortunate) difference is that instead of using
\texttt{interpret} in every case, we can use \texttt{runCoDivAp} to run our
\texttt{DivAp} in a covariant setting, and \texttt{runContraDivAp} to run our
\texttt{DivAp} in a contravariant setting (similarly for \texttt{runCoDecAlt}
and \texttt{runContraDecAlt}). Another small difference is that
\texttt{icollect} doesn't quite work properly on \texttt{DivAp}/\texttt{DecAlt},
so we have to convert them to \texttt{Ap} and \texttt{Dec} first.\footnote{These
  are unfortunate consequences of the fact that there is no general typeclass
  that contains both \texttt{Applicative} and \texttt{Divisible} together, or no
  typeclass that contains both \texttt{Plus} and \texttt{Conclude} together. If
  these existed, we could just use \texttt{interpret} for all four of those
  functions, and \texttt{icollect} would work fine as well.}

\begin{Shaded}
\begin{Highlighting}[]
\CommentTok{{-}{-} source: https://github.com/mstksg/inCode/tree/master/code{-}samples/functor{-}structures/invariant.hs\#L78{-}L154}

\NormalTok{schemaDoc}
\OtherTok{    ::} \DataTypeTok{String}       \CommentTok{{-}{-} \^{} name}
    \OtherTok{{-}>} \DataTypeTok{Schema}\NormalTok{ x     }\CommentTok{{-}{-} \^{} schema}
    \OtherTok{{-}>} \DataTypeTok{PP.Doc}\NormalTok{ a}
\NormalTok{schemaDoc title }\OtherTok{=}\NormalTok{ \textbackslash{}}\KeywordTok{case}
    \DataTypeTok{RecordType}\NormalTok{ fs }\OtherTok{{-}>}\NormalTok{ PP.vsep [}
\NormalTok{        PP.pretty (}\StringTok{"\{"} \OperatorTok{<>}\NormalTok{ title }\OperatorTok{<>} \StringTok{"\}"}\NormalTok{)}
\NormalTok{      , PP.indent }\DecValTok{2} \OperatorTok{.}\NormalTok{ PP.vsep }\OperatorTok{$}
\NormalTok{          icollect (\textbackslash{}fld }\OtherTok{{-}>} \StringTok{"*"} \OperatorTok{PP.<+>}\NormalTok{ PP.indent }\DecValTok{2}\NormalTok{ (fieldDoc fld)) (divApAp fs)}
\NormalTok{      ]}
    \DataTypeTok{SumType}\NormalTok{ cs    }\OtherTok{{-}>}\NormalTok{ PP.vsep [}
\NormalTok{        PP.pretty (}\StringTok{"("} \OperatorTok{<>}\NormalTok{ title }\OperatorTok{<>} \StringTok{")"}\NormalTok{)}
\NormalTok{      , }\StringTok{"Choice of:"}
\NormalTok{      , PP.indent }\DecValTok{2} \OperatorTok{.}\NormalTok{ PP.vsep }\OperatorTok{$}
\NormalTok{          icollect choiceDoc (decAltDec cs)}
\NormalTok{      ]}
    \DataTypeTok{SchemaLeaf}\NormalTok{ p  }\OtherTok{{-}>}\NormalTok{ PP.pretty (title }\OperatorTok{<>} \StringTok{":"}\NormalTok{)}
              \OperatorTok{PP.<+>}\NormalTok{ primDoc p}
  \KeywordTok{where}
\OtherTok{    fieldDoc ::} \DataTypeTok{Field}\NormalTok{ x }\OtherTok{{-}>} \DataTypeTok{PP.Doc}\NormalTok{ a}
\NormalTok{    fieldDoc }\DataTypeTok{Field}\NormalTok{\{}\OperatorTok{..}\NormalTok{\} }\OtherTok{=}\NormalTok{ schemaDoc fieldName fieldValue}
\OtherTok{    choiceDoc ::} \DataTypeTok{Choice}\NormalTok{ x }\OtherTok{{-}>} \DataTypeTok{PP.Doc}\NormalTok{ a}
\NormalTok{    choiceDoc }\DataTypeTok{Choice}\NormalTok{\{}\OperatorTok{..}\NormalTok{\} }\OtherTok{=}\NormalTok{ schemaDoc choiceName choiceValue}
\OtherTok{    primDoc ::} \DataTypeTok{Primitive}\NormalTok{ x }\OtherTok{{-}>} \DataTypeTok{PP.Doc}\NormalTok{ a}
\NormalTok{    primDoc }\OtherTok{=}\NormalTok{ \textbackslash{}}\KeywordTok{case}
      \DataTypeTok{PString}\NormalTok{ \_ \_ }\OtherTok{{-}>} \StringTok{"string"}
      \DataTypeTok{PNumber}\NormalTok{ \_ \_ }\OtherTok{{-}>} \StringTok{"number"}
      \DataTypeTok{PBool}\NormalTok{   \_ \_ }\OtherTok{{-}>} \StringTok{"bool"}

\NormalTok{schemaParser}
\OtherTok{    ::} \DataTypeTok{Schema}\NormalTok{ a}
    \OtherTok{{-}>} \DataTypeTok{A.Parse} \DataTypeTok{String}\NormalTok{ a}
\NormalTok{schemaParser }\OtherTok{=}\NormalTok{ \textbackslash{}}\KeywordTok{case}
    \DataTypeTok{RecordType}\NormalTok{ fs }\OtherTok{{-}>}\NormalTok{ runCoDivAp fieldParser fs}
    \DataTypeTok{SumType}\NormalTok{    cs }\OtherTok{{-}>}\NormalTok{ runCoDecAlt choiceParser cs}
    \DataTypeTok{SchemaLeaf}\NormalTok{ p  }\OtherTok{{-}>}\NormalTok{ primParser p}
  \KeywordTok{where}
\OtherTok{    choiceParser ::} \DataTypeTok{Choice}\NormalTok{ b }\OtherTok{{-}>} \DataTypeTok{A.Parse} \DataTypeTok{String}\NormalTok{ b}
\NormalTok{    choiceParser }\DataTypeTok{Choice}\NormalTok{\{}\OperatorTok{..}\NormalTok{\} }\OtherTok{=} \KeywordTok{do}
\NormalTok{      tag }\OtherTok{<{-}}\NormalTok{ A.key }\StringTok{"tag"}\NormalTok{ A.asString}
\NormalTok{      unless (tag }\OperatorTok{==}\NormalTok{ choiceName) }\OperatorTok{$}
\NormalTok{        A.throwCustomError }\StringTok{"Tag does not match"}
\NormalTok{      A.key }\StringTok{"contents"} \OperatorTok{$}\NormalTok{ schemaParser choiceValue}
\OtherTok{    fieldParser ::} \DataTypeTok{Field}\NormalTok{ b }\OtherTok{{-}>} \DataTypeTok{A.Parse} \DataTypeTok{String}\NormalTok{ b}
\NormalTok{    fieldParser }\DataTypeTok{Field}\NormalTok{\{}\OperatorTok{..}\NormalTok{\} }\OtherTok{=}\NormalTok{ A.key (T.pack fieldName) (schemaParser fieldValue)}
\OtherTok{    primParser ::} \DataTypeTok{Primitive}\NormalTok{ b }\OtherTok{{-}>} \DataTypeTok{A.Parse} \DataTypeTok{String}\NormalTok{ b}
\NormalTok{    primParser }\OtherTok{=}\NormalTok{ \textbackslash{}}\KeywordTok{case}
      \DataTypeTok{PString}\NormalTok{ \_ f }\OtherTok{{-}>}\NormalTok{ A.withString }\OperatorTok{$}
        \FunctionTok{maybe}\NormalTok{ (}\DataTypeTok{Left} \StringTok{"error validating string"}\NormalTok{) }\DataTypeTok{Right} \OperatorTok{.}\NormalTok{ f}
      \DataTypeTok{PNumber}\NormalTok{ \_ f }\OtherTok{{-}>}\NormalTok{ A.withScientific }\OperatorTok{$}
        \FunctionTok{maybe}\NormalTok{ (}\DataTypeTok{Left} \StringTok{"error validating number"}\NormalTok{) }\DataTypeTok{Right} \OperatorTok{.}\NormalTok{ f}
      \DataTypeTok{PBool}\NormalTok{ \_ f }\OtherTok{{-}>}\NormalTok{ A.withBool }\OperatorTok{$}
        \FunctionTok{maybe}\NormalTok{ (}\DataTypeTok{Left} \StringTok{"error validating bool"}\NormalTok{) }\DataTypeTok{Right} \OperatorTok{.}\NormalTok{ f}

\NormalTok{schemaToValue}
\OtherTok{    ::} \DataTypeTok{Schema}\NormalTok{ a}
    \OtherTok{{-}>}\NormalTok{ a}
    \OtherTok{{-}>} \DataTypeTok{Aeson.Value}
\NormalTok{schemaToValue }\OtherTok{=}\NormalTok{ \textbackslash{}}\KeywordTok{case}
    \DataTypeTok{RecordType}\NormalTok{ fs }\OtherTok{{-}>}\NormalTok{ Aeson.object}
                   \OperatorTok{.}\NormalTok{ getOp (runContraDivAp fieldToValue fs)}
    \DataTypeTok{SumType}\NormalTok{    cs }\OtherTok{{-}>}\NormalTok{ getOp (runContraDecAlt choiceToValue cs)}
    \DataTypeTok{SchemaLeaf}\NormalTok{ p  }\OtherTok{{-}>}\NormalTok{ primToValue p}
  \KeywordTok{where}
\OtherTok{    choiceToValue ::} \DataTypeTok{Choice}\NormalTok{ x }\OtherTok{{-}>} \DataTypeTok{Op} \DataTypeTok{Aeson.Value}\NormalTok{ x}
\NormalTok{    choiceToValue }\DataTypeTok{Choice}\NormalTok{\{}\OperatorTok{..}\NormalTok{\} }\OtherTok{=} \DataTypeTok{Op} \OperatorTok{$}\NormalTok{ \textbackslash{}x }\OtherTok{{-}>}\NormalTok{ Aeson.object}
\NormalTok{      [ }\StringTok{"tag"}      \OperatorTok{Aeson..=}\NormalTok{ T.pack choiceName}
\NormalTok{      , }\StringTok{"contents"} \OperatorTok{Aeson..=}\NormalTok{ schemaToValue choiceValue x}
\NormalTok{      ]}
\OtherTok{    fieldToValue ::} \DataTypeTok{Field}\NormalTok{ x }\OtherTok{{-}>} \DataTypeTok{Op}\NormalTok{ [}\DataTypeTok{Aeson.Pair}\NormalTok{] x}
\NormalTok{    fieldToValue }\DataTypeTok{Field}\NormalTok{\{}\OperatorTok{..}\NormalTok{\} }\OtherTok{=} \DataTypeTok{Op} \OperatorTok{$}\NormalTok{ \textbackslash{}x }\OtherTok{{-}>}
\NormalTok{        [T.pack fieldName }\OperatorTok{Aeson..=}\NormalTok{ schemaToValue fieldValue x]}
\OtherTok{    primToValue ::} \DataTypeTok{Primitive}\NormalTok{ x }\OtherTok{{-}>}\NormalTok{ x }\OtherTok{{-}>} \DataTypeTok{Aeson.Value}
\NormalTok{    primToValue }\OtherTok{=}\NormalTok{ \textbackslash{}}\KeywordTok{case}
      \DataTypeTok{PString}\NormalTok{ f \_ }\OtherTok{{-}>} \DataTypeTok{Aeson.String} \OperatorTok{.}\NormalTok{ T.pack }\OperatorTok{.}\NormalTok{ f}
      \DataTypeTok{PNumber}\NormalTok{ f \_ }\OtherTok{{-}>} \DataTypeTok{Aeson.Number} \OperatorTok{.}\NormalTok{ f}
      \DataTypeTok{PBool}\NormalTok{   f \_ }\OtherTok{{-}>} \DataTypeTok{Aeson.Bool} \OperatorTok{.}\NormalTok{ f}
\end{Highlighting}
\end{Shaded}

And there we have it --- a fully functional bidirectional parser schema type
that we assembled step-by-step, adding each piece incrementally and exploring
the space until we found something useful for us.

A cute function we could write to tie things together would be one that does a
round-trip, serializing and then parsing, to make sure things worked properly.

\begin{Shaded}
\begin{Highlighting}[]
\CommentTok{{-}{-} source: https://github.com/mstksg/inCode/tree/master/code{-}samples/functor{-}structures/invariant.hs\#L156{-}L160}

\NormalTok{testRoundTrip}
\OtherTok{    ::} \DataTypeTok{Schema}\NormalTok{ a}
    \OtherTok{{-}>}\NormalTok{ a}
    \OtherTok{{-}>} \DataTypeTok{Either}\NormalTok{ (}\DataTypeTok{A.ParseError} \DataTypeTok{String}\NormalTok{) a}
\NormalTok{testRoundTrip sch }\OtherTok{=}\NormalTok{ A.parseValue (schemaParser sch) }\OperatorTok{.}\NormalTok{ schemaToValue sch}
\end{Highlighting}
\end{Shaded}

\begin{Shaded}
\begin{Highlighting}[]
\NormalTok{ghci}\OperatorTok{>}\NormalTok{ testRoundTrip customerSchema (}\DataTypeTok{CPerson} \StringTok{"Sam"} \DecValTok{40}\NormalTok{)}
\DataTypeTok{Right}\NormalTok{ (}\DataTypeTok{CPerson}\NormalTok{ \{cpName }\OtherTok{=} \StringTok{"Sam"}\NormalTok{, cpAge }\OtherTok{=} \DecValTok{40}\NormalTok{\})}
\end{Highlighting}
\end{Shaded}

\hypertarget{an-alternative-invariant-strategy}{%
\section{An Alternative Invariant
Strategy}\label{an-alternative-invariant-strategy}}

The thought process ``I want to use both \texttt{Div} and \texttt{Ap}, let's
just look for \texttt{DivAp}'' is kind of nice and straightforward. However,
there's a major downside in using \texttt{DivAp} and \texttt{DecAlt} that make
their ergonomics not so great when building them up.

A major part about what makes \texttt{Ap} and \texttt{ListF} (and, to an extent,
\texttt{Div} and \texttt{Dec}) so nice to use is that are instances of popular
Haskell typeclasses like \texttt{Applicative} and \texttt{Alternative} (or
\texttt{Plus}) and using \texttt{Applicative} and \texttt{Alternative}
interfaces are pretty common in Haskell. Because of this, they are pretty
comfortable for most Haskellers to use.

However, \texttt{DivAp} and \texttt{DecAlt} aren't really instances of any
commonly used typeclass (aside from \texttt{Invariant}).\footnote{There
  \emph{could} be a typeclass for ``combination of \texttt{Applicative} and
  \texttt{Divisible}'' and ``combination of \texttt{Plus} and
  \texttt{Conclude}'':

\begin{Shaded}
\begin{Highlighting}[]
\KeywordTok{class} \DataTypeTok{DivisibleApplicative}\NormalTok{ f }\KeywordTok{where}
\OtherTok{  conquerpure ::}\NormalTok{ a }\OtherTok{{-}>}\NormalTok{ f a}
\OtherTok{  divideAp ::}\NormalTok{ (a }\OtherTok{{-}>}\NormalTok{ (b, c)) }\OtherTok{{-}>}\NormalTok{ (b }\OtherTok{{-}>}\NormalTok{ c }\OtherTok{{-}>}\NormalTok{ a) }\OtherTok{{-}>}\NormalTok{ f b }\OtherTok{{-}>}\NormalTok{ f c }\OtherTok{{-}>}\NormalTok{ f a}
\end{Highlighting}
\end{Shaded}

  And every \texttt{Applicative} and \texttt{Divisible} instance would be a
  valid instance of this. However, this doesn't really exist in any common
  Haskell libraries\ldots and I'm not sure it exists anywhere at all.} So you
really don't have any nice interface for them other than just using functions
specifically written for them, like \texttt{gather} and \texttt{swerve}, which
may feel ad-hoc.

Luckily, there's another way to achieve the same goals and also be able to take
advantage of our favorite familiar interfaces. We can ``add Contravariance''
directly into \texttt{Ap} itself, using
\href{https://hackage.haskell.org/package/functor-combinators/docs/Data-HFunctor-Route.html\#t:Pre}{\texttt{Pre}}.
This is a trick I first saw in the
\emph{\href{https://hackage.haskell.org/package/unjson}{unjson}} library. A
value of type:

\begin{Shaded}
\begin{Highlighting}[]
\DataTypeTok{Ap}\NormalTok{ (}\DataTypeTok{Pre}\NormalTok{ r }\DataTypeTok{Field}\NormalTok{) a}
\end{Highlighting}
\end{Shaded}

will ``produce'' \texttt{a}s covariantly\ldots but will ``consume'' \texttt{r}s
contravariantly. You can think of the \texttt{Pre\ r} as adding an ``tunnel'' to
guide the \texttt{r} to each \texttt{Field} in the \texttt{Ap}.

This means we can now use normal Applicative combinators to combine our fake
invariant type:

\begin{Shaded}
\begin{Highlighting}[]
\FunctionTok{pure}\OtherTok{ ::}\NormalTok{ a }\OtherTok{{-}>} \DataTypeTok{Ap}\NormalTok{ (}\DataTypeTok{Pre}\NormalTok{ r }\DataTypeTok{Field}\NormalTok{) a}

\NormalTok{(}\OperatorTok{<*>}\NormalTok{)}
\OtherTok{    ::} \DataTypeTok{Ap}\NormalTok{ (}\DataTypeTok{Pre}\NormalTok{ r }\DataTypeTok{Field}\NormalTok{) (a }\OtherTok{{-}>}\NormalTok{ b)}
    \OtherTok{{-}>} \DataTypeTok{Ap}\NormalTok{ (}\DataTypeTok{Pre}\NormalTok{ r }\DataTypeTok{Field}\NormalTok{) a}
    \OtherTok{{-}>} \DataTypeTok{Ap}\NormalTok{ (}\DataTypeTok{Pre}\NormalTok{ r }\DataTypeTok{Field}\NormalTok{) b}

\NormalTok{liftA2}
\OtherTok{    ::}\NormalTok{ (a }\OtherTok{{-}>}\NormalTok{ b }\OtherTok{{-}>}\NormalTok{ c)}
    \OtherTok{{-}>} \DataTypeTok{Ap}\NormalTok{ (}\DataTypeTok{Pre}\NormalTok{ r }\DataTypeTok{Field}\NormalTok{) a}
    \OtherTok{{-}>} \DataTypeTok{Ap}\NormalTok{ (}\DataTypeTok{Pre}\NormalTok{ r }\DataTypeTok{Field}\NormalTok{) b}
    \OtherTok{{-}>} \DataTypeTok{Ap}\NormalTok{ (}\DataTypeTok{Pre}\NormalTok{ r }\DataTypeTok{Field}\NormalTok{) c}
\end{Highlighting}
\end{Shaded}

We see that the \texttt{Applicative} combinators will recombine our ``output''
covariant types appropriately, but will keep the ``input'' contravariant type
constant\ldots which works out because each of the \texttt{Field}s inside could
work off of the same input type (remember that
\texttt{Div\ f\ a\ \textasciitilde{}\ {[}f\ a{]}}, it's just a list of things
that consume the same \texttt{a}).

We can make an \texttt{Ap\ (Pre\ r\ Field)\ a} using \texttt{injectPre}, which
asks us to provide that ``get an \texttt{a} from \texttt{r}'' function up-front:

\begin{Shaded}
\begin{Highlighting}[]
\OtherTok{injectPre ::}\NormalTok{ (r }\OtherTok{{-}>}\NormalTok{ a) }\OtherTok{{-}>}\NormalTok{ f a }\OtherTok{{-}>} \DataTypeTok{Ap}\NormalTok{ (}\DataTypeTok{Pre}\NormalTok{ r f) a}
\end{Highlighting}
\end{Shaded}

There's a useful newtype wrapper over \texttt{Pre} that makes consuming it
convenient by requiring the \texttt{r} and \texttt{a} to be the same:

\begin{Shaded}
\begin{Highlighting}[]
\KeywordTok{newtype} \DataTypeTok{PreT}\NormalTok{ t f a }\OtherTok{=} \DataTypeTok{PreT}\NormalTok{ (t (}\DataTypeTok{Pre}\NormalTok{ a f) a)}

\CommentTok{{-}{-} | interpret for PreT treats \textasciigrave{}PreT Ap f a\textasciigrave{} as if it were just \textasciigrave{}Ap f a\textasciigrave{}, so we}
\CommentTok{{-}{-} interpret into an \textasciigrave{}Applicative\textasciigrave{} context}
\NormalTok{interpret}
\OtherTok{    ::} \DataTypeTok{Applicative}\NormalTok{ g}
    \OtherTok{=>}\NormalTok{ (}\KeywordTok{forall}\NormalTok{ x}\OperatorTok{.}\NormalTok{ f x }\OtherTok{{-}>}\NormalTok{ g x)}
    \OtherTok{{-}>} \DataTypeTok{PreT} \DataTypeTok{Ap}\NormalTok{ f a}
    \OtherTok{{-}>}\NormalTok{ g a}

\CommentTok{{-}{-} | But we can also interpret into a \textasciigrave{}Divisible\textasciigrave{} context!  Just like as if we}
\CommentTok{{-}{-} had \textasciigrave{}Div f a\textasciigrave{}!}
\NormalTok{preDivisibleT}
\OtherTok{    ::} \DataTypeTok{Divisible}\NormalTok{ g}
    \OtherTok{=>}\NormalTok{ (}\KeywordTok{forall}\NormalTok{ x}\OperatorTok{.}\NormalTok{ f x }\OtherTok{{-}>}\NormalTok{ g x)}
    \OtherTok{{-}>} \DataTypeTok{PreT} \DataTypeTok{Ap}\NormalTok{ f a}
    \OtherTok{{-}>}\NormalTok{ g a}
\end{Highlighting}
\end{Shaded}

We see that \texttt{interpret} for \texttt{PreT\ Ap\ f\ a} works just like
\texttt{interpret} for \texttt{Ap\ f\ a}\ldots so we don't lose any power, it's
the same as always if we wanted to just use \texttt{Ap\ f\ a} covariantly to
interpret into a parser.

But, we magically gain \texttt{preDivisibleT}, which lets us \texttt{interpret}
into a contravariant \texttt{Divisible} context! Just like as if we had
\texttt{Div\ f\ a}!

So using \texttt{Pre} and \texttt{PreT}, we get to \emph{assemble} it using our
favorite \texttt{Applicative} combinators\ldots then when we wrap it in
\texttt{PreT}, we get to \emph{interpret} it in whatever way we want by choosing
different interpreters. It's the best of both worlds!

We can do the opposite thing with \texttt{Dec} as well: we can use
{[}\texttt{Post}{]}{[}Post{]} to embed covariant capabilities in \texttt{Dec}. A
value of type:

\begin{Shaded}
\begin{Highlighting}[]
\DataTypeTok{Dec}\NormalTok{ (}\DataTypeTok{Post}\NormalTok{ r }\DataTypeTok{Choice}\NormalTok{) a}
\end{Highlighting}
\end{Shaded}

will ``consume'' \texttt{a}s contravariantly (like a normal \texttt{Dec}), but
will also produce \texttt{r}s covariantly. You can think of the \texttt{Post\ r}
as adding an ``tunnel'' allowing the output of each \texttt{Choice} to exit out
of the \texttt{Dec}.

This means we can now use normal Decide contravariant typeclass-based
combinators to combine our fake invariant type:

\begin{Shaded}
\begin{Highlighting}[]
\NormalTok{decide}
\OtherTok{    ::}\NormalTok{ (a }\OtherTok{{-}>} \DataTypeTok{Either}\NormalTok{ b c)        }\CommentTok{{-}{-} \^{} break into branches}
    \OtherTok{{-}>} \DataTypeTok{Dec}\NormalTok{ (}\DataTypeTok{Post}\NormalTok{ r }\DataTypeTok{Choice}\NormalTok{) b    }\CommentTok{{-}{-} \^{} handle first branch}
    \OtherTok{{-}>} \DataTypeTok{Dec}\NormalTok{ (}\DataTypeTok{Post}\NormalTok{ r }\DataTypeTok{Choice}\NormalTok{) c    }\CommentTok{{-}{-} \^{} handle second branch}
    \OtherTok{{-}>} \DataTypeTok{Dec}\NormalTok{ (}\DataTypeTok{Post}\NormalTok{ r }\DataTypeTok{Choice}\NormalTok{) a    }\CommentTok{{-}{-} \^{} overall handler}
\end{Highlighting}
\end{Shaded}

We see that \texttt{decide} combinators will recombine our ``input''
contravariant types appropriately, but will keep the ``output'' covariant type
constant\ldots which works out because each of the \texttt{Choice}s inside could
be embedded into the same output type (remember that we used
\texttt{List\ f\ a\ \textasciitilde{}\ {[}f\ a{]}} for our contravariant choice
collection before, just a list of things that produce the same \texttt{a}).

Again, we can make a \texttt{Dec\ (Post\ r\ Choice)\ a} using
\texttt{injectPost}, which asks us to provide that ``embed the \texttt{a} in the
\texttt{r}'' function up-front:

\begin{Shaded}
\begin{Highlighting}[]
\OtherTok{injectPost ::}\NormalTok{ (a }\OtherTok{{-}>}\NormalTok{ r) }\OtherTok{{-}>}\NormalTok{ f a }\OtherTok{{-}>} \DataTypeTok{Dec}\NormalTok{ (}\DataTypeTok{Post}\NormalTok{ r f) a}
\end{Highlighting}
\end{Shaded}

And again, we have the newtype wrapper \texttt{PostT} that gives us convenient
interpreting functions:

\begin{Shaded}
\begin{Highlighting}[]
\KeywordTok{newtype} \DataTypeTok{PostT}\NormalTok{ t f a }\OtherTok{=} \DataTypeTok{PostT}\NormalTok{ (t (}\DataTypeTok{Post}\NormalTok{ a f) a)}

\CommentTok{{-}{-} | interpret for PostT treats \textasciigrave{}PostT Dec f a\textasciigrave{} as if it were just \textasciigrave{}Dec f a\textasciigrave{}, so we}
\CommentTok{{-}{-} interpret into a \textasciigrave{}Conclude\textasciigrave{} context like before}
\NormalTok{interpret}
\OtherTok{    ::} \DataTypeTok{Conclude}\NormalTok{ g}
    \OtherTok{=>}\NormalTok{ (}\KeywordTok{forall}\NormalTok{ x}\OperatorTok{.}\NormalTok{ f x }\OtherTok{{-}>}\NormalTok{ g x)}
    \OtherTok{{-}>} \DataTypeTok{PostT} \DataTypeTok{Dec}\NormalTok{ f a}
    \OtherTok{{-}>}\NormalTok{ g a}

\CommentTok{{-}{-} | But we can also interpret into a \textasciigrave{}Plus\textasciigrave{} context!  Just like as if we}
\CommentTok{{-}{-} had \textasciigrave{}ListF f a\textasciigrave{}!}
\NormalTok{postPlusT}
\OtherTok{    ::} \DataTypeTok{Plus}\NormalTok{ g}
    \OtherTok{=>}\NormalTok{ (}\KeywordTok{forall}\NormalTok{ x}\OperatorTok{.}\NormalTok{ f x }\OtherTok{{-}>}\NormalTok{ g x)}
    \OtherTok{{-}>} \DataTypeTok{PostT} \DataTypeTok{Choice}\NormalTok{ f a}
    \OtherTok{{-}>}\NormalTok{ g a}
\end{Highlighting}
\end{Shaded}

With these new tools, we can imagine a different invariant \texttt{Schema} type:

\begin{Shaded}
\begin{Highlighting}[]
\CommentTok{{-}{-} source: https://github.com/mstksg/inCode/tree/master/code{-}samples/functor{-}structures/routing.hs\#L47{-}L100}

\KeywordTok{data} \DataTypeTok{Schema}\NormalTok{ a }\OtherTok{=}
      \DataTypeTok{RecordType}\NormalTok{  (}\DataTypeTok{PreT}  \DataTypeTok{Ap}  \DataTypeTok{Field}\NormalTok{  a)}
    \OperatorTok{|} \DataTypeTok{SumType}\NormalTok{     (}\DataTypeTok{PostT} \DataTypeTok{Dec} \DataTypeTok{Choice}\NormalTok{ a)}
    \OperatorTok{|} \DataTypeTok{SchemaLeaf}\NormalTok{  (}\DataTypeTok{Primitive}\NormalTok{ a)}
  \KeywordTok{deriving} \DataTypeTok{Generic}

\KeywordTok{data} \DataTypeTok{Field}\NormalTok{ a }\OtherTok{=} \DataTypeTok{Field}
\NormalTok{    \{}\OtherTok{ fieldName  ::} \DataTypeTok{String}
\NormalTok{    ,}\OtherTok{ fieldValue ::} \DataTypeTok{Schema}\NormalTok{ a}
\NormalTok{    \}}
  \KeywordTok{deriving} \DataTypeTok{Generic}

\KeywordTok{data} \DataTypeTok{Choice}\NormalTok{ a }\OtherTok{=} \DataTypeTok{Choice}
\NormalTok{    \{}\OtherTok{ choiceName  ::} \DataTypeTok{String}
\NormalTok{    ,}\OtherTok{ choiceValue ::} \DataTypeTok{Schema}\NormalTok{ a}
\NormalTok{    \}}
  \KeywordTok{deriving} \DataTypeTok{Generic}

\KeywordTok{data} \DataTypeTok{Primitive}\NormalTok{ a }\OtherTok{=}
      \DataTypeTok{PString}\NormalTok{ (a }\OtherTok{{-}>} \DataTypeTok{String}\NormalTok{)     (}\DataTypeTok{String}     \OtherTok{{-}>} \DataTypeTok{Maybe}\NormalTok{ a)}
    \OperatorTok{|} \DataTypeTok{PNumber}\NormalTok{ (a }\OtherTok{{-}>} \DataTypeTok{Scientific}\NormalTok{) (}\DataTypeTok{Scientific} \OtherTok{{-}>} \DataTypeTok{Maybe}\NormalTok{ a)}
    \OperatorTok{|} \DataTypeTok{PBool}\NormalTok{   (a }\OtherTok{{-}>} \DataTypeTok{Bool}\NormalTok{)       (}\DataTypeTok{Bool}       \OtherTok{{-}>} \DataTypeTok{Maybe}\NormalTok{ a)}
  \KeywordTok{deriving} \DataTypeTok{Generic}

\OtherTok{customerSchema ::} \DataTypeTok{Schema} \DataTypeTok{Customer}
\NormalTok{customerSchema }\OtherTok{=} \DataTypeTok{SumType} \OperatorTok{.} \DataTypeTok{PostT} \OperatorTok{$}
\NormalTok{    decide (\textbackslash{}}\KeywordTok{case} \DataTypeTok{CPerson}\NormalTok{ x y }\OtherTok{{-}>} \DataTypeTok{Left}\NormalTok{ (x, y); }\DataTypeTok{CBusiness}\NormalTok{ x }\OtherTok{{-}>} \DataTypeTok{Right}\NormalTok{ x)}
\NormalTok{      (injectPost (}\FunctionTok{uncurry} \DataTypeTok{CPerson}\NormalTok{) }\DataTypeTok{Choice}
\NormalTok{        \{ choiceName }\OtherTok{=} \StringTok{"Person"}
\NormalTok{        , choiceValue }\OtherTok{=} \DataTypeTok{RecordType} \OperatorTok{.} \DataTypeTok{PreT} \OperatorTok{$}\NormalTok{ (,)}
            \OperatorTok{<$>}\NormalTok{ injectPre }\FunctionTok{fst} \DataTypeTok{Field}\NormalTok{ \{ fieldName }\OtherTok{=} \StringTok{"Name"}\NormalTok{, fieldValue }\OtherTok{=} \DataTypeTok{SchemaLeaf}\NormalTok{ pString \}}
            \OperatorTok{<*>}\NormalTok{ injectPre }\FunctionTok{snd} \DataTypeTok{Field}\NormalTok{ \{ fieldName }\OtherTok{=} \StringTok{"Age"}\NormalTok{ , fieldValue }\OtherTok{=} \DataTypeTok{SchemaLeaf}\NormalTok{ pInt    \}}
\NormalTok{        \}}
\NormalTok{      )}
\NormalTok{      (injectPost }\DataTypeTok{CBusiness}         \DataTypeTok{Choice}
\NormalTok{        \{ choiceName }\OtherTok{=} \StringTok{"Person"}
\NormalTok{        , choiceValue }\OtherTok{=} \DataTypeTok{RecordType} \OperatorTok{.}\NormalTok{ inject }\OperatorTok{$}
            \DataTypeTok{Field}\NormalTok{ \{ fieldName }\OtherTok{=} \StringTok{"Age"}\NormalTok{ , fieldValue }\OtherTok{=} \DataTypeTok{SchemaLeaf}\NormalTok{ pInt    \}}
\NormalTok{        \}}
\NormalTok{      )}
\end{Highlighting}
\end{Shaded}

And all of our running functions look pretty much the same as well:

\begin{Shaded}
\begin{Highlighting}[]
\CommentTok{{-}{-} source: https://github.com/mstksg/inCode/tree/master/code{-}samples/functor{-}structures/routing.hs\#L102{-}L182}

\NormalTok{schemaDoc}
\OtherTok{    ::} \DataTypeTok{String}       \CommentTok{{-}{-} \^{} name}
    \OtherTok{{-}>} \DataTypeTok{Schema}\NormalTok{ x     }\CommentTok{{-}{-} \^{} schema}
    \OtherTok{{-}>} \DataTypeTok{PP.Doc}\NormalTok{ a}
\NormalTok{schemaDoc title }\OtherTok{=}\NormalTok{ \textbackslash{}}\KeywordTok{case}
    \DataTypeTok{RecordType}\NormalTok{ fs }\OtherTok{{-}>}\NormalTok{ PP.vsep [}
\NormalTok{        PP.pretty (}\StringTok{"\{"} \OperatorTok{<>}\NormalTok{ title }\OperatorTok{<>} \StringTok{"\}"}\NormalTok{)}
\NormalTok{      , PP.indent }\DecValTok{2} \OperatorTok{.}\NormalTok{ PP.vsep }\OperatorTok{$}
\NormalTok{          icollect (\textbackslash{}fld }\OtherTok{{-}>} \StringTok{"*"} \OperatorTok{PP.<+>}\NormalTok{ PP.indent }\DecValTok{2}\NormalTok{ (fieldDoc fld)) fs}
\NormalTok{      ]}
    \DataTypeTok{SumType}\NormalTok{ cs    }\OtherTok{{-}>}\NormalTok{ PP.vsep [}
\NormalTok{        PP.pretty (}\StringTok{"("} \OperatorTok{<>}\NormalTok{ title }\OperatorTok{<>} \StringTok{")"}\NormalTok{)}
\NormalTok{      , }\StringTok{"Choice of:"}
\NormalTok{      , PP.indent }\DecValTok{2} \OperatorTok{.}\NormalTok{ PP.vsep }\OperatorTok{$}
\NormalTok{          icollect choiceDoc cs}
\NormalTok{      ]}
    \DataTypeTok{SchemaLeaf}\NormalTok{ p  }\OtherTok{{-}>}\NormalTok{ PP.pretty (title }\OperatorTok{<>} \StringTok{":"}\NormalTok{)}
              \OperatorTok{PP.<+>}\NormalTok{ primDoc p}
  \KeywordTok{where}
\OtherTok{    fieldDoc ::} \DataTypeTok{Field}\NormalTok{ x }\OtherTok{{-}>} \DataTypeTok{PP.Doc}\NormalTok{ a}
\NormalTok{    fieldDoc }\DataTypeTok{Field}\NormalTok{\{}\OperatorTok{..}\NormalTok{\} }\OtherTok{=}\NormalTok{ schemaDoc fieldName fieldValue}
\OtherTok{    choiceDoc ::} \DataTypeTok{Choice}\NormalTok{ x }\OtherTok{{-}>} \DataTypeTok{PP.Doc}\NormalTok{ a}
\NormalTok{    choiceDoc }\DataTypeTok{Choice}\NormalTok{\{}\OperatorTok{..}\NormalTok{\} }\OtherTok{=}\NormalTok{ schemaDoc choiceName choiceValue}
\OtherTok{    primDoc ::} \DataTypeTok{Primitive}\NormalTok{ x }\OtherTok{{-}>} \DataTypeTok{PP.Doc}\NormalTok{ a}
\NormalTok{    primDoc }\OtherTok{=}\NormalTok{ \textbackslash{}}\KeywordTok{case}
      \DataTypeTok{PString}\NormalTok{ \_ \_ }\OtherTok{{-}>} \StringTok{"string"}
      \DataTypeTok{PNumber}\NormalTok{ \_ \_ }\OtherTok{{-}>} \StringTok{"number"}
      \DataTypeTok{PBool}\NormalTok{   \_ \_ }\OtherTok{{-}>} \StringTok{"bool"}

\OtherTok{schemaParser ::} \DataTypeTok{Schema}\NormalTok{ a }\OtherTok{{-}>} \DataTypeTok{A.Parse} \DataTypeTok{ErrType}\NormalTok{ a}
\NormalTok{schemaParser }\OtherTok{=}\NormalTok{ \textbackslash{}}\KeywordTok{case}
    \DataTypeTok{RecordType}\NormalTok{ fs }\OtherTok{{-}>}\NormalTok{ interpret fieldParser fs}
    \DataTypeTok{SumType}\NormalTok{    cs }\OtherTok{{-}>}\NormalTok{ postPlusT choiceParser cs}
    \DataTypeTok{SchemaLeaf}\NormalTok{ p  }\OtherTok{{-}>}\NormalTok{ primParser p}
  \KeywordTok{where}
\OtherTok{    fieldParser ::} \DataTypeTok{Field}\NormalTok{ a }\OtherTok{{-}>} \DataTypeTok{A.Parse} \DataTypeTok{String}\NormalTok{ a}
\NormalTok{    fieldParser }\DataTypeTok{Field}\NormalTok{\{}\OperatorTok{..}\NormalTok{\} }\OtherTok{=}\NormalTok{ A.key (T.pack fieldName) (schemaParser fieldValue)}
\OtherTok{    choiceParser ::} \DataTypeTok{Choice}\NormalTok{ a }\OtherTok{{-}>} \DataTypeTok{A.Parse} \DataTypeTok{String}\NormalTok{ a}
\NormalTok{    choiceParser }\DataTypeTok{Choice}\NormalTok{\{}\OperatorTok{..}\NormalTok{\} }\OtherTok{=} \KeywordTok{do}
\NormalTok{      tag }\OtherTok{<{-}}\NormalTok{ A.key }\StringTok{"tag"}\NormalTok{ A.asString}
\NormalTok{      unless (tag }\OperatorTok{==}\NormalTok{ choiceName) }\OperatorTok{$}
\NormalTok{        A.throwCustomError }\StringTok{"Tag does not match"}
\NormalTok{      A.key }\StringTok{"contents"} \OperatorTok{$}\NormalTok{ schemaParser choiceValue}
\OtherTok{    primParser ::} \DataTypeTok{Primitive}\NormalTok{ a }\OtherTok{{-}>} \DataTypeTok{A.Parse} \DataTypeTok{String}\NormalTok{ a}
\NormalTok{    primParser }\OtherTok{=}\NormalTok{ \textbackslash{}}\KeywordTok{case}
      \DataTypeTok{PString}\NormalTok{ \_ f }\OtherTok{{-}>}\NormalTok{ A.withString }\OperatorTok{$}
        \FunctionTok{maybe}\NormalTok{ (}\DataTypeTok{Left} \StringTok{"error validating string"}\NormalTok{) }\DataTypeTok{Right} \OperatorTok{.}\NormalTok{ f}
      \DataTypeTok{PNumber}\NormalTok{ \_ f }\OtherTok{{-}>}\NormalTok{ A.withScientific }\OperatorTok{$}
        \FunctionTok{maybe}\NormalTok{ (}\DataTypeTok{Left} \StringTok{"error validating number"}\NormalTok{) }\DataTypeTok{Right} \OperatorTok{.}\NormalTok{ f}
      \DataTypeTok{PBool}\NormalTok{ \_ f }\OtherTok{{-}>}\NormalTok{ A.withBool }\OperatorTok{$}
        \FunctionTok{maybe}\NormalTok{ (}\DataTypeTok{Left} \StringTok{"error validating bool"}\NormalTok{) }\DataTypeTok{Right} \OperatorTok{.}\NormalTok{ f}

\NormalTok{schemaToValue}
\OtherTok{    ::} \DataTypeTok{Schema}\NormalTok{ a}
    \OtherTok{{-}>}\NormalTok{ a}
    \OtherTok{{-}>} \DataTypeTok{Aeson.Value}
\NormalTok{schemaToValue }\OtherTok{=}\NormalTok{ \textbackslash{}}\KeywordTok{case}
    \DataTypeTok{RecordType}\NormalTok{ fs }\OtherTok{{-}>}\NormalTok{ Aeson.object }\OperatorTok{.}\NormalTok{ getOp (preDivisibleT fieldToValue fs)}
    \DataTypeTok{SumType}\NormalTok{    cs }\OtherTok{{-}>}\NormalTok{ getOp (interpret choiceToValue cs)}
    \DataTypeTok{SchemaLeaf}\NormalTok{ p  }\OtherTok{{-}>}\NormalTok{ primToValue p}
  \KeywordTok{where}
\OtherTok{    fieldToValue ::} \DataTypeTok{Field}\NormalTok{ a }\OtherTok{{-}>} \DataTypeTok{Op}\NormalTok{ [}\DataTypeTok{Aeson.Pair}\NormalTok{] a}
\NormalTok{    fieldToValue }\DataTypeTok{Field}\NormalTok{\{}\OperatorTok{..}\NormalTok{\} }\OtherTok{=} \DataTypeTok{Op} \OperatorTok{$}\NormalTok{ \textbackslash{}x }\OtherTok{{-}>}
\NormalTok{        [T.pack fieldName }\OperatorTok{Aeson..=}\NormalTok{ schemaToValue fieldValue x]}
\OtherTok{    choiceToValue ::} \DataTypeTok{Choice}\NormalTok{ a }\OtherTok{{-}>} \DataTypeTok{Op} \DataTypeTok{Aeson.Value}\NormalTok{ a}
\NormalTok{    choiceToValue }\DataTypeTok{Choice}\NormalTok{\{}\OperatorTok{..}\NormalTok{\} }\OtherTok{=} \DataTypeTok{Op} \OperatorTok{$}\NormalTok{ \textbackslash{}x }\OtherTok{{-}>}\NormalTok{ Aeson.object}
\NormalTok{        [ }\StringTok{"tag"}      \OperatorTok{Aeson..=}\NormalTok{ T.pack choiceName}
\NormalTok{        , }\StringTok{"contents"} \OperatorTok{Aeson..=}\NormalTok{ schemaToValue choiceValue x}
\NormalTok{        ]}
\OtherTok{    primToValue ::} \DataTypeTok{Primitive}\NormalTok{ a }\OtherTok{{-}>}\NormalTok{ a }\OtherTok{{-}>} \DataTypeTok{Aeson.Value}
\NormalTok{    primToValue }\OtherTok{=}\NormalTok{ \textbackslash{}}\KeywordTok{case}
      \DataTypeTok{PString}\NormalTok{ f \_ }\OtherTok{{-}>}\NormalTok{ \textbackslash{}x }\OtherTok{{-}>} \DataTypeTok{Aeson.String}\NormalTok{ (T.pack (f x))}
      \DataTypeTok{PNumber}\NormalTok{ f \_ }\OtherTok{{-}>}\NormalTok{ \textbackslash{}x }\OtherTok{{-}>} \DataTypeTok{Aeson.Number}\NormalTok{ (f x)}
      \DataTypeTok{PBool}\NormalTok{   f \_ }\OtherTok{{-}>}\NormalTok{ \textbackslash{}x }\OtherTok{{-}>} \DataTypeTok{Aeson.Bool}\NormalTok{   (f x)}
\end{Highlighting}
\end{Shaded}

Just two separate styles for you to consider if we want to go into combining
\emph{both} covariant productin \emph{and} contravariant consumption!

\hypertarget{signoff}{%
\section{Signoff}\label{signoff}}

Hi, thanks for reading! You can reach me via email at
\href{mailto:justin@jle.im}{\nolinkurl{justin@jle.im}}, or at twitter at
\href{https://twitter.com/mstk}{@mstk}! This post and all others are published
under the \href{https://creativecommons.org/licenses/by-nc-nd/3.0/}{CC-BY-NC-ND
3.0} license. Corrections and edits via pull request are welcome and encouraged
at \href{https://github.com/mstksg/inCode}{the source repository}.

If you feel inclined, or this post was particularly helpful for you, why not
consider \href{https://www.patreon.com/justinle/overview}{supporting me on
Patreon}, or a \href{bitcoin:3D7rmAYgbDnp4gp4rf22THsGt74fNucPDU}{BTC donation}?
:)

\end{document}
