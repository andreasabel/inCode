\documentclass[]{article}
\usepackage{lmodern}
\usepackage{amssymb,amsmath}
\usepackage{ifxetex,ifluatex}
\usepackage{fixltx2e} % provides \textsubscript
\ifnum 0\ifxetex 1\fi\ifluatex 1\fi=0 % if pdftex
  \usepackage[T1]{fontenc}
  \usepackage[utf8]{inputenc}
\else % if luatex or xelatex
  \ifxetex
    \usepackage{mathspec}
    \usepackage{xltxtra,xunicode}
  \else
    \usepackage{fontspec}
  \fi
  \defaultfontfeatures{Mapping=tex-text,Scale=MatchLowercase}
  \newcommand{\euro}{€}
\fi
% use upquote if available, for straight quotes in verbatim environments
\IfFileExists{upquote.sty}{\usepackage{upquote}}{}
% use microtype if available
\IfFileExists{microtype.sty}{\usepackage{microtype}}{}
\usepackage[margin=1in]{geometry}
\usepackage{color}
\usepackage{fancyvrb}
\newcommand{\VerbBar}{|}
\newcommand{\VERB}{\Verb[commandchars=\\\{\}]}
\DefineVerbatimEnvironment{Highlighting}{Verbatim}{commandchars=\\\{\}}
% Add ',fontsize=\small' for more characters per line
\newenvironment{Shaded}{}{}
\newcommand{\AlertTok}[1]{\textcolor[rgb]{1.00,0.00,0.00}{\textbf{#1}}}
\newcommand{\AnnotationTok}[1]{\textcolor[rgb]{0.38,0.63,0.69}{\textbf{\textit{#1}}}}
\newcommand{\AttributeTok}[1]{\textcolor[rgb]{0.49,0.56,0.16}{#1}}
\newcommand{\BaseNTok}[1]{\textcolor[rgb]{0.25,0.63,0.44}{#1}}
\newcommand{\BuiltInTok}[1]{#1}
\newcommand{\CharTok}[1]{\textcolor[rgb]{0.25,0.44,0.63}{#1}}
\newcommand{\CommentTok}[1]{\textcolor[rgb]{0.38,0.63,0.69}{\textit{#1}}}
\newcommand{\CommentVarTok}[1]{\textcolor[rgb]{0.38,0.63,0.69}{\textbf{\textit{#1}}}}
\newcommand{\ConstantTok}[1]{\textcolor[rgb]{0.53,0.00,0.00}{#1}}
\newcommand{\ControlFlowTok}[1]{\textcolor[rgb]{0.00,0.44,0.13}{\textbf{#1}}}
\newcommand{\DataTypeTok}[1]{\textcolor[rgb]{0.56,0.13,0.00}{#1}}
\newcommand{\DecValTok}[1]{\textcolor[rgb]{0.25,0.63,0.44}{#1}}
\newcommand{\DocumentationTok}[1]{\textcolor[rgb]{0.73,0.13,0.13}{\textit{#1}}}
\newcommand{\ErrorTok}[1]{\textcolor[rgb]{1.00,0.00,0.00}{\textbf{#1}}}
\newcommand{\ExtensionTok}[1]{#1}
\newcommand{\FloatTok}[1]{\textcolor[rgb]{0.25,0.63,0.44}{#1}}
\newcommand{\FunctionTok}[1]{\textcolor[rgb]{0.02,0.16,0.49}{#1}}
\newcommand{\ImportTok}[1]{#1}
\newcommand{\InformationTok}[1]{\textcolor[rgb]{0.38,0.63,0.69}{\textbf{\textit{#1}}}}
\newcommand{\KeywordTok}[1]{\textcolor[rgb]{0.00,0.44,0.13}{\textbf{#1}}}
\newcommand{\NormalTok}[1]{#1}
\newcommand{\OperatorTok}[1]{\textcolor[rgb]{0.40,0.40,0.40}{#1}}
\newcommand{\OtherTok}[1]{\textcolor[rgb]{0.00,0.44,0.13}{#1}}
\newcommand{\PreprocessorTok}[1]{\textcolor[rgb]{0.74,0.48,0.00}{#1}}
\newcommand{\RegionMarkerTok}[1]{#1}
\newcommand{\SpecialCharTok}[1]{\textcolor[rgb]{0.25,0.44,0.63}{#1}}
\newcommand{\SpecialStringTok}[1]{\textcolor[rgb]{0.73,0.40,0.53}{#1}}
\newcommand{\StringTok}[1]{\textcolor[rgb]{0.25,0.44,0.63}{#1}}
\newcommand{\VariableTok}[1]{\textcolor[rgb]{0.10,0.09,0.49}{#1}}
\newcommand{\VerbatimStringTok}[1]{\textcolor[rgb]{0.25,0.44,0.63}{#1}}
\newcommand{\WarningTok}[1]{\textcolor[rgb]{0.38,0.63,0.69}{\textbf{\textit{#1}}}}
\ifxetex
  \usepackage[setpagesize=false, % page size defined by xetex
              unicode=false, % unicode breaks when used with xetex
              xetex]{hyperref}
\else
  \usepackage[unicode=true]{hyperref}
\fi
\hypersetup{breaklinks=true,
            bookmarks=true,
            pdfauthor={Justin Le},
            pdftitle={Abstract Validating Forms with Free Applicative/Alternative},
            colorlinks=true,
            citecolor=blue,
            urlcolor=blue,
            linkcolor=magenta,
            pdfborder={0 0 0}}
\urlstyle{same}  % don't use monospace font for urls
% Make links footnotes instead of hotlinks:
\renewcommand{\href}[2]{#2\footnote{\url{#1}}}
\setlength{\parindent}{0pt}
\setlength{\parskip}{6pt plus 2pt minus 1pt}
\setlength{\emergencystretch}{3em}  % prevent overfull lines
\setcounter{secnumdepth}{0}

\title{Abstract Validating Forms with Free Applicative/Alternative}
\author{Justin Le}

\begin{document}
\maketitle

\emph{Originally posted on
\textbf{\href{https://blog.jle.im/entry/forms-with-free-applicative-alternative.html}{in
Code}}.}

One tool I've been finding myself using a lot recently is the \emph{Free
Applicative} (and \emph{Free Alternative}), from the
\emph{\href{https://hackage.haskell.org/package/free}{free}} package.

Free Monads are great, and they're often used to implement the ``interpreter
pattern'' (although I personally prefer
\emph{\href{https://hackage.haskell.org/package/operational}{operational}}, as I
wrote about in a
\href{https://blog.jle.im/entry/interpreters-a-la-carte-duet.html}{previous blog
post}, for that design pattern). However, Free Applicatives are really a
completely different type of thing, and the use cases for each are pretty
disjoint.

If I had to make a general statement, I'll say that free monads are especially
good at representing the idea of abstract \emph{sequential} generators
(sequences that are chained dependently one after the other), and that free
applicatives are especially good at representing the idea of abstract
\emph{parallel} generators (things operating in parallel without any
interconnected data dependences).

For this post, I'll be talking about using the Free Applicative \texttt{Ap} (and
the Free Alternative, \texttt{Alt}) with an abstract representation of a form
element in order to generate an abstract representation of a validating form,
and leveraging this representation to realize these forms in terminal IO,
JSON/YAML, PDF documents, and even on the browser using \emph{ghcjs} and
\emph{\href{https://hackage.haskell.org/package/miso}{miso}}.

\hypertarget{overview}{%
\section{Overview}\label{overview}}

The general approach to utilizing the Free Applicative is to start with some
Functor \texttt{F} (\texttt{F\ a} represents the act of generating a value of
type \texttt{a}). Once you throw \texttt{F} into \texttt{Ap} to get
\texttt{Ap\ F}, you now are able to \emph{combine \texttt{F}s in parallel} with
\texttt{\textless{}\$\textgreater{}}, \texttt{\textless{}*\textgreater{}},
\texttt{liftA2}, \texttt{sequence}, \texttt{traverse}, etc., even though
\texttt{F} normally could not support such combinations. Then, finally, you have
the ability to provide a concrete generator function
\texttt{forall\ a.\ Applicative\ f\ =\textgreater{}\ F\ a\ -\textgreater{}\ f\ a}
(given \texttt{F\ a}, return an actual generator of \texttt{a}s in some
\texttt{Applicative}), and the magic of the Free Applicative will go in and
actually run all of your combined \texttt{F} actions ``in parallel''. The trick
is that, with the same value of \texttt{Ap\ F\ a}, you can \emph{run multiple
different concrete generators} on it, so you can realize \texttt{Ap\ F} in
multiple different contexts and situations, adapting it for whatever you need.

So, in our case, we're going to be making a Functor representing a form element:

\begin{Shaded}
\begin{Highlighting}[]
\KeywordTok{data} \DataTypeTok{FormElem}\NormalTok{ a}
\end{Highlighting}
\end{Shaded}

Where a \texttt{FormElem\ a} represents a \emph{single form element} producing
an \texttt{a}. A \texttt{FormElem\ Int}, for instance, will represent a single
form element producing an \texttt{Int}.

Then, we can create, using
\href{https://hackage.haskell.org/package/free/docs/Control-Applicative-Free.html}{Ap}:

\begin{Shaded}
\begin{Highlighting}[]
\KeywordTok{type} \DataTypeTok{Form} \FunctionTok{=} \DataTypeTok{Ap} \DataTypeTok{FormElem}
\end{Highlighting}
\end{Shaded}

And now we have a type where \texttt{Form\ a} is \emph{whole form with multiple
elements} that all work together to produce a value of type \texttt{a}!

For example, if we had \texttt{intElem\ ::\ FormElem\ Int}, then
\texttt{liftAp\ intElem\ ::\ Form\ Int}, a single-item form that makes an
\texttt{Int}:

\begin{Shaded}
\begin{Highlighting}[]
\OtherTok{intElem ::} \DataTypeTok{FormELem} \DataTypeTok{Int}

\OtherTok{intForm ::} \DataTypeTok{Form} \DataTypeTok{Int}
\NormalTok{intForm }\FunctionTok{=}\NormalTok{ liftAp intElem}
\end{Highlighting}
\end{Shaded}

We can now do all of our Applicativey stuff with it, to generate, for instance,
a form with two items that produces their sum:

\begin{Shaded}
\begin{Highlighting}[]
\CommentTok{-- | A form with two elements, whose overal result is the sum of the two}
\CommentTok{-- element's inputs}
\OtherTok{addingForm ::} \DataTypeTok{Form} \DataTypeTok{Int}
\NormalTok{addingForm }\FunctionTok{=}\NormalTok{ (}\FunctionTok{+}\NormalTok{) }\FunctionTok{<$>}\NormalTok{ intForm }\FunctionTok{<*>}\NormalTok{ intForm}
\end{Highlighting}
\end{Shaded}

Or, we can even generate a form with many `Int' elements, and produce a list of
all of their items:

\begin{Shaded}
\begin{Highlighting}[]
\CommentTok{-- | A form with five elements, whose result is a list of all of their inputs}
\OtherTok{bunchaInts ::} \DataTypeTok{Form}\NormalTok{ [}\DataTypeTok{Int}\NormalTok{]}
\NormalTok{bunchaInts }\FunctionTok{=}\NormalTok{ replicateM }\DecValTok{5}\NormalTok{ intForm}
\end{Highlighting}
\end{Shaded}

Of course, in real life, forms usually have \emph{Alternative} instances, which
allows you to use \texttt{\textless{}\textbar{}\textgreater{}} to ``chose'' a
result between potentially invalid entries, and also create ``optional'' entries
for free using \texttt{optional}. To do that, we actually use the \emph{Free
Alternative},
\emph{\href{https://hackage.haskell.org/package/free/docs/Control-Alternative-Free-Final.html}{Alt}},
instead:

\begin{Shaded}
\begin{Highlighting}[]
\KeywordTok{type} \DataTypeTok{Form} \FunctionTok{=} \DataTypeTok{Alt} \DataTypeTok{FormElem}

\OtherTok{intForm ::} \DataTypeTok{Form} \DataTypeTok{Int}
\NormalTok{intForm }\FunctionTok{=}\NormalTok{ liftAlt intElem}
\end{Highlighting}
\end{Shaded}

And now we get the ability to represent forms with multiple options with
\texttt{\textless{}\textbar{}\textgreater{}}:

\begin{Shaded}
\begin{Highlighting}[]
\OtherTok{eitherInt ::} \DataTypeTok{Form}\NormalTok{ (}\DataTypeTok{Either} \DataTypeTok{Int} \DataTypeTok{Int}\NormalTok{)}
\NormalTok{eitherInt }\FunctionTok{=}\NormalTok{ (}\DataTypeTok{Left} \FunctionTok{<$>}\NormalTok{ intForm) }\FunctionTok{<|>}\NormalTok{ (}\DataTypeTok{Right} \FunctionTok{<$>}\NormalTok{ intForm)}
\end{Highlighting}
\end{Shaded}

From multiple forms, with \texttt{choice}:

\begin{Shaded}
\begin{Highlighting}[]
\OtherTok{oneOfMany ::} \DataTypeTok{Form} \DataTypeTok{Int}
\NormalTok{oneOfMany }\FunctionTok{=}\NormalTok{ choice [intForm1, intForm2, intForm3]}
\end{Highlighting}
\end{Shaded}

And create ``optional'' entries:

\begin{Shaded}
\begin{Highlighting}[]
\OtherTok{optionalInt ::} \DataTypeTok{Form}\NormalTok{ (}\DataTypeTok{Maybe} \DataTypeTok{Int}\NormalTok{)}
\NormalTok{optionalInt }\FunctionTok{=}\NormalTok{ optional intForm}
\end{Highlighting}
\end{Shaded}

We get all of these capabilities \emph{for free}! All we did was \emph{define a
single form element}, and \texttt{Alt} gives us the ability to combine them with
\texttt{\textless{}*\textgreater{}}/\texttt{\textless{}\$\textgreater{}}, create
optional form items with \texttt{optional}, and create multiple form options
with \texttt{\textless{}\textbar{}\textgreater{}}!

\end{document}
